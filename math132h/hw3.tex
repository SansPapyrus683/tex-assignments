\documentclass[12pt]{article}

% a template that a friend gave, it's worked well enough for me
% i have added some packages and stuff that have proved useful

\usepackage{fancyhdr}
\usepackage{tipa}
\usepackage{fontspec}
\usepackage{amsfonts}
\usepackage{enumitem}
\usepackage[margin=1in]{geometry}
\usepackage{graphicx}
\usepackage{float}
\usepackage{amsmath}
\usepackage{braket}
\usepackage{amssymb}
\usepackage{booktabs}
\usepackage{hyperref}
\usepackage{mathtools}
\usepackage{xcolor}
\usepackage{float}
\usepackage{algpseudocodex}
\usepackage{titlesec}
\usepackage{bbm}
\usepackage{pythonhighlight}

\pagestyle{fancy}
\fancyhf{} % sets both header and footer to nothing
\lhead{Kevin Sheng}
\setmainfont{Comic Neue}
\renewcommand{\headrulewidth}{1pt}
\setlength{\headheight}{0.75in}
\setlength{\oddsidemargin}{0in}
\setlength{\evensidemargin}{0in}
\setlength{\voffset}{-.5in}
\setlength{\headsep}{10pt}
\setlength{\textwidth}{6.5in}
\setlength{\headwidth}{6.5in}
\setlength{\textheight}{8in}
\renewcommand{\headrulewidth}{0.5pt}
\renewcommand{\footrulewidth}{0.3pt}
\setlength{\textwidth}{6.5in}
\usepackage{setspace}
\usepackage{multicol}
\usepackage{float}
\setlength{\columnsep}{1cm}
\setlength\parindent{24pt}
\usepackage [english]{babel}
\usepackage [autostyle, english = american]{csquotes}
\MakeOuterQuote{"}

\setlength{\parskip}{6pt}
\setlength{\parindent}{0pt}

\titlespacing\section{0pt}{12pt plus 4pt minus 2pt}{0pt plus 2pt minus 2pt}
\titlespacing\subsection{0pt}{12pt plus 4pt minus 2pt}{0pt plus 2pt minus 2pt}
\titlespacing\subsubsection{0pt}{12pt plus 4pt minus 2pt}{0pt plus 2pt minus 2pt}

\hypersetup{colorlinks=true, urlcolor=blue}

\newcommand{\correction}[1]{\textcolor{red}{#1}}


\rhead{Math 132H}

\makeatletter
\def\@seccntformat#1{%
  \expandafter\ifx\csname c@#1\endcsname\c@section\else
  \csname the#1\endcsname\quad
  \fi}
\makeatother

\DeclareMathOperator{\cis}{cis}
\newcommand{\lra}{\xLeftrightarrow}
\newcommand{\ra}{\xRightarrow}

\begin{document}

All exercises here are in Chapter 2.

\section{Exercise 3}

Following the hint, let's integrate over the following three curves:
\begin{enumerate}
  \item $0$ to $R$ on $\R$: $\int_{0}^{R} e^{-Ax}\,dx=\frac{1}{A}$ as $R \to \infty$
  \item $0$ to $\omega$ on the arc of radius $R$: $\int_{0}^{\omega} e^{-ARe^{i\theta}} \cdot Rie^{i\theta}\,d\theta$
  \item $R$ to $0$ on $\theta=\omega$: $-\int_{0}^{R} e^{-Axe^{i\omega}}e^{i\omega}\,dx$
\end{enumerate}
Here, $A=\sqrt{a^2+b^2}$, and $\omega$ is the unique angle between
$-\frac{\pi}{2}$ and $\frac{\pi}{2}$ s.t. $\cos \omega = \frac{a}{A}$ and $\sin \omega = \frac{b}{A}$.
$e^{-Ax}$ is entire, so these three integrals always sum to $0$.

The second integral tends to zero.
To see this, note that
\begin{align*}
  \left|\int_{\partial C} e^{-Az}\,dz\right|
   & \le R|\omega| \cdot \sup_{\theta \in [0, \omega]} \left|e^{-ARe^{i\theta}}\right|               \\
   & = R|\omega| \cdot \exp\left(\sup_{\theta \in [0, \omega]} \Re\left(-ARe^{i\theta}\right)\right) \\
   & = R|\omega| \cdot \exp\left(\sup_{\theta \in [0, \omega]} -AR\cos(\theta)\right)                \\
   & = R|\omega| \cdot \exp\left(-AR\cos(\theta)\right)
\end{align*}
and since exponentials dominate polynomials, this term goes to $0$ as $R \to \infty$.

As for the third integral,
\begin{align*}
  -\int_{0}^{R} e^{-Axe^{i\omega}}e^{i\omega}\,dx
   & = -e^{i\omega} \int_{0}^{R} e^{-Axe^{i\omega}}\,dx                                                                    \\
   & = -e^{i\omega} \int_{0}^{R} e^{-Ax\cos(\omega)}e^{-iAx\sin(\omega)}\,dx                                               \\
   & = -\left(\frac{a}{A}+\frac{bi}{A}\right) \cdot \int_{0}^{R} e^{-ax}(\cos(-Ax\sin(\omega))+i\sin(-Ax\sin(\omega)))\,dx \\
   & = -\left(\frac{a}{A}+\frac{bi}{A}\right) \cdot \int_{0}^{R} e^{-ax}(\cos(bx)-i\sin(bx))\,dx                           \\
   & = -\frac{a+bi}{A}\left(\int_{0}^{R} e^{-ax}\cos(bx)\,dx-i\int_{0}^{R} e^{-ax}\sin(bx)\,dx\right)
\end{align*}
Taking $R$ to $\infty$ makes the two integrals inside the parenthesis are exactly what we want.

Putting it all together gives
\[\frac{1}{A}+0-\frac{a+bi}{A}\left(\int_{0}^{R} e^{-ax}\cos(bx)\,dx-i\int_{0}^{R} e^{-ax}\sin(bx)\,dx\right)=0\]
which when solved gets us
\begin{gather*}
  \int_{0}^{R} e^{-ax}\cos(bx)\,dx = \frac{a}{\sqrt{a^2+b^2}} \\
  \int_{0}^{R} e^{-ax}\sin(bx)\,dx = \frac{b}{\sqrt{a^2+b^2}}
\end{gather*}

\pagebreak

\section{Exercise 4}

Consider the integral of $\exp(-\pi z^2)$ around the following lines:
\begin{itemize}
  \item $-R$ to $R$: $\int_{-R}^{R} \exp(-\pi x^2)\,dx=1$ as $R \to \infty$.
        We were told we can use this fact on the test, so surely we can use it here.
  \item $R$ to $R+i\xi$: $i\xi \int_{0}^{1} \exp(-\pi(R+ix\xi)^2)\,dx$
  \item $R+i\xi$ to $-R+i\xi$: $-\int_{-R}^{R} \exp(-\pi(x+\xi i)^2)\,dx$
  \item $-R+i\xi$ to $-R$: $-\int_{0}^{\xi} \exp(-\pi(-R+ix)^2)\,dx$
\end{itemize}
Since $\exp(-\pi z^2)$ is entire, these four integrals sum to $0$.

Now consider the second integral:
\[\int_{0}^{\xi} \exp(-\pi(R+ix)^2)\,dx
  =\int_{0}^{\xi} \exp(-\pi R^2 - \pi 2Rix + \pi x^2)\,dx\]
Notice that for all $\epsilon > 0$ we can find an $R$ s.t. past it the magnitude of the
integrand is always less than $\epsilon$.
In other words, $\lim_{R \to \infty} \int_{0}^{\xi} \exp(-\pi(R+ix)^2)\,dx = 0$.

This is because the argument of the exponential is a polynomial in terms of $R$.
As $R$ grows large, the $-\pi R^2$ term dominates, and that term goes to $-\infty$,
making $\exp(\cdots) \to 0$.

By a virtually identical process, the fourth integral also converges to $0$ as $R \to \infty$.

Now, given that the second and fourth integrals tend to $0$, we finally have
\begin{align*}
             & 1 -\int_{-\infty}^{\infty} \exp(-\pi(x+\xi i)^2)\,dx=0                                                          \\
  \implies{} & 1 - e^{\pi \xi^2} \int_{-\infty}^{\infty} \exp(-\pi x^2 + 2\pi x\xi i)\,dx=0                                    \\
  \implies{} & \int_{-\infty}^{\infty} e^{-\pi x^2}e^{2\pi i x \xi}\,dx = \frac{1}{e^{\pi \xi^2}} = e^{-\pi \xi^2}\quad\square
\end{align*}

\pagebreak

\section{Exercise 8}

Given that we only consider $z \in \R$, $\overline{\{z': |z'-z| < 0.5\}} \subseteq \Omega$.

Also, node that on the boundary of this circle, the largest achievable magnitude is $|z|+0.5$.

Thus, by the inequality given in the premise,
\[\norm{f}_C \le A(1+(|z|+0.5))^\eta\]

Since $f$ is holomorphic, by Cauchy's Inequalities
\begin{align*}
  f^{(n)}(z)
   & \le \frac{n!}{0.5^n}\norm{f}_C                          \\
   & \le \frac{n!}{0.5^n}A(1+(|z|+0.5))^\eta                 \\
   & \le \frac{n!}{0.5^n}A(1.5+1.5|z|)^\eta                  \\
   & \le \frac{n!}{0.5^n}A \cdot 1.5^\eta \cdot (1+|z|)^\eta \\
   & = A_n(1+|z|)^\eta\quad\square
\end{align*}
Since we've succesfully defined $A_n$ in terms of $n$ and other constants alone.

\pagebreak

\section{Exercise 13}

I'll just use that $c_n n! = f^{(n)}(z_0)$ without proof,
since it follows basically straight from the power series derivation.
It was also given directly in the hint, so...

For all $z \in [0, 1]\ \exists n \in \N: c_n=0$, so by the above formula $f^{(n)}(z)=0$.

$[0, 1]$ isn't countable, so $\exists n: \in \N$ s.t. $f^{(n)}(z)=0$ for uncountably many $z$ in the interval.
Call the set of these $z$ $S$ for convenience.

Now lemme prove that for any uncountable subset $S$ of $[0, 1]$,
\[\exists x \in [0, 1]:\ \forall r > 0\ B_r(x) \cap (S \setminus \{x\}) \ne \varnothing\]
where $B_r(x)$ is just the ball of $x$ but limited to $[0, 1]$.

To show this, BWOC say $\forall x \in [0, 1]\ \exists r > 0: B_r(x) \cap (S \setminus \{x\}) = \varnothing$.

The interval's compact, so cover $x$ with finitely many of these balls.
This gives us $x_i$ and $r_i$ s.t.
\begin{align*}
  \varnothing
   & = \bigcup_{i=1}^n B_{r_i}(x_i) \cap (S \setminus \{x_i\})              \\
   & = \bigcup_{i=1}^n B_{r_i}(x_i) \cap (S \setminus \{x_1, \cdots, x_n\}) \\
   & = (S \setminus \{x_1, \cdots, x_n\}) \cap \bigcup_{i=1}^n B_{r_i}(x_i) \\
   & = (S \setminus \{x_1, \cdots, x_n\}) \cap [0, 1]
\end{align*}
which is a contradiction.

With this subpart proven, we can now find a sequence $a_n \subseteq S$ that converges to something
that isn't in the sequence itself.

As proven in class, if $f(a_i)=0$ for all $a_i$ in the sequence, then it must
be $0$ throughout the entire domain.
Applying this to our current situation yields that $f^{(n)}(z)=0\ \forall z$,
which forces $f(z)$ to be some kind of polynomial. $\square$

\pagebreak

\section{Exercise 15}

Consider the extension outside of $\overline{\mathbb{D}}$ defined by
$f(z)=\frac{1}{\overline{f(1/\bar{z})}}$.

Since $f$ is non-vanishing inside and on the disc, this extension is indeed well-defined.

Now let's examine $f$'s behavior on the circle $|z|=1$.
If we use the alternate formula,
\[\frac{1}{\overline{f(1/\bar{z})}}=\frac{1}{\overline{f(z)}}=f(z)\]
where the second equality is because $|f(z)|=1$.
This shows that the two versions of $f$ agree on the circle.

To prove that this new definition is holomorphic outside the circle, notice that
\begin{align*}
  \frac{1}{\overline{f(1/\bar{z})}}
   & = \frac{1}{\overline{\sum_{n=0}^{\infty} a_n \frac{1}{\bar{z}^n}}} \\
   & = \frac{1}{\sum_{n=0}^{\infty} \overline{a_n} \frac{1}{z^n}}       \\
   & = \frac{1}{g\left(\frac{1}{z}\right)}
\end{align*}
where $g$ is some analytic function.
Analytic functions are closed under chaining and division, so this new function
is still analytic and therefore $f$ is holomorphic outside $\mathbb{D}$.

Then, by a corollary of the symmetry principle we proved in class, this
shows that the "union" of $f$'s inside and outside is holomorphic.

Given it's entire and bounded (since it's nonvanishing in $\mathbb{D}$),
which means it must be constant everywhere, including in $\mathbb{D}$. $\square$

\pagebreak

\section{Problem 3a}

Any triangle in the complex plane can be decomposed into isosceles right triangles
whose sides are parallel to the coordinate axes.
Given this, it STP that the integral around this subclass of triangles is always going to be $0$.

The idea is to continuously fill up the interior of a triangle with smaller
and smaller squares until it's completely filled up, like so:
\begin{center}
  \includegraphics[width=6cm]{img/hw1/construction}
\end{center}
Please excuse the god-awful quality.

The sum of the integral along all these squares is $0$ and is equivalent to the
integral around their outer border.
As can be seen, the integral of the sides adjacent to the right angle
are obviously the same, but the same cannot be said for the hypotenuse.

To see this, first note that since the right triangle is closed and bounded in $\C$,
it's compact, making $f$ uniformly continous on it.

Then, note that at each step of breaking it down, the difference in integrals
can always be encapsulated by counterclockwise integrals around these tiny triangles:
\begin{center}
  \includegraphics[width=4cm]{img/hw1/err}
\end{center}

The bases correspond to a clockwise version of the square contour, while
the hypotenuse maps to the standard counterclockwise contour of the triangle.

WLOG let the coordinate of the lower left coordinate be $0$.
Since $f$ is UC, we an choose $a$ small enough s.t. $|f(z)-f(0)| < \epsilon$
for any $z$ in the triangle.

If we let the side length of these triangles be $a$, then the error
of a single one of these triangles is
\begin{align*}
        & \left|\left(\int_{0}^{a} f(t)\,dt - \int_{0}^{a} f(it)i\,dt\right) - \int_{0}^{a} f(t-i(a-t))(1-i)\,dt\right| \\
  ={}   & \left|\int_{0}^{a} f(t)-f(t-i(a-t))\,dt-i\int_{0}^{a} f(it)-f(t-i(a-t))\,dt\right|                            \\
  \le{} & \int_{0}^{a} \epsilon\,dt + \int_{0}^{a} \epsilon\,dt                                                         \\
        & = 2a\epsilon
\end{align*}
When $a$ becomes small, the difference of function values does too,
and so the overall error specified by this tiny triangle becomes small.

The sum of $a$ over all the triangles is finite as it's just the length of the square,
and so we can indeed make the integral of the square approximation equivalent too that of the triangle.

Since this approximation by squares is both equivalent to the integral
of the right triangle and $0$ by the premise, the integral around the right triangle
must also be $0$. $\square$

\pagebreak

\section{Problem 4}

\subsection{A Bounded Component Must Exist}

First let me show that $\C \setminus K$ has to have a bounded component.

BWOC let's say we could split $\C \setminus K$ into two unbounded components, $A$ and $B$.

Then, consider $z \in K$ and $r > 0$ s.t. $K \subseteq B_r(z)$.
$A$ and $B$ are both unbounded, so $A \setminus B_r(z)$ and $B \setminus B_r(z)$ are both nonempty.
Taking away elements from a set never connects any two sets, so notice that
\begin{align*}
  (A \setminus B_r(z)) \cup (B \setminus B_r(z))
   & = (A \cup B) \setminus B_r(z)     \\
   & = \C \setminus K \setminus B_r(z) \\
   & = \C \setminus B_r(z)
\end{align*}
indicating that we've split $\C \setminus B_r(z)$, a connected subset of $\C$,
into two disconnected components.
Contradiction.

\subsection{Actual Proof}

Take the domain $\Omega$ that's surrounded on all sides by $K$, which we just proved exists.

Consider the function $f(z)=\frac{1}{z-z_0}$, where $z_0 \in \Omega$,
and BWOC let us be able to approximate it uniformly on $K$ with polynomials.

In other words, $\forall \epsilon > 0\ \exists p(z): |p(z)-f(z)| < \epsilon\ \forall z \in K$.

$K$ is bounded, so $\sup_{z \in K} |z-z_0| < \infty$ and
\begin{align*}
             & |p(z)-f(z)| < \epsilon                       \\
  \implies{} & \left|p(z)-\frac{1}{z-z_0}\right| < \epsilon \\
  \implies{} & |p(z)(z-z_0)-1| < \epsilon|z-z_0|
\end{align*}
If we take $\epsilon$ small enough, then the RHS becomes less than $1$.

Now, we have that $\exists p(z)$ s.t. for all $z \in K$, $|p(z)(z-z_0)-1| < 1$.

Consider the function $g(z)=p(z)(z-z_0)-1$, which is entire.
By the maximum modulus principle, it cannot have a maximum on $\Omega$,
However, $\overline{\Omega}$ is compact, so there must be a maximum there.

In other words, $\exists k \in \partial \Omega \subseteq K: |g(z)| < |g(k)| < 1\ \forall z \in \Omega$.

This holds true even for $z_0$, developing into that $|g(z_0)|=|p(z_0)(z_0-z_0)-1|=1 < 1$, a contradiction. $\square$

\end{document}
