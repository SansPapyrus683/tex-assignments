\documentclass[12pt]{article}

% a template that a friend gave, it's worked well enough for me
% i have added some packages and stuff that have proved useful

\usepackage{fancyhdr}
\usepackage{tipa}
\usepackage{fontspec}
\usepackage{amsfonts}
\usepackage{enumitem}
\usepackage[margin=1in]{geometry}
\usepackage{graphicx}
\usepackage{float}
\usepackage{amsmath}
\usepackage{braket}
\usepackage{amssymb}
\usepackage{booktabs}
\usepackage{hyperref}
\usepackage{mathtools}
\usepackage{xcolor}
\usepackage{float}
\usepackage{algpseudocodex}
\usepackage{titlesec}
\usepackage{bbm}
\usepackage{pythonhighlight}

\pagestyle{fancy}
\fancyhf{} % sets both header and footer to nothing
\lhead{Kevin Sheng}
\setmainfont{Comic Neue}
\renewcommand{\headrulewidth}{1pt}
\setlength{\headheight}{0.75in}
\setlength{\oddsidemargin}{0in}
\setlength{\evensidemargin}{0in}
\setlength{\voffset}{-.5in}
\setlength{\headsep}{10pt}
\setlength{\textwidth}{6.5in}
\setlength{\headwidth}{6.5in}
\setlength{\textheight}{8in}
\renewcommand{\headrulewidth}{0.5pt}
\renewcommand{\footrulewidth}{0.3pt}
\setlength{\textwidth}{6.5in}
\usepackage{setspace}
\usepackage{multicol}
\usepackage{float}
\setlength{\columnsep}{1cm}
\setlength\parindent{24pt}
\usepackage [english]{babel}
\usepackage [autostyle, english = american]{csquotes}
\MakeOuterQuote{"}

\setlength{\parskip}{6pt}
\setlength{\parindent}{0pt}

\titlespacing\section{0pt}{12pt plus 4pt minus 2pt}{0pt plus 2pt minus 2pt}
\titlespacing\subsection{0pt}{12pt plus 4pt minus 2pt}{0pt plus 2pt minus 2pt}
\titlespacing\subsubsection{0pt}{12pt plus 4pt minus 2pt}{0pt plus 2pt minus 2pt}

\hypersetup{colorlinks=true, urlcolor=blue}

\newcommand{\correction}[1]{\textcolor{red}{#1}}


\rhead{Math 132H}

\DeclareMathOperator{\cis}{cis}
\newcommand{\lra}{\xLeftrightarrow}
\newcommand{\ra}{\xRightarrow}

\begin{document}

All exercises here are in Chapter 1.

\section{Exercise 1}

\begin{enumerate}[label=(\alph*)]
  \item The line perpendicular the to line segment formed by $z_1$ and $z_2$.
  \item Unit circle.
  \item In the xy-plane, this would be the same as the equation $x=3$. \label{item:vert_line}
  \item Everything strictly to the right of the vertical line at $c$.
  \item some kinda zone to the left or right of an arbitrary line idk man
  \item A horizontal parabola represented by the equation $x=\frac{y^2-1}{2}$.
  \item Same as \ref{item:vert_line}, only with $y=c$ (a horizontal line) instead.
\end{enumerate}

\section{Exercise 2}

I think it's easier if I let $z=a+bi$ and $w=c+di$, where all the components are real.

With this,
\begin{align*}
  \frac{1}{2}[(z, w) + (w, z)]
   & = \frac{1}{2}((ac+bd+i(bc-ad))+(ac+bd+i(ad-bc))) \\
   & = \frac{1}{2}(2ac+2bd)                           \\
   & = ac+bd                                          \\
   & = \braket{z, w}                                  \\
   & = \Re(ac+bd)                                     \\
   & = \Re((z, w))\quad\square
\end{align*}

\pagebreak

\section{Exercise 4}

First let's see if $i \succ 0$ works.

\begin{align*}
             & i \succ 0                 \\
  \implies{} & i \cdot i \cdot i \succ 0 \\
  \implies{} & -i \succ 0                \\
  \implies{} & i + (-i) \succ 0          \\
  \implies{} & 0 \succ 0
\end{align*}
which violates the first property specified.

By a similar chain of reasoning, $i \prec 0$ doesn't work either.
Thus, no ordering can exist in $\C$ that satisfies all the properties specified. $\square$

\section{Exercise 7}

\subsection{Part A}\label{sec:7a}

First lemme prove the case where $z \in \R$.

First, notice that for all $z \in \R$ and $w \in \C$ with magnitudes at most $1$,
\[(z^2-1)(|w|^2-1) \ge 0\]
where equality holds iff $|z|=1$ or $|w|=1$.

We can then manipulate this like so:
\begin{align*}
             & (z^2-1)(|w|^2-1) \ge 0                          \\
  \implies{} & 1-z^2-|w|^2+z^2|w|^2 \ge 0                      \\
  \implies{} & 1+z^2|w|^2 \ge z^2+|w|^2                        \\
  \implies{} & 1-zw-z\bar{w}+w|w| \ge z^2-zw-z\bar{w}+w\bar{w} \\
  \implies{} & (1-zw)(1-\bar{w}) \ge (z-w)(z-\bar{w})
\end{align*}
where equality again only holds with that previous condition.

Then, going back to our original inequality,
\begin{align*}
  \left|\frac{w-z}{1-\bar{w}z}\right|
   & = \left|\frac{(w-z)(1-wz)}{(1-\bar{w}z)(1-wz)}\right| \\
   & \le \left|\frac{(w-z)(1-wz)}{(z-w)(z-\bar{w})}\right| \\
   & = \left|\frac{1-wz}{z-\bar{w}}\right|                 \\
   & = \left|\frac{1-\bar{w}z}{w-z}\right|
\end{align*}
where the last line is because taking conjugates and flipping signs
doesn't do anything to the magnitude.

Lemme now just write the term as $a$ for convenience.

If equality doesn't hold ($|z|, |w| < 1$), then
\[|a| < \left|\frac{1}{a}\right| \implies |a|^2 < 1 \implies |a| < 1\]
and if it does hold (either $|z|$ or $|w|$ are $1$), by similar reasoning we have $|a|=1$.

Now if we let $z=re^{i\theta}$, then our case with real $z$ still proves this since
\begin{align*}
  \left|\frac{w-z}{1-\bar{w}z}\right|
   & = \left|\frac{w-re^{i\theta}}{1-\bar{w}re^{i\theta}}\right|                           \\
   & = \left|e^{i\theta} \cdot \frac{we^{-i\theta}-r}{1-r \cdot \bar{w}e^{i\theta}}\right| \\
   & = \left|\frac{we^{-i\theta}-r}{1-r \cdot \bar{w}e^{i\theta}}\right|                   \\
   & = \left|\frac{w'-r}{1-r \cdot \bar{w}'}\right|
\end{align*}
where $w'=we^{-i\theta}$ and $r$ is still just our real number.

\pagebreak

\subsection{Part B}

The disk part of (i) was already proven by \ref{sec:7a}.
It's also holomorphic since it's a composition of sums, multiplications, and divisions,
and the denominator isn't $0$ anywhere as $|\bar{w}z| < 1$.

We also have
\begin{align*}
  F(0) & = \frac{w-0}{1-\bar{w} \cdot 0} & F(w) & = \frac{w-w}{1-\bar{w}w} \\
       & = w                             &      & = 0
\end{align*}
proving (ii).

(iii) is also already proven by \ref{sec:7a}.

Finally, for (iv), notice that
\begin{align*}
  F(F(z))
  &= \frac{w-\frac{w-z}{1-\bar{w}z}}{1-\bar{w}\frac{w-z}{1-\bar{w}z}} \\
  &= \frac{w(1-\bar{w}z)-w+z}{1-\bar{w}z} \\
  &= \frac{w-w\bar{w}z-w+z}{1-\bar{w}z-\bar{w}w+z\bar{w}} \\
  &= \frac{z-w\bar{w}z}{1-w\bar{w}} \\
  &= z
\end{align*}
so the existence of an inverse of $F$ ($F$ itself) implies that it's a bijection.

\section{Exercise 13}

\subsection{Part A}

$\Re(f)$ is constant, so $\frac{\partial u}{\partial x} = \frac{\partial u}{\partial y} = 0$.

Since $f$ is holomorphic, it satisfies the CR equations, thus forcing
$\frac{\partial v}{\partial x} = \frac{\partial v}{\partial y} = 0$ as well.

The derivative is $0$ everywhere, so $f$ must be constant.

\subsection{Part B}

Basically the same as part A.

\subsection{Part C}

If $|f|$ is constant, then $u(z)^2+v(z)^2$ has to have zero derivative.

In other words,
\begin{gather*}
  2\frac{\partial u}{\partial x}+2\frac{\partial v}{\partial x}=0 \\
  2\frac{\partial u}{\partial y}+2\frac{\partial v}{\partial y}=0 
\end{gather*}
We can then chain equalities to get
\begin{align*}
  \frac{\partial u}{\partial x}
  &= \frac{\partial v}{\partial y} \\
  &= -\frac{\partial u}{\partial y} \\
  &= \frac{\partial v}{\partial x} \\
  &= -\frac{\partial u}{\partial x}
\end{align*}
which forces $\frac{\partial u}{\partial x}$ (and the rest of the partials) to be $0$. $\square$

\end{document}
