\documentclass[12pt]{article}

% a template that a friend gave, it's worked well enough for me
% i have added some packages and stuff that have proved useful

\usepackage{fancyhdr}
\usepackage{tipa}
\usepackage{fontspec}
\usepackage{amsfonts}
\usepackage{enumitem}
\usepackage[margin=1in]{geometry}
\usepackage{graphicx}
\usepackage{float}
\usepackage{amsmath}
\usepackage{braket}
\usepackage{amssymb}
\usepackage{booktabs}
\usepackage{hyperref}
\usepackage{mathtools}
\usepackage{xcolor}
\usepackage{float}
\usepackage{algpseudocodex}
\usepackage{titlesec}
\usepackage{bbm}
\usepackage{pythonhighlight}

\pagestyle{fancy}
\fancyhf{} % sets both header and footer to nothing
\lhead{Kevin Sheng}
\setmainfont{Comic Neue}
\renewcommand{\headrulewidth}{1pt}
\setlength{\headheight}{0.75in}
\setlength{\oddsidemargin}{0in}
\setlength{\evensidemargin}{0in}
\setlength{\voffset}{-.5in}
\setlength{\headsep}{10pt}
\setlength{\textwidth}{6.5in}
\setlength{\headwidth}{6.5in}
\setlength{\textheight}{8in}
\renewcommand{\headrulewidth}{0.5pt}
\renewcommand{\footrulewidth}{0.3pt}
\setlength{\textwidth}{6.5in}
\usepackage{setspace}
\usepackage{multicol}
\usepackage{float}
\setlength{\columnsep}{1cm}
\setlength\parindent{24pt}
\usepackage [english]{babel}
\usepackage [autostyle, english = american]{csquotes}
\MakeOuterQuote{"}

\setlength{\parskip}{6pt}
\setlength{\parindent}{0pt}

\titlespacing\section{0pt}{12pt plus 4pt minus 2pt}{0pt plus 2pt minus 2pt}
\titlespacing\subsection{0pt}{12pt plus 4pt minus 2pt}{0pt plus 2pt minus 2pt}
\titlespacing\subsubsection{0pt}{12pt plus 4pt minus 2pt}{0pt plus 2pt minus 2pt}

\hypersetup{colorlinks=true, urlcolor=blue}

\newcommand{\correction}[1]{\textcolor{red}{#1}}


\rhead{Math 132H}

\DeclareMathOperator{\cis}{cis}
\DeclareMathOperator{\res}{res}
\newcommand{\lra}{\xLeftrightarrow}
\newcommand{\ra}{\xRightarrow}

\begin{document}

All exercises here are in Chapter 3.

\section{Exercise 13}

\subsection{Dogwater Edge Case}

If $\epsilon \ge 1$, then for some radius $r$
\[f(z) \le A|z-z_0|^{-1+\epsilon} \le Ar^{-1+\epsilon}\]
which bounds $f$ in a disc and forces a removable singularity.

\subsection{Actual Proof}

Now we assume $0 < \epsilon < 1$.

BWOC let $f$ not have a removable singularity.

In other words, $f(z)=\sum_{j=1}^{\infty} a_{-j}(z-z_0)^{-j}+h(z)$, where $h$ is some holomorphic function
and there's at least one $a_{-j}$ that's nonzero.

Then, limiting ourselves to some small radius $r$, $h(z)$ is bounded.
If we let $H$ be the upper bound for $h$'s magnitude, then
\begin{align*}
             & |f(z)| \le A|z-z_0|^{-1+\epsilon}                                                  \\
  \implies{} & \left|\sum_{j=1}^{\infty} a_{-j}(z-z_0)^{-j}\right| \le A|z-z_0|^{-1+\epsilon} + H \\
  \implies{} & \begin{aligned}[t]
                 \left|\sum_{j=1}^{\infty} a_{-j}(z-z_0)^{-j+1}\right|
                  & \le (A|z-z_0|^{-1+\epsilon} + H) \cdot |z-z_0| \\
                  & \le A|z-z_0|^\epsilon + H|z-z_0|               \\
                  & \le Ar^\epsilon + Hr
               \end{aligned}
\end{align*}
The RHS is fixed, but the LHS is clearly unbounded, which is a contradiction. $\square$

\pagebreak

\section{Exercise 14}

$f$ is entire, so we can write $f(z)=\sum_{j=0}^{\infty} a_jz^j$ for all $z \in \C$.

\subsection{Finite Powers Case}

Suppose there were a finite amount of coefficients in the power series, forcing $f$ to be a polynomial.

We'll prove the contrapositive by assuming that $\deg(f) \ge 2$.

By the FTA, $f$ has at least $2$ roots.
They can't be different (or else $f$ wouldn't be injective), so $f(z)=(z-c)^n$ where $n \ge 2$.
However, even this function isn't injective since we can let
$(x-c)^n=1$ and get $n$ values of $x$ by roots of unity.

Thus, $\deg(f) \ge 2$ bans $f$ from being injective.

\subsection{Infinite Powers Case}

Now let the power series have an infinite amount of nonzero coefficients.

Consider the function $g(z)=f\left(\frac{1}{z}\right)$, which is holomorphic everywhere except $z=0$.

Since $g(z)=\sum_{j=0}^{\infty} a_jz^{-j}$, it has an essential singularity at $0$ too.

Then just use Big Picard to see that $g$ (and $f$ by extension) can't be injective. $\square$

\pagebreak

\section{Exercise 15}

WHY ARE THERE FOUR PROOFS IN THIS PROBLEM BRO OH MY GOD

\subsection{Part A}

What the bound is essentially saying is that $|f(z)| \le A|z|^k+B$.

By the Cauchy inequalities,
\begin{align*}
  \left|f^{(j)}(z_0)\right|
   & \le \frac{j!}{R^j} \sup_{|z-z_0|=R} |f(z)|       \\
   & \le \frac{j!}{R^j} \sup_{|z-z_0|=R} A|z-z_0|^k+B \\
   & \le \frac{j!}{R^j}\left(A(R+|z_0|)^k+B\right)
\end{align*}
When $j > k$, letting $R \to \infty$ forces $f^{j}(z_0)=0\ \forall 0$,
so the power series of $f$ cannot have any terms past $a_k(z-z_0)^k$. $\square$

\subsection{Part B}

\subsubsection{Stacked Products}

Let $\phi - \theta = \delta$.
With this, let $n \in \N$ be s.t. $\frac{2\pi}{n} < \frac{\delta}{2}$.
Assign $\alpha=\frac{2\pi}{n}$ for convenience.

Notice that $\forall z \in \mathbb{D}\ \exists 0 \le x \le n: \arg\left(ze^{ix\alpha}\right) \in (\theta, \varphi)$.
In more natural words, this is a bound on how many rotations of degree $\alpha$ we
have to do in order to bring any point in the disc into the given sector.

Now consider the function
\[g(z)=\prod_{j=0}^{n-1} f\left(ze^{ij\alpha}\right)\]
I propose that this function converges uniformly to $0$ as $|z| \to 1$ on the entire unit disc.
To prove this, fix an $\epsilon$ as always.

Since $f$ converges uniformly in the sector given by the problem, we can choose an $r < 1$ s.t.
$r < |z| < 1 \land \arg z \in [\theta, \varphi] \implies |f(z)| < \epsilon$.

Now $f(z)$ is bounded (say by some constant $M$),
so we can bound $g$ in the region \\ $\{z: r < |z| < 1\}$ like so:
\[|g(z)| = \prod_{j=0}^{n-1} \left|f\left(ze^{ij\alpha}\right)\right| \le \epsilon M^{n-1}\]
The $\epsilon$ is because of how we chose $n$ above.
At least one of the $ze^{ij\alpha}$s has to fall into the good sector,
and their magnitude is still greater than $r$.

$n$ and $M$ were both fixed before $\epsilon$, so this shows what we wanted.

\subsubsection{That Go to Zero}

Consider the function $h: \overline{\mathbb{D}} \to \C$ defined by the following formula:
\[h(z)=\begin{cases}
  g(z) & |z| < 1 \\
  0 & \text{otherwise}
\end{cases}\]
By what we just proved in the previous section, $h$ is a continous function.

$\mathbb{D}$ is compact, and this combined with the maximum modulus principle
gives us that it must attain a maximum on the boundary.

However, $h$ is $0$ everywhere on $\partial \mathbb{D}$, so $g(z)=0$.

\subsubsection{Going back to \texorpdfstring{$f$}{f}}

Take any ring in $\mathbb{D}$ $\{z: |z|=r\}$.

For every angle $\beta$ in $\left[0, \alpha\right]$,
we can find an $x$ s.t. $f\left(re^{i(x\alpha + \beta)}\right)=0$ since $g\left(e^{ix\alpha}\right)=0$.
All these are distinct since we're adding on to the angle in increments of $\alpha$.

All this means is that there's an uncountable amount of points on the ring that give $0$.

As proven in a previous homework, a compact subset with an uncountably infinite
subset must have a limit point.
Thus, we can obtain a sequence of convergent points
that all evaluate to $0$ and force $f(z)=0$ in its entirety. $\square$

\pagebreak

\subsection{Part C}

Consider the function defined on the unit \textit{disc}
\[f(z)=\prod_{j=1}^{n} z-w_j\]
Notice that $|f(z)|$ is exactly the product of distances that we want.

Also, since $|w_j|=1$, $|f(0)|=1$.

By the maximum modulus principle and the compactness of the unit circle
unioned with the unit disc, the maximum of $f$ must lie on the circle.

In other words, $\exists z: |z|=1 \land |f(z)| \ge |f(0)|=1$, so the must be a point
on the circle whose distance product with all the points has to be at least $1$.

Now, $|f|$ is continuous along the unit circle, and $|f(w_j)|=0$.
Thus, by the IVT, we can say that $\exists z: |z|=|f(z)|=1$. $\square$

\subsection{Part D}

By a stupid deus ex machina, consider the function $e^f$.
$\left|e^f\right|=e^{\Re(f)}$, and since $\Re(f)$ is bounded, $e^f$ is bounded too.
It's also entire, which means it must be constant.

The set of values in $\C$ that satisfy $e^x=c$ for some constant $c$
are all separate, so by continuity of $f$, $f$ must be constant. $\square$

\pagebreak

\section{Exercise 16}

\subsection{Part A}

The TA proved Rouche's Theorem (or at least a version of it), so I'm just gonna use
the textbook's version of it without proof here.

$f$ is nonvanishing on $\partial \mathbb{D}$, and the boundary is compact, so
we can find a positive lower bound for its magnitude on the circle $m$.
In the same vein, we can find an upper bound $G$ on the circle for $g$ as well.

Now notice that for all $0 < \epsilon < \frac{m}{G}$,
\begin{align*}
  |\epsilon g(z)| 
  &< \epsilon G \\
  &< \frac{m}{G} \cdot G \\
  &= m \\
  &< f(z)
\end{align*}
so as long as $\epsilon$ is small enough $\epsilon g(z)$ satisfies the conditions
of Rouche's Theorem.

$f$ has exactly one zero on $\mathbb{D}$, so $f_\epsilon=f+\epsilon g$ must have exactly one zero too.

\subsection{Part B}

Fix an $\epsilon$ small enough as well as an error tolerance $E$.

It suffices to find a $\delta$ s.t. $|\epsilon'-\epsilon| < \delta \implies |z_\epsilon - z_{\epsilon'}| < E$.

Notice that
\[f_{\epsilon'}(z) = (f(z) + \epsilon g(z)) + ((\epsilon' - \epsilon)g(z))\]

$f(z)+\epsilon g(z)$ has a zero at $z_\epsilon$, and we can take a radius $E$ (as long as $E$'s small enough)
s.t. $B_r(z_\epsilon) \subseteq \mathbb{D}$ to make it the only zero inside the ball.

On the compact border of this circle, we can once again upper bound $g$ with $G$
and lower bound $f(z)+\epsilon g(z)$ with $m$.
Forcing $\delta=\frac{m}{G}>0$, if $|\epsilon'-\epsilon| < \delta$ then
\begin{align*}
  |(\epsilon' - \epsilon)g(z)|
  &= |\epsilon' - \epsilon| |g(z)| \\
  &< |\epsilon' - \epsilon| G \\
  &< \frac{m}{G} \cdot G \\
  &< f(z)+\epsilon g(z)
\end{align*}
so we can apply Rouche's Theorem like in the previous part to see that
$f_{\epsilon'}(z)$ must have exactly one zero inside $B_E(z_\epsilon)$.
This zero is actually our $z_{\epsilon'}$, so $|z_{\epsilon'} - z_\epsilon| < E$. $\square$

\pagebreak

\section{Exercise 17}

\subsection{Part A}

I guess we'll first show that $f(z)=0$ has a solution inside $\mathbb{D}$.

BWOC say $f$ was nonvanishing in the entirety of $\mathbb{D}$.
This would make $1/f$ well-defined and holomorphic in $\mathbb{D}$ as well.
Notice that $|1/f|$ on $\partial \mathbb{D}$ is also $1$.

By the maximum modulus principle $|z| < 1 \implies |f(z)| < 1 \implies |1/f(z)| > 1$,
which is a contradiction since $1/f$ must also satisfy the principle.
Thus, $f(z)=0$ must have at least one solution in $\mathbb{D}$.

Now consider the constant function $g(z)=-w$ for any $|w| < 1$.
On $\partial \mathbb{D}$, $|g|=|w| < 1 = |f|$, so the conditions for Rouche's Theorem
are satisfied and $f(z)-w$ must have the same amount of zeros as $f$ in $\mathbb{D}$,
which as we've just shown is nonnegative.

In other words, $f(z)-w=0$ must have at least one solution in $\mathbb{D}$. $\square$

\subsection{Part B}

The argument goes basically the same as in part A.

BWOC say $f$ was nonvanishing in $\mathbb{D}$, once again making $1/f$ well-defined.

Here, on the border $|1/f| \le 1$.
However, $\exists z_0: |f(z_0)| < 1 \implies |1/f(z_0)| > 1$, violating
the maximum modulus principle and creating a contradiction.

After this, we can apply Rouche's Theorem again with $g(z)=-w$ to get the same result. $\square$

\end{document}
