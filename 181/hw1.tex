\documentclass[12pt]{article}

% a template that a friend gave, it's worked well enough for me
% i have added some packages and stuff that have proved useful

\usepackage{fancyhdr}
\usepackage{tipa}
\usepackage{fontspec}
\usepackage{amsfonts}
\usepackage{enumitem}
\usepackage[margin=1in]{geometry}
\usepackage{graphicx}
\usepackage{float}
\usepackage{amsmath}
\usepackage{braket}
\usepackage{amssymb}
\usepackage{booktabs}
\usepackage{hyperref}
\usepackage{mathtools}
\usepackage{xcolor}
\usepackage{float}
\usepackage{algpseudocodex}
\usepackage{titlesec}
\usepackage{bbm}
\usepackage{pythonhighlight}

\pagestyle{fancy}
\fancyhf{} % sets both header and footer to nothing
\lhead{Kevin Sheng}
\setmainfont{Comic Neue}
\renewcommand{\headrulewidth}{1pt}
\setlength{\headheight}{0.75in}
\setlength{\oddsidemargin}{0in}
\setlength{\evensidemargin}{0in}
\setlength{\voffset}{-.5in}
\setlength{\headsep}{10pt}
\setlength{\textwidth}{6.5in}
\setlength{\headwidth}{6.5in}
\setlength{\textheight}{8in}
\renewcommand{\headrulewidth}{0.5pt}
\renewcommand{\footrulewidth}{0.3pt}
\setlength{\textwidth}{6.5in}
\usepackage{setspace}
\usepackage{multicol}
\usepackage{float}
\setlength{\columnsep}{1cm}
\setlength\parindent{24pt}
\usepackage [english]{babel}
\usepackage [autostyle, english = american]{csquotes}
\MakeOuterQuote{"}

\setlength{\parskip}{6pt}
\setlength{\parindent}{0pt}

\titlespacing\section{0pt}{12pt plus 4pt minus 2pt}{0pt plus 2pt minus 2pt}
\titlespacing\subsection{0pt}{12pt plus 4pt minus 2pt}{0pt plus 2pt minus 2pt}
\titlespacing\subsubsection{0pt}{12pt plus 4pt minus 2pt}{0pt plus 2pt minus 2pt}

\hypersetup{colorlinks=true, urlcolor=blue}

\newcommand{\correction}[1]{\textcolor{red}{#1}}


\lhead{406-196-414}
\rhead{CS 181}

\begin{document}

\section{$\binom{3}{2}$-regular Languages}

\subsection*{Part A}

Let the parameters for DFA $M_i=(Q_i, F_i, q_{0i}, \Sigma, \delta_i)$.

Let's define our new DFA with these parameters:
\begin{itemize}[nolistsep]
    \item $Q=Q_1 \times Q_2 \times Q_3$
    \item $F=\{q \in Q \mid \text{exactly 2 states in $q$ are accepting in their machine}\}$
    \item $q'=\left(q_{01}, q_{02}, q_{03}\right)$
    \item $\Sigma$ stays the same
    \item $\delta: Q \times \Sigma \rightarrow Q$ defined as
          $\delta((q_1, q_2, q_3), s)=(\delta_1(q_1, s), \delta_2(q_2, s), \delta_3(q_3, s))$
\end{itemize}

If exactly two out of the three DFAs accept some input string, then this
new DFA $M$ will wind up in a state where exactly two of the three elements
are in their machine's $F$, resulting in $M$ itself accepting too.

\subsection*{Part B}

Say our regular language was implemeneted by a DFA $M$.
We can define the following:
\begin{itemize}[nolistsep]
    \item $M_1=M$
    \item $M_2=M$
    \item $M_3$ is a machine that never accepts.
          This can be done with the following simple DFA construction:
          \begin{itemize}[nolistsep]
              \item $Q=\{q_0\}$
              \item $F=\varnothing$
              \item $q_0=q_0$
              \item $\Sigma$ is the same
              \item $\delta: Q \times \Sigma \rightarrow Q$ defined as $\delta(q, s)=q$
          \end{itemize}
\end{itemize}

Any $L$ that's in the language will get accepted by $M_1$ and $M_2$ and rejected by $M_3$,
while any $L$ that isn't in the language will get rejected by all three.

\pagebreak

\section{Reversed Languages}

Let us construct an NFA $M'$ from the DFA $M$ that does the reversed version of the language,
as it was shown in class that any NFA can be reduced to a DFA.

If $M=(Q, F, q_0, \Sigma, \delta)$, then we can make our NFA as follows:
\begin{itemize}[nolistsep]
    \item $Q'=Q \cup \{q_0'\}$
    \item $F'=\{q_0\}$
    \item $q_0'$ is a new state that I've introduced.
    \item $\Sigma$ stays the same
    \item $\delta': Q' \times \Sigma \rightarrow S \subseteq Q$ defined as
          \[\delta'(q, s)=\begin{cases}
                  \{x \in Q \mid \delta(x, s)=q\} & q \ne q_0'                \\
                  F                               & q=q_0' \land s = \epsilon \\
                  \varnothing                     & \text{otherwise}
              \end{cases}\]
\end{itemize}

On input strings that $M$ accepts, the NFA starts at $q_0'$ and chooses
the states $M$ goes through but in reverse.
Due to the how I've constructed $\delta'$, the path back to $q_0$
will always be valid.

\section{Alternating Languages}

The idea is to make two copies of the original FSM that we have to alternate between.
One copy has all the states as usual, while the other appends a prime to the end.
\begin{itemize}[nolistsep]
    \item $Q'=Q \cup \{q' \mid q \in Q\}$
    \item $F'=F \cup \{f' \mid f \in F\}$
    \item $q_0=q_0$
    \item $\delta': F' \times \Sigma \rightarrow S \subseteq Q'$ defined by
          \begin{gather*}
              \delta'(q, s)=\{\delta(q, s)'\} \\
              \delta'\left(q', s\right)=\begin{cases}
                  \{\delta(q, i) \mid i \in \Sigma\} & s=\epsilon       \\
                  \varnothing                        & \text{otherwise}
              \end{cases}
          \end{gather*}
\end{itemize}

We start off at the initial copy, and then for each input token we consume,
we're \textit{forced} to make another transition in the copy of the FSM
that doesn't consume any input.

Notice that $F'$ has both the original $F$ and prime versions of the states
to account for the last character possibly being either in or not in
the alternating string.

\pagebreak

\section{Halved Languages}

We can go through all the intermediary states the two halves
of the string could meet up at and simulate the machine for each of them.
\begin{itemize}[nolistsep]
    \item $Q'=(Q \times Q \times Q) \cup \{q'_0\}$
    \item $F'=\{(m, a, b) \in Q' \mid a=m \land b \in F\}$
    \item $q_0=q'_0$
    \item $\delta': Q' \rightarrow S \subseteq Q'$ defined by
          \begin{gather*}
              \delta'(q'_0, s)=\begin{cases}
                  \{(m, q_0, m) \mid m \in Q\} & s=\epsilon       \\
                  \varnothing                  & \text{otherwise}
              \end{cases} \\
              \delta'((m, a, b), s)=\{(m, \delta(a, s), \delta(b, i)) \mid i \in \Sigma\}
          \end{gather*}
\end{itemize}

Say the halves of our string are $x$ and $y$ with $|x|=|y|$.
The initial machine accepts $xy$, so after taking $x$ it must end
up at some state $m$, and then after that it'll end up at some $q \in F$.

Our NFA simulates this by iterating over all possible intermediary states $m$.
During transition, the first half is known, so we call $\delta(a, s)$.
The second half could be anything, so $\delta'$ enumerates all possible $i \in \Sigma$
for what the second part of the state could end up being.

\end{document}
