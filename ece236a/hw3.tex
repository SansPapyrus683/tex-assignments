\documentclass[12pt]{article}

% a template that a friend gave, it's worked well enough for me
% i have added some packages and stuff that have proved useful

\usepackage{fancyhdr}
\usepackage{tipa}
\usepackage{fontspec}
\usepackage{amsfonts}
\usepackage{enumitem}
\usepackage[margin=1in]{geometry}
\usepackage{graphicx}
\usepackage{float}
\usepackage{amsmath}
\usepackage{braket}
\usepackage{amssymb}
\usepackage{booktabs}
\usepackage{hyperref}
\usepackage{mathtools}
\usepackage{xcolor}
\usepackage{float}
\usepackage{algpseudocodex}
\usepackage{titlesec}
\usepackage{bbm}

\pagestyle{fancy}
\fancyhf{} % sets both header and footer to nothing
\lhead{Kevin Sheng}
\setmainfont{Comic Neue}
\renewcommand{\headrulewidth}{1pt}
\setlength{\headheight}{0.75in}
\setlength{\oddsidemargin}{0in}
\setlength{\evensidemargin}{0in}
\setlength{\voffset}{-.5in}
\setlength{\headsep}{10pt}
\setlength{\textwidth}{6.5in}
\setlength{\headwidth}{6.5in}
\setlength{\textheight}{8in}
\renewcommand{\headrulewidth}{0.5pt}
\renewcommand{\footrulewidth}{0.3pt}
\setlength{\textwidth}{6.5in}
\usepackage{setspace}
\usepackage{multicol}
\usepackage{float}
\setlength{\columnsep}{1cm}
\setlength\parindent{24pt}
\usepackage [english]{babel}
\usepackage [autostyle, english = american]{csquotes}
\MakeOuterQuote{"}

\setlength{\parskip}{6pt}
\setlength{\parindent}{0pt}

\titlespacing\section{0pt}{12pt plus 4pt minus 2pt}{0pt plus 2pt minus 2pt}
\titlespacing\subsection{0pt}{12pt plus 4pt minus 2pt}{0pt plus 2pt minus 2pt}
\titlespacing\subsubsection{0pt}{12pt plus 4pt minus 2pt}{0pt plus 2pt minus 2pt}

\hypersetup{colorlinks=true, urlcolor=blue}

\newcommand{\correction}[1]{\textcolor{red}{#1}}


\rhead{ECE 236A}

\newcommand*{\vertbar}{\rule[-1ex]{0.5pt}{2.5ex}}
\newcommand*{\horzbar}{\rule[.5ex]{2.5ex}{0.5pt}}

\begin{document}

\begin{enumerate}
      \item The maximization problem is equivalent to the following:
            \begin{gather*}
                  \min_{\lambda} b^T\lambda\text{ s.t.} \\
                  A^T\lambda+c = \mathbf{0} \\
                  -\lambda \le \mathbf{0}
            \end{gather*}
            The Lagrangian is as follows ($x_1 \in \mathbb{R}^n$, $x_2 \in \mathbb{R}^m$):
            \begin{align*}
                  \mathcal{L}(\lambda, x_1, x_2)
                   & =b^T\lambda + x_1^T(A^T\lambda+c) + x_2^T(-\lambda) \\
                   & =(b^T+x_1^TA^T-x_2^T)\lambda+x_1^Tc
            \end{align*}
            $g(x_1, x_2)=x_1^Tc$ when $b^T+x_1^TA^T-x_2^T=\mathbf{0}$ and $-\infty$ otherwise.

            Now we check to see what constraints on $x$ must be necessary for
            $\mathcal{L}$ to be a lower bound:
            \begin{align*}
                  g(x) \le \mathcal{L}(\lambda^*, x_1, x_2)
                   & = b^T\lambda^* + x_1^T(A^T\lambda^*+c) - x_2^T\lambda^* \\
                   & = b^T\lambda^* - x_2^T\lambda^*
            \end{align*}
            For $\mathcal{L}(\lambda^*, x_1, x_2) \le b^T\lambda^*$,
            $x^T\lambda^* \ge 0$, so we need $x_2 \ge \mathbf{0}$.

            The dual of the dual is thus
            \begin{gather*}
                  \max_{x_1, x_2} c^Tx_1\text{ s.t.} \\
                  b+Ax_1-x_2=\mathbf{0} \\
                  x_2 \ge \mathbf{0}
            \end{gather*}
            We still need to make some observations to see that this is equivalent to the primal.

            Note that $x_2$ serves as nothing more than a slack variable of sorts, so
            we can rewrite the two constraints as just
            \[Ax_1 \ge -b \to -Ax_1 \le b\]
            Next, we can make the substitution $x=-x_1$ to turn this constraint into
            \[Ax \le b\]
            $c^Tx_1=-c^Tx$, so the maximization objective then turns back into a minimization one.
            With these observations, we have turned the dual of the dual back into the primal. $\square$

            \pagebreak

      \item We first derive the Lagrangian ($\lambda_1, \lambda_2, \lambda_3 \in \mathbb{R}^m$):
            \begin{align*}
                   & \hphantom{={}} \mathcal{L}(s, x, \lambda_1, \lambda_2, \lambda_3)                                                                            \\
                   & = \mathbf{1}^Ts+\lambda_1^T(Ax-b-\mathbf{1}-s)+\lambda_2^T(-Ax+b-\mathbf{1}-s)+\lambda_3^T(-s)                                               \\
                   & = (\mathbf{1}^T-\lambda_1^T-\lambda_2^T-\lambda_3^T)s+(\lambda_1^T-\lambda_2^T)Ax-\mathbf{1}^T(\lambda_1+\lambda_2)+b^T(\lambda_2-\lambda_1)
            \end{align*}
            For $\mathcal{L}$ to be a finite value,
            we need $\sum \lambda_i = \mathbf{1}$ and $A^T(\lambda_1-\lambda_2)=\mathbf{0}$.

            Now we check what other conditions the $\lambda$s need to hold:
            \begin{align*}
                  g(x, s) & \le \mathcal{L}(x^*, s^*, \lambda_1, \lambda_2, \lambda_3)                                                \\
                          & =\mathbf{1}^Ts^*+\lambda_1^T(Ax^*-b-\mathbf{1}-s^*)+\lambda_2^T(-Ax^*+b-\mathbf{1}-s^*)+\lambda_3^T(-s^*)
            \end{align*}
            The coefficients of $\lambda_1$, $\lambda_3$, and $\lambda_3$ are all nonpositive.
            Thus, we need all the $\lambda$s to be non\textit{negative}.

            Our final dual problem is as follows:
            \begin{gather*}
                  \max_{\lambda_1, \lambda_2, \lambda_3} -\mathbf{1}^T(\lambda_1+\lambda_2)-b^T(\lambda_1-\lambda_2)\text{ s.t.} \\
                  \lambda_1+\lambda_2+\lambda_3=\mathbf{1} \\
                  A^T(\lambda_1-\lambda_2)=\mathbf{0} \\
                  \lambda_1, \lambda_2, \lambda_3 \ge \mathbf{0}
            \end{gather*}
            $\lambda_3$ is just a slack variable, so we can remove it
            and change the second constraint to \[\lambda_1+\lambda_2 \le \mathbf{1}\]

            With $\lambda_3$ out of the way, we see that this is equivalent
            to the LP problem in the exercise with $z=\lambda_1-\lambda_2$ and $|z|=\lambda_1+\lambda_2$.
            This is because optimizers won't set both $(\lambda_1)_i$ and $(\lambda_2)_i$ to be nonzero.

            From there, we can see that $\mathbf{1}^T (\lambda_1+\lambda_2)=||z||_1$
            and that $||z||_\infty \le 1 \leftrightarrow \lambda_1 + \lambda_2 \le \mathbf{1}$.

      \item \begin{enumerate}
                  \item Let $v \in \mathbb{R}^n$ be a vector s.t. $v_i=1$ if $a_i \ge \alpha$ and $0$ otherwise.
                        Then, our linear program is
                        \begin{gather*}
                              \max_{p} v^Tp \text{ s.t.} \\
                              1^Tp = 1 \\
                              p \ge \mathbf{0} \\
                              a^Tp = b
                        \end{gather*}
                  \item Before we start, we can change the maximization objective into
                        a minimization one to make things more standard.

                        Lagrangian ($\lambda_1, \lambda_3 \in \mathbb{R}, \lambda_2 \in \mathbb{R}^n$):
                        \begin{align*}
                              \mathcal{L}(p, \lambda_1, \lambda_2, \lambda_3)
                               & = -v^Tp+\lambda_1(\mathbf{1}^Tp-1)+\lambda_2^T(-p)+\lambda_3(a^Tp-b)              \\
                               & = (v^T+\lambda_1 \mathbf{1}^T - \lambda_2^T+\lambda_3 a^T)p-\lambda_1-\lambda_3 b \\
                               & \therefore v+\lambda_1\mathbf{1} - \lambda_2 + \lambda_3 a = 0
                        \end{align*}

                        Additional conditions:
                        \begin{align*}
                              g(\lambda_1, \lambda_2, \lambda_3)
                               & \le \mathcal{L}(p^*, \lambda_1, \lambda_2, \lambda_3)                       \\
                               & = v^Tp^*+\lambda_1(\mathbf{1}^Tp^*-1)+\lambda_2^T(-p^*)+\lambda_3(a^Tp^*-b) \\
                               & = v^Tp^*-\lambda_2^T p^*                                                    \\
                               & \therefore \lambda_2 \ge \mathbf{0}
                        \end{align*}

                        Final LP:
                        \begin{gather*}
                              \max_{\lambda_1, \lambda_2, \lambda_3} -\lambda_1-\lambda_3 b\text{ s.t.} \\
                              v + \lambda_1\mathbf{1} - \lambda_2+\lambda_3 a = \mathbf{0} \\
                              \lambda_2 \ge \mathbf{0}
                        \end{gather*}

                        First, maximizing a value is the same as minimizing its negative,
                        so we can turn the objective function into
                        \[\min_{\lambda_1, \lambda_2, \lambda_3} \lambda_1+\lambda_3 b\]

                        We see that $\lambda_2$ is nothing more than a slack variable, so we can reformulate the first constraint as
                        \[v + \lambda_1\mathbf{1} + \lambda_3 a \ge \mathbf{0}\]

                        Recall that $v$ is an "indicator" vector of sorts that indicates whether $a_i \ge \alpha$.
                        Breaking the constraint down, we see that when $v_i=1$, it's
                        \[-1+\lambda_1+a_i \lambda_3 \ge 0 \to \lambda_1+a_i \lambda_3 \ge 1\]
                        Similarly, when $v_i=0$, the constraint turns to
                        \[\lambda_1+a_i \lambda_3 \ge 0\]

                        Putting all these observations together, we get our transformed LP:
                        \begin{gather*}
                              \min_{\lambda_1, \lambda_3} \lambda_1+\lambda_3 b\text{ s.t.} \\
                              \lambda_1+a_i \lambda_3 \ge 1\ \forall a_i \ge \alpha \\
                              \lambda_1+a_i \lambda_3 \ge 0\ \forall a_i < \alpha
                        \end{gather*}
                        which by inspection is equivalent to the one that was given in the problem.
            \end{enumerate}

            \pagebreak

      \item \begin{enumerate}
                  \item The constraint of the associated dual is
                        \[\begin{bmatrix}
                                    -1 & -1 & 0   & -6  & 1  \\
                                    -6 & -2 & 3   & -11 & 6  \\
                                    1  & 7  & -10 & -2  & -1 \\
                                    3  & 1  & -1  & 12  & -3
                              \end{bmatrix} \begin{bmatrix}
                                    \lambda_1 \\ \lambda_2 \\ \lambda_3 \\ \lambda_4 \\ \lambda_5
                              \end{bmatrix} = \begin{bmatrix}
                                    -47 \\ -93 \\ -17 \\ 93
                              \end{bmatrix}\]

                        $J(x)=\{1, 2, 3, 4\}$, so $\lambda_5 = 0$.
                        Solving the set of linear equations with that constraint, we get
                        \[\lambda = \begin{bmatrix}3 & 2 & 2 & 7 & 0\end{bmatrix}^T\]
                        Since a feasible $\lambda$ exists for the given $x$
                        that satisfies complementary slackness,
                        $x$ is optimal.

                  \item The constraints of the associated dual are
                        \begin{gather*}
                              \begin{bmatrix}
                                    1  & 4  & 2  & 3  \\
                                    3  & 2  & 4  & 1  \\
                                    5  & -2 & 4  & 2  \\
                                    -2 & 1  & -2 & -1 \\
                                    3  & 1  & 5  & -2
                              \end{bmatrix}\lambda_1 - \lambda_2 = \begin{bmatrix}
                                    -7 \\ -6 \\ -5 \\ 2 \\ -3
                              \end{bmatrix} \\
                              \lambda_2 \ge \mathbf{0}
                        \end{gather*}

                        $x$ meets the constraints of the 1st, 2nd, and 4th rows with equality,
                        and its 1st and 5th elements are $0$.
                        This constrains our $\lambda$s to the following:
                        \[\begin{bmatrix}
                                    1  & 4  & 2  & 3  \\
                                    3  & 2  & 4  & 1  \\
                                    5  & -2 & 4  & 2  \\
                                    -2 & 1  & -2 & -1 \\
                                    3  & 1  & 5  & -2
                              \end{bmatrix}\begin{bmatrix}
                                    (\lambda_1)_1 \\ (\lambda_1)_2 \\ 0 \\ (\lambda_1)_4
                              \end{bmatrix} - \begin{bmatrix}
                                    (\lambda_2)_1 \\ 0 \\ 0 \\ 0 \\ (\lambda_2)_5
                              \end{bmatrix} = \begin{bmatrix}
                                    -7 \\ -6 \\ -5 \\ 2 \\ -3
                              \end{bmatrix}\]
                        Solving, we get $\lambda_1 = \begin{bmatrix}-1 & -1 & 0 & -1\end{bmatrix}^T$
                        and $\lambda_2 = \begin{bmatrix}1 & 0 & 0 & 0 & -1\end{bmatrix}^T$.
                        However, this violates $\lambda_2 \ge \mathbf{0}$.

                        Since no feasible $\lambda$ exists that satisfies
                        complementary slackness for this $x$, this $x$ isn't optimal.
            \end{enumerate}

            \pagebreak

      \item Suppose that the optimal solutions are $x^*$ and $y^*$.
            By strong duality, $c^Tx^*=-b^Ty^*$.
            Using that $x^*$ and $y^*$ inherently satisfy the constraints,
            \begin{gather*}
                  c^Tx^*=-b^Ty^* \\
                  c^Tx^*=-(Ax^*)^Ty^* \\
                  (x^*)^T c + (x^*)^T A^T y^*=0 \\
                  (x^*)^T (c + A^T y^*) = 0
            \end{gather*}
            $x^*, c+A^Ty^* \ge \mathbf{0}$, so at least one of the terms must be nonzero for each $i$.

            Let us extend $x_I^*$ to a solution for the original LP
            inserting $0$s in all the indices that weren't included in $I$.
            No numbers were removed from $y_I^*$, so it stays the same.

            Let us check feasibility.
            $A_I x_I = b$, and we've only added $0$s to the vector,
            so $Ax$ stays the same.
            That $c+A^Ty \ge 0$ is given as part of the premise, so $y$ is also feasible.

            Now it remains to show that $x^T(c+A^T y)=0$, since we can deduce optimality from that.
            For this, it suffices to show that $x_i(c_i+a^T_i y)=0\ \forall i=1, \cdots, n$.

            If $i \in I$, then $x_i(c_i+a^T_i y)=0$ since $x_i$ and $y$
            were optimal in the reduced problem and must satisfy complementary slackness.
            OTOH, if $i \notin I$, $x_i = 0$ and the equation is trivially true.

            Complementary slackness holds, so this $x^*$ is optimal.
            My friend is right. $\square$
\end{enumerate}
\end{document}
