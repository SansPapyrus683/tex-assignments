\documentclass[12pt]{article}

% a template that a friend gave, it's worked well enough for me
% i have added some packages and stuff that have proved useful

\usepackage{fancyhdr}
\usepackage{tipa}
\usepackage{fontspec}
\usepackage{amsfonts}
\usepackage{enumitem}
\usepackage[margin=1in]{geometry}
\usepackage{graphicx}
\usepackage{float}
\usepackage{amsmath}
\usepackage{braket}
\usepackage{amssymb}
\usepackage{booktabs}
\usepackage{hyperref}
\usepackage{mathtools}
\usepackage{xcolor}
\usepackage{float}
\usepackage{algpseudocodex}
\usepackage{titlesec}
\usepackage{bbm}

\pagestyle{fancy}
\fancyhf{} % sets both header and footer to nothing
\lhead{Kevin Sheng}
\setmainfont{Comic Neue}
\renewcommand{\headrulewidth}{1pt}
\setlength{\headheight}{0.75in}
\setlength{\oddsidemargin}{0in}
\setlength{\evensidemargin}{0in}
\setlength{\voffset}{-.5in}
\setlength{\headsep}{10pt}
\setlength{\textwidth}{6.5in}
\setlength{\headwidth}{6.5in}
\setlength{\textheight}{8in}
\renewcommand{\headrulewidth}{0.5pt}
\renewcommand{\footrulewidth}{0.3pt}
\setlength{\textwidth}{6.5in}
\usepackage{setspace}
\usepackage{multicol}
\usepackage{float}
\setlength{\columnsep}{1cm}
\setlength\parindent{24pt}
\usepackage [english]{babel}
\usepackage [autostyle, english = american]{csquotes}
\MakeOuterQuote{"}

\setlength{\parskip}{6pt}
\setlength{\parindent}{0pt}

\titlespacing\section{0pt}{12pt plus 4pt minus 2pt}{0pt plus 2pt minus 2pt}
\titlespacing\subsection{0pt}{12pt plus 4pt minus 2pt}{0pt plus 2pt minus 2pt}
\titlespacing\subsubsection{0pt}{12pt plus 4pt minus 2pt}{0pt plus 2pt minus 2pt}

\hypersetup{colorlinks=true, urlcolor=blue}

\newcommand{\correction}[1]{\textcolor{red}{#1}}


\rhead{ECE 236A}

\newcommand*{\vertbar}{\rule[-1ex]{0.5pt}{2.5ex}}
\newcommand*{\horzbar}{\rule[.5ex]{2.5ex}{0.5pt}}

\begin{document}

\begin{enumerate}
      \item \begin{enumerate}
                  \item Let there be $n$ players and $m$ opssible matchups respectively.

                        For variables, let $g \in \mathbb{Z}^m$ and $A \in \mathbb{Z}^{n \times m}$
                        be a matrix defined as follows:
                        \[A_{ij}=\begin{cases}
                                    1 & \text{Player $i$ is in matchtup $j$} \\
                                    0 & \text{otherwise}
                              \end{cases}\]

                        Now our ILP is as follows:
                        \begin{gather*}
                              \max_{g} \mathbf{1^T}g \text{ s.t.} \\
                              Ag \le \mathbf{1} \\
                              \mathbf{0} \le g \le \mathbf{1}
                        \end{gather*}

                        Unfortunately, the feasible set described by the matrix is not TUM.
                        Consider the following $A$ with $n=m=3$:
                        \[\begin{bmatrix}
                                    1 & 1 & 0 \\
                                    0 & 1 & 1 \\
                                    1 & 0 & 0
                              \end{bmatrix}\]
                        Its determinant is $2 \notin \{-1, 0, 1\}$.
                  \item We can make two more copies of $g$ for the following LP:
                        \begin{gather*}
                              \max_{g_1, g_2, g_3} \mathbf{1^T}(g_1+g_2+g_3) \text{ s.t.} \\
                              Ag_1, Ag_2, Ag_3 \le \mathbf{1} \\
                              g_1^Tg_2=g_1^Tg_3=g_2^Tg_3=0 \\
                              \mathbf{0} \le g_1, g_2, g_3 \le \mathbf{1}
                        \end{gather*}
            \end{enumerate}

      \item The statement is equivalent saying that one of the following is true:
            \begin{gather}
                  \exists x \in \mathbb{R}^n: Ax \ge b \\
                  \exists \lambda \in \mathbb{R}^m: A^T \lambda \le \mathbf{0}, \lambda^Tb > 0, \lambda \ge \mathbf{0}
            \end{gather}
            where $A \in \mathbb{R}^{m \times n}$ and $b \in \mathbb{R}^m$.

            The prove this, notice that the first part is equivalent to the first
            part of the theorem of alternatives with $-A$ and $-b$:
            \[Ax \ge b \rightarrow -Ax \le -b\]
            Now by this part $\exists z \in \mathbb{R}^m: z \ge 0, (-A)^Tz = 0, (-b)^Tz < 0$.
            Taking out the negative signs, we get $z \ge 0, A^Tz=0 \le 0, b^Tz > 0$,
            which correspond exactly to the conditions described in (2). $\square$

      \item \begin{enumerate}
                  \item The constraint of the max-flow LP is
                        \[\begin{bmatrix}
                                    \horzbar & I_{|E|}    & \horzbar & \mathbf{0} \\
                                    \horzbar & m^T        & \horzbar              \\
                                    \horzbar & -I_{|E|+1} & \horzbar
                              \end{bmatrix}\]

                        Since concatenating $e_x$ or $\mathbf{0}$ to a TUM matrix preserves its TUM-ness
                        (see \href{https://math.stackexchange.com/a/3618732/713952}{here}),
                        the new constraint matrix is still TUM.

                  \item The dual of the max-flow LP is
                        \begin{gather*}
                              \min_{p \in \mathbb{R}^{|V|}, d \in \mathbb{R}^{|E|}} c^Td \text{ s.t.} \\
                              p_s - p_t \ge 1 \\
                              d_{ij} \ge p_i - p_j \\
                              d_{ij}, p_i, p_j \ge 0
                        \end{gather*}

                        If we stack $d$ on top of $p$ for the formal inequality form,
                        the constraint matrix is as follows:
                        \begin{itemize}
                              \item One row with one $1$ and one $-1$ somewhere in the right $|V|$ elements.
                              \item $-I_{|E|}$ on the left side with some vectors on the right
                                    such that there's exactly one $1$ and one $-1$ in each row not including the ID matrix.
                              \item $I_{|E| + |V|}$
                        \end{itemize}

                        Notice that the last $|V|$ columns of the first $|E|+1$ rows
                        is equivalent to $m$ in the previous problem.
                        Thus, by the same reasoning, the constraint matrix
                        here is still TUM and our solution is guaranteed to be integral.

                        Also, it's never strictly more optimal to have a solution
                        with elements greater than $1$ because we can always
                        reach a solution that's in $\{0, 1\}$ from there. $\square$
            \end{enumerate}

      \pagebreak

      \item \begin{enumerate}
                  \item Assuming the roads can receive fractional funding:
                        \begin{gather*}
                              \min_{r} \mathbf{1}^Tr\text{ s.t.} \\
                              r \ge \mathbf{0} \\
                              Ar \ge \mathbf{1} \\
                              A_{ij}=\begin{cases}
                                    1 & \text{road $j$ connects city $i$} \\
                                    0 & \text{otherwise}
                              \end{cases}
                        \end{gather*}
                  \item Lagrangian:
                        \begin{align*}
                              \mathcal{L}(\lambda_1, \lambda_2)
                               & =\mathbf{1}^Tr+\lambda_1^T(-r)+\lambda_2^T(\mathbf{1}-Ar)              \\
                               & =(\mathbf{1}^T-\lambda_1^T-\lambda_2^TA)r+\lambda_2^T \cdot \mathbf{1}
                        \end{align*}

                        For $g(\lambda_1, \lambda_2)$ to be a valid lower bound,
                        we also need $\lambda_1, \lambda_2 \ge 0$.

                        Thus, the dual is
                        \begin{gather*}
                              \max_{\lambda_1, \lambda_2} \mathbf{1}^T\lambda_2\text{ s.t.} \\
                              \mathbf{1}-\lambda_1-A^T\lambda_2=\mathbf{0} \\
                              \lambda_1, \lambda_2 \ge 0
                        \end{gather*}
                        We can further simplify this by noticing that $\lambda_1$ serves only as a slack variable:
                        \begin{gather*}
                              \max_{\lambda} \mathbf{1}^T\lambda\text{ s.t.} \\
                              A^T\lambda \le \mathbf{1} \\
                              \lambda \ge 0
                        \end{gather*}
                  \item Contraining $\lambda \in \{0, 1\}$ makes the dual the MIS ILP problem.
                  \item I propose that the slackness condition is
                  \[\boxed{\lambda^T(Ar-\mathbf{1})=\mathbf{0}}\]
                  and will prove that this is true when and only when strong duality holds.

                  $\lambda \ge 0$ and $Ar-1 \ge \mathbf{0}$, so their dot product must be nonnegative.
                  As for nonpositivity, notice that
                  \begin{align*}
                        \lambda^T(Ar-\mathbf{1})
                        &= (A^T\lambda)^Tr-\lambda^T\mathbf{1} \\
                        &\le \mathbf{1}^Tr-\lambda^T\mathbf{1} \\
                        &= \mathbf{0}
                  \end{align*}
                  From this we can see that the expression must equal $\mathbf{0}$.

                  OTOH, say that the equality above is what we are given.
                  By a similar chain of equalities, we can obtain
                  \[(A^T\lambda)^Tr=\mathbf{1}^T\lambda \rightarrow \mathbf{1}^Tr \le \mathbf{1}^T\lambda\]
                  Through weak duality $\mathbf{1}^Tr \ge \mathbf{1}^T\lambda$, so
                  with this it must be that $\mathbf{1}^Tr=\mathbf{1}^T\lambda$. $\square$

            \end{enumerate}

      \item This is a grossly inefficient formulation, but it's still
            polynomial in terms of the input.

            Let $A \in \mathbb{Z}^{m \times n}$ be defined as follows:
            \[A_{ij}=\begin{cases}
                        1 & \text{Lecturer $j$ offers class $i$} \\
                        0 & \text{otherwise}
                  \end{cases}\]

            Also, let $C_{i} \in \mathbb{Z}^{m \times n}$ be the classes offered on the $i$th day
            and $d \in \mathbb{Z}^{mn}$ indicate all $C_i$s with nonzero sums.

            Our LP is as follows:
            \begin{gather*}
                  \max_{C_{i}, d} \mathbf{1}^T(\mathbf{1}-d) \text{ s.t.} \\
                  \sum_{i=1}^{mn} C_{i}=A \\
                  C_i\mathbf{1} \le 1, C_i^T\mathbf{1} \le 1 \\
                  \left(C_i\mathbf{1}\right)^T\mathbf{1} \le K \\
                  \left(C_i\mathbf{1}\right)^T\mathbf{1} \cdot d_i = 0 \\
                  0 \le C_i \le 1, \mathbf{0} \le d \le \mathbf{1}
            \end{gather*}
\end{enumerate}
\end{document}
