\documentclass[12pt]{article}

% a template that a friend gave, it's worked well enough for me
% i have added some packages and stuff that have proved useful

\usepackage{fancyhdr}
\usepackage{tipa}
\usepackage{fontspec}
\usepackage{amsfonts}
\usepackage{enumitem}
\usepackage[margin=1in]{geometry}
\usepackage{graphicx}
\usepackage{float}
\usepackage{amsmath}
\usepackage{braket}
\usepackage{amssymb}
\usepackage{booktabs}
\usepackage{hyperref}
\usepackage{mathtools}
\usepackage{xcolor}
\usepackage{float}
\usepackage{algpseudocodex}
\usepackage{titlesec}
\usepackage{bbm}

\pagestyle{fancy}
\fancyhf{} % sets both header and footer to nothing
\lhead{Kevin Sheng}
\setmainfont{Comic Neue}
\renewcommand{\headrulewidth}{1pt}
\setlength{\headheight}{0.75in}
\setlength{\oddsidemargin}{0in}
\setlength{\evensidemargin}{0in}
\setlength{\voffset}{-.5in}
\setlength{\headsep}{10pt}
\setlength{\textwidth}{6.5in}
\setlength{\headwidth}{6.5in}
\setlength{\textheight}{8in}
\renewcommand{\headrulewidth}{0.5pt}
\renewcommand{\footrulewidth}{0.3pt}
\setlength{\textwidth}{6.5in}
\usepackage{setspace}
\usepackage{multicol}
\usepackage{float}
\setlength{\columnsep}{1cm}
\setlength\parindent{24pt}
\usepackage [english]{babel}
\usepackage [autostyle, english = american]{csquotes}
\MakeOuterQuote{"}

\setlength{\parskip}{6pt}
\setlength{\parindent}{0pt}

\titlespacing\section{0pt}{12pt plus 4pt minus 2pt}{0pt plus 2pt minus 2pt}
\titlespacing\subsection{0pt}{12pt plus 4pt minus 2pt}{0pt plus 2pt minus 2pt}
\titlespacing\subsubsection{0pt}{12pt plus 4pt minus 2pt}{0pt plus 2pt minus 2pt}

\hypersetup{colorlinks=true, urlcolor=blue}

\newcommand{\correction}[1]{\textcolor{red}{#1}}


\begin{document}
\begin{enumerate}
    \item First, let's transform this equation into the form $y'=a(t)y+f(t)$:
          \[y'=-2ty+5t\]
          Solving the homogeneous version of this equation $y'=y(-2t)$, we get $y_h=\exp(-t^2)$.
          To get $v(t)$, we know that $v'(t)=\frac{5t}{y_h}=5t \exp(t^2)$ and thus $v(t)=\frac{5}{2}\exp(t^2)+C$.
          Thus, our final answer is \[v(t)y_h=\boxed{\frac{5}{2}+C\exp(-t^2)}\]
    \item Putting our equation in the same form as above, we get that $a(t)=m$ and $f(t)=c_1e^{mx}$.
          We then find a possible $u(x)=e^{-mx}$ and rearrange the equation to get the following: \label{list:2}
          \begin{gather*}
              u(x)(y'-my)=c_1 \\
              (uy)'=c_1 \\
              uy=c_1x+C \\
              \boxed{y=e^{mx}(c_1x+C)}
          \end{gather*}
    \item Since equations go both ways, we can start from the end and work our way backwards.
          \begin{gather*}
              z'=(1-n)a(t)z+(1-n)f(t) \\
              (1-n)x^{-n} \cdot x'=(1-n)a(t)x^{1-n}+(1-n)f(t) \\
              x^{-n} \cdot x' = a(t)x^{1-n}+f(t) \\
              x'=a(t)x+f(t)x^n \quad \square
          \end{gather*}
    \item Letting $z=y^{1-2}=\frac{1}{y}$, we transform our equation into
          \[z'=(1-2)\left(-\frac{1}{x}\right)z+(1-2)x=\frac{z}{x}-x\]
          Using the same method as in (\ref{list:2}), we first solve for $u(x)$:
          \begin{align*}
              u(x) & =\exp\left(-\int a(x)\,dx\right) \\
                   & =\exp(-\ln |x|)                  \\
                   & =\frac{1}{x}
          \end{align*}
          We can remove the absolute value since we're given that $x>0$.

          We then multiply both sides of our initial equation by $\frac{1}{x}$ and go from there:
          \begin{gather*}
              (uz)'=-x \cdot \frac{1}{x} \\
              uz=-x+C \\
              z=\frac{-x+C}{1/x} \\
              z=-x^2+Cx \\
              \boxed{y=\frac{1}{-x^2+Cx}}
          \end{gather*}
    \item Setting up our differential equations, we have
          \begin{align*}
               & \frac{dA}{dt}=-\frac{A}{20}                        &  & A(0)=20 \\
               & \frac{dB}{dt}=-\frac{dA}{dt}-\frac{2.5B}{200+2.5t} &  & B(0)=40
          \end{align*}
          where $A$ and $B$ represent the amount of salt in the tanks specified by their names.

          The first equation is linear separable and is thus trivial to solve:
          \begin{gather*}
              \frac{dA}{A}=-\frac{dt}{20} \\
              \ln |A|=-\frac{t}{20}+C \\
              A=C\exp(-\frac{t}{20})=20\exp\left(-\frac{t}{20}\right)
          \end{gather*}
          Now that we have $A(t)$, we solve for $B(t)$, which has now simplified to
          \[B'=\exp\left(-\frac{t}{20}\right)-B\frac{2.5}{200+2.5t}\]

          Since $a(t)=-\frac{2.5}{200+2.5t}$, we have
          \[u(t)=\exp\left(-\int -\frac{2.5}{200+2.5t}\,dt\right)=\exp(\ln |t+80|)=t+80\]
          Note that we omit the absolute value sign since $t \ge 0$.

          Then we multiply both sides by $u(t)$ and solve from there.
          \begin{align*}
              (Bu)' & =\exp\left(-\frac{t}{20}\right) \cdot u(t)                                           \\
              Bu    & =\int \exp\left(-\frac{t}{20}\right)(t+80)\,dt                                       \\
                    & =(t+80)(-20\exp\left(-\frac{t}{20}\right))+\int 20\exp\left(-\frac{t}{20}\right)\,dt \\
                    & =-20\exp\left(-\frac{t}{20}\right)(t+80)-400\exp\left(-\frac{t}{20}\right)+C         \\
                    & =-20\exp\left(-\frac{t}{20}\right)(t+100)+C                                          \\
              B     & =\frac{''}{t+80}
          \end{align*}

          Plugging in the initial condition $B(0)=40$, we get that $C=5200$,
          so our final answer is $B(20)=\boxed{52-\frac{24}{e}}$.

    \item We know $y_1$ and $y_2$ both work for the specified differential equation,
          so let's specify a third equation $y=C_1y_1+C_2y_2$.

          Evaluating the LHS of the given equation, we have
          \begin{align*}
              y' & =C_1y_1'+C_2y_2'                       \\
                 & =C_1y_1a(t)+C_1f(t)+C_2y_2a(t)+C_2f(t)
          \end{align*}
          Doing the same on the right side, we have
          \[a(t)y+f(t) = C_1y_1a(t)+C_2y_2a(t)+f(t)\]
          For these two to be equal, $C_1+C_2$ has to be $1$.
\end{enumerate}
\end{document}
