\documentclass[12pt]{article}

% a template that a friend gave, it's worked well enough for me
% i have added some packages and stuff that have proved useful

\usepackage{fancyhdr}
\usepackage{tipa}
\usepackage{fontspec}
\usepackage{amsfonts}
\usepackage{enumitem}
\usepackage[margin=1in]{geometry}
\usepackage{graphicx}
\usepackage{float}
\usepackage{amsmath}
\usepackage{braket}
\usepackage{amssymb}
\usepackage{booktabs}
\usepackage{hyperref}
\usepackage{mathtools}
\usepackage{xcolor}
\usepackage{float}
\usepackage{algpseudocodex}
\usepackage{titlesec}
\usepackage{bbm}
\usepackage{pythonhighlight}

\pagestyle{fancy}
\fancyhf{} % sets both header and footer to nothing
\lhead{Kevin Sheng}
\setmainfont{Comic Neue}
\renewcommand{\headrulewidth}{1pt}
\setlength{\headheight}{0.75in}
\setlength{\oddsidemargin}{0in}
\setlength{\evensidemargin}{0in}
\setlength{\voffset}{-.5in}
\setlength{\headsep}{10pt}
\setlength{\textwidth}{6.5in}
\setlength{\headwidth}{6.5in}
\setlength{\textheight}{8in}
\renewcommand{\headrulewidth}{0.5pt}
\renewcommand{\footrulewidth}{0.3pt}
\setlength{\textwidth}{6.5in}
\usepackage{setspace}
\usepackage{multicol}
\usepackage{float}
\setlength{\columnsep}{1cm}
\setlength\parindent{24pt}
\usepackage [english]{babel}
\usepackage [autostyle, english = american]{csquotes}
\MakeOuterQuote{"}

\setlength{\parskip}{6pt}
\setlength{\parindent}{0pt}

\titlespacing\section{0pt}{12pt plus 4pt minus 2pt}{0pt plus 2pt minus 2pt}
\titlespacing\subsection{0pt}{12pt plus 4pt minus 2pt}{0pt plus 2pt minus 2pt}
\titlespacing\subsubsection{0pt}{12pt plus 4pt minus 2pt}{0pt plus 2pt minus 2pt}

\hypersetup{colorlinks=true, urlcolor=blue}

\newcommand{\correction}[1]{\textcolor{red}{#1}}


\rhead{Math 114C}

\makeatletter
\def\@seccntformat#1{%
  \expandafter\ifx\csname c@#1\endcsname\c@section\else
  \csname the#1\endcsname\quad
  \fi}
\makeatother

\DeclareMathOperator{\ch}{Char}
\DeclareMathOperator{\proj}{proj}
\DeclareMathOperator{\len}{len}
\newcommand{\pto}{\rightharpoonup}

\setcounter{section}{-1}  % since bro decided to add a problem 0 LMAO

\begin{document}

\section{Problem 0}

I believe my solution to problem 2 in the previous homework also works for this,
but I'll just paste it here again:

{
\color{blue}
Consider the following function:
\[f(e, x)=\begin{cases}
    \uparrow & x=e              \\
    0        & \text{otherwise}
  \end{cases}\]
By the Second Recursion Theorem $\exists e^*: \varphi_e(x)=f(e, x)$,
so $\varphi_e(e)\uparrow$ and $\varphi_e(x)\downarrow=0$ for everything else.
This means $e \in A$.

However, we've shown that every program has infinitely many variations that do the same thing.

Thus, $\exists e' \ne e: \varphi_e'=\varphi_e \implies \varphi_e'(e')\downarrow \implies e' \notin A$. $\square$
}

\section{Problem 1}\label{sec:prob1}

I'll just prove part B, since it implies part A.

Consider the relation $R \subseteq \N^3$ defined by
\[R(z, e, x) \iff \exists y: y \in W_e \land \varphi_z(y)\downarrow=x\]
By the closure properties of semirecursive relations shown in class, $R$ should also be semirecursive.

Using the s-m-n theorem for these relations gives a total $v: \N^2 \to \N$ s.t.
\begin{align*}
  W_{v(z, e)}(x)
  &\iff R(z, e, x) \\
  &\iff \exists y: y \in W_e \land \varphi_z(y)\downarrow=x \\
  &\iff x \in \varphi_z[W_e]\quad\square
\end{align*}

\pagebreak

\section{Problem 2}

\subsection{Part A}

For containment, fix an r.e. subset $A \subseteq \N$ and let $B \subseteq \N^2$ be its recursive witness.

To show that this is the range of some recursive function, consider the function
\[f(x)=x + 0 \cdot \mu y[(x, y) \in B]\]
that returns its input if the minimizer succeeds and diverges in all other cases.

If we let $e_f$ be the code for $f$, we have
\begin{align*}
  x \in R_{e_f}
  &\iff f(x)=x \\
  &\iff \exists y: (x, y) \in B \\
  &\iff x \in A
\end{align*}

Also, the relation $Q(e, x)$ is uniformly semirecursive since
\[x \in R_e \iff \exists y: \varphi_e(y)\downarrow = x \iff \exists t, y: T(e, x, t) \land U(t)=x\]
where $T$ is recursive and $U$ is total and recursive.

\subsection{Part B}

To get $u$, notice that $R_e=\varphi_e[W_e]$, so we can use the function we
proved exists in \ref{sec:prob1} and plug in $e$ for both of the inputs.

We can make $v$ a function that takes in $e$ and generates a code s.t.
\[\varphi_{v(e)}(x)=\mu y[T(e, x, y) \land U(y)=x] \cdot 0 + x\]
Once again, this is basically just a restricted identity function that
returns its input iff the minimizer within finds an input.

Then,
\begin{align*}
  x \in R_{v(e)}
  &\iff \exists y: T(e, x, y) \land U(y)=x \\
  &\iff \varphi_e(x)\downarrow = U(y) \\
  &\iff x \in W_e\quad\square
\end{align*}

% As for $v$, consider the relation $G$ defined by
% \[G(e, x) \iff x \in R_e \iff \exists y: \varphi_e(y)=x\]
% By the Second Recursion Theorem for r.e. sets, $\exists  e^*$

\pagebreak

\section{Problem 3}

$A$ and $B$ are co-r.e., which means $A^C$ and $B^C$ are themselves r.e.

Since r.e. sets have the reduction property, we can separate $A^C \cup B^C$
into a pair of disjoint r.e. sets $A^*$ and $B^*$, where $A^C \subseteq A^*$ and $B^C \subseteq B^*$.

However, note that since $A \cap B=\varnothing$, $A^C \cup B^C=\N$ and so $B^*=(A^*)^C$.

$A^*$ and $B^*$ are both semirecursive, so this means they're both \textit{recursive} too,
as they're complements of each other.
I propose that $B^*$ (or $A^*$, it doesn't matter) is the separation set that we want.

Indeed,
\begin{align*}
  x \in A
  &\implies x \notin B \\
  &\implies x \in B^C \subseteq B^* \\
  &\implies A \subseteq B^*
\end{align*}
and
\begin{align*}
  x \in B
  &\implies x \notin A \\
  &\implies x \in A^* \\
  &\implies x \notin B^* \\
  &\implies B \cap B^* = \varnothing\quad\square
\end{align*}

\pagebreak

\section{Problem 4}

\subsection{Reflexivity}

If $f(x)=x$, then
\[n \in A \implies f(n)=n \in A\]
Yeah.

\subsection{Symmetric-ness}

Indeed,
\[A \equiv_m B \implies A \le_m B \land B \le_m A \implies B \equiv_m A\]
{\footnotesize (lol)}

\subsection{Transitivity}

Suppose $A \equiv_m B$ and $B \equiv_m C$.

I'll prove that $A \le_m C$.
Since $A \le_m B$ and $B \le_m C$, we have
\begin{gather*}
  \exists f: \N \to \N: x \in A \implies f(x) \in B \\
  \exists g: \N \to \N: x \in B \implies g(x) \in C
\end{gather*}
If we let $h=g \circ f$, then
\[x \in A \implies f(x) \in B \implies f(g(x))=h(x) \in C\]
The proof that $C \le_m A$ goes much the same way.

\pagebreak

\section{Problem 5}

\subsection{Part A}

Consider the function $g(x, y)=\varphi_x(x)$ (the second argument is unused).

By the s-m-n theorem, $\exists s: \N \to \N$ s.t. $\varphi_{s(x)}(y)=g(x,y)=\varphi_x(x)$.

This $s$ is the reduction we want, since
\begin{align*}
  K(e)
  &\iff \varphi_e(e)\downarrow \\
  &\iff \varphi_{s(e)}(x)\downarrow = \varphi_{e}(e) \\
  &\iff R(s(e))
\end{align*}

\subsection{Part B}

Perform the same construction as in the previous part, except now $g(x, y)=0 \cdot \varphi_x(x) + y$.

If $K(e)$, then $\varphi_e(e)\downarrow$ and $\varphi_{s(x)}(y)\downarrow=y$ for all $y \in \N$.
This makes its range all of $\N$ to, since it's the identity function.

OTOH, if $\lnot K(e)$, then $\varphi_e(e)\uparrow$ and
$\varphi_{s(x)}(y)\uparrow$ for all $y \in \N$, making its range the null set.

\pagebreak

\section{Problem 6}

\subsection{Part A}

Consider the function
\[g(e, x)=\ch_T(e, e, x)\]
and the s-m-n function $s: \N \to \N$ s.t. $\varphi_{s(e)}(x)=g(e, x)$.

If $\varphi_e(e)\uparrow$, then $\varphi_{s(e)}(x)=\ch_T(e, e, x)=0$ for all $x \in \N$,
which makes $x \in A$.

OTOH, if not, then $\exists x: \ch_T(e, e, x)=1$, which forces $s(e) \notin A$.

Thus, $\lnot K \le_m A$, which then means $A$ isn't r.e.

\subsection{Part B}

(am i using s-m-n too much? seems a bit overkill, but better safe than sorry, ig)

Consider the function
\[g(e, x)=\begin{cases}
  \phi_e(x) & x \ne e \\
  \uparrow & \text{otherwise}
\end{cases}\]
and (again) the s-m-n function $s: \N \to \N$ s.t. $\varphi_{s(e)}(x)=g(e, x)$.
Note that by construction $W_{s(e)}=W_e \setminus \{e\}$.

Now consider $f: \N \to \N$ defined by $f(e)=\braket{e, s(e)}$.

If $\varphi_e(e)\uparrow$, $e \notin W_e \implies W_e \subseteq W_{s(e)}$, and so $e \in B$.

If not, then $e \in W_e$ but $e \notin W_{s(e)}$, so $e \notin B$ as well.

And the conclusion is the same.
$\lnot K \le_m B$, and $B$ isn't r.e.

\pagebreak

\section{Problem 7}

$A \cup C$ is r.e. since $A$ and $C$ are both r.e.

$A$ is r.e., so $\exists a \in \N: A=W_a$.

To show that $(A \cup C)^C$ is productive, consider the following function:
\[g(e, x)=\mu y[T(e, x, y) \lor T(a, x, y)]\]
Using s-m-n, $\exists h: \N \to \N$ s.t. $\varphi_{h(e)}(x)=g(x)$.

Notice that $W_{h(e)}=W_e \cup W_a$, since if at least one of
$\varphi_e(x)$ or $\varphi_a(x)$ halts $g(e, x)$ will too.

By creativity of $C$, we've a recursive total $f: \N \to \N$ s.t. $W_e \subseteq C^C \implies f(e) \in C^C \setminus W_e$.

I propose that $f \circ h$ is our productive function.

Indeed,
\begin{align*}
  W_e \subseteq (A \cup C)^C
  &\implies W_{h(e)} = W_e \cup W_a \subseteq C^C \\
  &\implies f(h(e)) \in C^C \setminus W_{h(e)} \\
  &\implies f(h(e)) \in C^C \setminus (A \cup W_e) \\
  &\implies f(h(e)) \in (A \cup C)^C \setminus W_e\quad\square
\end{align*}

\section{Problem 8}

Again, we've a recursive total $f: \N \to \N$ s.t. $W_e \subseteq B \implies f(e) \in B \setminus W_e$.


Since $B$ is recursive, we can define the following function:
\[g(e, x)=\begin{cases}
  \uparrow & x \in B \\
  \varphi_e(x) & \text{otherwise}
\end{cases}\]
and make an $s: \N \to \N$ s.t. $\varphi_{h(e)}(x)=g(e, x)$.

Notice how $W_{h(e)}=W_e \setminus B$.

From here, $f \circ h$ is once again our productive function since
\begin{align*}
  W_e \subseteq A \cup B
  &\implies W_{h(e)} \subseteq A \\
  &\implies f(h(e)) \in A \setminus W_{h(e)} \\
  &\implies f(h(e)) \in (A \cup B) \setminus W_{h(e)}\quad\square
\end{align*}

\pagebreak

\section{Problem 9}

That $A \otimes B$ is r.e. is true under closure properties of r.e. sets.

I'll focus on the two interesting properties we have to prove first.

\subsection{Infinite Negation}

For convenience, let $A'=\{2x \mid x \in A\}$ and $B'=\{2x+1 \mid x \in B\}$.

Notice that $A' \cap B'=\varnothing$, since
one set only has odds and the other only has evens.

We'll prove the contrapositive.
If $(A' \cup B')^C$ is finite, it would be recursive.
This means $R(x) \iff x \in A' \cup B'$ is decidable, and since
the parity of a number is also decidable,
this would imply that $A$ and $B$ are themselves decidable.

\subsection{No Infinite R.E. Subset}

Let's prove the contrapositive again.

Assume we have an infinite r.e. subset $C \subseteq (A' \cup B')^C$.

$C$ must either have an infinite amount of even numbers or an infinite amount of odds.
We'll assume it has infinite evens first.

I propose that $D=\left\{\frac{c}{2} \mid 2 \mid c \land c \in C\right\}$ is
an infinite r.e. subset of $A^C$.

That $D$ is infinite and r.e. comes from our construction of $C$ and closure properties.

As for the actual subset-ness, note that
\begin{align*}
  x \in D
  &\implies 2x \in C \\
  &\implies 2x \notin A' \\
  &\implies x \notin A
\end{align*}

If $C$ has infinite odds, the proof goes much the same way.
We conclude by symmetry. $\square$

\end{document}
