\documentclass[12pt]{article}

% a template that a friend gave, it's worked well enough for me
% i have added some packages and stuff that have proved useful

\usepackage{fancyhdr}
\usepackage{tipa}
\usepackage{fontspec}
\usepackage{amsfonts}
\usepackage{enumitem}
\usepackage[margin=1in]{geometry}
\usepackage{graphicx}
\usepackage{float}
\usepackage{amsmath}
\usepackage{braket}
\usepackage{amssymb}
\usepackage{booktabs}
\usepackage{hyperref}
\usepackage{mathtools}
\usepackage{xcolor}
\usepackage{float}
\usepackage{algpseudocodex}
\usepackage{titlesec}
\usepackage{bbm}
\usepackage{pythonhighlight}

\pagestyle{fancy}
\fancyhf{} % sets both header and footer to nothing
\lhead{Kevin Sheng}
\setmainfont{Comic Neue}
\renewcommand{\headrulewidth}{1pt}
\setlength{\headheight}{0.75in}
\setlength{\oddsidemargin}{0in}
\setlength{\evensidemargin}{0in}
\setlength{\voffset}{-.5in}
\setlength{\headsep}{10pt}
\setlength{\textwidth}{6.5in}
\setlength{\headwidth}{6.5in}
\setlength{\textheight}{8in}
\renewcommand{\headrulewidth}{0.5pt}
\renewcommand{\footrulewidth}{0.3pt}
\setlength{\textwidth}{6.5in}
\usepackage{setspace}
\usepackage{multicol}
\usepackage{float}
\setlength{\columnsep}{1cm}
\setlength\parindent{24pt}
\usepackage [english]{babel}
\usepackage [autostyle, english = american]{csquotes}
\MakeOuterQuote{"}

\setlength{\parskip}{6pt}
\setlength{\parindent}{0pt}

\titlespacing\section{0pt}{12pt plus 4pt minus 2pt}{0pt plus 2pt minus 2pt}
\titlespacing\subsection{0pt}{12pt plus 4pt minus 2pt}{0pt plus 2pt minus 2pt}
\titlespacing\subsubsection{0pt}{12pt plus 4pt minus 2pt}{0pt plus 2pt minus 2pt}

\hypersetup{colorlinks=true, urlcolor=blue}

\newcommand{\correction}[1]{\textcolor{red}{#1}}


\rhead{Math 114C}

\makeatletter
\def\@seccntformat#1{%
  \expandafter\ifx\csname c@#1\endcsname\c@section\else
  \csname the#1\endcsname\quad
  \fi}
\makeatother

\DeclareMathOperator{\ch}{Char}

\begin{document}

\section{Problem 1}

\subsection{Part A}

This is just casework, no?
\begin{enumerate}
  \item If $R(\vec{x}) \land Q(\vec{x})$,
        $\ch_R(\vec{x})=\ch_Q(\vec{x})=1$ and $\ch_{R\&Q}(\vec{x})=1$ as well.
  \item If $\lnot R(\vec{x}) \land \lnot Q(\vec{x})$,
        $\ch_R(\vec{x})=\ch_Q(\vec{x})=0$ and $\ch_{R\&Q}(\vec{x})=0$.
  \item If $\lnot R(\vec{x}) \land Q(\vec{x})$,
  $\ch_R(\vec{x})=0$ and $\ch_Q(\vec{x})=1$, while $\ch_{R\&Q}(\vec{x})=0$.
   \item If $R(\vec{x}) \land \lnot Q(\vec{x})$,
  $\ch_R(\vec{x})=1$ and $\ch_Q(\vec{x})=0$, while $\ch_{R\&Q}(\vec{x})=0$.
\end{enumerate}
In all four cases, $\ch_{R\&Q}(\vec{x})=\ch_R(\vec{x}) \cdot \ch_Q(\vec{x})$.

\subsection{Part B}

Same gist.
I'll omit the values of $\ch_R$ and $\ch_Q$ since they were already stated in part A.
\begin{enumerate}
  \item $R(\vec{x}) \land Q(\vec{x}) \implies \ch_{R\&Q}(\vec{x})=1$.
  \item $\lnot R(\vec{x}) \land \lnot Q(\vec{x}) \implies \ch_{R\&Q}(\vec{x})=0$.
  \item $\lnot R(\vec{x}) \land Q(\vec{x}) \implies \ch_{R\&Q}(\vec{x})=1$.
  \item $R(\vec{x}) \land \lnot Q(\vec{x}) \implies \ch_{R\&Q}(\vec{x})=1$.
\end{enumerate}
Again, by basic arithmetic we can verify that
$\ch_{R\&Q}(\vec{x})=\min(1, \ch_R(\vec{x}) + \ch_Q(\vec{x}))$.

\section{Problem 2}

\subsection{Part A}

For the forward direction, notice that since $A=B$, $A(\vec{x}) \iff B(\vec{x})$.

Then, for all $\vec{x} \in \N^n$, $\ch_A(\vec{x})=0 \iff A(\vec{x}) \iff B(\vec{x}) \iff \ch_B(\vec{x})=0$.
A similar chain of reasoning holds true for when $\ch_A(\vec{x})=1$.

To prove the converse, consider the case where $A \ne B$,
so WLOG assume $\exists \vec{x}: A(\vec{x}) \land \lnot B(\vec{x})$.
(If $A \subseteq B$ then we can just swap the two sets and play the same game.)

Here, $\ch_A(\vec{x})=1$ while $\ch_B(\vec{x})=0$, which clearly isn't the same.

\subsection{Part B}

To prove existence, consider the set $A=\{n \in \N: f(n)=1\}$.

Then we have two cases for $\ch_A$:
\begin{enumerate}
  \item $\ch_A(x)=1 \iff A(x) \iff f(x)=1$
  \item $\ch_A(x)=0 \iff \lnot A(x) \iff f(x) \ne 0 \iff f(x)=0$
\end{enumerate}
so we see that this $A$ indeed works.

As for uniqueness, consider two $A, B \subseteq \N$ s.t. $f=\ch_A$ and $f=\ch_B$.

Since $\ch_A=f=\ch_B$, by part A $A=B$ and so there must be at most one.

\section{Problem 3}

\subsection{Part A}

$g(x)$ diverges for all $x$, so $f \circ g$ must too.
I guess we'd write $(f \circ g)(x)=\uparrow$?

\subsection{Part B}

The formula is
\[(f \circ g)(x)=\begin{cases}
  1 & \text{$x$ is even and $x \ge 10$} \\
  \uparrow & \text{otherwise}
\end{cases}\]
If both conditions are met, then $g(x)=2x \ge 20$ and $f(x)$ would then output $1$.

If $x$ is odd, then $g$ would diverge, while if $x < 10$, then $2x < 20$ and $f$ would diverge.

\subsection{Part C}

The formula is
\[(f \circ g)(x)=\begin{cases}
  x & \text{$x$ is a multiple of $15$} \\
  \uparrow & \text{otherwise}
\end{cases}\]
Both $f$ and $g$ just act as filters that spit back the input only if they meet a certain condition.

To meet both conditions, $x$ has to be both a multiple of $3$ and $5$,
which is equivalent to it being a multiple of $15$.

\section{Problem 4}\label{sec:p4}

I'm 90\% sure what the question meant by the conditions was that
\begin{gather*}
  M(x, 0) = g(x) \\
  M(x, y + 1) = h(x, M(x, y), y)
\end{gather*}
since there's no way for $M(x, y)=g(x)$ in general.

Consider the following formulas:
\begin{gather*}
  g(x)=0 \\
  h(x, t, y) = t + x
\end{gather*}

I'll prove that these with the above conditions make things work through induction on $y$.

Our base case is trivial, as $x \cdot 0 = 0\ \forall x$.

Now, assuming that our $g$ and $h$ work for all $x \in \N$ and $y \le N$,
it suffices to show that it works for all $x \in \N$ and $y=N+1$.

This is true because
\begin{align*}
  M(x, N+1)
  &= h(x, M(x, N), N) \\
  &= M(x, N) + N \\
  &= xN + N \\
  &= x(N+1)
\end{align*}
which is exactly what we want. $\square$

\section{Problem 5}

If we let $g(x)=x$ and $h(x, t, y)=f(t)$, then $F$ defined by primitive
recursion on $g$ and $h$ should give us what we want.

The proof is again by induction on $y$.

If $y=0$, $F(x, 0)=g(x)=x$.

Now, assuming this works for all $x \in \N$ and $y \le N$,
\begin{align*}
  F(x, N+1)
  &= h(x, F(x, N), N) \\
  &= f(F(x, N)) \\
  &= f(f^N(x)) \\
  &= f^{N+1}(x)
\end{align*}
As for convergence conditions, we just $f^{k}(x) \in \mathbb{D}\ \forall k < N$.

\section{Problem 6}

\begin{enumerate}
  \item $f(1) \uparrow$, since $1$ is the only number that divides $1$ but it isn't greater than itself.
  \item $f(4)=2$, as $2$ is the smallest number that goes into $4$ besides $1$.
  \item $f(7)=7$, since $7$ is prime and only has $1$ and itself as divisors.
  \item $f(15)=3$, as $3 \mid 15$ and $2 \nmid 15$.
\end{enumerate}

\section{Problem 7}

\subsection{Construction}

Consider the following functions:
\begin{gather*}
  s(x)=x^2 \\
  c(x, y)=1-(s(y) + 1 - x) \\
  g(x)=\mu y[c(x, y)=0]
\end{gather*}
We proved that multiplication was recursive in \ref{sec:p4}, so $s$ is recursive.

$c$ is a composition of substitutions, addition, and truncated subtraction, so
it's also recursive.

Finally, $g$ is just minimization on $c$, so we're fine there.

\subsection{Proof of Behavior}

If $\exists a \in \N: a^2=x$, then $s(a)+1-x=1$ and $c(x, a)=0$.

OTOH, for all $a' < a$, $a'^2 < x$, so $s(a') < x$, $s(a')+1 \le x$, and $c(x, a')=1$.
Notice that the second inequality relies on that all our variables are in $\N$.

As for numbers which aren't perfect squares, they return the lowest natural number
whose square is at least $x$, so we get $\ceil{\sqrt{x}}$.

\pagebreak

\section{Problem 8}

Consider this construction by primitive recursion:
\begin{gather*}
  a(\vec{x})=1 \\
  b(x, t, y)=f(\vec{x}, y) \cdot t \\
  h(\vec{x}, 0)=a(\vec{x}) \\
  h(\vec{x}, y+1)=b(\vec{x}, h(\vec{x}, y), y)
\end{gather*}
$a$ only uses substitution, while $b$ uses it plus multiplication,
which we proved was also recursive.

The proof is by induction on $y$.

For our base case, $h(\vec{x}, 0)=1$, which lines up with the formula given.

Assuming that it holds for all $y \le N$, then
\begin{align*}
  h(\vec{x}, N+1)
  &= b(\vec{x}, h(\vec{x}, N), N) \\
  &= f(\vec{x}, N) \cdot h(\vec{x}, N) \\
  &= f(\vec{x}, N) \cdot \prod_{i=0}^{N-1} f(\vec{x}, i) \\
  &= \prod_{i=0}^{N} f(\vec{x}, i)
\end{align*}
which again lines up with the formula given.

As for the convergence implications, the forward direction should be obvious,
as evaluating $h(\vec{x}, y+1)$ necessitates evaluation of $f(\vec{x}, 0)$ all the way to $f(\vec{x}, y)$.

For the backward direction, we need induction on $y$ again.
The base case where $y=0$ is true since $h(\vec{x}, 1)$ only depends on $h(\vec{x}, 0)=0$ and $f(\vec{x}, 0)$.
Thus, $f(\vec{x}, 0) \downarrow \implies h(\vec{x}, 1) \downarrow$.

Now, we assume that
$f(\vec{x}, 0) \downarrow \land \cdots \land f(\vec{x}, N-1) \implies h(\vec{x}, N) \downarrow$.

Since $h(\vec{x}, N+1)=f(\vec{x}, N) \cdot h(\vec{x}, N)$,
we have the implication that $f(\vec{x}, N) \downarrow \land h(\vec{x}, N) \downarrow \implies h(\vec{x}, N+1) \downarrow$.
However, by our hypothesis, this implication can also be written as 
$f(\vec{x}, 0) \downarrow \land \cdots \land f(\vec{x}, N) \implies h(\vec{x}, N+1) \downarrow$,
thus completing the inductive step and proving the backwards direction.

\end{document}
