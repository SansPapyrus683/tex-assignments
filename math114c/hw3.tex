\documentclass[12pt]{article}

% a template that a friend gave, it's worked well enough for me
% i have added some packages and stuff that have proved useful

\usepackage{fancyhdr}
\usepackage{tipa}
\usepackage{fontspec}
\usepackage{amsfonts}
\usepackage{enumitem}
\usepackage[margin=1in]{geometry}
\usepackage{graphicx}
\usepackage{float}
\usepackage{amsmath}
\usepackage{braket}
\usepackage{amssymb}
\usepackage{booktabs}
\usepackage{hyperref}
\usepackage{mathtools}
\usepackage{xcolor}
\usepackage{float}
\usepackage{algpseudocodex}
\usepackage{titlesec}
\usepackage{bbm}
\usepackage{pythonhighlight}

\pagestyle{fancy}
\fancyhf{} % sets both header and footer to nothing
\lhead{Kevin Sheng}
\setmainfont{Comic Neue}
\renewcommand{\headrulewidth}{1pt}
\setlength{\headheight}{0.75in}
\setlength{\oddsidemargin}{0in}
\setlength{\evensidemargin}{0in}
\setlength{\voffset}{-.5in}
\setlength{\headsep}{10pt}
\setlength{\textwidth}{6.5in}
\setlength{\headwidth}{6.5in}
\setlength{\textheight}{8in}
\renewcommand{\headrulewidth}{0.5pt}
\renewcommand{\footrulewidth}{0.3pt}
\setlength{\textwidth}{6.5in}
\usepackage{setspace}
\usepackage{multicol}
\usepackage{float}
\setlength{\columnsep}{1cm}
\setlength\parindent{24pt}
\usepackage [english]{babel}
\usepackage [autostyle, english = american]{csquotes}
\MakeOuterQuote{"}

\setlength{\parskip}{6pt}
\setlength{\parindent}{0pt}

\titlespacing\section{0pt}{12pt plus 4pt minus 2pt}{0pt plus 2pt minus 2pt}
\titlespacing\subsection{0pt}{12pt plus 4pt minus 2pt}{0pt plus 2pt minus 2pt}
\titlespacing\subsubsection{0pt}{12pt plus 4pt minus 2pt}{0pt plus 2pt minus 2pt}

\hypersetup{colorlinks=true, urlcolor=blue}

\newcommand{\correction}[1]{\textcolor{red}{#1}}


\rhead{Math 114C}

\makeatletter
\def\@seccntformat#1{%
  \expandafter\ifx\csname c@#1\endcsname\c@section\else
  \csname the#1\endcsname\quad
  \fi}
\makeatother

\DeclareMathOperator{\ch}{Char}
\DeclareMathOperator{\proj}{proj}
\DeclareMathOperator{\len}{len}
\newcommand{\pto}{\rightharpoonup}

\begin{document}

\section{Problem 1}

I mean, we only proved the version stated in class for $n=m=1$,
but since there's a bijection between $\N$ and $\N*$ anyways, I think
it should be valid to prove this using the theorem for arbitrary $n$ and $m$.
:clueless:

If $e$ is any code, we can let $f_e: \N^{m+n} \pto \N$ be the function corresponding to it.

By the version of the s-m-n theorem shown in class,
$\exists S_e: \N^m \to \N: \varphi^{(n)}_{S_e(\vec{x})}(\vec{y})=f_e(\vec{x}, \vec{y})$.

Let $s_n^m: \N \times \N^m \to \N$ be defined by $s_n^m(e, \vec{x})=S_e(\vec{x})$.
As we've just proved that such an $S_e$ is guaranteed to exist for all $e$,
this function is well-defined and is total.

Finally,
\begin{align*}
  \varphi_{s_n^m(e, \vec{x})}^{(n)}(\vec{y})
   & = \varphi_{S_e(\vec{x})}^{(n)}(\vec{y})          \\
   & = f_e(\vec{x}, \vec{y})                          \\
   & = \varphi_e{(m+n)}(\vec{x}, \vec{y})\quad\square
\end{align*}
so this $s_n^m$ we've constructed indeed satisfies the conditions laid out. $\square$

\section{Problem 2}

Define $g: \N \times \N^n \to \N$ by $g(e, \vec{x})=\varphi_{f(e)}^{(n)}(\vec{x})$.

By the Second Recursion Theorem,
$\exists e^* \in \N: \varphi_{e^*}^{(n)}(\vec{x})=g(e^*, \vec{x})=\varphi_{f(e^*)}^{(n)}(\vec{x})$.

(did i miss something???)

\section{Problem 3}

Fix an $f: \N \times \N^n \to \N$.

By the s-m-n theorem, $\exists s: \N \to \N: \varphi_{s(e)}^{(n)}(\vec{x})=f(e, \vec{x})$.

Finally, by Roger's, $\exists e^* \in \N: \varphi_{e^*}{(n)}=\varphi_{s(e^*)}^{(n)}=f(e^*, \vec{x})$.

\pagebreak

\section{Problem 4}

\subsection{Part A}

Let $f: \N \times \N \pto \N$ be defined by
\[f(x, y)=\begin{cases}
    0        & y \mid x         \\
    \uparrow & \text{otherwise}
  \end{cases}\]
By the s-m-n theorem, $\exists u: \N \to \N: \varphi_{u(x)}(y)=f(x, y)$.

For any $y$ that aren't divisible by $x$, $f(x, y)\uparrow$ and so does $\varphi_{u(x)}(y)$.

OTOH, if $x \mid y$, then $\varphi_{u(x)}(y)\downarrow=0$.

\subsection{Part B}

Let $f$ be as it was in the previous part.

By the Recursion Theorem, $\exists e^* \in \N: \varphi_{e^*}(x)=f(e^*, x)$.

It should be obvious that this function converges only when $e^* \mid x$. 

\section{Problem 5}

\subsection{Part A}

Let $f: \N \times \N$ be defined by $f(x, y)=x^y$.

By the s-m-n theorem, $\exists S: \N \to \N: \varphi_{S(x)}(y)=x^y$.

By construction, this function outputs all the powers of $x$ (just set $y$ to the power)
and \textit{only} the powers of $x$.

\subsection{Part B}

Let $f$ stay unchanged.

By the Recursion Theorem, $\exists x^* \in \N: \varphi_{x^*}(y)=f(x^*, y)=(x^*)^y$.

Using the same logic as in the previous part, $R_{x^*}$ has all of powers of $x^*$ and nothing more.

\end{document}
