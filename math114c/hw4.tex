\documentclass[12pt]{article}

% a template that a friend gave, it's worked well enough for me
% i have added some packages and stuff that have proved useful

\usepackage{fancyhdr}
\usepackage{tipa}
\usepackage{fontspec}
\usepackage{amsfonts}
\usepackage{enumitem}
\usepackage[margin=1in]{geometry}
\usepackage{graphicx}
\usepackage{float}
\usepackage{amsmath}
\usepackage{braket}
\usepackage{amssymb}
\usepackage{booktabs}
\usepackage{hyperref}
\usepackage{mathtools}
\usepackage{xcolor}
\usepackage{float}
\usepackage{algpseudocodex}
\usepackage{titlesec}
\usepackage{bbm}
\usepackage{pythonhighlight}

\pagestyle{fancy}
\fancyhf{} % sets both header and footer to nothing
\lhead{Kevin Sheng}
\setmainfont{Comic Neue}
\renewcommand{\headrulewidth}{1pt}
\setlength{\headheight}{0.75in}
\setlength{\oddsidemargin}{0in}
\setlength{\evensidemargin}{0in}
\setlength{\voffset}{-.5in}
\setlength{\headsep}{10pt}
\setlength{\textwidth}{6.5in}
\setlength{\headwidth}{6.5in}
\setlength{\textheight}{8in}
\renewcommand{\headrulewidth}{0.5pt}
\renewcommand{\footrulewidth}{0.3pt}
\setlength{\textwidth}{6.5in}
\usepackage{setspace}
\usepackage{multicol}
\usepackage{float}
\setlength{\columnsep}{1cm}
\setlength\parindent{24pt}
\usepackage [english]{babel}
\usepackage [autostyle, english = american]{csquotes}
\MakeOuterQuote{"}

\setlength{\parskip}{6pt}
\setlength{\parindent}{0pt}

\titlespacing\section{0pt}{12pt plus 4pt minus 2pt}{0pt plus 2pt minus 2pt}
\titlespacing\subsection{0pt}{12pt plus 4pt minus 2pt}{0pt plus 2pt minus 2pt}
\titlespacing\subsubsection{0pt}{12pt plus 4pt minus 2pt}{0pt plus 2pt minus 2pt}

\hypersetup{colorlinks=true, urlcolor=blue}

\newcommand{\correction}[1]{\textcolor{red}{#1}}


\rhead{Math 114C}

\makeatletter
\def\@seccntformat#1{%
  \expandafter\ifx\csname c@#1\endcsname\c@section\else
  \csname the#1\endcsname\quad
  \fi}
\makeatother

\DeclareMathOperator{\ch}{Char}
\DeclareMathOperator{\proj}{proj}
\DeclareMathOperator{\len}{len}
\newcommand{\pto}{\rightharpoonup}

\begin{document}

\section{Problem 1}

Is this just Cantor's Diagonalization in a beard and wig?

BWOC say there existed a surjection $\theta$ from $\N$ to all unary functions $f: \N \to \N$.

Then consider the function $f': \N \to \N$ defined by $f'(x)=\theta_x(x)+1$.

Notice that this function is not included in the enumeration, since for all
$x \in \N$ $f'$ differs from $\theta_x$ when you plug in $x$.
Contradiction. $\square$

\section{Problem 2}

Consider the following function:
\[f(e, x)=\begin{cases}
    \uparrow & x=e              \\
    0        & \text{otherwise}
  \end{cases}\]
By the Second Recursion Theorem $\exists e^*: \varphi_e(x)=f(e, x)$,
so $\varphi_e(e)\uparrow$ and $\varphi_e(x)\downarrow=0$ for everything else.
This means $e \in A$.

However, we've shown that every program has infinitely many variations that do the same thing.

Thus, $\exists e' \ne e: \varphi_e'=\varphi_e \implies \varphi_e'(e')\downarrow \implies e' \notin A$. $\square$

\section{Problem 3}

\begin{enumerate}[label=(\alph*)]
  \item True. $\exists e: \varphi_e=\ch_{\lnot R}$, then we can define $\ch_R(x)=1-\ch_{\lnot R}(x)$,
        so $\lnot R$ must be undecidable by this reduction.

  \item False. If we let $Q(\vec{x})=\lnot R(\vec{x})$, then $\ch_{R \land Q}(\vec{x})=0$.

  \item False. Let $Q(\vec{x})=\lnot R(\vec{x})$ again to see that $\ch_{R \lor Q}(\vec{x})=1$.

  \item False. Let $R(x, y)$ hold iff $\varphi_{x+y}(0)\downarrow$.
        While this relation is undecidable, $\ch_P(x)=1$ since we can always choose a $y$ s.t. $x+y$ is some trivial machine.
\end{enumerate}

\pagebreak

\section{Problem 4}

I just realized that semirecursiveness is what CS 181 calls recognizability.

\subsection{Part A}

$R$ isn't decidable since we can reduce the halting problem to it.

However, it is semirecursive.
Let $S_e(\Braket{x, s})$ hold iff $\varphi_e$ halts on input $x$ in $s$ steps.
This relation is decidable, since we can just run $\varphi_e$ for $s$ steps and see what happens.

Then, $R(e) \iff \exists n: S_e(n)$, since $W_e \neq \varnothing$ implies
the existence of an $x$ that makes $\varphi_e$ run for a finite amount of steps ($s$).

\subsection{Part B}

$Q$ isn't decidable since we can define $\phi_{e'}(x)=0 \cdot \phi_e(x) + x$.
$\phi_{e'}(x)\downarrow \iff \phi_{e'}(x)=x$,
so $Q(e')$ is equivalent to $W_e \ne \varnothing$, which again is undecidable.

This one is also semirecursive, though.

Let $M_e(\Braket{x, s})$ hold iff $\varphi_e$ halts on input $x$ with output $x$ in $s$ steps.
We can play much the same game as we did with $R$ to see that $Q(e) \iff \exists x: M_e(x)$.

\subsection{Part C}

$P$ isn't decidable since we can also reduce the halting problem to it.
It also isn't semirecursive.

Let $H(\braket{e, x, s})$ hold iff $\phi_e$ halts on input $x$ in $s$ steps.
The existence of a universal machine for URMs by the Kleene predicate
allows this relation to be recursive.

Then we can define $\lnot P(e, x) \iff \exists s: H(\braket{e, x, s})$, proving the semirecursiveness of $\lnot P$.

\subsection{Part D}

$W_f$ is recursive, so we can define $\ch_{G_f}(x, y)$ like so:
\[\ch_{G_f}(x, y)=\begin{cases}
    1 & x \in W_f \land f(x)\downarrow = y \\
    0 & \text{otherwise}
  \end{cases}\]
The idea is to not even try to evaluate $f(x)$ if $x \notin W_f$ so we don't diverge.

But yeah, this function is recursive, so $G_f$ is too.

\pagebreak

\section{Problem 5}

\subsection{Part A}

\subsubsection{Forward Direction}

$R$'s semirecursive, so $\exists R_0: R(\vec{x}) \iff \exists y: R_0(\vec{x}, y)$, where $R_0$ is recursive.

Then I'm pretty sure we can just define
\[\ch_R^p(\vec{x})=\min(\max(\mu y[R_0(\vec{x}, y)], 1), 1)\]
If $R(\vec{x})$, then the minimizer will give a $y$ and the min and max will force the output to be $1$.

OTOH, if $\lnot R_0(\vec{x}, y)\ \forall y$, $\ch_R^p(\vec{x})\uparrow$ since the minimizer will never stop.

\subsubsection{Backward Direction}

Let $e$ be the code for $\ch_R^p$.

We can define the relation $R_0(\vec{x}, y)$ to hold iff
$\ch_R^p(\vec{x})=\varphi_e$ halts in $y$ steps (and has output $1$, but that's guaranteed).

If $\exists y: R_0(\vec{x}, y)$, then $\ch_R^p(\vec{x})\downarrow = 1$ in $y$ steps and $R(\vec{x})$ holds.

OTOH, if $R(\vec{x})$ holds, then $\ch_R^p(\vec{x})\downarrow = 1$, and $\varphi_e$
must halt in some finite number of steps, forcing a $y$ where $R_0(\vec{x}, y)$ holds. $\square$

\subsection{Part B}

By the previous part, we know there exists an $\ch_R^p$ with the specified behavior.

Now consider the construction
\[g(\vec{x})=\ch_R^p(\vec{x}) \cdot f(\vec{x})\]
If $R(\vec{x})$, $\ch_R^p(\vec{x})=1$ and $g(\vec{x})=f(\vec{x})$.

Otherwise, $\ch_R^p(\vec{x})\uparrow$ and by extension $g(\vec{x})\uparrow$ too. $\square$

\pagebreak

\section{Problem 6}

$A \subseteq \N$ and it's infinite, so we can enumerate them like $a_0, a_1, a_2, \cdots$, where $a_i < a_{i+1}$.

I forgot if there was anything between finite-ness and countably infinite.
That whether there's anything between $\N$ and $\R$ is unsolvable, but IDK about this one.

Regardless, by the premise $f(x)=a_x$ is computable.

Lemme first prove that $f(x) \ge x$.
The base case, $f(0) \ge 0$, is trivial.
As for the inductive step, notice that $f(n+1) > f(n) \ge n$.
Since we're in $\N$, this forces $f(n+1) \ge n+1$ as well.

Now we can define the relation
\[A(x) \iff \exists y \le x: f(y)=x\]
where the RHS is recursive since it only uses a bounded existence statement.

To show that this equivalence is true, note that $A(x) \iff \exists y: f(y)=x$ is \textit{definitely} true.

If $A(x)$, $\exists y: f(y)=x$.
Note that if $y > x$, then $f(y) \ge y > x \implies f(y) \ne x$, so $y \ge x$.

OTOH, $\exists y \le x: f(y)=x \implies A(x)$ since $f: \N \to A$. $\square$

\pagebreak

\section{Problem 7}

\subsection{First Part}

$A$ is r.e., so $\exists g: \N \to A$ that's surjective.

It STP that $f^{-1}[A]$ is semirecursive.
Let $R(x, \braket{a, b})$ hold iff $f(x)\downarrow$ in $a$ steps and $g(b)=f(x)$.
This relation is recurisve since $g$ is total and we can use the Kleene predicate
to check if $f(x)$ terminates in $a$ steps.

  {\footnotesize (ig we should also reject if the second argument isn't a valid sequence code but that's just implementation)}

Then,
\[x \in f^{-1}[A] \iff \exists y: R(x, y)\]
If $x \in f^{-1}[A]$, then $f(x) \in A$.
This means that $\exists a$ s.t. $f(x)$ halts in $a$ steps \textit{and} $\exists b: g(b)=f(x)$.
Set $y=\braket{a, b}$ to see that $R(x, y)$ holds by construction.

The reverse direction goes basically the same way as the forward but in reverse.

\subsection{Second Part}

Let $s(e)$ be the program coded by $\phi_e \circ f$.
Recursive functions are closed under substitution, so this should be fine.

Then,
\begin{align*}
  x \in W_{s(e)}
   & \iff \varphi_{s(e)}(x)\downarrow                  \\
   & \iff \phi_e(f(x))\downarrow                       \\
   & \iff f(x)\downarrow \land\ \phi_e(f(x))\downarrow \\
   & \iff f(x)\downarrow \in W_e                       \\
   & \iff x \in f^{-1}[W_e]\quad\square
\end{align*}

\pagebreak

\section{Problem 8}

\subsection{Part A}

I'm assuming $A \ne \varnothing$ here.

But yeah, if $A$ is r.e. then there exists an $f: \N \to A$ which is total and surjective.

This is the $f$ we want, because
\begin{align*}
  x \in X
   & \iff \exists e: x \in W_e          \\
   & \iff \exists n: x \in W_{f(n)}     \\
   & \iff x \in \bigcup_{n=0}^\infty W_{f(n)}
\end{align*}

\subsection{Part B}

It STP that $X$ is semirecursive.

Let $f$ be the same $f$ from the previous part.

Consider the relation $R(x, \braket{n, s})$ which holds iff $\varphi_{f(n)}(x)$ terminates after $s$ steps.
$f$ is total and recursive, so this relation is recursive.

Then,
\begin{align*}
  x \in X
  &\iff x \in \bigcup_{n=0}^\infty W_{f(n)} \\
  &\iff \exists n: \varphi_{f(n)}(x)\downarrow \\
  &\iff \exists n, s: R(x, \braket{n, s}) \\
  &\iff \exists y: R(x, y)\quad\square
\end{align*}

\subsection{Part C}

Well we can write $\lnot K$ (or any subset of $\N$, really) as
\[\lnot K = \bigcup_{x \in \lnot K} \{x\}\]
{\scriptsize honestly idk what the question was asking here...}

To show that $\lnot K$ can't be written as an effective union, it STP that it isn't r.e.

This follows from the fact that $K$ \textit{is} r.e. and Kleene's Theorem.

\end{document}
