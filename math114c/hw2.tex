\documentclass[12pt]{article}

% a template that a friend gave, it's worked well enough for me
% i have added some packages and stuff that have proved useful

\usepackage{fancyhdr}
\usepackage{tipa}
\usepackage{fontspec}
\usepackage{amsfonts}
\usepackage{enumitem}
\usepackage[margin=1in]{geometry}
\usepackage{graphicx}
\usepackage{float}
\usepackage{amsmath}
\usepackage{braket}
\usepackage{amssymb}
\usepackage{booktabs}
\usepackage{hyperref}
\usepackage{mathtools}
\usepackage{xcolor}
\usepackage{float}
\usepackage{algpseudocodex}
\usepackage{titlesec}
\usepackage{bbm}
\usepackage{pythonhighlight}

\pagestyle{fancy}
\fancyhf{} % sets both header and footer to nothing
\lhead{Kevin Sheng}
\setmainfont{Comic Neue}
\renewcommand{\headrulewidth}{1pt}
\setlength{\headheight}{0.75in}
\setlength{\oddsidemargin}{0in}
\setlength{\evensidemargin}{0in}
\setlength{\voffset}{-.5in}
\setlength{\headsep}{10pt}
\setlength{\textwidth}{6.5in}
\setlength{\headwidth}{6.5in}
\setlength{\textheight}{8in}
\renewcommand{\headrulewidth}{0.5pt}
\renewcommand{\footrulewidth}{0.3pt}
\setlength{\textwidth}{6.5in}
\usepackage{setspace}
\usepackage{multicol}
\usepackage{float}
\setlength{\columnsep}{1cm}
\setlength\parindent{24pt}
\usepackage [english]{babel}
\usepackage [autostyle, english = american]{csquotes}
\MakeOuterQuote{"}

\setlength{\parskip}{6pt}
\setlength{\parindent}{0pt}

\titlespacing\section{0pt}{12pt plus 4pt minus 2pt}{0pt plus 2pt minus 2pt}
\titlespacing\subsection{0pt}{12pt plus 4pt minus 2pt}{0pt plus 2pt minus 2pt}
\titlespacing\subsubsection{0pt}{12pt plus 4pt minus 2pt}{0pt plus 2pt minus 2pt}

\hypersetup{colorlinks=true, urlcolor=blue}

\newcommand{\correction}[1]{\textcolor{red}{#1}}


\rhead{Math 114C}

\makeatletter
\def\@seccntformat#1{%
  \expandafter\ifx\csname c@#1\endcsname\c@section\else
  \csname the#1\endcsname\quad
  \fi}
\makeatother

\DeclareMathOperator{\ch}{Char}
\DeclareMathOperator{\proj}{proj}
\DeclareMathOperator{\len}{len}

\begin{document}

\section{Problem 1}

Notice that
\[\text{$x$ is prime} \iff \forall 2 \le y \le x - 1\ \lnot \text{Divides}(y + 1, x)\]
Or in other words, $x$ is prime iff none of the numbers between $2$ and $x-1$ go into it.

As was shown in class, the boundedness of $y$ and the recursiveness of the
divisibility relation indicate that $\text{Prime}(x)$ itself is also recursive. $\square$

\section{Problem 2}

Consider the following function defined recursively:
\begin{gather*}
  f(0)=\mu y[1 - \ch_A(y) = 0] \\
  f(n+1)=f(n) + 1 + \mu y[1 - \ch_A(f(n)+1+y) = 0]
\end{gather*}
It should be evident that $f(0)$ returns the lowest element in $A$.

As for the inductive step, notice that the minimizer returns the lowest $y$ s.t. $\ch_A(f(n)+1+y)=1$.
The argument is guaranteed to be greater than $f(n)$, and by the properties of the minimizer
we know it's also the lowest one that's in $A$ as well.

This, combined with the fact that we add $f(n)+1$ back to $y$ in it makes
this $f$ satisfy all the specified properties. $\square$

\section{Problem 3}

Given that we have the projection function,
\[R(u) \iff \forall 0 \le i \le n-2\ (\proj(u, i) < \text{proj}(u, i + 1))\]
which is basically the chained inequality given in the problem broken
down into a bunch of simpler inequalities.

The individual inequalities are recursive, and we're only checking a bounded
number of them, so $R$ overall should still be recursive. $\square$

\pagebreak

\section{Problem 4}

If $R$ is recursive, then
\[R'(u) \iff \text{Seq}(u) \land R(\proj(u, 0), \proj(u, 1), \cdots, \proj(u, n-1))\]
where all of the functions on the RHS are recursive, so yeah.

OTOH, if $R'$ is recursive, then
\[R(\vec{x}) \iff R'(\braket{x_0, x_1, \cdots, x_{n-1}})\]
since the angle brackets always give us a valid sequence code.

Again, everything on the RHS is recursive. $\square$

\section{Problem 5}

It STP that the function $F(\vec{x}, y)=\Braket{f_0(\vec{x}, y), f_1(\vec{x}, y)}$ is recursive.
If that's recursive, then we can just index into the results and call it a day.

Notice that it can be defined using primitive recursion like so:
\begin{gather*}
  F(\vec{x}, 0)=\Braket{g_0(\vec{x}), g_1(\vec{x})} \\
  F(\vec{x}, y+1)=\Braket{h_0(\vec{x}, y, \proj(F(\vec{x}, y), 0), \proj(F(\vec{x}, y), 1)), h_1(\cdots)}
\end{gather*}
where the ellipsis in $h_1$ mean that it's just a repeat of the arguments in $h_0$.

But yeah, we have this definition, so all's good! $\square$

\pagebreak

\section{Problem 6}

\subsection{Part A}

We have
\begin{align*}
  A(0+1, 0+1)
   & = A(0, A(0+1, 0)) \\
   & = A(0+1, 0)+1     \\
   & = A(0, 1)+1       \\
   & = (1+1)+1         \\
   & = \boxed{3}
\end{align*}

\subsection{Part B}

First I'll show that for $m=1$, $A(m, n) \downarrow\forall n$ by induction on $n$.

The base case was already shown.
As for the inductive step, we have
\begin{align*}
  A(0+1, n+1)
   & = A(0, A(0+1, n)) \\
   & = A(0+1, n)+1     \\
   & = A(1, n)+1
\end{align*}
so as long as $A(1, n)\downarrow$, then $A(1, n+1)\downarrow$ as well.

Now it suffices to show that if $A(m, n)\downarrow\forall n$, then $A(m+1, n)\downarrow\forall n$ as well.

For this we need to induct on $n$ once again.
The base case here is trivial, since $A(m+1, 0)=A(m, 1)$ which we know converges
by our inductive hypothesis.

Now, assuming $A(m+1, n)\downarrow$,
\[A(m+1,n+1)=A(m, A(m+1, n))\]
The inner argument converges, giving us $A(m, N)$ where $N \in \N$.
Thankfully, this also converges by our other hypothesis.

Thus, for all $m, n \in \N$, $A(m, n)\downarrow$. $\square$

\pagebreak

\section{Problem 7}

\subsection{Part A}

\subsubsection{Forward Direction}

Lemme first show that the forward direction holds for all $n$ when $m=0$.

In this case, $A(0, n)=n+1$, and the one-element sequence $(0, n+1, n)$ proves existence.

Now, assuming the hypothesis for a certain $m$, we induct on $n$.

The base case is showing the direction for $A(m+1, 0)=A(m,1)$.
We know $\exists \vec{x}_0, \cdots \vec{x}_{k-1}$ with $\vec{x}_{k-1}=(m, 1, A(m, 1))$.
Given this, we can append $x_k=(m+1, 0, A(m,1))$, which fulfills condition $2$
in the definition and shows the base case.

For the final inductive step, we now have sequences for $A(m+1, n)$ and $A(m, A(m+1, n))$.
Concatenating these sequences still gives us another valid one.

Now we have the following tuples:
\begin{itemize}[nolistsep]
  \item $(m+1, n, A(m+1, n))$
  \item $(m, A(m+1, n), A(m, A(m+1, n)))$
\end{itemize}
Notice that we can add another tuple $(m+1, n+1, A(m, A(m+1, n)))$ to the end now,
as it satisfies condition 3 in the definition with tuple $j$ being the first
and tuple $l$ being the second.

But yeah, following these steps should give us a sequence for all $m$ and $n$!

\subsubsection{Backwards Direction}

Following the hint, we induct on the length of the sequence.

If $n=1$, then the sequence can only be of the form $(0, n, n+1)$, and indeed, $A(0, n)=n+1$.

Now we assume this is true for all sequences with length at \textit{or below} $n$.

For any sequence of length $n+1$, the last element must satisfy any one of the three conditions.
\begin{enumerate}
  \item If it's of the form $(0, n, n+1)$, then it's trivial.
  \item If $\exists m: \exists i: x_i=(m, 1, y) \land x_k=(m+1, 0, y)$,
        then by our hypothesis we know $A(m, 1)=y$.
        Also, $A(m+1, 0)=y=A(m, 1)$, so this new element is still valid.
  \item The final case goes similarly- it's just more pattern matching.
\end{enumerate}
And that's the inductive step! $\square$

\pagebreak

\subsection{Part B}

Notice that the decision problem of checking whether
$\Braket{\Braket{m_0, n_0, y_0}, \cdots, \Braket{m_{k-1}, n_{k-1}, y_{k-1}}}$
encodes a valid Ackermann computation is recursive.

First let us define $\texttt{ValidInd}(\Braket{\Braket{m_0, n_0, y_0}, \cdots, \Braket{m_{k-1}, n_{k-1}, y_{k-1}}}, i)$
as met iff $\vec{x}_{i}$ exists and meets the definition's conditions.

This can be coded with the following characteristic function:
\begin{align*}
  \ch_{\texttt{ValidInd}}(\cdots)
   & = (m_i=0 \land y_i=n_i+1) \\
   & \ \lor (n_i=0 \land \exists j < i: m_i=m_j+1 \land n_j=1 \land y_i=y_j) \\
   & \ \lor (\exists j, l < i: m_j=m_i \land n_j+1=n_i \land m_l+1=m_i \land n_l=y_j \land y_l=y_i)
\end{align*}

With this function, we can define
\[\ch_{\texttt{ValidAck}}(u)=\forall 0 \le i < \len(u)\ (\texttt{ValidInd}(u, i))\]

And now we can finally define $A$ as follows:
\begin{align*}
  A(m, n)=\mu y[ & \ch_{\texttt{ValidAck}}(y)       \\
  \land          & \proj(\proj(y, \len(y)-1), 0)=m  \\
  \land          & \proj(\proj(y, \len(y)-1), 1)=n]
\end{align*}
Or in other words, it tries to find the least $y$ s.t. $(m, n, y)$ appears
at the end of some valid Ackermann computation sequence. $\square$

\pagebreak

\section{Problem 8}

\subsection{Part A}

I've translated the instructions into a sort of pseudocode:
\begin{algorithmic}[1]
  \State $R_1 \gets R_0$
  \State $R_0 \gets R_0 + 1$
  \State $R_2 \gets R_2 + 1$
  \If{$R_1 = R_2$}
    \State Exit
  \EndIf
  \State Go to line 2
\end{algorithmic}
When $R_0$ starts at $2$, the machine sets $R_1$ to $2$ as well.
It then simultaneously increments $R_2$ and $R_0$ until the former hits $2$,
at which point $\boxed{R_0=4}$.

\subsection{Part B}

I believe $\boxed{f_P^{(1)}(x)=2x}$.

The machine increments $R_0$ and $R_2$ for as many times as it takes $R_2$
to reach $R_1$'s value, which is precisely the initial value of $R_0$.
Since $R_0$ is incremented this many times as well, it doubles
by the time the program terminates.

\section{Problem 9}

Here's the pseudocode:
\begin{algorithmic}[1]
  \State $R_1 \gets R_1 + 1$
  \State $R_1 \gets R_1 + 1$
  \If{$R_0 = R_1$}
    \State Exit
  \EndIf
  \State Go to line 1
\end{algorithmic}

As can be seen, the program adds $2$ to $R_1$ (which starts at $0$) until it equals that of $R_0$.

For $f_P^{(1)}(x)\downarrow$, we need $x$ to be even.
If $R_0$ is odd, the two registers will never be equal and the program will not halt.

\pagebreak

\section{Problem 10}

\subsection{Part A}

True.

Notice that assuming a fixed program, the state of a URM can be completely
identified by the instruction number it's at, $x$, along with $R_i$, it's registers.

The program starts with $x=0$ and $R_i$ being whatever input we gave it.
However, after $J(0, 0, 0)$, $x$ is still $0$ and no changes have been made to $R_i$ whatsoever.
Given that the machine is still in the exact same state after one transition,
we can say that it won't ever be able to leave it.

\subsection{Part B}

True.

The starting state of $f_P^{n+k}(x_0, \cdots, x_{n-1}, 0, \cdots, 0)$ is the following:
\[R_i=\begin{cases}
  x_i & i < n \\
  0 & n \le i < n+k \\
  0 & \text{otherwise}
\end{cases}\]
OTOH, the start of $f_P^{n+k}(x_0, \cdots, x_{n-1})$ is this:
\[R_i=\begin{cases}
  x_i & i < n \\
  0 & \text{otherwise}
\end{cases}\]
It should be obvious that these two starting registers sets are exactly the same.
Thus, the convergence of one implies the convergence of the other and vice versa.

\subsection{Part C}

True.

For any program $P$, we can always add some nonsense instruction that won't
affect the execution of the program, like $J(n, n, -1)$.

Any time the program jumps to that instruction, the original $P$ would've halted.
This program would first take an extra step (comparing $R_n$ to $R_n$) and then
jumping to $I_{-1}$, a nonexistent instruction.
This doesn't modify any data and won't be referred to in the original program
except to exit, so the overall functionality remains the same.

However, we can choose any $n$ for this instruction,
so technically there's infinitely many $P'$s that function the same as $P$.

\end{document}
