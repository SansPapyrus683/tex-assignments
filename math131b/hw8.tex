\documentclass[12pt]{article}

% a template that a friend gave, it's worked well enough for me
% i have added some packages and stuff that have proved useful

\usepackage{fancyhdr}
\usepackage{tipa}
\usepackage{fontspec}
\usepackage{amsfonts}
\usepackage{enumitem}
\usepackage[margin=1in]{geometry}
\usepackage{graphicx}
\usepackage{float}
\usepackage{amsmath}
\usepackage{braket}
\usepackage{amssymb}
\usepackage{booktabs}
\usepackage{hyperref}
\usepackage{mathtools}
\usepackage{xcolor}
\usepackage{float}
\usepackage{algpseudocodex}
\usepackage{titlesec}
\usepackage{bbm}

\pagestyle{fancy}
\fancyhf{} % sets both header and footer to nothing
\lhead{Kevin Sheng}
\setmainfont{Comic Neue}
\renewcommand{\headrulewidth}{1pt}
\setlength{\headheight}{0.75in}
\setlength{\oddsidemargin}{0in}
\setlength{\evensidemargin}{0in}
\setlength{\voffset}{-.5in}
\setlength{\headsep}{10pt}
\setlength{\textwidth}{6.5in}
\setlength{\headwidth}{6.5in}
\setlength{\textheight}{8in}
\renewcommand{\headrulewidth}{0.5pt}
\renewcommand{\footrulewidth}{0.3pt}
\setlength{\textwidth}{6.5in}
\usepackage{setspace}
\usepackage{multicol}
\usepackage{float}
\setlength{\columnsep}{1cm}
\setlength\parindent{24pt}
\usepackage [english]{babel}
\usepackage [autostyle, english = american]{csquotes}
\MakeOuterQuote{"}

\setlength{\parskip}{6pt}
\setlength{\parindent}{0pt}

\titlespacing\section{0pt}{12pt plus 4pt minus 2pt}{0pt plus 2pt minus 2pt}
\titlespacing\subsection{0pt}{12pt plus 4pt minus 2pt}{0pt plus 2pt minus 2pt}
\titlespacing\subsubsection{0pt}{12pt plus 4pt minus 2pt}{0pt plus 2pt minus 2pt}

\hypersetup{colorlinks=true, urlcolor=blue}

\newcommand{\correction}[1]{\textcolor{red}{#1}}


\rhead{Math 131B}

\makeatletter
\def\@seccntformat#1{%
  \expandafter\ifx\csname c@#1\endcsname\c@section\else
  \csname the#1\endcsname\quad
  \fi}
\makeatother

\DeclareMathOperator{\Fr}{Fr}
\newcommand{\lra}{\xLeftrightarrow}
\newcommand{\ra}{\xRightarrow}
\newcommand{\N}{\mathbb{N}}
\newcommand{\R}{\mathbb{R}}
\newcommand{\Z}{\mathbb{Z}}
\newcommand{\Q}{\mathbb{Q}}
\newcommand{\C}{\mathbb{C}}
\newcommand{\norm}[1]{\left\lVert#1\right\rVert}

\begin{document}

\setcounter{section}{2}

\section{Problem 3}

\subsection{Periodicity}\label{sec:p3p1}

We apply the addition formula:
\begin{align*}
  \sin(x+\pi)
   & = \sin(x)\cos(\pi) + \sin(\pi)\cos(x) \\
   & = -\sin(x) + 0                        \\
   & = -\sin(x)                            \\ \\
  \cos(x+\pi)
   & = \cos(x)\cos(\pi)-\sin(x)\sin(\pi)   \\
   & = -\cos(x) - 0                        \\
   & = -\cos(x)\quad\square
\end{align*}
Periodicity is quite straightforward given this, as
\[\sin(x+2\pi)=\sin((x+\pi)+\pi)=-\sin(x+\pi)=\sin(x)\]

\subsection{Zeroes of Sine}

I'll prove that $\sin(n\pi)=0\ \forall n \in \N$ by induction.
The base case where $n=1$ is true since we straight up defined $\sin(\pi)=0$.

Then, if the statement is true for $n$, it must be true for $n+1$ since
\[\sin((n+1)\pi)=\sin(n\pi+\pi)=-\sin(n\pi)=0\]
By similar reasoning, the same must be true if we extend to the negative integers.

For the other direction, we start off with that $\sin(x) \ne 0\ \forall x \in (0, \pi)$.
With this being the case, then, for any $x \in (\pi, 2\pi)$
\[\exists y \in (0, \pi): x=y+\pi \implies \sin(x)=\sin(y+\pi)=-\sin(y) \ne 0\]
and we repeat until we get all the non-integer multiples of $\pi$. $\square$

\pagebreak

\subsection{And Cosine}\label{sec:p3p3}

We again proceed by induction, but now with
$\cos\left(\left(n-\frac{1}{2}\right)\pi\right)=0\ \forall n \in \N$.

For the base case, notice that
\begin{align*}
             & \cos\left(\pi-\frac{\pi}{2}\right)
  = \cos(\pi)\cos\left(\frac{\pi}{2}\right) + \sin(\pi)\sin\left(\frac{\pi}{2}\right) \\
  \implies{} & \cos\left(\frac{\pi}{2}\right) = -\cos\left(\frac{\pi}{2}\right) + 0   \\
  \implies{} & \cos\left(\frac{\pi}{2}\right) = -\cos\left(\frac{\pi}{2}\right)       \\
  \implies{} & \cos\left(\frac{\pi}{2}\right) = 0
\end{align*}
Then we do the exact same thing with regards to the application of \ref{sec:p3p1}
to see that the statement for cosine also holds true.

As for the other direction, notice that $\cos(x) > 0\ \forall x \in \left(0, \frac{\pi}{2}\right)$.

This is because if there was such an $x$ s.t. $\cos(x) \le 0$, then
by the MVT there would exist some $y \in \left(x, \frac{\pi}{2}\right)$
with nonnegative derivative, which contradicts that $\frac{d}{dx} \cos(x)=-\sin(x)$
is strictly positive on the specified interval.

But with this interval proven,
the rest goes the exact same as the previous section. $\square$

\pagebreak

\setcounter{section}{4}

\section{Problem 5}

 (i'm assuming we don't have to prove 4.7.4 and can just use it)

Fix $a+bi \in \C$.

Letting $r=|a+bi|$, we see that
\[\left(\frac{a}{r}\right)^2+\left(\frac{b}{r}\right)^2=1\]
allowing us to apply the result from Exercise 4.7.4 to $\left(\frac{a}{r}, \frac{b}{r}\right) \in \R^2$.

There's exactly one (1) $\theta$ s.t. $\frac{a}{r}=\cos(\theta)$ and $\frac{b}{r}=\sin(\theta)$.
Multiplying both sides by $r$ gets us $a=r\cos(\theta)$ and $b=r\sin(\theta)$.

Thus,
\begin{align*}
  re^{i\theta}
   & = r(\cos(\theta)+i\sin(\theta)) \\
   & = r\cos(\theta)+ir\sin(\theta)  \\
   & = a+bi
\end{align*}

Notice that if $r \ne |a+bi|$, no matter what $\theta$ we choose
\begin{align*}
  \left|re^{i\theta}\right|
   & = \left|r\cos(\theta)+ir\sin(\theta)\right|
   & = \sqrt{r^2\sin^2(\theta)+r^2\cos^2(\theta)} \\
   & = r                                          \\
   & \ne |a+bi|
\end{align*}
so no value of $\theta$ could rectify the difference in $r$.

Thus, there's one and only one pair of $r$ and $\theta$ that work.

\pagebreak

\section{Problem 6}

Let $I=\left(-\frac{\pi}{2}, \frac{\pi}{2}\right)$ for convenience.

By a step in \ref{sec:p3p3}, $\cos(x)$ is nonzero on $I$, so
$\tan$ is indeed differetiable on $i$ with derivative
\begin{align*}
  \tan'(x)
   & = \frac{\sin'(x)\cos(x)-\cos'(x)\sin(x)}{\cos^2(x)}       \\
   & = \frac{\cos^2(x)+\sin^2(x)}{\cos^2(x)}                   \\
   & = \frac{\cos^2(x)}{\cos^2(x)}+\frac{\sin^2(x)}{\cos^2(x)} \\
   & = 1+\tan^2(x)
\end{align*}
As this is strictly positive, $\tan$ is indeed strictly increasing on $I$.

Since $\cos\left(\frac{\pi}{2}\right)=1$
and sine is positive on $(0, \pi)$, $\sin\left(\frac{\pi}{2}\right)=1$.
and as sine is continuous $\lim_{n \to \frac{\pi}{2}} \sin(x) = 1$ as well.
However, the same limit for cosine goes to $0$, so $\lim_{n \to \frac{\pi}{2}} \tan(x)=+\infty$.
By similar logic we have $\lim_{n \to -\frac{\pi}{2}} \tan(x)=-\infty$.

$\tan$ is increasing and therefore injective, which means it's a bijection onto its range,
which as we've just shown is $(-\infty, \infty)=\R$.
Thus, we can create an inverse function $\arctan(x): \R \to I$.

Since $\tan'(x) \ne 0$, $\arctan$ is differetiable with derivative
\begin{align*}
  \arctan'(x)
   & =\frac{1}{\tan'(\arctan(x))}    \\
   & =\frac{1}{1+\tan^2(\arctan(x))} \\
   & =\frac{1}{1+x^2}
\end{align*}

\pagebreak

\section{Problem 7}

\subsection{Bounded}

Consider any function $f \in C(\R/\Z, \C)$

I'll first show that $f$ is bounded on $[0, 1]$.

Since the interval is compact, its image $f([0, 1]) \subseteq \C$ must be compact and therefore bounded.
This means $\exists M \in \R: |x-y| < M\ \forall x, y \in f([0, 1])$

Then, if we fix any $y$ in the image, by the triangle inequality
\[|x| < |x-y| + |y| < M + |y|\]
so indeed $\exists M' \in \R: |x| < M'$ across all $x$ and $f([0, 1])$ is bounded.

(i'm not sure if all this was actually needed but better safe than sorry)

Now, since $f$ is $\Z$-periodic, for all $x$ $\exists y \in [0, 1]: f(x)=f(y) < M'$.
Thus, $f$ on the entirety of $\R$ is bounded as well. $\square$

\subsection{Vector Space/Algebra Properties}

For addition we have that
\begin{align*}
  (f+g)(x+1)
   & = f(x+1)+g(x+1) \\
   & = f(x)+g(x)     \\
   & = (f+g)(x)
\end{align*}
so $f+g$ is still $\Z$-periodic.

For multiplication, you can replace the addition with multiplication
and nothing would change.

Also, since all constant functions are $\Z$-periodic,
constant multiplcation is just a special case of function multiplcation. $\square$

\subsection{Closed in \texorpdfstring{$C(\R/\Z, \C)$}{C(R, C)}}

Fix a sequence $f_n \subseteq C(\R/\Z, \C)$ and let its limit be $f$.

It STP $f(x)=f(x+1)$ for any $x$, so fix an $x$ as well.

Since $f_n$ uniformly converges to $f$, we have
\begin{align*}
  \lim_{n \to \infty} f_n(x)=f(x) && \lim_{n \to \infty} f_n(x+1)=f(x+1)
\end{align*}
However, since all the $f_n$s are $\Z$-periodic,
the two sequences are identical and their limits must be too, making $f(x)=f(x+1)$. $\square$

\section{Problem 8}

\subsection{Is Metric Space}

This proof works for $C(\R, \C)$ too actually!

But anyways, $d_\infty$ is clearly nonnegative.

If $f=g$ then $|f(x)-g(x)|=0\ \forall x \in \R$ and $d_\infty(f, g)=0$.

OTOH, if $f \ne g$ then $\exists x: f(x) \ne g(x) \implies |f(x)-g(x)| > 0$, so $d_\infty(f, g) > 0$.

As for the triangle inequality, fix $f$, $g$, and $h$.
Then
\begin{align*}
  d_\infty(f, g)
  &= \sup_{x \in \R} |f(x)-g(x)| \\
  &\le \sup_{x \in \R} |f(x)-h(x)|+|h(x)-g(x)| \\
  &\le \sup_{x \in \R} |f(x)-h(x)| + \sup_{x \in \R} |h(x)-g(x)| \\
  &= d_\infty(f, h) + d_\infty(h, g)\quad\square
\end{align*}

\subsection{And Complete}

Didn't we show this earlier?
$\C$ is complete, so $C(\R, \C)$ is complete under $d_\infty$.

$C(\R/\Z, \C)$ is closed within a complete set,
so it must be complete itself. $\square$

\end{document}
