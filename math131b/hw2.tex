\documentclass[12pt]{article}

% a template that a friend gave, it's worked well enough for me
% i have added some packages and stuff that have proved useful

\usepackage{fancyhdr}
\usepackage{tipa}
\usepackage{fontspec}
\usepackage{amsfonts}
\usepackage{enumitem}
\usepackage[margin=1in]{geometry}
\usepackage{graphicx}
\usepackage{float}
\usepackage{amsmath}
\usepackage{braket}
\usepackage{amssymb}
\usepackage{booktabs}
\usepackage{hyperref}
\usepackage{mathtools}
\usepackage{xcolor}
\usepackage{float}
\usepackage{algpseudocodex}
\usepackage{titlesec}
\usepackage{bbm}
\usepackage{pythonhighlight}

\pagestyle{fancy}
\fancyhf{} % sets both header and footer to nothing
\lhead{Kevin Sheng}
\setmainfont{Comic Neue}
\renewcommand{\headrulewidth}{1pt}
\setlength{\headheight}{0.75in}
\setlength{\oddsidemargin}{0in}
\setlength{\evensidemargin}{0in}
\setlength{\voffset}{-.5in}
\setlength{\headsep}{10pt}
\setlength{\textwidth}{6.5in}
\setlength{\headwidth}{6.5in}
\setlength{\textheight}{8in}
\renewcommand{\headrulewidth}{0.5pt}
\renewcommand{\footrulewidth}{0.3pt}
\setlength{\textwidth}{6.5in}
\usepackage{setspace}
\usepackage{multicol}
\usepackage{float}
\setlength{\columnsep}{1cm}
\setlength\parindent{24pt}
\usepackage [english]{babel}
\usepackage [autostyle, english = american]{csquotes}
\MakeOuterQuote{"}

\setlength{\parskip}{6pt}
\setlength{\parindent}{0pt}

\titlespacing\section{0pt}{12pt plus 4pt minus 2pt}{0pt plus 2pt minus 2pt}
\titlespacing\subsection{0pt}{12pt plus 4pt minus 2pt}{0pt plus 2pt minus 2pt}
\titlespacing\subsubsection{0pt}{12pt plus 4pt minus 2pt}{0pt plus 2pt minus 2pt}

\hypersetup{colorlinks=true, urlcolor=blue}

\newcommand{\correction}[1]{\textcolor{red}{#1}}


\rhead{Math 131B}

\makeatletter
\def\@seccntformat#1{%
  \expandafter\ifx\csname c@#1\endcsname\c@section\else
  \csname the#1\endcsname\quad
  \fi}
\makeatother

\DeclareMathOperator{\Fr}{Fr}
\newcommand{\lra}{\xLeftrightarrow}
\newcommand{\ra}{\xRightarrow}
\newcommand{\N}{\mathbb{N}}
\newcommand{\R}{\mathbb{R}}
\newcommand{\Z}{\mathbb{Z}}
\newcommand{\Q}{\mathbb{Q}}
\newcommand{\norm}[1]{\left\lVert#1\right\rVert}

\begin{document}

\section{Problem 1}

\subsection{Limit Point Implies Subsequence}

Consider the subsequence $x_{k_n}$ where
$k_n$ is any $n$ s.t. $d(x_{k_n}, L) < \frac{1}{n}$ and $k_n > k_{n-1}$.
If $n=1$, the second constraint doesn't apply.

By construction, this is both a valid subsequence and converges to $L$.

\subsection{Subsequence Implies Limit Point}

Fix $N \in \N$ and $\epsilon > 0$.

Since $k_n$ is a subsequence, $\exists M_1: k_n \ge M_1\ \forall n \ge M_1$.
$x_{k_n}$ also converges, so $\exists M_2: d(x_{k_m}, L) < \epsilon\ \forall m \ge M_2$.

Taking $n = \max(M_1, M_2)$, we see that this number is both greater than $N$
and its distance from $L$ is less than $\epsilon$. $\square$

\section{Problem 2}

\subsection{Complete Implies Closed}

Consider any convergent sequence in $Y$.
All convergent sequences are Cauchy, and since $Y$ is complete
this sequence must converge in itself, making $Y$ closed. $\square$

\subsection{Closed In Complete}

Consider any Cauchy sequence in $Y$.
Since $X$ is complete and $Y \subseteq X$, this must converge in $X$.
Also, $Y$ is closed, so any convergent sequence (including this one) in it
must converge in itself as well. $\square$

\pagebreak

\section{Problem 3}

BWOC say $\bigcap_{n \in N} K_n = \varnothing$.

Then $K_1 \cap \bigcap_{n=2}^\infty K_n = \varnothing$, so
$K_1$ is contained in the complement of the big cap.

All $K_n$s are compact and therefore closed, so their negation must be open.
Also,
\[\left(\bigcap_{n=2}^\infty K_n\right)^C=\bigcup_{n=2}^\infty K_n^C\]
so the complements of the $K_n$s must form an open cover of $K_1$.

This then implies a finite subcover $S \subseteq \N \setminus \{1\}$, so
\[K_1 \subseteq \bigcup_{n \in S} K_n^C = \left(\bigcap_{n \in S} K_n\right)^C
\implies K_1 \cap \bigcap_{n \in S} K_n = \varnothing\]
However, the finite number of sets are all nonempty and nested within each other,
which is a contradiction. $\square$

\pagebreak

\section{Problem 4}

\subsection{Part A}

The forward direction comes for free since compactness always implies closedness.

For the backward direction, assume $Z$ is closed and fix any sequence within $Z$.
This sequence is also in $Y$, so there's a subsequence within that converges in $Y$.
However, since $Z$ is closed, any sequence in it that converges in $Y$ must also converge in itself.
Thus, this convergent subsequence also converges in $Z$, so $Z$ is compact. $\square$

\subsection{Part B}

Consider any open cover of $\bigcup_{i=1}^n Y_i$.

This also covers each $Y_i$ individually, so there exists a finite subcover for each of them.

Notice that we can take the union of all these to get a finite subcover for $\bigcup_{i=1}^n Y_i$.
A finite union of finite subcovers is still finite,
so the overall union of $Y_i$s is compact too. $\square$

\subsection{Part C}

Let $Y$ be our finite set and $\{X_\alpha\}$ be any open cover of it.

Since $Y$ is finite, lemme just enumerate them from $y_1$ to $y_n$, where $n=|Y|$.

We can always construct a finite subcover by letting
$Y \subseteq \bigcup_{i=1}^n X_i$, where $X_i$ is any set in the cover that has $y_i$.

Any $y_i$ is accounted for in the corresponding $X_i$ and there's obviously a finite amount.
Also, each $X_i$ is guaranteed to exist because if no $X_\alpha$ contained $y_i$,
then $\{X_\alpha\}$ wouldn't be considered a valid cover. $\square$

\pagebreak

\section{Problem 5}

I'll prove the contrapositive.
If $X$ wasn't compact, there's a sequence $x_n$ in it with no convergent subsequence i.e. no limit point.

Lemme first prove that this means
\[\forall x \in X\ \exists r > 0: |B_r(x) \cap (x_n)| \le 1\]
by showing the contrapositive.

Assume $\exists x \in X: \forall r > 0: |B_r(x) \cap (x_n)| > 1$.

Fix $\epsilon > 0$ and $N \in \N$.
Take $\epsilon' = \min_{i=1, \cdots, n} d(x, x_i)$ and $r=\min(\epsilon, \epsilon')$.

Then by our initial assumption $\exists n: d(x, x_n) < r$ and since $r \le \epsilon'$ $n \ge N$ as well,
which means $x$ is indeed a limit point of $(x_n)$.

With this, we can construct a cover $\{Y_\alpha\} = \{B_r(x) \mid x \in X\}$ where $r$
is the radius that we just proved always exists.
Each ball can has at most one element from $(x_n)$,
so any finite subfamily will always miss an element in $(x_n)$.$\square$

\section{Problem 6}

Let $R = \inf(\{d(x_0, y) \mid y \in E\})$.
By definition of the infimum, $\forall \epsilon > 0\ \exists y \in E: d(x_0, y) < R + \epsilon$.

Thus, we can construct a sequence $(x_n)$ where $x_n$ is any $y \in E: d(x_0, y) < R + \frac{1}{n}$.

$E$ is compact, so there must exist some subsequence $x_{k_n}$ that converges in $E$ as well.
It remains to prove that this limit point $x$ satisfies $d(x_0, x) = R$.

BWOC let $d(x_0, x) - R = \epsilon > 0$.
By construction, $\exists N \in \N: d(x_{k_n}, x_0) < R+\frac{\epsilon}{2}\ \forall n \ge N$.

Then by the triangle inequality we have
\begin{align*}
  & d(x, x_0) < d(x, x_{k_n}) + d(x_{k_n}, x_0) \\
  \implies{} & d(x, x_{k_n}) > d(x, x_0) - d(x_{k_n}, x_0) \\
  \implies{} & d(x, x_{k_n}) > (R + \epsilon) - \left(R + \frac{\epsilon}{2}\right) > 0
\end{align*}
which contradicts that $\lim_{n \to \infty} x_{k_n} = x$. $\square$

\pagebreak

\section{Problem 7}

\subsection{Openness Preservation Implies Continuity}

Fix $x \in X$ and $\epsilon > 0$.

$B_\epsilon(f(x))$ is open, so $I=f^{-1}(B_\epsilon(f(x)))$ is too.

$x$ is actually in that set since $f(x) \in I$, so $\exists \delta: B_\delta(x) \subseteq I$.

For all $x': d_X(x, x') < \delta$, $x' \in B_\delta(x) \subseteq I$
which means $d_Y(f(x'), f(x)) < \epsilon$. $\square$

\subsection{Also Closedness Preservation}

First lemme prove the following lemma for all $S \subseteq Y$:
\[\left(f^{-1}(S^C)\right)^C = f^{-1}(S)\]
As always, it STP subset-ness in both directions.

If $x \notin f^{-1}(S^C)$, $f(x) \notin S^C \implies f(x) \in S \implies x \in f^{-1}(S)$.

OTOH, if $x \in f^{-1}(S)$, $f(x) \in S \implies f(x) \notin S^C \implies x \notin f^{-1}(S^C) \implies x \notin \left(f^{-1}(S^C)\right)^C$.

Now consider any closed set $F \subseteq Y$.
$F^C \subseteq Y$ is open, so $f^{-1}(F^C)$ is open
and $\left(f^{-1}(F^C)\right)^C = f^{-1}(F)$ is closed. $\square$

\subsection{And Vice Versa}

Consider any open set $V \subseteq Y$.
$V^C$ is closed, so $f^{-1}\left(V^C\right)$ is also closed
and $\left(f^{-1}\left(V^C\right)\right)^C = f^{-1}(V)$ is open. $\square$

\subsection{Open Set to Nonopen Set}

Consider $f: \R \to \R$ where $f(x) = |x|$ and take $U = \R$ itself.

Though $U$ is open, $f(U) = [0, \infty)$, which isn't.

\subsection{Closed Set to Nonclosed Set}

Consider $f: \R \to \R$ where $f(x)=e^{-x}$ and take $K = [0, \infty)$.

Though $K$ is closed, $f$ maps it to $(0, 1]$, which isn't.

\end{document}
