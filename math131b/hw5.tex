\documentclass[12pt]{article}

% a template that a friend gave, it's worked well enough for me
% i have added some packages and stuff that have proved useful

\usepackage{fancyhdr}
\usepackage{tipa}
\usepackage{fontspec}
\usepackage{amsfonts}
\usepackage{enumitem}
\usepackage[margin=1in]{geometry}
\usepackage{graphicx}
\usepackage{float}
\usepackage{amsmath}
\usepackage{braket}
\usepackage{amssymb}
\usepackage{booktabs}
\usepackage{hyperref}
\usepackage{mathtools}
\usepackage{xcolor}
\usepackage{float}
\usepackage{algpseudocodex}
\usepackage{titlesec}
\usepackage{bbm}

\pagestyle{fancy}
\fancyhf{} % sets both header and footer to nothing
\lhead{Kevin Sheng}
\setmainfont{Comic Neue}
\renewcommand{\headrulewidth}{1pt}
\setlength{\headheight}{0.75in}
\setlength{\oddsidemargin}{0in}
\setlength{\evensidemargin}{0in}
\setlength{\voffset}{-.5in}
\setlength{\headsep}{10pt}
\setlength{\textwidth}{6.5in}
\setlength{\headwidth}{6.5in}
\setlength{\textheight}{8in}
\renewcommand{\headrulewidth}{0.5pt}
\renewcommand{\footrulewidth}{0.3pt}
\setlength{\textwidth}{6.5in}
\usepackage{setspace}
\usepackage{multicol}
\usepackage{float}
\setlength{\columnsep}{1cm}
\setlength\parindent{24pt}
\usepackage [english]{babel}
\usepackage [autostyle, english = american]{csquotes}
\MakeOuterQuote{"}

\setlength{\parskip}{6pt}
\setlength{\parindent}{0pt}

\titlespacing\section{0pt}{12pt plus 4pt minus 2pt}{0pt plus 2pt minus 2pt}
\titlespacing\subsection{0pt}{12pt plus 4pt minus 2pt}{0pt plus 2pt minus 2pt}
\titlespacing\subsubsection{0pt}{12pt plus 4pt minus 2pt}{0pt plus 2pt minus 2pt}

\hypersetup{colorlinks=true, urlcolor=blue}

\newcommand{\correction}[1]{\textcolor{red}{#1}}


\rhead{Math 131B}

\makeatletter
\def\@seccntformat#1{%
  \expandafter\ifx\csname c@#1\endcsname\c@section\else
  \csname the#1\endcsname\quad
  \fi}
\makeatother

\DeclareMathOperator{\Fr}{Fr}
\newcommand{\lra}{\xLeftrightarrow}
\newcommand{\ra}{\xRightarrow}
\newcommand{\N}{\mathbb{N}}
\newcommand{\R}{\mathbb{R}}
\newcommand{\Z}{\mathbb{Z}}
\newcommand{\Q}{\mathbb{Q}}
\newcommand{\norm}[1]{\left\lVert#1\right\rVert}

\begin{document}

\setcounter{section}{1}

\section{Problem 2}\label{sec:p2}

Consider any Cauchy sequence of functions $\left(f^{(n)}\right)$ from $X$ to $Y$.

I propose that they uniformly converge to the function $f$ defined by
\[f(x)=\lim_{n \to \infty} f^{(n)}(x)\]

\subsection{Limit Existence}

First, though, we need to show that this sequence of limits actually exists.

As $Y$ is complete, it STP the sequence is Cauchy, so let's fix an $\epsilon$ and
the $x_0$ on which we're applying $f$.

By our assumptions, $\left(f^{(n)}\right)$ is Cauchy and
\[\forall \epsilon > 0\ \exists N: d_\infty\left(f^{(n)}, f^{(m)}\right) < \epsilon\ \forall n, m \ge N\]
Notice that this is the same $N$ we need for our original sequence.

This is because
\begin{align*}
    & d_\infty\left(f^{(n)}, f^{(m)}\right) < \epsilon \\
    \implies{} & \sup_{x \in X} d_Y\left(f^{(n)}(x), f^{(m)}(x)\right) < \epsilon \\
    \implies{} & d_Y\left(f^{(n)}(x_0), f^{(m)}(x_0)\right) < \epsilon\ \forall n, m \ge N
\end{align*}

\subsection{Function Convergence}

Fix an $\epsilon$ and choose $N$ s.t.
$d_\infty\left(f^{(n)}, f^{(m)}\right) < \frac{\epsilon}{2}\ \forall n, m \ge N$.

I'm pretty sure this $N$ will work for our uniform convergence.
At any point $x$, $\exists N': d_Y\left(f^{(n)}(x), f(x)\right) < \frac{\epsilon}{2}\ \forall n \ge N'$.

If we let $m=\max(N, N')$, then for all $n \ge N$ we have
\begin{align*}
    d_Y\left(f^{(n)}(x), f(x)\right)
    &\le d_Y\left(f^{(n)}(x), f^{(m)}(x)\right) + d_Y\left(f^{(m)}(x), f(x)\right) \\
    &< \frac{\epsilon}{2} + \frac{\epsilon}{2} \\
    &= \epsilon
\end{align*}
which implies $d_\infty\left(f^{(n)}, f\right) < \epsilon$.

Since uniform convergence preserves boundedness and continuity, $f \in C(X, Y)$.
Thus, we've shown that $C(X, Y)$ is complete. $\square$

\pagebreak

\section{Problem 3}

First off, notice that for any two functions $f$ and $g$ $d_\infty(f, f+g) = \norm{g}_\infty$.
This is because at any point $f(x)$ and $f(x)+g(x)$ differ by exactly $g(x)$.

But anyways, it STP $\sum_{n=1}^\infty f^{(n)}$ is Cauchy by what we've just shown in \ref{sec:p2}.
In other words, for any $\epsilon > 0$ we need an $N$ s.t. for all $m > n \ge N$
\[d_\infty\left(\sum_{i=1}^n f^{(i)}, \sum_{i=1}^m f^{(i)}\right) < \epsilon
\implies \norm{\sum_{i=n+1}^m f^{(i)}}_\infty < \epsilon\]

Since $\sum_{i=1}^\infty \norm{f^{(i)}}_\infty$ converges,
it satisfies the Cauchy criterion, so we can find an $N$ s.t. for all $m > n \ge N$
\[\sum_{i=n+1}^m \norm{f^{(i)}}_\infty < \epsilon\]

This is the $N$ that we want, since
\begin{align*}
    \epsilon 
    &> \sum_{i=n+1}^m \sup_{x \in X} |f^{(i)}(x)| \\
    &\ge \sup_{x \in X} \sum_{i=n+1}^m |f^{(i)}(x)| \\
    &= \norm{\sum_{i=n+1}^m f^{(i)}}_\infty \quad\square
\end{align*}

\section{Problem 4}

This is just chasing some definitions.
\begin{align*}
    \sum_{n=1}^\infty \int_a^b f^{(n)}
    &= \lim_{N \to \infty} \sum_{n=1}^N \int_a^b f^{(n)} \\
    &= \lim_{N \to \infty} \int_a^b \sum_{n=1}^N f^{(n)} & \text{finite sum of integrals is integral of sums} \\
    &= \int_a^b \lim_{N \to \infty} \sum_{n=1}^N f^{(n)} & \text{by the theorem we were told to use} \\
    &= \int_a^b \sum_{n=1}^\infty f^{(n)}\quad\square
\end{align*}

\pagebreak

\section{Problem 5}

\subsection{Differentiable}

Applying differentiation rules gives us
\[\frac{d}{dx} \sqrt{\frac{1}{n^2}+x^2}=\frac{1}{2\sqrt{\frac{1}{n^2}+x^2}} \cdot 2x=\boxed{\frac{1}{\sqrt{\frac{1}{n^2}+x^2}}}\]

\subsection{Bounded}

Indeed, for the left inequality we see that
\begin{align*}
    & x^2 \le \frac{1}{n^2}+x^2 && \text{$n$ is positive} \\
    \implies{} & \sqrt{x^2} \le f^{(n)}(x) && \text{square root is monotonic} \\
    \implies{} & |x| \le f^{(n)}(x)
\end{align*}
and for the right side we similarly have
\begin{align*}
    & \frac{1}{n^2}+x^2 \le \left(|x|+\frac{1}{n}\right)^2 \\
    \implies{} & f^{(n)}(x) \le |x|+\frac{1}{n}
\end{align*}

\subsection{Limit Not Differentiable}

If $h > 0$, $\frac{|h|-|0|}{h}=\frac{h}{h}=1$..
However, if $h < 0$, $\frac{|h|-|0|}{h}=\frac{-h}{h}=-1$.
Since the left hand and right hand limits aren't equivalent, $f(x)=|x|$ isn't differentiable at $0$.

\pagebreak

\section{Problem 6}

I'll just write $f(x)=L-\int_{a}^{x_0} g(t)\,dt+\int_{a}^{x} g(t)\,dt$ instead of what Tao does.

\subsection{Uniform Convergence}

Fix an $\epsilon$ and choose $N$ large enough s.t. the following hold true for any $n \ge N$:
\begin{itemize}[nolistsep]
    \item $d_\infty(f_n, g) < \epsilon$
    \item $|f_n(x_0)-L| < \epsilon$
\end{itemize}

Now, consider $|f_n(x)-f(x)|$ across all $x$:
\begin{align*}
    |f_n(x)-f(x)|
    &= \left|\left(L-\int_{a}^{x_0} g(t)\,dt\right)+\int_{a}^{x} g(t)\,dt-f_n(x)\right| \\
    &= \left|\left(L-\int_{a}^{x_0} g(t)\,dt\right)+\int_{a}^{x} g(t)\,dt-\int_{a}^{x} f_n'(t)\,dt-f_n(a)\right| \\
    &\le \left|L-\int_{a}^{x_0} g(t)\,dt-f_n(a)\right|+\left|\int_{a}^{x} g(t)\,dt-\int_{a}^{x} f_n'(t)\,dt\right| \\
    &\le \left|L-\int_{a}^{x_0} g(t)\,dt-f_n(a)\right|+\left|\int_{a}^{x} g(t)-f_n'(t)\,dt\right| \\
    &\le \left|L-\int_{a}^{x_0} g(t)\,dt-f_n(a)\right|+\epsilon(b-a)
\end{align*}

The first term requires a bit more work.
Notice that by the triangle inequality,
\begin{align*}
    &\left|L-\int_{a}^{x_0} g(t)\,dt-f_n(a)\right| \\
    \le{} & |(L-f_n(a))-(f_n(x_0)-f_n(a))|+\left|(f_n(x_0)-f_n(a))-\int_{a}^{x_0} g(t)\,dt\right| \\
    \le{} & |L-f_n(a)|+\left|(f_n(x_0)-f_n(a))-\int_{a}^{x_0} g(t)\,dt\right| \\
    \le{} & \epsilon+\left|\int_{a}^{x_0} f_n'(t)\,dt-\int_{a}^{x_0} g(t)\,dt\right| \\
    \le{} & \epsilon + (b-a)\epsilon
\end{align*}

So we see that for $n$ chosen wisely $|f_n(x)-f(x)|<(2b-2a+1)\epsilon$ for all $x$.
The left coefficient is constant, so as $\epsilon \to 0$ the distance should to.
Thus, $f_n$ uniformly converges to $f$.

\subsection{Derivative is \texorpdfstring{$g$}{g}}

Thankfully, this part is pretty simple by the FTC.

Since both of the first two terms are constant w.r.t $x$, their derivative becomes $0$.
That just leaves
\[\frac{d}{dx} \int_{a}^{x} g(t)\,dt = g(x)\quad\square\]

\end{document}
