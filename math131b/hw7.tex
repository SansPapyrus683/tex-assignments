\documentclass[12pt]{article}

% a template that a friend gave, it's worked well enough for me
% i have added some packages and stuff that have proved useful

\usepackage{fancyhdr}
\usepackage{tipa}
\usepackage{fontspec}
\usepackage{amsfonts}
\usepackage{enumitem}
\usepackage[margin=1in]{geometry}
\usepackage{graphicx}
\usepackage{float}
\usepackage{amsmath}
\usepackage{braket}
\usepackage{amssymb}
\usepackage{booktabs}
\usepackage{hyperref}
\usepackage{mathtools}
\usepackage{xcolor}
\usepackage{float}
\usepackage{algpseudocodex}
\usepackage{titlesec}
\usepackage{bbm}
\usepackage{pythonhighlight}

\pagestyle{fancy}
\fancyhf{} % sets both header and footer to nothing
\lhead{Kevin Sheng}
\setmainfont{Comic Neue}
\renewcommand{\headrulewidth}{1pt}
\setlength{\headheight}{0.75in}
\setlength{\oddsidemargin}{0in}
\setlength{\evensidemargin}{0in}
\setlength{\voffset}{-.5in}
\setlength{\headsep}{10pt}
\setlength{\textwidth}{6.5in}
\setlength{\headwidth}{6.5in}
\setlength{\textheight}{8in}
\renewcommand{\headrulewidth}{0.5pt}
\renewcommand{\footrulewidth}{0.3pt}
\setlength{\textwidth}{6.5in}
\usepackage{setspace}
\usepackage{multicol}
\usepackage{float}
\setlength{\columnsep}{1cm}
\setlength\parindent{24pt}
\usepackage [english]{babel}
\usepackage [autostyle, english = american]{csquotes}
\MakeOuterQuote{"}

\setlength{\parskip}{6pt}
\setlength{\parindent}{0pt}

\titlespacing\section{0pt}{12pt plus 4pt minus 2pt}{0pt plus 2pt minus 2pt}
\titlespacing\subsection{0pt}{12pt plus 4pt minus 2pt}{0pt plus 2pt minus 2pt}
\titlespacing\subsubsection{0pt}{12pt plus 4pt minus 2pt}{0pt plus 2pt minus 2pt}

\hypersetup{colorlinks=true, urlcolor=blue}

\newcommand{\correction}[1]{\textcolor{red}{#1}}


\rhead{Math 131B}

\makeatletter
\def\@seccntformat#1{%
  \expandafter\ifx\csname c@#1\endcsname\c@section\else
  \csname the#1\endcsname\quad
  \fi}
\makeatother

\DeclareMathOperator{\Fr}{Fr}
\newcommand{\lra}{\xLeftrightarrow}
\newcommand{\ra}{\xRightarrow}

\begin{document}

\setcounter{section}{3}

\section{Problem 4}

Fix an $a \in (0, \infty)$.
We just have to find a power series with a nonzero radius of convergence.

\subsection{Initial Equality}

Differentiating $\ln (a-x)$, we see that
\begin{align*}
    \frac{d}{dx} \ln (a-x)
     & = -\frac{1}{a-x}                                             \\
     & = \frac{1/a}{1-x/a}                                          \\
     & = -\sum_{n=0}^{\infty} \frac{1}{a}\left(\frac{x}{a}\right)^n
\end{align*}
with the final equality only being true if $\left|\frac{x}{a}\right| < 1 \implies |x| < a$.

Notice that the series of functions $f_n(x)=\frac{1}{a}\left(\frac{x}{a}\right)^n$ converges
uniformly on $[-b, b]$ where $0 < b < a$ by the M-Test, since $\norm{f_n}_\infty=\frac{b^n}{a^{n+1}}$ and
\[\sum_{n=0}^{\infty} \frac{b^n}{a^{n+1}}=\frac{1}{a} \sum_{n=0}^{\infty} \left(\frac{b}{a}\right)^n\]
which conveges as $b < a$.

Fix an $x_0 \in (-a, a)$ and WLOG assume it's positive.
Integrating both sides gives us
\begin{align*}
    \left.\left(\ln(a-x)\right)\right|^{x_0}_0
     & = \int_{0}^{x_0} -\sum_{n=0}^{\infty} \frac{1}{a}\left(\frac{x}{a}\right)^n\,dx                 \\
     & = -\frac{1}{a} \int_{0}^{x_0} \sum_{n=0}^{\infty} \left(\frac{x}{a}\right)^n\,dx                \\
     & = -\frac{1}{a} \sum_{n=0}^{\infty} \int_{0}^{x_0} \left(\frac{x}{a}\right)^n\,dx                \\
     & = -\frac{1}{a} \sum_{n=0}^{\infty} \left.\left(\frac{x^{n+1}}{(n+1)a^n}\right)\right|^{x_0}_{0} \\
     & = - \sum_{n=0}^{\infty} \frac{x_0^{n+1}}{(n+1)(a+1)^n}                                          \\
     & = -\sum_{n=1}^{\infty} \frac{x_0}{n a^n}
\end{align*}
where we can swap the integral and the summation by uniform convergence.

\subsection{Actually Using It}

The previous section showed for all $x \in (-a, a)$ that
\[\ln(a-x) = \ln a -\sum_{n=1}^{\infty} \frac{x^n}{na^n}\]
(I implicitly evaluated the LHS over there and moved a log over, my bad).

If we let $x'=a-x$, we see that across all $x' \in (0, 2a)$
\begin{align*}
    \ln x'
     & = \ln a - \sum_{n=1}^{\infty} \frac{(a-x')^n}{na^n}           \\
     & = \ln a - \sum_{n=1}^{\infty} \frac{(-1)^n(x'-a)^n}{na^n}     \\
     & = \ln a + \sum_{n=1}^{\infty} \frac{(-1)^{n+1}}{na^n}(x'-a)^n
\end{align*}
finally giving us our power series.
The random log in the front is just $c_0$. $\square$

\pagebreak

\section{Problem 5}

$f$ is analytic at $0$, so there's a $c_n$ s.t. the power series
\[\sum_{n=0}^{\infty} c_n(x-0)^n\]
has a positive radius of convergence.

By Taylor's formula, for any $x$ within this radius
\begin{align*}
    f(x)
     & = \sum_{n=0}^{\infty} c_n x^n                   \\
     & = \sum_{n=0}^{\infty} \frac{f^{(n)}(0)}{n!} x^n \\
     & = \sum_{n=0}^{\infty} \frac{f(0)}{n!}x^n        \\
     & = f(0) \sum_{n=0}^{\infty} \frac{x^n}{n!}       \\
     & = f(0) e^x
\end{align*}
We wind up seeing that the radius of convergence is actually $\infty$
since $c_n=\frac{f(0)}{n!}$, so this formula must be true for all $x$.

Thus, $f(x)=Ce^x$, where $C=f(0)$. $\square$

\pagebreak

\section{Problem 6}

First off, notice that
\begin{align*}
    \frac{d}{dx} \frac{e^x}{x^m}
     & = \frac{e^x x^m - mx^{m-1}e^x}{x^{2m}}         \\
     & = \frac{e^x x^{m-1}\left(x - m\right)}{x^{2m}}
\end{align*}
so as long as $x \ge m$, the derivative is nonnegative and the function is increasing.

Consider the sequence $a_n=\frac{e^n}{n^m}$.
The ratio between consecutive elements is
\begin{align*}
    \frac{a_{n+1}}{a_n}
     & = \frac{e^{n+1} / (n+1)^m}{e^n / n^m}  \\
     & = e \cdot \left(\frac{n}{n+1}\right)^m
\end{align*}

Choosing $n$ s.t. $n > m$ and $2^{1/m}n > n+1$
(which is always possible since $2^{1/m} > 1$),
\begin{align*}
    e \cdot \left(\frac{n}{n+1}\right)^m
     & > e \cdot \left(\frac{1}{2^{1/m}}\right)^m \\
     & > e \cdot \frac{1}{2}                      \\
     & > 1
\end{align*}

Since the sequence is nonnegative and past a certain point
has a common ratio greater than $1$ past a certain point, it must be unbounded.
The actual function itself is increasing past this point as well,
so it too must be unbounded. $\square$

\end{document}
