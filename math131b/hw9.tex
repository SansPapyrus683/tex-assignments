\documentclass[12pt]{article}

% a template that a friend gave, it's worked well enough for me
% i have added some packages and stuff that have proved useful

\usepackage{fancyhdr}
\usepackage{tipa}
\usepackage{fontspec}
\usepackage{amsfonts}
\usepackage{enumitem}
\usepackage[margin=1in]{geometry}
\usepackage{graphicx}
\usepackage{float}
\usepackage{amsmath}
\usepackage{braket}
\usepackage{amssymb}
\usepackage{booktabs}
\usepackage{hyperref}
\usepackage{mathtools}
\usepackage{xcolor}
\usepackage{float}
\usepackage{algpseudocodex}
\usepackage{titlesec}
\usepackage{bbm}

\pagestyle{fancy}
\fancyhf{} % sets both header and footer to nothing
\lhead{Kevin Sheng}
\setmainfont{Comic Neue}
\renewcommand{\headrulewidth}{1pt}
\setlength{\headheight}{0.75in}
\setlength{\oddsidemargin}{0in}
\setlength{\evensidemargin}{0in}
\setlength{\voffset}{-.5in}
\setlength{\headsep}{10pt}
\setlength{\textwidth}{6.5in}
\setlength{\headwidth}{6.5in}
\setlength{\textheight}{8in}
\renewcommand{\headrulewidth}{0.5pt}
\renewcommand{\footrulewidth}{0.3pt}
\setlength{\textwidth}{6.5in}
\usepackage{setspace}
\usepackage{multicol}
\usepackage{float}
\setlength{\columnsep}{1cm}
\setlength\parindent{24pt}
\usepackage [english]{babel}
\usepackage [autostyle, english = american]{csquotes}
\MakeOuterQuote{"}

\setlength{\parskip}{6pt}
\setlength{\parindent}{0pt}

\titlespacing\section{0pt}{12pt plus 4pt minus 2pt}{0pt plus 2pt minus 2pt}
\titlespacing\subsection{0pt}{12pt plus 4pt minus 2pt}{0pt plus 2pt minus 2pt}
\titlespacing\subsubsection{0pt}{12pt plus 4pt minus 2pt}{0pt plus 2pt minus 2pt}

\hypersetup{colorlinks=true, urlcolor=blue}

\newcommand{\correction}[1]{\textcolor{red}{#1}}


\rhead{Math 131B}

\makeatletter
\def\@seccntformat#1{%
  \expandafter\ifx\csname c@#1\endcsname\c@section\else
  \csname the#1\endcsname\quad
  \fi}
\makeatother

\DeclareMathOperator{\Fr}{Fr}
\newcommand{\lra}{\xLeftrightarrow}
\newcommand{\ra}{\xRightarrow}
\newcommand{\N}{\mathbb{N}}
\newcommand{\R}{\mathbb{R}}
\newcommand{\Z}{\mathbb{Z}}
\newcommand{\Q}{\mathbb{Q}}
\newcommand{\C}{\mathbb{C}}
\newcommand{\norm}[1]{\left\lVert#1\right\rVert}

\begin{document}

\setcounter{section}{1}

\section{Problem 2}

I'll assume Lemmas 5.2.5 and 5.2.7 without proof.
(5.2.7 is proved in \ref{sec:p4} anyways, so\dots)

The metric's range is a subset of the square root function, which is nonnegative.

\subsection{0 Iff the Same}

Pretty straightforward:
\[\norm{f-g}_2=0 \iff f(x)-g(x)=0\ \forall x \iff f(x)-g(x)\ \forall x\]

\subsection{Triangle Inequality}

Suppose we've three functions in $C(\R/\Z, \C)$ $f$, $g$, and $h$.
Then
\begin{align*}
  \norm{f-g}_2
   & = \norm{f+(-g)}_2                     \\
   & \le \norm{f}_2+\norm{-g}_2            \\
   & = \norm{f}_2 + \norm{g}_2\quad\square
\end{align*}

\section{Problem 3}

Since $|f(x)|^2$ is nonnegative,
\[\int_{0}^{1} |f(x)|^2\,dx \le (1-0)\sup_{x \in [0, 1]} |f(x)|^2 = \sup_{x \in [0, 1]} |f(x)|^2\]
This then implies
\[\norm{f}_2^2 \le \sup_{x \in [0, 1]} |f(x)|^2\]
Taking square roots of both sides preserves the inequality since everything's positive
and gives us $\norm{f}_2$ on the LHS.

On the RHS, notice that
$\sqrt{\sup_{x \in [0, 1]} |f(x)|^2} = \sup_{x \in [0, 1]} |f(x)| = \norm{f}_\infty$
since the square root is monotonic and everything within the set we're taking the
supremum of is positive.

Thus, we get $\norm{f}_2 \le \norm{f}_\infty$. $\square$

\pagebreak

\section{Problem 4}\label{sec:p4}

\subsection{Non-degeneracy}

For the forwards direction, we have
\begin{align*}
             & \norm{f}_2 = 0                                                                   \\
  \implies{} & \int_{0}^{1} |f(x)|^2\,dx = 0 & \text{unpack defn \& square both sides}          \\
  \implies{} & |f(x)|^2 = 0                  & \text{integrand is positive so it has to be $0$} \\
  \implies{} & f(x)=0                        & \text{magnitude is $0$}
\end{align*}
The backwards direction is trivial; $\int_{0}^{1} 0\,dx = 0$. $\square$

\subsection{Cauchy-Schwarz}\label{sec:p4p2}

Funny, in 131BH a homework problem was to prove Holder's, a more general version.
Unfortunately the proof was scattered across like three pages and was only for reals, so\dots

Only consider situations where $\norm{g} > 0$;
if it's $0$ $f+g=f$ and the inequality holds trivially.

Consider the function $h=f\norm{g}_2^2 - \braket{f, g}g$.

By some stuff in Lemma 5.2.5 and the fact that $\norm{g} \in \R$
\begin{align*}
             & \Braket{h, h} \ge 0                                                                \\
  \implies{} & \Braket{f\norm{g}_2^2 - \braket{f, g}g, h} \ge 0                                   \\
  \implies{} & \Braket{f\norm{g}_2^2, h} \ge \Braket{\braket{f, g}g, h}                           \\
  \implies{} & \Braket{f\norm{g}_2^2, f\norm{g}_2^2} - \Braket{f\norm{g}_2^2, \braket{f, g}g}
  \ge \Braket{\braket{f, g}g, f\norm{g}_2^2} - \Braket{\braket{f, g}g, \braket{f, g}g}            \\
  \implies{} & \norm{g}_2^4 \Braket{f, f} - \norm{g}_2^2\overline{\braket{f, g}}\braket{f, g}
  \ge \braket{f, g}\norm{g}_2^2\Braket{g, f} - \braket{f, g}\overline{\braket{f, g}}\Braket{g, g} \\
  \implies{} & \norm{g}_2^4\norm{f}_2^2 \ge |\braket{f, g}|^2\norm{g}_2^2                         \\
  \implies{} & \norm{g}_2\norm{f}_2 \ge |\braket{f, g}|\quad\square
\end{align*}

\pagebreak

\subsection{Triangle Inequality}

Since $\Re(\braket{f, g}) \le |\braket{f, g}|$, we have by what we just showed in \ref{sec:p4p2}
\begin{align*}
             & 2\Re(\braket{f, g}) \le 2\norm{f}_2\norm{g}_2                                          \\
  \implies{} & \braket{f, g} + \overline{\braket{f, g}} \le 2\sqrt{\braket{f, f}}\sqrt{\braket{g, g}} \\
  \implies{} & \braket{f, g} + \braket{g, f} \le 2\sqrt{\braket{f, f}\braket{g, g}}                   \\
  \implies{} & \braket{f, f} + \braket{f, g} + \braket{g, f} + \braket{g, g}
  \le \braket{f, f} + 2\sqrt{\braket{f, f}\braket{g, g}} + \braket{g, g}                              \\
  \implies{} & \braket{f + g, f + g} \le \left(\sqrt{\braket{f, f}} + \sqrt{\braket{g, g}}\right)     \\
  \implies{} & \sqrt{\braket{f + g, f + g}} \le \norm{f}_2 + \norm{g}_2                               \\
  \implies{} & \norm{f+g}_2 \le \norm{f}_2+\norm{g}_2\quad\square
\end{align*}

\subsection{Pythagoras' Theorem}

Unlucky, a lot of this is copy paste from the last part:
\begin{align*}
  \norm{f}_2^2 + \norm{g}_2^2
   & = \braket{f, f} + \braket{g, g}                                 \\
   & = \braket{f, f} + \braket{f, g} + \braket{g, f} + \braket{g, g} \\
   & = \braket{f + g, f + g}                                         \\
   & = \norm{f+g}_2^2\quad\square
\end{align*}

\subsection{Homogeneity}

Finally, an easy one to finish off:
\begin{align*}
  \norm{cf}_2
   & = \sqrt{\braket{cf, cf}}    \\
   & = \sqrt{|c|^2\braket{f, f}} \\
   & = |c|\sqrt{\braket{f, f}}   \\
   & = |c|\norm{f}_2\quad\square
\end{align*}

\pagebreak

\section{Problem 5}\label{sec:p5}

I mean, just evaluate everything directly; if $n=m$,
\begin{align*}
  \Braket{e_n, e_m}
   & = \braket{e_n, e_n}                                \\
   & = \int_{0}^{1} \left|e^{2\pi inx}\right|^2\,dx     \\
   & = \int_{0}^{1} \cos(2\pi nx)^2+\sin(2\pi nx)^2\,dx \\
   & = \int_{0}^{1} 1\,dx                               \\
   & = 1
\end{align*}
while if $n \ne m$, we can let $k=n-m$ where $k \ne 0$ and
\begin{align*}
  \int_{0}^{1} e^{2\pi inx}\overline{e^{2\pi imx}}\,dx
   & = \int_{0}^{1} e^{2\pi inx-2\pi imx}\,dx                                                       \\
   & = \int_{0}^{1} e^{i(2\pi(n-m))}                                                                \\
   & = \int_{0}^{1} \cos(2\pi kx)+i\sin(2\pi kx)\,dx                                                \\
   & = \left.\left(\frac{1}{2\pi kx}\sin(2\pi kx) - \frac{i}{2\pi k}\cos(2\pi kx)\right)\right|^1_0 \\
   & = 0\quad\square
\end{align*}

\pagebreak

\section{Problem 6}

\subsection{Coefficient Extraction Formula}

We have
\begin{align*}
  \Braket{f, e_n}
   & = \Braket{\sum_{n'=-N}^{N} c_n'e_n', e_n} \\
   & = \sum_{n=-N}^{N} \Braket{c_n'e_n', e_n}  \\
   & = \sum_{n=-N}^{N} c_n'\Braket{e_n', e_n}  \\
   & = c_n\Braket{e_n, e_n}                    \\
   & = c_n
\end{align*}

\subsection{Norm Identity}

\subsubsection{Some Random Equality}

I'll first prove
\[\norm{\sum_{n=-N}^{N} c_ne_n}_2^2 = \sum_{n=-N}^{N} \norm{c_ne_n}_2^2\]
by induction; the base case where $N=0$ is trivial.

Then, assuming the validity of the statement for $N$, for $N+1$ we've
\[\norm{\sum_{n=-N-1}^{N+1} c_ne_n}_2^2
  = \norm{c_{-N-1}e_{-N-1}+\sum_{n=-N}^{N+1} c_ne_n}_2^2\]
and by \ref{sec:p5}
\begin{align*}
  \Braket{c_{N-1}e_{N-1}, \sum_{n=-N}^{N+1} c_ne_n}
   & = c_{N-1}\Braket{e_{N-1}, \sum_{n=-N}^{N+1} c_ne_n}            \\
   & = c_{N-1}\sum_{n=-N}^{N+1} \Braket{e_{N-1}, c_ne_n}            \\
   & = c_{N-1}\sum_{n=-N}^{N+1} \overline{c_n}\Braket{e_{N-1}, e_n} \\
   & = 0
\end{align*}
so we can apply Pythagoras' Theorem like so:
\[\norm{c_{-N-1}e_{-N-1}+\sum_{n=-N}^{N+1} c_ne_n}_2^2
  =\norm{c_{-N-1}e_{-N-1}}_2^2+\norm{\sum_{n=-N}^{N+1} c_ne_n}_2^2\]
By identical reasoning we can conclude
\[\Braket{c_{N+1}e_{N+1}, \sum_{n=-N}^{N} c_ne_n}=0\]
so
\begin{align*}
      & \norm{c_{-N-1}c_{-N-1}}_2^2+\norm{\sum_{n=-N}^{N+1} c_ne_n}_2^2                         \\
  ={} & \norm{c_{-N-1}c_{-N-1}}_2^2+\norm{c_{N+1}e_{N+1} + \sum_{n=-N}^{N} c_ne_n}_2^2          \\
  ={} & \norm{c_{-N-1}c_{-N-1}}_2^2+\norm{c_{N+1}e_{N+1}}_2^2+\norm{\sum_{n=-N}^{N} c_ne_n}_2^2 \\
  ={} & \sum_{n=-N-1}^{N+1} \norm{c_ne_n}_2^2
\end{align*}

\subsubsection{Actual Proof}

We can then apply this identity like so:
\begin{align*}
  \norm{f}_2^2
   & = \norm{\sum_{n=-N}^{N} c_ne_n}_2^2     \\
   & = \sum_{n=-N}^{N} \norm{c_ne_n}_2^2     \\
   & = \sum_{n=-N}^{N} |c_n|^2\norm{e_n}_2^2 \\
   & = \sum_{n=-N}^{N} |c_n|^2\quad\square
\end{align*}

\pagebreak

\section{Problem 7}

Wow, a lot of this stuff was shown in ECE 102\dots

\subsection{Closure}

\subsubsection{\texorpdfstring{$\Z$}{Z}-Periodic}

Just plug in the numbers like so:
\begin{align*}
  (f * g)(x+1)
   & = \int_{0}^{1} f(y)g((x+1)-y)\,dx \\
   & = \int_{0}^{1} f(y)g(x-y)\,dx     \\
   & = (f*g)(x)
\end{align*}

\subsubsection{Continuous}

Notice that $g$ is uniformly continuous.
We start from the fact that it's continuous on the compact $[0, 2]$, which makes it UC there.

In other words, for any $\epsilon$
$\exists \delta: x, y \in [0, 2] \land |x-y| < \delta \implies |g(x)-g(y)| < \epsilon$.

To show UC-ness on $\R$, note that if we pick $\delta'=\min(\delta, 2)$, then
\[\forall x, y \in \R: \exists a, b \in [0, 2], z \in \Z: x=a+z \land y=b+z\]
In other words, we can represent any pair of real numbers with
a pair of numbers in $[0, 2]$.

Since $g$ is $\Z$-periodic,
\[|f(x)-f(y)|=|f(a)-f(b)|<\epsilon\]
and is UC on the entirety of $\R$.

(proof continued on next page)

\pagebreak

With that proven, we can move onto the actual proof of continuity.

Fix an $\epsilon$ and choose $\delta$ s.t. $|x-y| < \delta \implies |g(x)-g(y)| < \epsilon$.

Then, for any $|d| < \delta$ we've
\begin{align*}
  |(f*g)(x+d)-(f*g)(x)|
   & = \left|\int_{0}^{1} f(y)(g(x-y)-g(x+d-y))\,dy\right| \\
   & \le \int_{0}^{1} |f(y)||g(x-y)-g(x+d-y)|\,dy          \\
   & \le \int_{0}^{1} \epsilon|f(y)|\,dy                   \\
   & \le \epsilon \int_{0}^{1} |f(y)|\,dy
\end{align*}
Notice that the integral doesn't depend on $\epsilon$ at all and is bounded,
so with sufficiently small $\epsilon$ the error term goes to $0$. $\square$

\subsection{Commutativity}

I'll assume we can do changes of variables.

Let $u=x-y$, so $y=x-u$ and $\frac{dy}{du}=-1$.

Then
\begin{align*}
  (f*g)(x)
   & = \int_{0}^{1} f(y)g(x-y)\,dy    \\
   & = \int_{x}^{x-1} -f(x-u)g(u)\,du \\
   & = \int_{x-1}^{x} f(x-u)g(u)\,du
\end{align*}
If $x \in \Z$, then this integral by periodicity is equal to $(g * f)(x)$.

If it isn't, then $\exists z \in \Z: x-1 < z < x$ and
\begin{align*}
  \int_{x-1}^{x} f(x-u)g(u)\,du
  &= \int_{x-1}^{z} f(x-u)g(u)\,du + \int_{z}^{x} f(x-u)g(u)\,du \\
  &= \int_{x}^{z+1} f(x-u)g(u)\,du + \int_{z}^{x} f(x-u)g(u)\,du \\
  &= \int_{z}^{z+1} f(x-u)g(u)\,du \\
  &= (g * f)(x)\quad\square
\end{align*}

\pagebreak

\subsection{Bilinearity}

For addition, we have
\begin{align*}
  (f * (g + h))(x)
   & = \int_{0}^{1} f(y)(g(x-y)+h(x-y))\,dy                      \\
   & = \int_{0}^{1} f(y)g(x-y)\,dy + \int_{0}^{1} f(y)h(x-y)\,dy \\
   & = (f * g)(x) + (f * h)(x)
\end{align*}
and for the other way
\begin{align*}
  (f + g) * h
   & = h * (f + g)   \\
   & = h * f + h * g \\
   & = f * h + f * g
\end{align*}

For constant multiplication, we've
\begin{align*}
  c(f * g)
   & = c\int_{0}^{1} f(y)g(x-y)\,dy   \\
   & = \int_{0}^{1} (cf(y))g(x-y)\,dy \\
   & = (cf) * g
\end{align*}
and by commutativity
\begin{align*}
  c(f * g)
   & = c(g *f)  \\
   & = (cg) * f \\
   & = f * (cg)
\end{align*}
so yeah, that's every property proven! $\square$

\end{document}
