\documentclass[12pt]{article}

% a template that a friend gave, it's worked well enough for me
% i have added some packages and stuff that have proved useful

\usepackage{fancyhdr}
\usepackage{tipa}
\usepackage{fontspec}
\usepackage{amsfonts}
\usepackage{enumitem}
\usepackage[margin=1in]{geometry}
\usepackage{graphicx}
\usepackage{float}
\usepackage{amsmath}
\usepackage{braket}
\usepackage{amssymb}
\usepackage{booktabs}
\usepackage{hyperref}
\usepackage{mathtools}
\usepackage{xcolor}
\usepackage{float}
\usepackage{algpseudocodex}
\usepackage{titlesec}
\usepackage{bbm}
\usepackage{pythonhighlight}

\pagestyle{fancy}
\fancyhf{} % sets both header and footer to nothing
\lhead{Kevin Sheng}
\setmainfont{Comic Neue}
\renewcommand{\headrulewidth}{1pt}
\setlength{\headheight}{0.75in}
\setlength{\oddsidemargin}{0in}
\setlength{\evensidemargin}{0in}
\setlength{\voffset}{-.5in}
\setlength{\headsep}{10pt}
\setlength{\textwidth}{6.5in}
\setlength{\headwidth}{6.5in}
\setlength{\textheight}{8in}
\renewcommand{\headrulewidth}{0.5pt}
\renewcommand{\footrulewidth}{0.3pt}
\setlength{\textwidth}{6.5in}
\usepackage{setspace}
\usepackage{multicol}
\usepackage{float}
\setlength{\columnsep}{1cm}
\setlength\parindent{24pt}
\usepackage [english]{babel}
\usepackage [autostyle, english = american]{csquotes}
\MakeOuterQuote{"}

\setlength{\parskip}{6pt}
\setlength{\parindent}{0pt}

\titlespacing\section{0pt}{12pt plus 4pt minus 2pt}{0pt plus 2pt minus 2pt}
\titlespacing\subsection{0pt}{12pt plus 4pt minus 2pt}{0pt plus 2pt minus 2pt}
\titlespacing\subsubsection{0pt}{12pt plus 4pt minus 2pt}{0pt plus 2pt minus 2pt}

\hypersetup{colorlinks=true, urlcolor=blue}

\newcommand{\correction}[1]{\textcolor{red}{#1}}


\rhead{Math 131B}

\makeatletter
\def\@seccntformat#1{%
  \expandafter\ifx\csname c@#1\endcsname\c@section\else
  \csname the#1\endcsname\quad
  \fi}
\makeatother

\DeclareMathOperator{\Fr}{Fr}
\newcommand{\lra}{\xLeftrightarrow}
\newcommand{\ra}{\xRightarrow}

\begin{document}

\section{Problem 1}

\subsection{Series Convergence}

Let $F_n(x)=\sum_{i=1}^n f_i(x)$.
Each of the finitely many $f_i$s are differentiable, so $F_n'(x)=\sum_{i=1}^n f_i'(x)$.

We're given that $\sum_{i=n}^\infty \norm{f_n'}$ converges,
so by the M-Test $F_n'$ must converge uniformly to some $D$.

Also by assumptions, $\lim_{n \to \infty} F_n(x_0)$ exists.

Now we see that the \textit{sequence} of functions and their associated derivatives
satisfy the conditions for Theorem 3.7.1.
This means $F_n$ converges to some $F$ and by extension $\sum_{n=1}^\infty f_n$ exists.

\subsection{Limit Swapping}

Again by the theorem we just used,
\begin{align*}
    \frac{d}{dx} \sum_{n=1}^\infty f_n(x)
    &=\frac{d}{dx} \lim_{n \to \infty} F_n(x) \\
    &=\frac{d}{dx} F(x) \\
    &= D(x) \\
    &= \lim_{n \to \infty} \frac{d}{dx} F_n(x) \\
    &= \sum_{n=1}^\infty \frac{d}{dx} f_n(x)\quad\square
\end{align*}

\pagebreak

\section{Problem 2}

% yes, this is basically ripped straight from the bh version. sue me.

Let $f_n(x)=4^{-n} \cos\left(32^n\pi x\right)$ for convenience.

$|\cos x| \le 1$, so $|f_n(x)| \le 4^{-n}$ and the series does indeed converge for all $x$.
Denote the limit of the series of functions as $f$.

To show that it \textit{uniformly} converges, fix an $\epsilon$.
Since $f_n$ is bounded by a geometric series, we have
\begin{align*}
    \left|f(x)-\sum_{i=1}^n f_i(x)\right|
    &= \left|\sum_{i=n+1}^\infty f_i(x)\right| \\
    &\le \sum_{i=n+1}^\infty 4^{-n} \\
    &= \frac{4^{-n-1}}{1-\frac{1}{4}}
\end{align*}

We can certainly choose an $n$ s.t. the bound above is less than $\epsilon$, so there we go. $\square$

\section{Problem 3}

\subsection{Divergence and Abs. Convergence}\label{sec:p3p1}

If $|x-a| > R$, then $|x-a| \cdot \limsup_{n \to \infty} |c_n|^{1/n} > 1$.

Applying the root test to the series requires us to compute
\[\limsup_{n \to \infty} \left|c_n|x-a|^n\right|^{1/n}
=\limsup_{n \to \infty} |c_n|^{1/n}|x-a|\]
Notice that this is precisely the value above that was greater
than $1$ by our assumptions, so the series must diverge. $\square$

The proof goes much the same for absolute convergence,
except we see that the limsup of the roots is less than $1$.

\subsection{Uniform Convergence on Compact Sets}

Consider $\norm{c_n |x-a|^n}_\infty$.
The largest magnitude the function within achieves is if $|x-a|=r$, so the norm must be $|c_n|r^n$.

By the same reasoning as in \ref{sec:p3p1}, $\sum_{n=0}^\infty |c_n|r^n$ converges absolutely,
so by the M-Test the series of functions itself converges to $f$ uniformly. $\square$

\pagebreak

\section{Problem 4}

\subsection{Differentiation}\label{sec:p4p1}

Let $f_n(x)=c_n|x-a|^n$ for convenience.

The series of $f_n$s already uniformly converge, so they certainly pointwise converge as well.
All we need to do is prove the convergence of the derivatives.

By the M-Test, it STP the convergence of
\[\sum_{n=0}^\infty \norm{f_n'}_\infty = \sum_{n=0}^\infty nc_nr^{n-1}\]

Then applying the root test, we see that we then NTS that this limit:
\begin{align*}
    \limsup_{n \to \infty} \left|nc_nr^{n-1}\right|^{1/n}
    &= \limsup_{n \to \infty} r\left|\frac{nc_n}{r}\right|^{1/n} \\
    &= \limsup_{n \to \infty} r |c_n|^{1/n} \cdot \limsup_{n \to \infty} \left(\frac{n}{r}\right)^{1/n}
\end{align*}
is less than $1$.
The first term is already less than $1$, so as long as the second one isn't greater than that we're golden.

\subsubsection{Limit Derivation}

The second limit actually does converge exactly to $1$, so I'll be using that limit from here on out:
\begin{align*}
    \lim_{n \to \infty} \left(\frac{n}{r}\right)^{1/n}
    &= \lim_{n \to \infty} \exp \frac{\ln \left(\frac{n}{r}\right)}{n} \\
    &= \exp \lim_{n \to \infty} \frac{\ln \left(\frac{n}{r}\right)}{n} & \text{$\exp$ is continuous} \\
    &= \exp \lim_{n \to \infty} \frac{\frac{r}{n}}{1} & \text{LHopital's} \\
    &= \exp 0 \\
    &= 1
\end{align*}

With all that, we see that the sequence of derivatives does indeed uniformly converge,
satisfying all the conditions needed to apply Theorem 3.7.1.
Thus, the series $\sum_{n=0}^\infty f_n'(x)$ indeed converges to $f'$. $\square$

\subsection{Integration}

This is just chasing some definitions:
\begin{align*}
    \int_y^z f(x)\,dx
    &= \int_y^z \sum_{n=0}^\infty f_n(x)\,dx \\
    &= \sum_{n=0}^\infty \int_y^z f_n(x)\,dx & \text{by 3.6.2} \\
    &= \sum_{n=0}^\infty \left.\left(\frac{c_n}{n+1}(a-x)^{n+1}\right)\right|^y_z \\
    &= \sum_{n=0}^\infty \frac{c_n}{n+1}\left((a-x)^{n+1}-(a-y)^{n+1}\right)\quad\square
\end{align*}

\pagebreak

\section{Problem 5}\label{sec:p5}

\subsection{Base Case}

$k=0$ is pretty easy, since $f^{(0)}(x)=f(x)$
\[f^{(0)}=\sum_{n=0}^\infty c_{n+0}\frac{(n+0)!}{n!}(x-a)^n=\sum_{n=0}^\infty c_n(x-a)^n\]

\subsection{Inductive Step}

Assume the statement holds for $k$; now we need to show it holds for $k+1$ as well.

We have
\[f^{(k)}(x)=\sum_{n=0}^\infty c_{n+k}\frac{(n+k)!}{n!}(x-a)^n\]
which converges on $(a-r, a+r)$.

By \ref{sec:p4p1}, we can take $c'_n=c_{n+k}\frac{(n+k)!}{n!}$ to see that
\begin{align*}
    f^{(k+1)}(x)
    &=\sum_{n=0}^\infty nc'_n(x-a)^{n-1} \\
    &=\sum_{n=1}^\infty nc_{n+k}\frac{(n+k)!}{n!}(x-a)^{n-1} \\
    &=\sum_{n=0}^\infty (n+1)c_{n+k+1}\frac{(n+k+1)!}{(n+1)!}(x-a)^n
\end{align*}
where we take out the first term in the summation since it's equal to $0$.

The theorem also states that it converges on $(a-r, a+r)$ as well,
so this is indeed the $k+1$th derivative. $\square$

\pagebreak

\section{Problem 6}

Plugging in $a$ to the formula we proved in \ref{sec:p5},
\begin{align*}
    f^{(k)}(a)
    &=\sum_{n=0}^\infty c_{n+k}\frac{(n+k)!}{n!}(a-a)^n \\
    &=c_k \cdot k! \cdot 0^0 \\
    &= c_k \cdot k!
\end{align*}

Which then implies
\begin{align*}
    f(x)
    &=\sum_{n=0}^\infty c_n(x-a)^n \\
    &=\sum_{n=0}^\infty \frac{c_n \cdot n!}{n!}(x-a)^n \\
    &=\sum_{n=0}^\infty \frac{f^{(n)}(a)}{n!}(x-a)^n\quad\square
\end{align*}

\section{Problem 7}

By the binomial formula, we have
\begin{align*}
    (x-a)^n
    &= (x-b+b-a)^n \\
    &= ((x-b)+(b-a))^n \\
    &= \sum_{m=0}^n \frac{n!}{(n-m)!m!} (x-b)^m(b-a)^{n-m}\quad\square
\end{align*}

\end{document}
