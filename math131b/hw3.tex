\documentclass[12pt]{article}

% a template that a friend gave, it's worked well enough for me
% i have added some packages and stuff that have proved useful

\usepackage{fancyhdr}
\usepackage{tipa}
\usepackage{fontspec}
\usepackage{amsfonts}
\usepackage{enumitem}
\usepackage[margin=1in]{geometry}
\usepackage{graphicx}
\usepackage{float}
\usepackage{amsmath}
\usepackage{braket}
\usepackage{amssymb}
\usepackage{booktabs}
\usepackage{hyperref}
\usepackage{mathtools}
\usepackage{xcolor}
\usepackage{float}
\usepackage{algpseudocodex}
\usepackage{titlesec}
\usepackage{bbm}
\usepackage{pythonhighlight}

\pagestyle{fancy}
\fancyhf{} % sets both header and footer to nothing
\lhead{Kevin Sheng}
\setmainfont{Comic Neue}
\renewcommand{\headrulewidth}{1pt}
\setlength{\headheight}{0.75in}
\setlength{\oddsidemargin}{0in}
\setlength{\evensidemargin}{0in}
\setlength{\voffset}{-.5in}
\setlength{\headsep}{10pt}
\setlength{\textwidth}{6.5in}
\setlength{\headwidth}{6.5in}
\setlength{\textheight}{8in}
\renewcommand{\headrulewidth}{0.5pt}
\renewcommand{\footrulewidth}{0.3pt}
\setlength{\textwidth}{6.5in}
\usepackage{setspace}
\usepackage{multicol}
\usepackage{float}
\setlength{\columnsep}{1cm}
\setlength\parindent{24pt}
\usepackage [english]{babel}
\usepackage [autostyle, english = american]{csquotes}
\MakeOuterQuote{"}

\setlength{\parskip}{6pt}
\setlength{\parindent}{0pt}

\titlespacing\section{0pt}{12pt plus 4pt minus 2pt}{0pt plus 2pt minus 2pt}
\titlespacing\subsection{0pt}{12pt plus 4pt minus 2pt}{0pt plus 2pt minus 2pt}
\titlespacing\subsubsection{0pt}{12pt plus 4pt minus 2pt}{0pt plus 2pt minus 2pt}

\hypersetup{colorlinks=true, urlcolor=blue}

\newcommand{\correction}[1]{\textcolor{red}{#1}}


\rhead{Math 131B}

\makeatletter
\def\@seccntformat#1{%
  \expandafter\ifx\csname c@#1\endcsname\c@section\else
  \csname the#1\endcsname\quad
  \fi}
\makeatother

\DeclareMathOperator{\Fr}{Fr}
\newcommand{\lra}{\xLeftrightarrow}
\newcommand{\ra}{\xRightarrow}

\begin{document}

\section{Problem 1}

\subsection{Openness Preservation Implies Continuity}

Fix $x \in X$ and $\epsilon > 0$.

$B_\epsilon(f(x))$ is open, so $I=f^{-1}(B_\epsilon(f(x)))$ is too.

$x$ is actually in that set since $f(x) \in I$, so $\exists \delta: B_\delta(x) \subseteq I$.

For all $x': d_X(x, x') < \delta$, $x' \in B_\delta(x) \subseteq I$
which means $d_Y(f(x'), f(x)) < \epsilon$. $\square$

\subsection{Also Closedness Preservation}

First lemme prove the following lemma for all $S \subseteq Y$:
\[\left(f^{-1}(S^C)\right)^C = f^{-1}(S)\]
As always, it STP subset-ness in both directions.

If $x \notin f^{-1}(S^C)$, $f(x) \notin S^C \implies f(x) \in S \implies x \in f^{-1}(S)$.

OTOH, if $x \in f^{-1}(S)$, $f(x) \in S \implies f(x) \notin S^C \implies x \notin f^{-1}(S^C) \implies x \notin \left(f^{-1}(S^C)\right)^C$.

Now consider any closed set $F \subseteq Y$.
$F^C \subseteq Y$ is open, so $f^{-1}(F^C)$ is open
and $\left(f^{-1}(F^C)\right)^C = f^{-1}(F)$ is closed. $\square$

\subsection{And Vice Versa}

Consider any open set $V \subseteq Y$.
$V^C$ is closed, so $f^{-1}\left(V^C\right)$ is also closed
and $\left(f^{-1}\left(V^C\right)\right)^C = f^{-1}(V)$ is open. $\square$

\subsection{Open Set to Nonopen Set}

Consider $f: \R \to \R$ where $f(x) = |x|$ and take $U = \R$ itself.

Though $U$ is open, $f(U) = [0, \infty)$, which isn't.

\subsection{Closed Set to Nonclosed Set}

Consider $f: \R \to \R$ where $f(x)=e^{-x}$ and take $K = [0, \infty)$.

Though $K$ is closed, $f$ maps it to $(0, 1]$, which isn't.

\pagebreak

\section{Problem 2}

Fix $\epsilon > 0$ and let $\epsilon' = \frac{\epsilon}{\sqrt{2}}$.

Choose $\delta_f$ s.t. $|x-x_0| < \delta_f \implies |f(x)-f(x_0)| < \epsilon'$, and an analagous $\delta_g$.

Then,
\begin{align*}
  d_{l_2}((f(x_0), g(x_0)), (f(x), g(x)))
   & = \sqrt{(f(x_0)-f(x))^2+(g(x_0)-g(x))^2} \\
   & < \sqrt{2 \cdot (\epsilon')^2}           \\
   & = \epsilon\quad\square
\end{align*}

\section{Problem 3}

\subsection{\texorpdfstring{$f$}{f} is Bounded}

It STP $f(X)$ is compact; consider any open cover $\{O_\alpha\}$ of $f(X)$.

Since $f$ is continuous, $f^{-1}(O_\alpha)$ is also open for all $\alpha$.

Notice that $\left\{f^{-1}(O_\alpha)\right\}$ forms an open cover of $X$
since the inverse image of continuous functions map open sets to open sets.
Also, $\forall x \in X\ f(x) \in f(X) \implies \exists \alpha: f(x) \in O_\alpha$.
Thus, it must have a finite subcover $f^{-1}(O_1), \cdots, f^{-1}(O_n)$.

$O_1$ through $O_n$ form a finite subcover of $f(X)$.
This is because for all $y \in f(X)$,
\begin{align*}
             & \exists x \in X: f(x)=y      \\
  \implies{} & \exists i: x \in f^{-1}(O_i) \\
  \implies{} & f(x)=y \in O_i\quad\square
\end{align*}

\subsection{Max and Min}

Our codomain is $\R$, so $f(X)$ certainly has a sup; it remains to show that $\sup f(X) \in f(X)$.

BWOC say $\sup f(X) \notin X$.

We then construct $(a_n)$ where $a_n$ is any $y \in f(X): y > \sup f(X) - \frac{1}{n}$.
Such a $y$ is guaranteed to exist by the definition of the supremum.

Though this sequence converges to $\sup f(X)$, the limit isn't actually in $f(X)$.
This contradicts the fact that $f(X)$ is compact (which we just proved), so
\[\sup f(X) \in f(X) \implies \exists x \in X: f(x) = \sup f(X)\quad\square\]
The proof for the existence of the minimum goes basically the same way as the proof for the maximum.

\pagebreak

\section{Problem 4}

Fix an $\epsilon > 0$. We know the following since $f$ and $g$ are both UC:
\begin{itemize}
  \item $\exists \delta_Y:\ \forall a, b \in Y\ d_Y(a, b) < \delta_Y \implies d_Z(g(a), g(b)) < \epsilon$
  \item $\forall \epsilon_Y > 0\ \exists \delta_X: \forall a, b \in X\ d_X(a, b) < \delta_X \implies d_Y(f(a), f(b)) < \epsilon_Y$
\end{itemize}
With this, take $\delta_Y$ and its corresponding $\delta_X$ if we set $\epsilon_Y=\delta_Y$.

Now, $d_X(a, b) < \delta_X \implies d_Y(f(a), f(b)) < \delta_Y \implies d_Z(g(f(a)), g(f(b))) < \epsilon$. $\square$

\section{Problem 5}

\subsection{Continuous Across One}

Since the function is symmetric w.r.t. $x$ and $y$, I'll just show
that it's continuous if we fix $x$.

So, fixing $x \ne 0$, we have $f(y)=\frac{xy}{x^2+y^2}$ all the way through.
Since $x \ne 0$, we just have to show that $\frac{y}{x^2+y^2}$ is continuous.
Both the numerator and denominator are continuous, and since the denominator
is nonzero, the quotient of the two must be as well.

In the case where $x=0$, $f(y)=0$, which is obviously continuous.

\subsection{Not Jointly Continuous}

Consider the sequence $a_n = \left(\frac{1}{n}, \frac{1}{n}\right)$ which converges to $(0, 0)$.

Though $f((0, 0)) = 0$, $f(a_n)=\frac{1/n^2}{2/n^2}=\frac{1}{2}$ for all $n$,
so $(f(a_n))$ doesn't converge to the same value that $(a_n)$ does.

\end{document}
