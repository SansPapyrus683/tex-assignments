\documentclass[12pt]{article}

% a template that a friend gave, it's worked well enough for me
% i have added some packages and stuff that have proved useful

\usepackage{fancyhdr}
\usepackage{tipa}
\usepackage{fontspec}
\usepackage{amsfonts}
\usepackage{enumitem}
\usepackage[margin=1in]{geometry}
\usepackage{graphicx}
\usepackage{float}
\usepackage{amsmath}
\usepackage{braket}
\usepackage{amssymb}
\usepackage{booktabs}
\usepackage{hyperref}
\usepackage{mathtools}
\usepackage{xcolor}
\usepackage{float}
\usepackage{algpseudocodex}
\usepackage{titlesec}
\usepackage{bbm}
\usepackage{pythonhighlight}

\pagestyle{fancy}
\fancyhf{} % sets both header and footer to nothing
\lhead{Kevin Sheng}
\setmainfont{Comic Neue}
\renewcommand{\headrulewidth}{1pt}
\setlength{\headheight}{0.75in}
\setlength{\oddsidemargin}{0in}
\setlength{\evensidemargin}{0in}
\setlength{\voffset}{-.5in}
\setlength{\headsep}{10pt}
\setlength{\textwidth}{6.5in}
\setlength{\headwidth}{6.5in}
\setlength{\textheight}{8in}
\renewcommand{\headrulewidth}{0.5pt}
\renewcommand{\footrulewidth}{0.3pt}
\setlength{\textwidth}{6.5in}
\usepackage{setspace}
\usepackage{multicol}
\usepackage{float}
\setlength{\columnsep}{1cm}
\setlength\parindent{24pt}
\usepackage [english]{babel}
\usepackage [autostyle, english = american]{csquotes}
\MakeOuterQuote{"}

\setlength{\parskip}{6pt}
\setlength{\parindent}{0pt}

\titlespacing\section{0pt}{12pt plus 4pt minus 2pt}{0pt plus 2pt minus 2pt}
\titlespacing\subsection{0pt}{12pt plus 4pt minus 2pt}{0pt plus 2pt minus 2pt}
\titlespacing\subsubsection{0pt}{12pt plus 4pt minus 2pt}{0pt plus 2pt minus 2pt}

\hypersetup{colorlinks=true, urlcolor=blue}

\newcommand{\correction}[1]{\textcolor{red}{#1}}


\rhead{ECE 236A}

\begin{document}

\begin{enumerate}
      \item \begin{enumerate}
                  \item $P^*=-3$, and the set of optimal solutions is $\{(3, 0, 0)\}$.
                  \item $P^*=0$. This time, the optimal set is infinite:
                  $\{(x_1, 0, x_3)\,|\,1 \le x_1 \le 3, x_3 \in \mathbb{R}^+\}$.
                  \item We can make $x_3$ arbitrarily large in the negative direction, so $P^*=-\infty$
                  and the optimal set is empty.
            \end{enumerate}
      \item Let $x \in \mathbb{R}^n$ be the vector of money the company invests in each person.
            Also, let $m$ be the maximum of the differences between the money a person gets and the money any of their friends get.
            \begin{gather*}
                  \min_{x, m} m\text{ s.t.} \\
                  -m \le a_{ij}(x_i-x_j) \le m, 1 \le i, j \le n \\
                  C_1 \le \mathbf{A}x \le C_2 \\
                  \mathbf{1}^T x = 10^4 \\
                  x \ge \mathbf{0}
            \end{gather*}
      \item We define the overall objective maximum we're trying to minimize as $M$.
            Then, for each of the infinity norms, we define their maximum as $m_{i}$.
            To handle the absolute values in the infinity norm definition, we take
            a new matrix $H_i$ for $1 \le i \le p$.

            Thus, our objective becomes
            \begin{gather*}
                  \min_{m, H_i, X} m\text{ s.t.} \\
                  m \ge \mathbf{1}^T (H_i)_j^T, 1 \le i \le p, 1 \le j \le n \\
                  -H_i \le A_iX-I \le H_i, 1 \le i \le p
            \end{gather*}
            In the third constraint, matrix comparison is done elementwise.
            $(H_i)_j$ is the $j$-th row of $H_i$.
      \item Although it's difficult to define a solution in formal mathematical terms,
            I can give an efficient algorithm for coming up with the solution.
            In both these algorithms, $x$ is initially a vector of all $0$s that we slowly fill up.
            \begin{enumerate}[label=\roman*.]
                  \item Iterate through the indices of $1$ through $n$ sorted by their corresponding value of $c_i$.
                        While $\mathbf{1}^T x < k$, set $x_i=\min(2, k-\mathbf{1}^Tx)$.
                  \item If $d^T \mathbf{1} < \alpha$, then there is no feasible solution.

                        Otherwise, iterate through the indices of $1$ through $n$
                        sorted by their corresponding value of $\frac{c_i}{d_i}$.
                        While $d^T x < \alpha$, set $x_i=\min(1, \alpha - d^T x)$.
            \end{enumerate}
      \item \begin{enumerate}
                  \item $a$ represents the absolute value of $x$.
                        \begin{gather*}
                              \min_{x, a} \mathbf{1}^T a\text{ s.t.} \\
                              -a \le x \le a \\
                              -\mathbf{1} \le Ax-b \le \mathbf{1}
                        \end{gather*}
                  \item $a$ represents the absolute value of $Ax-b$, and $m$ is the infinity norm of $x$.
                        \begin{gather*}
                              \min_{x, a, m} \mathbf{1}^T a+m\text{ s.t.} \\
                              -m \le x \le m \\
                              -a \le Ax-b \le a
                        \end{gather*}
            \end{enumerate}
      \item The only ambiguity expressed in the problem statement is whether
            we want to minimize the \textit{sum} of the two radii or the \textit{maximum}
            of the two radii, as both could be plausible.

            Besides that, the constraints can be made to line up exactly with the problem statement.
            Let $p_1, p_2 \in \mathbb{R}^3$ be the locations of the first and second lamps respectively
            (not to be confused with $p$, the maximum radius a lamp can be).
            \begin{gather*}
                  \min_{r_1, r_2, p_1, p_2, d_1, d_2, d_3, d_4} r_1+r_2\text{ s.t.} \\
                  0 \le r_1, r_2 \le p \\
                  -d_1 \le \mathbf{b}^{(1)}-p_1 \le d_1, r_1 \ge \mathbf{1}^T d_1 \\
                  -d_2 \le \mathbf{b}^{(2)}-p_1 \le d_2, r_1 \ge \mathbf{1}^T d_2 \\
                  -d_3 \le \mathbf{b}^{(2)}-p_2 \le d_3, r_2 \ge \mathbf{1}^T d_3 \\
                  -d_4 \le \mathbf{b}^{(3)}-p_2 \le d_4, r_2 \ge \mathbf{1}^T d_4
            \end{gather*}
            The $d$ variables are the absolute values of the distances from the lamps to the bats.
\end{enumerate}
\end{document}
