\documentclass[12pt]{article}

% a template that a friend gave, it's worked well enough for me
% i have added some packages and stuff that have proved useful

\usepackage{fancyhdr}
\usepackage{tipa}
\usepackage{fontspec}
\usepackage{amsfonts}
\usepackage{enumitem}
\usepackage[margin=1in]{geometry}
\usepackage{graphicx}
\usepackage{float}
\usepackage{amsmath}
\usepackage{braket}
\usepackage{amssymb}
\usepackage{booktabs}
\usepackage{hyperref}
\usepackage{mathtools}
\usepackage{xcolor}
\usepackage{float}
\usepackage{algpseudocodex}
\usepackage{titlesec}
\usepackage{bbm}
\usepackage{pythonhighlight}

\pagestyle{fancy}
\fancyhf{} % sets both header and footer to nothing
\lhead{Kevin Sheng}
\setmainfont{Comic Neue}
\renewcommand{\headrulewidth}{1pt}
\setlength{\headheight}{0.75in}
\setlength{\oddsidemargin}{0in}
\setlength{\evensidemargin}{0in}
\setlength{\voffset}{-.5in}
\setlength{\headsep}{10pt}
\setlength{\textwidth}{6.5in}
\setlength{\headwidth}{6.5in}
\setlength{\textheight}{8in}
\renewcommand{\headrulewidth}{0.5pt}
\renewcommand{\footrulewidth}{0.3pt}
\setlength{\textwidth}{6.5in}
\usepackage{setspace}
\usepackage{multicol}
\usepackage{float}
\setlength{\columnsep}{1cm}
\setlength\parindent{24pt}
\usepackage [english]{babel}
\usepackage [autostyle, english = american]{csquotes}
\MakeOuterQuote{"}

\setlength{\parskip}{6pt}
\setlength{\parindent}{0pt}

\titlespacing\section{0pt}{12pt plus 4pt minus 2pt}{0pt plus 2pt minus 2pt}
\titlespacing\subsection{0pt}{12pt plus 4pt minus 2pt}{0pt plus 2pt minus 2pt}
\titlespacing\subsubsection{0pt}{12pt plus 4pt minus 2pt}{0pt plus 2pt minus 2pt}

\hypersetup{colorlinks=true, urlcolor=blue}

\newcommand{\correction}[1]{\textcolor{red}{#1}}


\rhead{ECE 133A}

\DeclareMathOperator{\std}{std}
\DeclareMathOperator{\avg}{avg}
\DeclareMathOperator{\Tr}{Tr}

\begin{document}

\section{Exercise A1.8}

\subsection{Part A}

Taking derivatives wrt $c_1$ gives
\begin{align*}
    \frac{d}{dc_1} J
     & = \frac{d}{dc_1} \frac{(c_1\mathbf{1}+c_2a-b)^T(c_1\mathbf{1}+c_2a-b)}{n(1+c_2^2)}                                        \\
     & = \frac{1}{n(1+c_2^2)} \frac{d}{dc_1} \left(c_1^2n+2c_1\mathbf{1}^T(c_2a-b)+\norm{c_2a-b}\right) \\
     & = \frac{1}{n(1+c_2^2)} \left(2nc_1+2 \cdot \mathbf{1}^T(c_2a-b)\right)
\end{align*}
Setting this equal to $0$ and solving gives
\[\frac{2c_1}{(1+c_2^2)}=-\frac{2 \cdot \mathbf{1}^T(c_2a-b)}{n(1+c_2^2)}\]
so removing the common $\frac{2}{(1+c_2^2)}$ gets us
\begin{align*}
    c_1
     & =-\frac{\mathbf{1}^T(c_2a-b)}{n}                    \\
     & =\frac{\mathbf{1}^Tb}{n}-c_2\frac{\mathbf{1}^Ta}{n} \\
     & =m_b-m_ac_2\quad\square
\end{align*}

\pagebreak

\subsection{Part B}

Simplifying gives
\begin{align*}
    J
     & = \frac{\norm{c_2(a-m_a\mathbf{1})-(b-m_b\mathbf{1})}^2}{n(1+c_2^2)} \\
     & = \frac{1}{n(1+c_2^2)}\left(c_2^2\norm{a-m_a\mathbf{1}}^2+\norm{b-m_b\mathbf{1}}^2-2c_2(a-m_a\mathbf{1})^T(b-m_b\mathbf{1})\right) \\
     &= \frac{c_2^2\norm{a-m_a\mathbf{1}}^2}{n(1+c_2)}+\frac{\norm{b-m_b\mathbf{1}}^2}{n(1+c_2^2)}-\frac{2c_2(a-m_a\mathbf{1})^T(b-m_b\mathbf{1})}{n(1+c_2^2)} \\
     &= \frac{c_2^2s_a^2}{1+c_2^2}+\frac{s_b^2}{1+c_2}-\frac{(a-m_a\mathbf{1})^T(b-m_b\mathbf{1})}{n}\frac{2c_2}{1+c_2^2} \\
     &= \frac{c_2^2s_a^2}{1+c_2^2}+\frac{s_b^2}{1+c_2}-\rho s_a s_b\frac{2c_2}{1+c_2^2} \\
     &= \frac{c_2^2s_a^2+s_b^2-2\rho s_a s_b c_2}{1+c_2^2}
\end{align*}

Then, differentiating wrt $c_2$ gives
\begin{align*}
    \frac{dJ}{dc_2}
    &= \frac{1}{(1+c_2^2)^2} \cdot \left((2c_2s_a^2-2\rho s_a s_b)(1+c_2^2) - 2c_2(c_2^2s_a^2+s_b^2-2\rho s_a s_b c_2)\right)
\end{align*}
While this looks bad, only the term in parenthesis really matters since we need it to be zero
and the fraction in front is always nonzero.

Setting that equal to $0$ and simplifying gives
\begin{align*}
    & (2c_2s_a^2-2\rho s_a s_b)(1+c_2^2) - 2c_2(c_2^2s_a^2+s_b^2-2\rho s_a s_b c_2) = 0 \\
    \implies{} & \left(2c_2s_a^2+2c_2^3s_a^2-2\rho s_a s_b - 2c_2^2\rho s_a s_b\right) - \left(2c_2^3s_a^2 + 2c_2s_b^2 - 4\rho s_a s_b c_2^2\right) = 0 \\
    \implies{} & 2\rho s_a s_b c_2^2 + 2c_2s_a^2 - 2c_2s_b^2 - 2\rho s_a s_b = 0 \\
    \implies{} & \rho c_2^2 + c_2\left(\frac{s_a}{s_b}-\frac{s_b}{s_a}\right) - \rho = 0
\end{align*}

Assuming $\rho > 0$ first, the parabola is upward-facing, so it's increasing at the second root.
Since $\frac{1}{(1+c_2^2)^2}$ is always positive, the actual derivative is also increasing at this point
and the root with positive discriminant is the minimum.
It's obvious that this root is also positive.

By similar reasoning (reversing all the signs and whatnot), the negative root
is the minimum when $\rho < 0$ as well. $\square$

\pagebreak

\section{Exercise A2.4}

Let $G_A'$ be $G_A$ but with self-loops for all the nodes.

Notice that $I+A$ is the matrix associated with $G_A'$, since
after the addition of $I$ every node is guaranteed to have an edge to itself.

\subsection{A Quick Graph Theory Tangent}

I propose that $G_A'$ is strongly connected iff for any pair of nodes $a$ and $b$
there exists a path of length $n-1$ between them.

For the forward direction, notice that any path between $a$ and $b$ 
can't contain more than $n-1$ edges, because any past that would imply
a cycle in the path that we can remove without consequence.
For paths of length less than $n-1$, we can use the self-loop on
either $a$ or $b$ added previously to add as many edges as necessary
to achieve a length of $n-1$.

The reverse implication is thankfully quite simple, since a path existing between
any two nodes is the very definition of being strongly connected.

\subsection{Main Proof}

After the above, it STP that $((I+A)^x)_{ij}$ is positive iff there's a path of length $x$ from $i$ to $j$.

The proof is by induction, where the base case of $x=1$ is given by construction.

For the inductive step, notice that
\begin{align*}
    (I+A)^{x+1}_{ij}
    &= ((I+A)^x(I+A))_{ij} \\
    &= \sum_{k=1}^{n} (I+A)^x_{ik}(I+A)_{kj}
\end{align*}
If this is positive, then since all elements of $A$ are nonnegative there's at least one $k$ s.t.
$(I+A)^x_{ik}(I+A)_{kj} > 0$, implying the existence of
\begin{enumerate}[nolistsep]
    \item A node $k$ there $i$ can reach in $x$ edges
    \item An edge from $k$ to $j$
\end{enumerate}
and thus a path of length $x+1$ from $i$ to $j$.

OTOH, a path of length $x+1$ from $i$ to $j$ implies both the things in the above list
as well, and thus holds true by our inductive hypothesis.

Given what we just proved, then $G_A'$ is strongly connected iff $A$ is irreducible.
Since adding self-loops never boosts or reduces connectivity, we also have
the same result for $G_A$. $\square$

\section{Exercise T10.11}

\subsection{Part A}\label{sec:matTr}

The $x$th element in the diagonal is
\[(A^TB)_{xx}=\sum_{i=1}^{m} A_{ix}B_{ix}\]
The indices in $A$ are reversed since we're multiplying by $A^T$.

Summing over all $x$ from $1$ to $n$ then gets us
\[\sum_{x=1}^{n} (A^TB)_{xx} = \sum_{x=1}^{n} \sum_{i=1}^{m} A_{ix}B_{ix}\quad\square\]
Comput ing this value takes $\boxed{O(mn)}$ time.

\subsection{Part D}

Applying the formula from \ref{sec:matTr} to $BA^T$ gives us
\begin{align*}
    \Tr(BA^T)
     & = \sum_{i=1}^{n} \sum_{j=1}^{m} (B^T)_{ij}(A^T)_{ij} \\
     & = \sum_{i=1}^{n} \sum_{j=1}^{m} B_{ji}A_{ji}         \\
     & = \sum_{i=1}^{m} \sum_{j=1}^{n} A_{ij}B_{ij}         \\
     & = \Tr(A^TB)\quad\square
\end{align*}
The jump from the second nested summation to the third is simply a reordering
of the indices.
At the end of the day it's still a "dot product" of the matrices.

\pagebreak

\section{Exercise A2.8}

Let $c_i \in \R^n$ be defined by $c_i=Bx_i$.

We can compute each $c_i$ in $O(n^2)$ time, so doing all of them would take $O(n^3)$ time.

Then, if we split $y$ into $y_1, \cdots, y_n$ the same way we did for $x$, then
\[y_x=\sum_{i=1}^{n} A_{xi}Bx_i=\sum_{i=1}^{n} A_{xi}c_i\]
We're adding up $n$ vectors that are just scaled differently,
so this would also take $O(n^2)$ time for one $x$ and $O(n^3)$ overall.

Notice that this step is done independently from the computation of $C_i$,
and so due to how big-O notation works the overall complexity is still $\boxed{O(n^3)}$,
which is way better than the $O(n^4)$ complexity of the naive algorithm.

\end{document}
