\documentclass[12pt]{article}

% a template that a friend gave, it's worked well enough for me
% i have added some packages and stuff that have proved useful

\usepackage{fancyhdr}
\usepackage{tipa}
\usepackage{fontspec}
\usepackage{amsfonts}
\usepackage{enumitem}
\usepackage[margin=1in]{geometry}
\usepackage{graphicx}
\usepackage{float}
\usepackage{amsmath}
\usepackage{braket}
\usepackage{amssymb}
\usepackage{booktabs}
\usepackage{hyperref}
\usepackage{mathtools}
\usepackage{xcolor}
\usepackage{float}
\usepackage{algpseudocodex}
\usepackage{titlesec}
\usepackage{bbm}
\usepackage{pythonhighlight}

\pagestyle{fancy}
\fancyhf{} % sets both header and footer to nothing
\lhead{Kevin Sheng}
\setmainfont{Comic Neue}
\renewcommand{\headrulewidth}{1pt}
\setlength{\headheight}{0.75in}
\setlength{\oddsidemargin}{0in}
\setlength{\evensidemargin}{0in}
\setlength{\voffset}{-.5in}
\setlength{\headsep}{10pt}
\setlength{\textwidth}{6.5in}
\setlength{\headwidth}{6.5in}
\setlength{\textheight}{8in}
\renewcommand{\headrulewidth}{0.5pt}
\renewcommand{\footrulewidth}{0.3pt}
\setlength{\textwidth}{6.5in}
\usepackage{setspace}
\usepackage{multicol}
\usepackage{float}
\setlength{\columnsep}{1cm}
\setlength\parindent{24pt}
\usepackage [english]{babel}
\usepackage [autostyle, english = american]{csquotes}
\MakeOuterQuote{"}

\setlength{\parskip}{6pt}
\setlength{\parindent}{0pt}

\titlespacing\section{0pt}{12pt plus 4pt minus 2pt}{0pt plus 2pt minus 2pt}
\titlespacing\subsection{0pt}{12pt plus 4pt minus 2pt}{0pt plus 2pt minus 2pt}
\titlespacing\subsubsection{0pt}{12pt plus 4pt minus 2pt}{0pt plus 2pt minus 2pt}

\hypersetup{colorlinks=true, urlcolor=blue}

\newcommand{\correction}[1]{\textcolor{red}{#1}}


\rhead{ECE 133A}

\DeclareMathOperator{\std}{std}
\DeclareMathOperator{\avg}{avg}
\DeclareMathOperator{\Tr}{Tr}
\newcommand*{\vertbar}{\rule[-1ex]{0.5pt}{2.5ex}}
\newcommand*{\horzbar}{\rule[.5ex]{2.5ex}{0.5pt}}

\begin{document}

\section{Exercise A4.15}

\subsection{Part A}

The first set of equations with $p$ itself is just:
\[\begin{bmatrix}
        1      & t_1    & t_1^2  & \cdots & t_1^{n-1} \\
        1      & t_2    & t_2^2  & \cdots & t_2^{n-1} \\
        \vdots & \vdots & \vdots &        & \vdots    \\
        1      & t_m    & t_m^2  & \cdots & t_m^{n-1}
    \end{bmatrix}x = y\]

For the second set, notice that $p'$ is also a polynomial:
\[p'(x)=x_2+2x_3t+3x_4t^2+\cdots+(n-1)x_{n}t^{n-2}\]
so we can represent that set as another matrix multiplication:
\[\begin{bmatrix}
        0      & 1      & 2t_1   & \cdots & (n-1)t_1^{n-2} \\
        0      & 1      & 2t_2   & \cdots & (n-1)t_2^{n-2} \\
        \vdots & \vdots & \vdots &        & \vdots         \\
        0      & 1      & 2t_m   & \cdots & (n-1)t_m^{n-2}
    \end{bmatrix}x = s\]

The two matrices above have the same width, so we can just stack
them on top of each other and call it a day:
\[\begin{bmatrix}
        \text{the first one} \\
        \text{the second one}
    \end{bmatrix}x=\begin{bmatrix}
        y \\
        s
    \end{bmatrix}\]

\pagebreak

\subsection{Part B}

It STP that the only solution to the above linear system when $y=s=\mathbf{0}$ is $x=\mathbf{0}$ as well.

If the outputs are all $0$, then $p(t_i)=p'(t_i)=0$ for all $i=1, \cdots, m$.

I propose that this implies $t_i$ is a double root for $p$.

Suppose $t_i$ was just a single root, so we can set $p(x)=(x-t_i)h(x)$ where $h(t_i) \ne 0$.
Then,
\begin{align*}
    p'(t_i)
     & = h(t_i)+(t_i-t_i)h'(t_i) \\
     & = h(t_i)                  \\
     & \ne 0
\end{align*}
which is a contradiction.

All the $t_i$s must occur at least twice in the factored version of $p$.
However, the only way a polynomial of degree $n-1$ can have $n$ roots (including duplicates)
is if it's $0$ \textit{everywhere},

Thus, the only solution to the above system is
$x=\mathbf{0}$ and our above matrix is nonsingular. $\square$.

\subsection{Part C}

Solving the system with \texttt{np.linalg.solve} gives
\[x^T \approx \begin{bmatrix}
        -2        &
        0         &
        134.735   &
        -75.308   &
        -1859.645 &
        4962.037  &
        -4696.586 &
        1537.766
    \end{bmatrix}\]
and the plot of $p(x)$ looks like this:
\begin{center}
    \includegraphics[width=8cm]{img/hw3/poly}
\end{center}
Note the scale on the y-axis.

\pagebreak

\section{Exercise T8.11}

\subsection{General System}

Following the hint, we square the equations:
\[\rho_i^2 = \norm{x-a_i}^2 = x^Tx + a_i^Ta_i - 2x^Ta_i\]
Subtracting the first equation from the others gives
\begin{align*}
    \rho_i^2-\rho_1^2
     & = \left(x^Tx + a_i^Ta_i - 2x^Ta_i\right) - \left(x^Tx + a_1^Ta_1 - 2x^Ta_1\right) \\
     & = \norm{a_i}^2 - \norm{a_1}^2 - 2x^T(a_i-a_1)
\end{align*}
and moving all the constant crap to the LHS gives us
\[(a_1-a_i)x^T = \frac{1}{2}\left(\rho_i^2 - \rho_1^2 - \norm{a_i}^2 + \norm{a_1}^2\right)\]
which we can stack all together into a linear system:
\[\begin{bmatrix}
        \horzbar & a_1 - a_2 & \horzbar \\
        \horzbar & a_1 - a_3 & \horzbar \\
        \horzbar & a_1 - a_4 & \horzbar
    \end{bmatrix}x = \frac{1}{2}\begin{bmatrix}
        \rho_2^2 - \rho_1^2 - \norm{a_2}^2 + \norm{a_1}^2 \\
        \rho_3^2 - \rho_1^2 - \norm{a_3}^2 + \norm{a_1}^2 \\
        \rho_4^2 - \rho_1^2 - \norm{a_4}^2 + \norm{a_1}^2
    \end{bmatrix}\]

\subsection{Specific Instance}

Plugging in the numbers and using \texttt{np.linalg.solve} gives
\[x \approx \begin{bmatrix}
        0.6047 \\
        0.4047 \\
        -0.5035
    \end{bmatrix}\]

\pagebreak

\section{Exercise A4.3}

I mean, no other choice than to multiply it out...
\begin{align*}
        & \left(A+uv^T\right)\left(A^{-1}-\frac{1}{1+v^TA^{-1}u}A^{-1}uv^TA^{-1}\right)                    \\
    ={} & \left(A+uv^T\right)\left(I-\frac{1}{1+v^TA^{-1}u}\left(A^{-1}uv^T\right)\right)A^{-1}            \\
    ={} & \left(A-\frac{1}{1+v^TA^{-1}u}AA^{-1}uv^T+uv^T-\frac{1}{1+v^TA^{-1}u}uv^TA^{-1}uv^T\right)A^{-1} \\
    ={} & I-\frac{1}{1+v^TA^{-1}u}uv^TA^{-1}+uv^TA^{-1}-\frac{1}{1+v^TA^{-1}u}uv^TA^{-1}uv^TA^{-1}         \\
    ={} & I+\left(-\frac{I}{1+v^TA^{-1}u}+I-\frac{1}{1+v^TA^{-1}u}uv^TA^{-1}\right)uv^TA^{-1}              \\
    ={} & I+\left(\frac{v^TA^{-1}u}{1+v^TA^{-1}u}-\frac{1}{1+v^TA^{-1}u}uv^TA^{-1}\right)uv^TA^{-1}        \\
    ={} & I+\frac{1}{1+v^TA^{-1}u}\left(v^TA^{-1}uI-uv^TA^{-1}\right)uv^TA^{-1}
\end{align*}
Now it just STP that $\left(v^TA^{-1}uI-uv^TA^{-1}\right)uv^T=\mathbf{0}$,
since that would make the entire RHS reduce to just $I$.

That expression is true because
\begin{align*}
        & \left(v^TA^{-1}uI-uv^TA^{-1}\right)uv^T       \\
    ={} & v^TA^{-1}u \cdot uv^T-uv^TA^{-1}uv^T          \\
    ={} & v^TA^{-1}u \cdot uv^T-u(v^TA^{-1}u)v^T        \\
    ={} & v^TA^{-1}u \cdot uv^T - v^TA^{-1}u \cdot uv^T \\
    ={} & \mathbf{0}\quad\square
\end{align*}
where the trick is to pinch the scalar out from the middle in the second term.

\pagebreak

\section{Exercise A4.18}

Uh, yeah, just some equation chaining:
\begin{align*}
               & A^T=A^T                             \\
    \implies{} & A^T=I \cdot A^T                     \\
    \implies{} & A^T=(XA)^T A^T                      \\
    \implies{} & A^T = A^TX^TA^T                     \\
    \implies{} & A^T = A^T(AX)^T                     \\
    \implies{} & A^T = A^TAX                         \\
    \implies{} & (A^TA)^{-1}A^T = (A^TA)^{-1}(A^TA)X \\
    \implies{} & A^\dagger = X\quad\square
\end{align*}

\end{document}
