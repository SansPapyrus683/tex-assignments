\documentclass[12pt]{article}

% a template that a friend gave, it's worked well enough for me
% i have added some packages and stuff that have proved useful

\usepackage{fancyhdr}
\usepackage{tipa}
\usepackage{fontspec}
\usepackage{amsfonts}
\usepackage{enumitem}
\usepackage[margin=1in]{geometry}
\usepackage{graphicx}
\usepackage{float}
\usepackage{amsmath}
\usepackage{braket}
\usepackage{amssymb}
\usepackage{booktabs}
\usepackage{hyperref}
\usepackage{mathtools}
\usepackage{xcolor}
\usepackage{float}
\usepackage{algpseudocodex}
\usepackage{titlesec}
\usepackage{bbm}
\usepackage{pythonhighlight}

\pagestyle{fancy}
\fancyhf{} % sets both header and footer to nothing
\lhead{Kevin Sheng}
\setmainfont{Comic Neue}
\renewcommand{\headrulewidth}{1pt}
\setlength{\headheight}{0.75in}
\setlength{\oddsidemargin}{0in}
\setlength{\evensidemargin}{0in}
\setlength{\voffset}{-.5in}
\setlength{\headsep}{10pt}
\setlength{\textwidth}{6.5in}
\setlength{\headwidth}{6.5in}
\setlength{\textheight}{8in}
\renewcommand{\headrulewidth}{0.5pt}
\renewcommand{\footrulewidth}{0.3pt}
\setlength{\textwidth}{6.5in}
\usepackage{setspace}
\usepackage{multicol}
\usepackage{float}
\setlength{\columnsep}{1cm}
\setlength\parindent{24pt}
\usepackage [english]{babel}
\usepackage [autostyle, english = american]{csquotes}
\MakeOuterQuote{"}

\setlength{\parskip}{6pt}
\setlength{\parindent}{0pt}

\titlespacing\section{0pt}{12pt plus 4pt minus 2pt}{0pt plus 2pt minus 2pt}
\titlespacing\subsection{0pt}{12pt plus 4pt minus 2pt}{0pt plus 2pt minus 2pt}
\titlespacing\subsubsection{0pt}{12pt plus 4pt minus 2pt}{0pt plus 2pt minus 2pt}

\hypersetup{colorlinks=true, urlcolor=blue}

\newcommand{\correction}[1]{\textcolor{red}{#1}}


\rhead{ECE 133A}

\newcommand{\norm}[1]{\left\lVert#1\right\rVert}
\DeclareMathOperator{\std}{std}
\DeclareMathOperator{\avg}{avg}
\DeclareMathOperator{\rms}{rms}
\DeclareMathOperator{\Tr}{Tr}
\DeclareMathOperator{\diag}{diag}
\newcommand{\R}{\mathbb{R}}
\newcommand{\wilde}{\widetilde}
\newcommand*{\vertbar}{\rule[-1ex]{0.5pt}{2.5ex}}
\newcommand*{\horzbar}{\rule[.5ex]{2.5ex}{0.5pt}}

\begin{document}

\section{Exercise A10.9}

\subsection{Part A}

It STP that $\exists z \in \R$ s.t.
\[\begin{bmatrix}
        A^TA  & e_i \\
        e_i^T & 0
    \end{bmatrix}\begin{bmatrix}
        x \\ z
    \end{bmatrix}=\begin{bmatrix}
        A^Tb \\ 0
    \end{bmatrix}\]
where $x$ is the expression given in the problem.

This system breaks down into the following linear equations:
\begin{align*}
    A^TAx + ze_i = A^Tb &  & x_i = 0
\end{align*}

Given that $\hat{x}=(A^TA)^{-1}A^Tb$, we have
\begin{align*}
    A^TAx + ze_i
     & = A^TA\left(\hat{x}-\frac{\hat{x}_i}{(A^TA)^{-1}_{ii}}(A^TA)^{-1}e_i\right)+ze_i \\
     & = A^Tb - \frac{\hat{x}_i}{(A^TA)^{-1}_{ii}} e_i + ze_i                           \\
     & = A^Tb - \left(\frac{\hat{x}_i}{(A^TA)^{-1}_{ii}} - z\right)e_i
\end{align*}
so just setting $z=\frac{\hat{x}_i}{(A^TA)^{-1}_{ii}}$ makes the LHS of the first equation $A^Tb$.

I suppose we also have to check that $x_i$ is actually $0$.

Notice that $(A^TA)^{-1}e_i$ is just the $i$th column of the matrix, and
when we divide it by the scalar $(A^TA)^{-1}_{ii}$ we make the $i$th element of that vector $1$.
Then, multiplying it by $\hat{x}_i$ ensures the $i$th element to be $0$.

Subtracting this resulting vector from $\hat{x}$ is guaranteed to make $x_i=0$.

\subsection{Part B}

Assuming we already have the QR factorization of $A$,
we can first compute $A^TA=R^TR$ in $O(mn^2)$ time.

Using the normal equations, we can calculate $\hat{x}=R^{-1}Q^Tb$
using back (or was it forward?) substitution in $O(n^2+nm)$ time.

After that, the only thing left to do is get $(A^TA)^{-1}e_i$
by solving the linear system $R^TRx=e_i$, which can be done in $O(n^2)$ time.
This vector also happens to be the $i$th column of $(A^TA)^{-1}$, so
the $i$th element of this vector is also the number in the denominator.

After that it's just a bunch of elementwise stuff that takes linear time,
so our overall time complexity is \boxed{O(mn^2+n^2)}.

\section{Exercise A10.19}

\subsection{Part A}

It suffices to show that $\hat{x}(\lambda)$ satisfies the normal equation with
\begin{align*}
    A'=\begin{bmatrix}
           \sqrt{\lambda}a_1^T \\
           a_2^T               \\
           \vdots              \\
           a_m^T
       \end{bmatrix} &  &
    b'=\begin{bmatrix}
           \sqrt{\lambda}b_1 \\
           b_2               \\
           \vdots            \\
           b_m
       \end{bmatrix}
\end{align*}
In other words, we need to prove that
\[(A'^TA') \cdot \hat{x}(\lambda) = A'^Tb'\]

First lemme see if $(A'^TA')$ can be simplified in any way:
\begin{align*}
    (A'^TA')_{ij}
     & = \sum_{k=1}^{m} A'^T_{ik}A'_{kj}                    \\
     & = \sum_{k=1}^{m} A'_{ki}A'_{kj}                      \\
     & = \lambda A_{1i}A_{1j} + \sum_{k=2}^{m} A_{ki}A_{kj} \\
     & = (A^TA)_{ij} + (\lambda - 1)A_{1i}A_{1j}
\end{align*}
so extending this to all elements gives
\[A'^TA' = A^TA + (\lambda - 1)a_1 a_1^T\]
and if we let $\xi$ be that horrid fraction given in the problem,
\begin{align*}
        & (A'^TA') \cdot \hat{x}(\lambda)                                                                                                \\
    ={} & \left(A^TA + (\lambda - 1)a_1 a_1^T\right) \cdot \left((A^TA)^{-1}A^Tb + \xi (A^TA)^{-1}a_1\right)                             \\
    ={} & A^TA\left((A^TA)^{-1}A^Tb + \xi (A^TA)^{-1}a_1\right) + (\lambda-1)a_1 a_1^T \left((A^TA)^{-1}A^Tb + \xi (A^TA)^{-1}a_1\right) \\
    ={} & A^Tb + \xi a_1 + (\lambda-1)a_1 a_1^T \left((A^TA)^{-1}A^Tb + \xi (A^TA)^{-1}a_1\right)                                        \\
    ={} & A^Tb + \xi a_1 + (\lambda-1)a_1 a_1^T \left(\hat{x} + \xi (A^TA)^{-1}a_1\right)
\end{align*}

Ok, well, that doesn't seem to nice.
Lemme do some reduction on the RHS, $A'^Tb'$, to see if they have anything in common:
\begin{align*}
    (A'^Tb')_i
     & = \sum_{k=1}^{m} A'^T_{ik}b'_k                 \\
     & = \sum_{k=1}^{m} A'_{ki}b'_k                   \\
     & = \lambda A_{1i}b_1 + \sum_{k=2}^{m} A_{ki}b_k \\
     & = (A^Tb)_i + (\lambda - 1)A_{1i}b_1
\end{align*}
and so $A'^Tb' = A^Tb + (\lambda - 1)b_1a_1$.

Well, at least that remvoes the common term of $A^Tb$.
It remains to show
\[\xi a_1 + (\lambda-1)a_1 a_1^T \left(\hat{x} + \xi (A^TA)^{-1}a_1\right) = (\lambda - 1)b_1a_1\]

For convenience, let $c=a_1^T(A^TA)^{-1}$ be an $1 \times n$ matrix.

Then on the RHS we have
\begin{align*}
        & \xi a_1 + (\lambda-1)a_1 a_1^T \left(\hat{x} + \xi (A^TA)^{-1}a_1\right) \\
    ={} & \xi a_1 + (\lambda - 1)a_1cA^Tb + (\lambda - 1)\xi a_1 c a_1             \\
    ={} & \left(\xi + (\lambda - 1)cA^Tb + (\lambda - 1)\xi ca_1\right)a_1
\end{align*}
so we can reduce that nasty equation above to "just"
\[\xi + (\lambda - 1)cA^Tb + (\lambda - 1)\xi ca_1 = (\lambda - 1)b_1\]
or, after dividing both sides by $\lambda-1$,
\[\frac{\xi}{\lambda - 1} + cA^Tb + \xi ca_1 = b_1\]

Having $c$ as an intermediary also allows us to write
\[\xi = \frac{(\lambda-1)(b_1-cA^Tb)}{1+(\lambda-1)ca_1}\]

But now it's time to bite the bullet and expand everything out:
\begin{align*}
        & \frac{\xi}{\lambda - 1} + cA^Tb + \xi ca_1                                                                                                          \\
    ={} & \frac{b_1-cA^Tb}{1+(\lambda-1)ca_1} + cA^Tb + \frac{(\lambda-1)(b_1-cA^Tb)ca_1}{1+(\lambda-1)ca_1}                                                  \\
    ={} & \frac{b_1-cA^Tb}{1+(\lambda-1)ca_1} + \frac{cA^Tb + (\lambda-1)cA^Tbca_1}{1+(\lambda-1)ca_1} + \frac{(\lambda-1)(b_1-cA^Tb)ca_1}{1+(\lambda-1)ca_1} \\
    ={} & \frac{b_1 - cA^Tb + cA^Tb + (\lambda-1)cA^Tbca_1 + (\lambda-1)(-cA^Tb)ca_1 + (\lambda-1)b_1ca_1}{1+(\lambda-1)ca_1}                                 \\
    ={} & \frac{b_1+(\lambda-1)b_1ca_1}{1+(\lambda-1)ca_1}                                                                                                    \\
    ={} & \frac{b_1(1 + (\lambda - 1)ca_1)}{1+(\lambda-1)ca_1}                                                                                                \\
    ={} & b_1\quad\square
\end{align*}
Funny how things work out.

\subsection{Part B}

Assuming we already have $A=QR$, $A^TA=R^TR$ and we can do the following:
\begin{enumerate}
    \item Compute $(A^TA)^{-1}a_1$ by solving $R^TRy=a_1$ in $O(n^2)$ time using back \& forward substitution.
    \item Given the above, we can get $a_1^T(A^TA)^{-1}a_1$ in $O(n)$ time since it's a vector dot product.
\end{enumerate}

\subsection{Part C}

This is a constrained least squares problem where $A$ is missing its first row and $C$ is just that missing first row.

If we redefine $A'$ and $b'$ to be their originals without the first row, then
by a method similar to what we did in part A it can be shown that
$A'^TA' = A^TA - a_1a_1^T$ and $A'^Tb' = A^Tb - b_1a_1$

So now it suffices to show $\exists z \in \R$ s.t.
\[\begin{bmatrix}
        A^TA - a_1a_1^T & a_1 \\
        a_1^T           & 0
    \end{bmatrix}\begin{bmatrix}
        x' \\ z
    \end{bmatrix}=\begin{bmatrix}
        A^T - b_1a_1 \\
        a_1
    \end{bmatrix}\]
where $x'$ is the solution given in the problem.

Once again, let $c=a_1^T(A^TA)^{-1}$, so the diabolical fraction
\[\frac{b_1-a_1^T\hat{x}}{a_1^T(A^TA)^{-1}a_1}=\frac{b_1-cA^Tb}{ca_1}\]

First let's just check that the second system works out:
\begin{align*}
        & a_1^T\left(\hat{x}+\frac{b_1-cA^Tb}{ca_1}(A^TA)^{-1}a_1\right) \\
    ={} & a_1^T\hat{x} + \frac{b_1-cA^Tb}{ca_1}a_1^T(A^TA)^{-1}a_1       \\
    ={} & cA^Tb + \frac{b_1-cA^Tb}{ca_1}ca_1                             \\
    ={} & b_1
\end{align*}
Alright, all seems good here.

Now in preparation for the first system, let's simplify $(A^TA-a_1a_1^T)x'$:
\begin{align*}
        & (A^TA-a_1a_1^T)x'                                                                            \\
    ={} & (A^TA-a_1a_1^T)\left((A^TA)^{-1}A^Tb + \frac{b_1-cA^Tb}{ca_1}(A^TA)^{-1}a_1\right)           \\
    ={} & A^Tb + \frac{b_1-cA^Tb}{ca_1}a_1 - a_1c\left(A^Tb+\frac{b_1-cA^Tb}{ca_1}a_1\right)           \\
    ={} & A^Tb + \left(\frac{b_1-cA^Tb}{ca_1} - c\left(A^Tb+\frac{b_1-cA^Tb}{ca_1}a_1\right)\right)a_1
\end{align*}
As we can see, it's $A^Tb$ plus some cursed multiple of $a_1$.
No matter what that scalar comes out to be, we can always
tune $z$ s.t. it plus the scalar winds up being $-b_1$.

Thus, the propose solution is indeed optimal. $\square$

\pagebreak

\section{Exercise A11.8}

\subsection{Part C}

At the first step, we've
\begin{align*}
    R_{11}=\sqrt{1}=1 &  & R_{1,2:n}=A_{1,2:n}=\begin{bmatrix}0 & 1\end{bmatrix}
\end{align*}

Now we take the outer product and subtract it from $A$:
\[A_{2:n,2:n}-R^T_{1,2:n}R_{1,2:n}=\begin{bmatrix}
        1 & 1 \\
        1 & a
    \end{bmatrix}-\begin{bmatrix}
        0 & 0 \\
        0 & 1
    \end{bmatrix}=\begin{bmatrix}
        1 & 1   \\
        1 & a-1
    \end{bmatrix}\]

At the second step, we have
\begin{align*}
    R_{22}=1 &  & R_{23}=1
\end{align*}

And subtracting this outer product (which is just a scalar at this point) forces
\[R_{33}=\sqrt{a-2}\]

Both both these square roots to be real, we need $a > 2$.

If it is, our Cholseky factor is
\[R=\begin{bmatrix}
        1 & 0 & 1          \\
        0 & 1 & 1          \\
        0 & 0 & \sqrt{a-2}
    \end{bmatrix}\]

\pagebreak

\subsection{Part D}

This goes much the same way.

At the first step, we've
\begin{align*}
    R_{11}=\sqrt{1}=1 &  & R_{1,2:n}=A_{1,2:n}=\begin{bmatrix}0 & a\end{bmatrix}
\end{align*}

Now we take the outer product and subtract it from $A$:
\[A_{2:n,2:n}-R^T_{1,2:n}R_{1,2:n}=\begin{bmatrix}
        1 & 0 \\
        0 & 1
    \end{bmatrix}-\begin{bmatrix}
        0 & 0   \\
        0 & a^2
    \end{bmatrix}=\begin{bmatrix}
        1 & 0     \\
        0 & 1-a^2
    \end{bmatrix}\]

At the second step, we have
\begin{align*}
    R_{22}=1 &  & R_{23}=0
\end{align*}

Since $R_{23}=0$, there's nothing to subtract from the final element and we have $R_{33}=\sqrt{1-a^2}$.

This implies the condition that $a < 1$, and our Cholseky factor is thus
\[R=\begin{bmatrix}
        1 & 0 & a            \\
        0 & 1 & 0            \\
        0 & 0 & \sqrt{1-a^2}
    \end{bmatrix}\]

\pagebreak

\section{Exercise A11.30}

\subsection{Part A}

For any $x \ne \mathbf{0}$, if we let $v=C^Tx$, then
\begin{align*}
    x^TAx
     & = x^TCD^{-1}C^Tx                              \\
     & = v^TD^{-1}v                                  \\
     & = \sum_{i=1}^{m} v_i^2 \cdot \frac{1}{D_{ii}} \\
     & > 0
\end{align*}
The last inequality is due to a couple things:
\begin{itemize}[nolistsep]
    \item $C^T$ has linearly independent columns, so $x \ne \mathbf{0} \implies C^Tx \ne \mathbf{0}$.
    \item $D_{ii}$ has positive elements only, so all elements in the summation are nonnegative.
\end{itemize}

\subsection{Part B}

We need to solve these two equations:
\begin{align*}
    Dx+C^Ty=\mathbf{0} &  & Cx=b
\end{align*}
With some manipulation on the first equation we have
\begin{align*}
               & Dx+C^Ty=\mathbf{0}        \\
    \implies{} & x+D^{-1}C^Ty=\mathbf{0}   \\
    \implies{} & Cx+CD^{-1}C^Ty=\mathbf{0} \\
    \implies{} & CD^{-1}C^Ty=-b            \\
    \implies{} & Ay=-b
\end{align*}
which can be easily solved given the Cholseky factorization of $A$ in $O(n^2)$ time.

To do that, we have to first calculate $A$ ($O(nm^2)$) and then actually factor it ($O(n^3)$).

$x$ is trivially solved, since $Dx=-C^Ty$ and $D$ is a diagonal matrix.
It takes $O(mn)$ flops to calculate $-C^Ty$, and $m$ flops to divide each element by $D_{ii}$.

Putting all this together gives us a final TC of \boxed{O(nm^2+n^3)}
(I've omitted the other terms since these cubic terms dominate).

\pagebreak

\subsection{Part C}

This time, we have the system
\begin{gather*}
    Dx+C^Ty=\mathbf{0} \\
    dz+f^Ty=0 \\
    Cx+zf=b
\end{gather*}
where $d, z \in \R$ and $f \in R^n$.

In a similar manner as the previous part, we can manipulate the first equation to get
\[b-zf+Ay=\mathbf{0} \implies y=A^{-1}(zf-b)\]
and we can isolate $z$ in the second equation to get
\begin{align*}
    z
     & =-\frac{f^Ty}{d}                    \\
     & = -\frac{f^TA^{-1}(zf-b)}{d}        \\
     & = -\frac{zf^TA^{-1}f-f^TA^{-1}b}{d}
\end{align*}

All we need to get $z$ is $f^TA^{-1}f$ and $f^TA^{-1}b$.
Thankfully, we don't have to calculate any inverse, since
\begin{align*}
    f^TA^{-1}b
     & = f^TR^{-1}R^{-T}b          \\
     & = (R^{-T}f)^T \cdot R^{-T}b
\end{align*}
and we can get both terms in $O(n^2)$ time since $R$ is upper triangular.
The process for $f^TA^{-1}f$ goes in much the same manner.

With $z$ solved, $Ay=zf-b$ can be solved in $O(n^2)$ time and $x=-D^{-1}C^Ty$
can be calculated in the same way as it was previously in $O(mn)$ time.

Putting all this together gives us a time complexity of \boxed{O(n^2+mn)},
assuming we have all the matrices from the previous part ready.

\end{document}
