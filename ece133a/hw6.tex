\documentclass[12pt]{article}

% a template that a friend gave, it's worked well enough for me
% i have added some packages and stuff that have proved useful

\usepackage{fancyhdr}
\usepackage{tipa}
\usepackage{fontspec}
\usepackage{amsfonts}
\usepackage{enumitem}
\usepackage[margin=1in]{geometry}
\usepackage{graphicx}
\usepackage{float}
\usepackage{amsmath}
\usepackage{braket}
\usepackage{amssymb}
\usepackage{booktabs}
\usepackage{hyperref}
\usepackage{mathtools}
\usepackage{xcolor}
\usepackage{float}
\usepackage{algpseudocodex}
\usepackage{titlesec}
\usepackage{bbm}
\usepackage{pythonhighlight}

\pagestyle{fancy}
\fancyhf{} % sets both header and footer to nothing
\lhead{Kevin Sheng}
\setmainfont{Comic Neue}
\renewcommand{\headrulewidth}{1pt}
\setlength{\headheight}{0.75in}
\setlength{\oddsidemargin}{0in}
\setlength{\evensidemargin}{0in}
\setlength{\voffset}{-.5in}
\setlength{\headsep}{10pt}
\setlength{\textwidth}{6.5in}
\setlength{\headwidth}{6.5in}
\setlength{\textheight}{8in}
\renewcommand{\headrulewidth}{0.5pt}
\renewcommand{\footrulewidth}{0.3pt}
\setlength{\textwidth}{6.5in}
\usepackage{setspace}
\usepackage{multicol}
\usepackage{float}
\setlength{\columnsep}{1cm}
\setlength\parindent{24pt}
\usepackage [english]{babel}
\usepackage [autostyle, english = american]{csquotes}
\MakeOuterQuote{"}

\setlength{\parskip}{6pt}
\setlength{\parindent}{0pt}

\titlespacing\section{0pt}{12pt plus 4pt minus 2pt}{0pt plus 2pt minus 2pt}
\titlespacing\subsection{0pt}{12pt plus 4pt minus 2pt}{0pt plus 2pt minus 2pt}
\titlespacing\subsubsection{0pt}{12pt plus 4pt minus 2pt}{0pt plus 2pt minus 2pt}

\hypersetup{colorlinks=true, urlcolor=blue}

\newcommand{\correction}[1]{\textcolor{red}{#1}}


\rhead{ECE 133A}

\newcommand{\norm}[1]{\left\lVert#1\right\rVert}
\DeclareMathOperator{\std}{std}
\DeclareMathOperator{\avg}{avg}
\DeclareMathOperator{\rms}{rms}
\DeclareMathOperator{\Tr}{Tr}
\DeclareMathOperator{\diag}{diag}
\newcommand{\R}{\mathbb{R}}
\newcommand{\wilde}{\widetilde}
\newcommand*{\vertbar}{\rule[-1ex]{0.5pt}{2.5ex}}
\newcommand*{\horzbar}{\rule[.5ex]{2.5ex}{0.5pt}}

\begin{document}

\section{Exercise A9.4}

\subsection{Part A}

\subsubsection{Manipulating the LHS}

Let's first figure out what $T(X)^TT(X)$ is:
\begin{align*}
    T(X)^TT(X)
    &= T(X)^HT(X) \\
    &= \left(\frac{1}{n^2}\wilde{W}^H\diag(\wilde{W}x)\wilde{W}\right)^H\left(\frac{1}{n^2}\wilde{W}^H\diag(\wilde{W}x)\wilde{W}\right) \\
    &= \frac{1}{n^4} \wilde{W}^H\diag(\wilde{W}x)^H\wilde{W} \cdot \wilde{W}^H\diag(\wilde{W}x)\wilde{W} \\
    &= \frac{1}{n^2} \wilde{W}^H\diag(\wilde{W}x)^H\diag(\wilde{W}x)\wilde{W} \\
    &= \frac{1}{n^2} \wilde{W}^H\diag\left(\wilde{W}x \odot (\wilde{W}x)^H\right)\wilde{W}
\end{align*}
where $x$ is the corresponding $n^2$-vector of $X$.

We can calculate the middle matrix in $O(n^2 \log n)$ time by doing the following:
\begin{enumerate}
    \item Get $\wilde{W}x$ by applying the 2D FFT in $O(n^2 \log n)$.
    \item Do an elementwise dot product with hitself and get $\diag(\cdots)$ in $O(n)$.
\end{enumerate}

Then, adding all three transpose products on the LHS gives us:
\begin{align*}
     & (A^TA + \lambda D_V^T D_V + \lambda D_H^T D_H)x \\
    ={} & \frac{1}{n^2} \wilde{W}^H\diag\left(\wilde{W}b \odot \overline{\wilde{W}b} + \lambda\wilde{W}e \odot \overline{\wilde{W}e} + \lambda\wilde{W}t \odot \overline{\wilde{W}t}\right)\wilde{W}x
\end{align*}
where $e$ and $t$ are the vectorizations of $E$ and $E^T$ respectively.,

From now on, I'll use $d$ to represent the vector inside the $\diag$ function from now on.

\subsubsection{Manipulating the RHS}

The RHS "simplifies" like so:
\begin{align*}
    A^Ty
    &= T(B)^Hy \\
    &= \left(\frac{1}{n^2}\wilde{W}^H\diag(\wilde{W}b)\wilde{W}\right)^Hy \\
    &= \frac{1}{n^2} \wilde{W}^H\diag(\wilde{W}b)^H\wilde{W}y
\end{align*}

\subsubsection{Putting Them Together}

We now have this equation to solve:
\[\frac{1}{n^2}\wilde{W}^H\diag(d)\wilde{W}
=\frac{1}{n^2} \wilde{W}^H\diag(\wilde{W}b)^H\wilde{W}y\]

This can be simplified like so:
\begin{align*}
    & \frac{1}{n^2}\wilde{W}^H\diag(d)\wilde{W}x
=\frac{1}{n^2} \wilde{W}^H\diag(\wilde{W}b)^H\wilde{W}y \\
\implies{} & \diag(d)\wilde{W}x=\diag(\wilde{W}b)^H\wilde{W}y \\
\implies{} & \wilde{W}x=\diag(1 \oslash d)\diag(\wilde{W}b)^H\wilde{W}y \\
\implies{} & x=\frac{1}{n^2}W^H \diag(1 \oslash d)\diag(\wilde{W}b)^H\wilde{W}y
\end{align*}
where $\oslash$ represent elementwise division.
I'm pretty sure it's not too bad to assume that all elements of $d$ are nonzero.

We can calculate this in the following manner:
\begin{enumerate}
    \item Get $\wilde{W}y$ using 2D FFT in $O(n^2 \log n)$ time.
    \item $\diag(\wilde{W}b)$ (and by extension its transpose) was already obtained when doing the LHS.
    \item Merge the above two elements in $O(n^2)$ time since it's just an elementwise product.
    \item Multiply the above result with $\diag(1 \oslash d)$ in $O(n^2)$ time by similar reasoning. 
    \item Finally put $\frac{1}{n^2}\wilde{W}^H$ on the right with a 2D IFFT.
\end{enumerate}

Combining these two steps finally gets us a linear system of $n^2$ equations and $n^2$ variables,
which we can solve like any other lienar system.


This algorithm is implemented in the attached Jupyter notebook.

\subsection{Part B}

Also seee the notebook for the deblurring implementation.

\pagebreak

\section{Exercise A10.2}

It's easier to make the first vehicle $1$ unit ahead of the second vehicle
instead of having them start $1$ unit apart.

We just have to find the least-norm solution that meets the following constraints:
\begin{enumerate}
    \item $s_1(20)-p_1(20)=1$
    \item $s_2(20)=0$
    \item $p_2(20)=0$
\end{enumerate}

Let's do some bashing to see what the equations actually are:
\begin{align*}
     & s_1(1)=0                                    &  & s_2(1)=0.1u(0)                                              \\
     & s_1(2)=0.1u(0)                              &  & s_2(2)=0.95 \cdot 0.1u(0) + 0.1u(1)                         \\
     & s_1(3)=u(0)(0.1 + 0.95 \cdot 0.1) + 0.1u(1) &  & s_2(3)=0.95^2 \cdot 0.1u(0) + 0.95 \cdot 0.1 u(1) + 0.1u(2)
\end{align*}

We can do some pattern recognition to see that
\begin{gather*}
    s_1(n)=\sum_{i=0}^{n-2} \left(u(i)\sum_{j=0}^{n-i-2} 0.1 \cdot 0.95^j\right) \\
    s_2(n)=\sum_{i=0}^{n-1} 0.95^{n-1-i} \cdot 0.1u(i)
\end{gather*}

A similar expression is the formula for $p_1(n)$ and $p_2(n)$ only
with $0.8$ and $0.2$ instead of $0.95$ and $0.1$ respectively.

Regardless, all of these expressions are linear in terms of $u(t)$ and $v(t)$,
so this can indeed be formulated as a least norm problem.

See the attached Jupyter notebook for my solution.

\end{document}
