\documentclass[12pt]{article}

% a template that a friend gave, it's worked well enough for me
% i have added some packages and stuff that have proved useful

\usepackage{fancyhdr}
\usepackage{tipa}
\usepackage{fontspec}
\usepackage{amsfonts}
\usepackage{enumitem}
\usepackage[margin=1in]{geometry}
\usepackage{graphicx}
\usepackage{float}
\usepackage{amsmath}
\usepackage{braket}
\usepackage{amssymb}
\usepackage{booktabs}
\usepackage{hyperref}
\usepackage{mathtools}
\usepackage{xcolor}
\usepackage{float}
\usepackage{algpseudocodex}
\usepackage{titlesec}
\usepackage{bbm}
\usepackage{pythonhighlight}

\pagestyle{fancy}
\fancyhf{} % sets both header and footer to nothing
\lhead{Kevin Sheng}
\setmainfont{Comic Neue}
\renewcommand{\headrulewidth}{1pt}
\setlength{\headheight}{0.75in}
\setlength{\oddsidemargin}{0in}
\setlength{\evensidemargin}{0in}
\setlength{\voffset}{-.5in}
\setlength{\headsep}{10pt}
\setlength{\textwidth}{6.5in}
\setlength{\headwidth}{6.5in}
\setlength{\textheight}{8in}
\renewcommand{\headrulewidth}{0.5pt}
\renewcommand{\footrulewidth}{0.3pt}
\setlength{\textwidth}{6.5in}
\usepackage{setspace}
\usepackage{multicol}
\usepackage{float}
\setlength{\columnsep}{1cm}
\setlength\parindent{24pt}
\usepackage [english]{babel}
\usepackage [autostyle, english = american]{csquotes}
\MakeOuterQuote{"}

\setlength{\parskip}{6pt}
\setlength{\parindent}{0pt}

\titlespacing\section{0pt}{12pt plus 4pt minus 2pt}{0pt plus 2pt minus 2pt}
\titlespacing\subsection{0pt}{12pt plus 4pt minus 2pt}{0pt plus 2pt minus 2pt}
\titlespacing\subsubsection{0pt}{12pt plus 4pt minus 2pt}{0pt plus 2pt minus 2pt}

\hypersetup{colorlinks=true, urlcolor=blue}

\newcommand{\correction}[1]{\textcolor{red}{#1}}


\rhead{ECE 133A}

\DeclareMathOperator{\std}{std}
\DeclareMathOperator{\avg}{avg}
\DeclareMathOperator{\rms}{rms}
\DeclareMathOperator{\Tr}{Tr}
\DeclareMathOperator{\diag}{diag}
\newcommand*{\vertbar}{\rule[-1ex]{0.5pt}{2.5ex}}
\newcommand*{\horzbar}{\rule[.5ex]{2.5ex}{0.5pt}}

\begin{document}

\section{Exercise A5.11}

\subsection{Part A}

The first $n-1$ rows just specify $A_{ii}x_i = b_i$, which is doable with just one computation.

The only row that requires any real work is the last one, and since
we already now the first $n-1$ elements of $b$, the last element is already
pinned down and specified; we just have to calculate it:
\[b_n = \frac{1}{A_{nn}}\sum_{i=1}^{n-1} A_{ni}b_i\]
This computation that takes $O(n)$ time combined the the other variables
that all together take $O(n)$ time as well makes the overall complexity still just \boxed{O(n)}.

\subsection{Part B}

Computing the inverse is equivalent to solving the equation for $e_1$ through $e_n$.

The thing is, for all of these vectors, we know there can be at most \textit{one} nonzero
element out of $b_1, \cdots, b_{n-1}$ by how the matrix is constructed.
If we're solving for $e_1$, then it has to be $b_1$, and solving for $e_2$
requires $b_2$ to be the nonzero one, so on and so forth.

Thus, we can actually solve each equation here in a number of flops
constant with respect to $n$, making the overall complexity of inverting $A$ \boxed{O(n)}.

\pagebreak

\section{Exercise A5.6}

\subsection{Part A}

Notice that
\begin{gather*}
    (DX)_{ij} = \sum_{k=1}^{n} D_{ik}X_{kj} = D_{ii}X_{ij} \\
    (XD)_{ij} = \sum_{k=1}^{n} X_{ik}D_{kj} = X_{ij}D_{jj}
\end{gather*}
so really we just have to solve $n^2$ equations of the form
\[X_{ij}D_{ii}+X_{ij}D_{jj}=B_{ij}\]
which will always have a solution since the problem guarantees $D_{ii}+D_{jj} \ne 0$.

Each of these equations takes constant time to solve, so
the overall complexity of our algorithm is just \boxed{O(n^2)}.

\pagebreak

\subsection{Part B}

If we focus our attention on the first column, we see that
\begin{align*}
    B_{i1}
     & = \sum_{k=1}^{n} L_{ik}X_{k1}+\sum_{k=1}^{n} (L^T)_{k1}X_{ik} \\
     & = \sum_{k=1}^{n} L_{ik}X_{k1}+\sum_{k=1}^{1} (L^T)_{k1}X_{ik} \\
     & = \sum_{k=1}^{n} L_{ik}X_{k1} + L_{11}X_{i1}
\end{align*}
which then implies that
\[b_1 = L x_1 + L_{11}x_1\]
where $b_1$ and $x_1$ are the first columns for $B$ and $X$ respectively.

Notice that since $L$ is triangular, this linear system for the first column of $X$
and $B$ can be solved in $O(n^2)$ time.

For the second column, we have something similar:
\begin{align*}
               & B_{i2} = \sum_{k=1}^{n} L_{ik}X_{k2} + \sum_{k=1}^{2} (L^T)_{k2}X_{ik} \\
    \implies{} & b_2 = Lx_2 + L_{21}x_1 + L_{22}x_2                                     \\
    \implies{} & b_2 - L_{21}x_1 = Lx_2 + L_{22}x_2
\end{align*}
From the previous step we already know what $x_1$ is, so
we can move it to the LHS and solve for the RHS just like we did before
since $L$ is still lower triangular.

We do this all the way until $x_n$.
At each step we do have to compute the LHS, which in the worst case takes $O(n^2)$ for each step.
However, solving the triangular matrix equation also takes $O(n^2)$,
so it doesn't really impact the overall complexity.

In the end, we do an $O(n^2)$ computation $n$ times, once for each column of $X$,
resulting in a final complexity of $\boxed{O(n^3)}$.

\pagebreak

\section{Exercise A6.3}

\subsection{Part A}

Since $S^T=-S$, $S+S^T=0$, and $x^T(S+S^T)x = 0$.
Doing some equation manipulation gives
\begin{align*}
    0
     & =x^T(S+S^T)x          \\
     & = x^TSx + x^TS^Tx     \\
     & = x^TSx + (Sx)^Tx     \\
     & = x^TSx + (x^T(Sx))^T \\
     & = x^TSx + x^TSx
\end{align*}
where the last step is true due to the output being a scalar.

It should be pretty clear that $2x^TSx = 0 \implies x^TSx = 0$.

Now as for the actual proof,
\begin{align*}
               & (I-S)x = \mathbf{0}        \\
    \implies{} & x^T(I-S)x = \mathbf{0}     \\
    \implies{} & x^TIx - x^TSx = \mathbf{0} \\
    \implies{} & x^Tx = \mathbf{0}          \\
    \implies{} & x = \mathbf{0}\quad\square
\end{align*}

\subsection{Part B}

This is just some really scuffed algebra:
\begin{align*}
               & S-S^2 = (I-S)S                                      \\
    \implies{} & (I-S)^{-1}S - (I-S)^{-1}S^2 = S                     \\
    \implies{} & (I-S)^{-1}S(I-S) = S                                \\
    \implies{} & (I-S)^{-1}S = S(I-S)^{-1}                           \\
    \implies{} & (I-S)^{-1} + (I-S)^{-1}S = (I-S)^{-1} + S(I-S)^{-1} \\
    \implies{} & (I-S)^{-1}(I+S) = (I+S)(I-S)^{-1}\quad\square
\end{align*}

\pagebreak

\subsection{Part C}

For convenience, sometimes I'll let $X=(I-S)^{-1}$.

It STP that $AA^T=I$:
\begin{align*}
    AA^T
     & = (I+S)X((I+S)X)^T                 \\
     & = X(I+S)(X(I+S))^T                 \\
     & = X(I+S)(I+S)^TX^T                 \\
     & = X(I+S)(I-S)X^T                   \\
     & = X(I-S)(I+S)X^T                   \\
     & = (I+S)X^T                         \\
     & = (I-S)^T\left((I-S)^T\right)^{-1} \\
     & = I\quad\square
\end{align*}

\pagebreak

\section{Exercise A6.9}

\subsection{Part A}

On the LHS, notice that putting $S$ on the right just rotates the rows to the left:
\begin{align*}
    (WS)_{ab}
     & = \sum_{c=1}^{k} W_{ac}S_{cb} \\
     & = \begin{cases}
             W_{a(b+1)} & b < n \\
             W_{a1}     & b = n
         \end{cases}
\end{align*}
We see that each cell takes on the value of its right neighbor, with the
cells at the end wrapping around to the beginning.
With this being the case, doing the multiplication $k-1$ times will rotate
each row in $W$ $k-1$ times as well.

As fot the RHS, notice that multiplying a diagonal by a matrix
does an elementwise multiplication of each column with the elements along the diagonal, so
\begin{align*}
    (\diag(We_k)W)_{ab}
     & = \omega^{-(a-1)(b-1)}\omega^{-(a-1)(k-1)} \\
     & = \omega^{-(a-1)(b+k-2)}
\end{align*}
Notice that when $k=2$, $(\diag(We_k)W)_{ab}=W_{a(b+1)}$, with the edge case
of the last column turning into all $1$s since the exponent would become a multiple of $n$.
For arbitrary $k$, the idea is that $(\diag(We_k)W)_{ab}=W_{a(b+k-1)}$ where $b+k-1$
is calculated mod $n$.

Regardless, these two operations are doing the same thing, which is circularly
rotating each row to the left by $1$ for each increase in $k$. $\square$

\pagebreak

\subsection{Part B}

If we expand out $S$ using the expression given, then
\[T(a)=\begin{bmatrix}
        \frac{1}{n}W^H\diag(We_1)Wa & \cdots & \frac{1}{n}W^H\diag(We_n)Wa
    \end{bmatrix}\]
where we can then strip out the common prefix to get
\[T(a)=\frac{1}{n}W^H\begin{bmatrix}
        \diag(We_1)Wa & \cdots & \diag(We_n)Wa
    \end{bmatrix}\]
It STP that the remaining matrix is equivalent to $\diag(Wa)W$.

Fix a column $c$.
On the $T(a)$ side, we have
\begin{align*}
    (\diag(We_c)Wa)_i
     & = \sum_{k=1}^{n} (\diag(We_c)W)_{ik} a_k   \\
     & = \sum_{k=1}^{n} \omega^{-(i-1)(k+c-2)}a_k
\end{align*}
and on the other (where $w_c$ represents the $c$th column of $w$)
\begin{align*}
    (\diag(Wa)w_c)_i
     & = \diag(Wa)_{ii} W_{ic}                                           \\
     & = \sum_{k=1}^{n} W_{ik}c_k \cdot W_{ic}                           \\
     & = \sum_{k=1}^{n} \omega^{(i-1)(k-1)}c_k \cdot \omega^{(i-1)(c-1)} \\
     & = \sum_{k=1}^{n} \omega^{(i-1)(c+k-2)}c_k                         \\
     & = (\diag(We_c)Wa)_i\quad\square
\end{align*}

There's almost certainly a better way to do this, but hey, if it works, it works!

\pagebreak

\subsection{Part C}

Expanding $T(a)x$ gives
\[T(a)x = \frac{1}{n}W^H\diag(Wa)Wx\]
We can first compute $Wx$ and $Wa$ using the FFT.

All $\diag(Wa)Wx$ does is an elementwise product, so that can be computed in $O(n)$ time.

After that, we have $\frac{1}{n}W^H \cdot \text{whatever vector that gives}$,
which is an inverse DFT that can again be done with FFT.

Overall, we have three linearithmic operations and one linear operation, so the total TC is just $O(n \log n)$.

\subsection{Part D}

We have
\begin{align*}
               & T(a)x = b                          \\
    \implies{} & \frac{1}{n}W^H\diag(Wa)Wx = b      \\
    \implies{} & \diag(Wa)Wx = Wb                   \\
    \implies{} & Wx = \diag(Wa)^{-1}Wb              \\
    \implies{} & x = \frac{1}{n}W^H\diag(Wa)^{-1}Wb
\end{align*}
The inverse of a diagonal matrix is the inverse of the elements along the diagonal,
so we can invert $\diag(Wa)$ given $Wa$ in $O(n)$ time.

With that in mind, we can compute the following in this order:
\begin{itemize}
    \item $Wb$ and $\diag(Wa)$ using the FFT in $O(n \log n)$
    \item $\diag(Wa)^{-1}Wb$ in $O(n)$
    \item $\frac{1}{n}W^H\left(\diag(Wa)^{-1}Wb\right)$ in $O(n \log n)$
\end{itemize}

An empirical comparison of the speed is in the attached Jupyter Notebook.

\pagebreak

\section{Exercise A6.25}

\subsection{Part A}

Let $q_i$ be the $i$th column of $Q$.

Since $R_{11} q_1 = \mathbf{1}$ and $\norm{q_1} = 1$,
\begin{align*}
    \norm{R_{11}q_1}
     & = R_{11}\norm{q_1} \\
     & = R_{11}           \\
     & = \sqrt{n}
\end{align*}
Solving for $n$ yields $n = \boxed{R_{11}^2}$.

This also implies that $q_1 = \frac{1}{\sqrt{n}}\mathbf{1}$.

\subsection{Part B}

Letting $r_i$ be the $i$th column of $R$, we have
\[a=\norm{Qr_1}=\norm{r_1}\]
since matrices with orthonormal columns preserve norms.
In a similar vein, $b=\norm{r_2}$.

\subsection{Part C}

From the second column,
\begin{align*}
               & R_{12}q_1 + R_{22}q_2 = a                         \\
    \implies{} & R_{12}q_1^Tq_1 = \frac{1}{\sqrt{n}}\mathbf{1}^T a \\
    \implies{} & \avg(a) = \boxed{\frac{R_{12}}{\sqrt{n}}}
\end{align*}
we can do something similar for the third column to see that $\avg(b) = \boxed{\frac{R_{13}}{\sqrt{n}}}$.

\subsection{Part D}

Given that we know the above quantities, $\std(a)$ (and by extension $\std(b)$) is just:
\begin{align*}
    \std(a)
     & = \rms(a-\avg(a)\mathbf{1})                                                                     \\
     & = \frac{1}{\sqrt{n}}\sqrt{(a-\avg(a)\mathbf{1})^T(a-\avg(a)\mathbf{1})}                                \\
     & = \frac{1}{\sqrt{n}}\sqrt{\left(a^Ta - 2\avg(a)\mathbf{1}^Ta + \avg(a)^2\mathbf{1}^T\mathbf{1}\right)} \\
     & = \frac{1}{\sqrt{n}}\sqrt{\norm{a}^2 - \avg(a)\mathbf{1}^Ta}                              \\
     & = \boxed{\sqrt{\frac{\norm{a}^2-n\avg(a)^2}{n}}}
\end{align*}

\subsection{Part E}

From class, we know that
\begin{align*}
    \rho_{ab}
     & = \frac{1}{n}\sum_{i=1}^{n} \frac{a_i-\avg(a)}{\std(a)}\frac{b_i-\avg(b)}{\std(b)}                                                \\
     & = \frac{1}{n\std(a)\std(b)}\sum_{i=1}^{n} (a_i-\avg(a))(b_i-\avg(b))                                                              \\
     & = \frac{1}{n\std(a)\std(b)} (a-\avg(a)\mathbf{1})^T(b-\avg(b)\mathbf{1})                                                          \\
     & = \frac{1}{n\std(a)\std(b)}\left(a^Tb - \avg(a)\mathbf{1}^Tb - \avg(b)\mathbf{1}^Ta + \avg(a)\avg(b)\mathbf{1}^T\mathbf{1}\right) \\
     & = \frac{1}{n\std(a)\std(b)}\left(a^Tb - n\avg(a)\avg(b)\right)
\end{align*}

All of the quantities except for $a^Tb$ can be figured out from the previous parts.

We get the dot product by using their expanded form in the QR:
\begin{align*}
    a^Tb
     & = (R_{12}q_1+R_{22}q_2)^T(R_{13}q_1 + R_{23}q_2 + R_{33}q_3) \\
     & = R_{12}R_{13}q_1^Tq_1 + R_{22}R_{23}q_2^Tq_2                \\
     & = R_{12}R_{13}+R_{22}R_{23}
\end{align*}
Plugging this newfound term back into the above expression gives us $\rho$.

\end{document}
