\documentclass[12pt]{article}

% a template that a friend gave, it's worked well enough for me
% i have added some packages and stuff that have proved useful

\usepackage{fancyhdr}
\usepackage{tipa}
\usepackage{fontspec}
\usepackage{amsfonts}
\usepackage{enumitem}
\usepackage[margin=1in]{geometry}
\usepackage{graphicx}
\usepackage{float}
\usepackage{amsmath}
\usepackage{braket}
\usepackage{amssymb}
\usepackage{booktabs}
\usepackage{hyperref}
\usepackage{mathtools}
\usepackage{xcolor}
\usepackage{float}
\usepackage{algpseudocodex}
\usepackage{titlesec}
\usepackage{bbm}
\usepackage{pythonhighlight}

\pagestyle{fancy}
\fancyhf{} % sets both header and footer to nothing
\lhead{Kevin Sheng}
\setmainfont{Comic Neue}
\renewcommand{\headrulewidth}{1pt}
\setlength{\headheight}{0.75in}
\setlength{\oddsidemargin}{0in}
\setlength{\evensidemargin}{0in}
\setlength{\voffset}{-.5in}
\setlength{\headsep}{10pt}
\setlength{\textwidth}{6.5in}
\setlength{\headwidth}{6.5in}
\setlength{\textheight}{8in}
\renewcommand{\headrulewidth}{0.5pt}
\renewcommand{\footrulewidth}{0.3pt}
\setlength{\textwidth}{6.5in}
\usepackage{setspace}
\usepackage{multicol}
\usepackage{float}
\setlength{\columnsep}{1cm}
\setlength\parindent{24pt}
\usepackage [english]{babel}
\usepackage [autostyle, english = american]{csquotes}
\MakeOuterQuote{"}

\setlength{\parskip}{6pt}
\setlength{\parindent}{0pt}

\titlespacing\section{0pt}{12pt plus 4pt minus 2pt}{0pt plus 2pt minus 2pt}
\titlespacing\subsection{0pt}{12pt plus 4pt minus 2pt}{0pt plus 2pt minus 2pt}
\titlespacing\subsubsection{0pt}{12pt plus 4pt minus 2pt}{0pt plus 2pt minus 2pt}

\hypersetup{colorlinks=true, urlcolor=blue}

\newcommand{\correction}[1]{\textcolor{red}{#1}}


\rhead{ECE 133A}

\DeclareMathOperator{\std}{std}
\DeclareMathOperator{\avg}{avg}
\DeclareMathOperator{\rms}{rms}
\DeclareMathOperator{\Tr}{Tr}
\DeclareMathOperator{\diag}{diag}
\newcommand*{\vertbar}{\rule[-1ex]{0.5pt}{2.5ex}}
\newcommand*{\horzbar}{\rule[.5ex]{2.5ex}{0.5pt}}

\begin{document}

\section{Exercise A8.3}

We can rearrange the equation to get
\[\alpha t_i + \beta = \ln\left(\frac{y_i}{1-y_i}\right)\]
which is a linear transformation of $\alpha$ and $\beta$.
Since $y_i \ne 0$, division by $0$ won't happen.

Thus, the parameters for our least squares regression are
\begin{align*}
    A = \begin{bmatrix}
            t & \mathbf{1}
        \end{bmatrix}
     &  &
    b=\ln(y \oslash (1-y))
\end{align*}
where $\oslash$ denotes elementwise division.
The variable is just the 2D vector $\begin{bmatrix}\alpha \\ \beta\end{bmatrix}$.

A test of this formulation can be found in the attached Jupyter Notebook.

\pagebreak

\section{Exercise A8.6}

\subsection{Part A}

The term inside the summation can be "simplified" to
\[(u_i-u_c)^2+(v_i-v_c)^2-R^2=(u_c^2+v_c^2-R^2)-2u_cu_i+u_i^2-2v_cv_i+v_i^2\]

Thus, we can define this as a least squares with
\begin{align*}
    A=\begin{bmatrix}
          -2u & -2v & \mathbf{1}
      \end{bmatrix} &  &
    b=-u^2-v^2              &  & x=\begin{bmatrix}u_c \\ v_c \\ w\end{bmatrix}
\end{align*}
where $u$ and $v$ are the vectors containing $u_i$ and $v_i$ respectively.

\subsection{Part B}

Let $w > u_c^2+v_c^2$.
It suffices to show that there's a better value of $w$ given a $u_c$ and $v_c$.

Then
\begin{align*}
    (u_i-u_c)^2+(v_i-v_c)^2-R^2
     & = w-2u_cu_i+u_i^2-2v_cv_i+v_i^2           \\
     & > u_c^2+v_c^2-2u_cu_i+u_i^2-2v_cv_i+v_i^2 \\
     & = (u_c-u_i)^2+(v_c-v_i)^2                 \\
     & \ge 0
\end{align*}
so across all things in the summation the term inside the parenthesis
that's being squared is strictly positive.

This means that by decreasing $w$ we can reduce the magnitude across
all terms and by extension the norm of the residual, giving us a more optimal solution. $\square$

Testing of this formulation can be found in the attached Jupyter Notebook.

\section{Exercise T12.12}

\subsection{Part A}

Nothing a little equation manipulation can't do:
\begin{align*}
    \mathcal{D}(u)+\mathcal{D}(v)
     & = \sum_{e=1}^{L} ((p_{i_e})_1-(p_{j_e})_1)^2+ \sum_{e=1}^{L} ((p_{i_e})_2-(p_{j_e})_2)^2 \\
     & = \sum_{e=1}^{L} ((p_{i_e})_1-(p_{j_e})_1)^2 + ((p_{i_e})_2-(p_{j_e})_2)^2               \\
     & = \sum_{e=1}^{L} \norm{p_{i_e}-p_{j_e}}^2\quad\square
\end{align*}
Notice how it doesn't matter which direction the edges go.

\subsection{Part B}

\subsubsection{Setup}

Let $B_1$ be an $L \times N-K$ matrix with each row corresponding to an edge,
and $c_1$ be a vector of length $L$ with corresponding entries.

For each edge $(i_e, j_e)$, we define the rows like so:
\begin{itemize}[nolistsep]
    \item If $i_e$ and $j_e$ are both free nodes, then $(B_1)_{e(i_e)}=1$ and $(B_1)_{e(j_e)}=-1$.
    \item If $i_e$ is free and $j_e$ is fixed, then $(B_1)_{e(i_e)}=1$ and $(c_1)_e=-(p_e)_1$.
          We do a similar thing if it's vice versa.
    \item If both are fixed, $(c_1)_e=(p_{i_e})_1-(p_{j_e})_1$.
\end{itemize}
Any entry that isn't mentioned is set to $0$.

We also define $B_2 \in \R^{L \times N-K}$ and $c_2 \in \R^L$
using a similar procedure, only using the second coordinates instead of the first.

I propose that minimizing
\begin{equation} \label{eq:1}
    \norm{\begin{bmatrix}B_1 & 0 \\ 0 & B_2\end{bmatrix}x-\begin{bmatrix}c_1 \\ c_2\end{bmatrix}}
\end{equation}
solves the problem.

\subsubsection{Proof This Works and I'm Not Crazy}

To do this, I'll first show that $\norm{B_1u-c_1}^2=\mathcal{D}(u)$.

Let the resulting vector be $r \in \R^L$.

Each edge $(i_e, j_e)$ can be done separately and according to the cases I've laid out before:
\begin{itemize}[nolistsep]
    \item If $i_e$ and $j_e$ are both fixed, $r_e=1 \cdot (p_{i_e})_1 - 1 \cdot (p_{j_e})_1=(p_{i_e})_1 - (p_{j_e})_1$.
    \item If $i_e$ is free and $j_e$ is fixed, $r_e=1 \cdot (p_{i_e})_1 -(c_1)_e=(p_{i_e})_1 - (p_{j_e})_1$.
          The reversed case is basically identical.
    \item If both are fixed, the formula is just $-(c_1)_e$.
\end{itemize}

The square of all of these components are $((p_{i_e})_1 - (p_{j_e})_1)^2$,
which when summed is the same as $\mathcal{D}(u)$.

For the same reasons, $\norm{B_2v-c_2}^2=\mathcal{D}(v)$.

Since \ref{eq:1} takes the norm of the \textit{concatenated} vectors,
not their sum, it is equivalent to $\mathcal{D}(u)+\mathcal{D}(v)$, which is
precisely what we're trying to minimize. $\square$

\subsection{Part C}

The best coordinates for the first $6$ points are:
\[\begin{bmatrix}
        0.355 & 0.348 \\
        0.531 & 0.799 \\
        0.513 & 0.564 \\
        0.553 & 0.48  \\
        0.612 & 0.63  \\
        0.791 & 0.527
    \end{bmatrix}\]
and when plotted the graph looks like this:
\begin{center}
    \includegraphics[width=10cm]{img/hw5/placement}
\end{center}

\section{Exercise A8.11}

\subsection{Part A}

Fully expanded, the normal equation is
\[\begin{bmatrix}
        1 & 10^{-k} & 0       \\
        1 & 0       & 10^{-k}
    \end{bmatrix}\begin{bmatrix}
        1       & 1       \\
        10^{-k} & 0       \\
        0       & 10^{-k}
    \end{bmatrix}x=\begin{bmatrix}
        1 & 10^{-k} & 0       \\
        1 & 0       & 10^{-k}
    \end{bmatrix}\begin{bmatrix}
        -10^{-k} \\ 1+10^{-k} \\ 1-10^{-k}
    \end{bmatrix}\]
which simplifies to the linear system
\[\begin{bmatrix}
        1+10^{-2k} & 1          \\
        1          & 1+10^{-2k}
    \end{bmatrix}x=\begin{bmatrix}
        10^{-2k} \\ -10^{-2k}
    \end{bmatrix}\]

Solving this analytically gives \boxed{x=\begin{bmatrix}1 \\ -1\end{bmatrix}}.

\subsection{Part B}

The results for $k=6$, $k=7$, and $k=8$ are as follows:
\begin{align*}
    \begin{bmatrix}
        0.999999999860908 \\
        -0.999999999860908
    \end{bmatrix} &  &
    \begin{bmatrix}
        1.000000001838782 \\
        -1.000000001838781
    \end{bmatrix} &  &
    \begin{bmatrix}
        1.000000006592782 \\
        -1.000000006592782
    \end{bmatrix}
\end{align*}
I'm not sure how many decimal places are needed, so I just put
all the decimals \texttt{format long} gave me.

\subsection{Part C}

If we calculate it with \texttt{(A' * A) \textbackslash\ (A' * b)}, the results are
\begin{align*}
    \begin{bmatrix}
        0.999911107241502 \\
        -0.999911107241502
    \end{bmatrix} &  &
    \begin{bmatrix}
        1.000799917512465 \\
        -1.000799917512465
    \end{bmatrix} &  &
    \begin{bmatrix}
        \infty \\
        -\infty
    \end{bmatrix}
\end{align*}
Evidently, this method is more numerically unstable than the previous one.

\pagebreak

\section{Exercise A8.14}

oh my god bruh

\subsection{Part A}

It suffices to show that the given solution satisfies the normal equation
\[\begin{bmatrix}A & a\end{bmatrix}^T
    \begin{bmatrix}
        A & a
    \end{bmatrix}
    \begin{bmatrix}
        \hat{y} \\ \hat{z}
    \end{bmatrix}=\begin{bmatrix}A & a\end{bmatrix}^Tb\]
which "simplifies" to
\[\begin{bmatrix}
        A^TA & A^Ta \\
        a^TA & a^Ta
    \end{bmatrix}\begin{bmatrix}
        \hat{y} \\ \hat{z}
    \end{bmatrix}=\begin{bmatrix}
        A^T \\ a^T
    \end{bmatrix}b\]

Anyways, we can first multiply by $\begin{bmatrix}\hat{x} \\ 0\end{bmatrix}$:
\[\begin{bmatrix}
        A^TA & A^Ta \\
        a^TA & a^Ta
    \end{bmatrix}\begin{bmatrix}
        \hat{x} \\ 0
    \end{bmatrix}=\begin{bmatrix}
        A^TA\hat{x}+0 \cdot A^TA \\
        a^TA\hat{x}+0 \cdot a^Ta
    \end{bmatrix}=\begin{bmatrix}
        A^Tb \\
        a^TA\hat{x}
    \end{bmatrix}\]
where the last step is because $\hat{x}$ satisfies the normal equation wrt $A$ and $b$.

Then we multiply by $\begin{bmatrix}(A^TA)^{-1}A^Ta \\ -1\end{bmatrix}$:
\begin{align*}
    \begin{bmatrix}
        A^TA & A^Ta \\
        a^TA & a^Ta
    \end{bmatrix}\begin{bmatrix}
                     (A^TA)^{-1}A^Ta \\ -1
                 \end{bmatrix}
     & = \begin{bmatrix}
             A^TA(A^TA)^{-1}A^Ta-A^Ta \\
             a^TA(A^TA)^{-1}A^Ta-a^Ta
         \end{bmatrix} \\
     & = \begin{bmatrix}
             A^Ta-A^Ta \\
             a^T(A(A^TA)^{-1}A^Ta-a)
         \end{bmatrix}   \\
     & = \begin{bmatrix}
             0 \\
             a^T(A(A^TA)^{-1}A^T-I)a
         \end{bmatrix}
\end{align*}
Putting that nasty scalar in front gives us
\[-\frac{a^T(b-A\hat{x})}{a^T(I-A(A^TA)^{-1}A^T)a}\begin{bmatrix}
        0 \\
        a^T(A(A^TA)^{-1}A^T-I)a
    \end{bmatrix}=\begin{bmatrix}
        0 \\
        a^T(b-A\hat{x})
    \end{bmatrix}\]

Adding these two gives
\[\begin{bmatrix}
    A^Tb \\ a^TA\hat{x}
\end{bmatrix}+\begin{bmatrix}
    0 \\ a^T(b-A\hat{x})
\end{bmatrix}=\begin{bmatrix}
    A^Tb \\ a^Tb
\end{bmatrix}\]
which is exactly what we want. $\square$

\subsection{Part B}

We first calculate $\hat{x}:$
\begin{align*}
    \hat{x}
    &= (A^TA)^{-1}A^Tb \\
    &= ((QR)^TQR)^{-1}(QR)^Tb \\
    &= (R^TQ^TQR)^{-1}R^TQ^Tb \\
    &= (R^TR)^{-1}R^TQ^Tb \\
    &= R^{-1}R^{-T}R^TQ^Tb \\
    &= R^{-1}Q^Tb
\end{align*}
so we just have to solve the linear system $R\hat{x}=Q^Tb$.

Calculating $Q^Tb$ takes $m(2n-1)$ flops, and the back substitution for $R$ takes $n^2$ flops,
giving us a total flop count of around \boxed{2mn+n^2} flops.

To solve for $\hat{y}$ and $\hat{z}$, there's a couple things we need to solve.

The element in the second matrix, $A^\dagger a$, can be computed in much the same way
we did for $A^\dagger b$ in $2mn+n^2$ flops.

The scalar in the numerator, $a^T(b-A\hat{x})$, doesn't need anything fancy.
Doing all the muliplications through brute force takes around $2mn$ flops.

Finally, the denominator can first be simplified like so:
\begin{align*}
     a^T(I-A(A^TA)^{-1}A^T)a
     &= a^T(I-QR \cdot R^{-1}Q^T)a \\
     &= a^T(I-QQ^T)a
\end{align*}
Here, we need around $m^2n$ flops to compute $(I-QQ^T)$, and about $m^2$ flops to calculate $a^T(I-QQ^T)a$
by first doing the matrix-vector multiplication on the right.

It doesn't cost too much to put these elements together, since it's all just elementwise stuff at this point.
This gives us a flop count of about \boxed{2mn+n^2+m^2(n+1)} to compute $\hat{y}$ and $\hat{z}$.

\end{document}
