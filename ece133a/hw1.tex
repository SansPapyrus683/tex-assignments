\documentclass[12pt]{article}

% a template that a friend gave, it's worked well enough for me
% i have added some packages and stuff that have proved useful

\usepackage{fancyhdr}
\usepackage{tipa}
\usepackage{fontspec}
\usepackage{amsfonts}
\usepackage{enumitem}
\usepackage[margin=1in]{geometry}
\usepackage{graphicx}
\usepackage{float}
\usepackage{amsmath}
\usepackage{braket}
\usepackage{amssymb}
\usepackage{booktabs}
\usepackage{hyperref}
\usepackage{mathtools}
\usepackage{xcolor}
\usepackage{float}
\usepackage{algpseudocodex}
\usepackage{titlesec}
\usepackage{bbm}
\usepackage{pythonhighlight}

\pagestyle{fancy}
\fancyhf{} % sets both header and footer to nothing
\lhead{Kevin Sheng}
\setmainfont{Comic Neue}
\renewcommand{\headrulewidth}{1pt}
\setlength{\headheight}{0.75in}
\setlength{\oddsidemargin}{0in}
\setlength{\evensidemargin}{0in}
\setlength{\voffset}{-.5in}
\setlength{\headsep}{10pt}
\setlength{\textwidth}{6.5in}
\setlength{\headwidth}{6.5in}
\setlength{\textheight}{8in}
\renewcommand{\headrulewidth}{0.5pt}
\renewcommand{\footrulewidth}{0.3pt}
\setlength{\textwidth}{6.5in}
\usepackage{setspace}
\usepackage{multicol}
\usepackage{float}
\setlength{\columnsep}{1cm}
\setlength\parindent{24pt}
\usepackage [english]{babel}
\usepackage [autostyle, english = american]{csquotes}
\MakeOuterQuote{"}

\setlength{\parskip}{6pt}
\setlength{\parindent}{0pt}

\titlespacing\section{0pt}{12pt plus 4pt minus 2pt}{0pt plus 2pt minus 2pt}
\titlespacing\subsection{0pt}{12pt plus 4pt minus 2pt}{0pt plus 2pt minus 2pt}
\titlespacing\subsubsection{0pt}{12pt plus 4pt minus 2pt}{0pt plus 2pt minus 2pt}

\hypersetup{colorlinks=true, urlcolor=blue}

\newcommand{\correction}[1]{\textcolor{red}{#1}}


\rhead{ECE 133A}

\DeclareMathOperator{\std}{std}
\DeclareMathOperator{\avg}{avg}

\begin{document}

\section{Exercise T2.8}

\subsection{Part A}

I'm assuming that $\alpha$ and $\beta$ are fixed.

Evaluating the integral gives
\begin{align*}
    \int_\alpha^\beta p(x)\,dx
     & = \sum_{i=0}^{n-1} \int_{\alpha}^{\beta} c_{i+1}x^i                                  \\
     & = \sum_{i=0}^{n-1} \left.\left(\frac{c_{i+1}}{i+1}x^{i+1}\right)\right|^\beta_\alpha \\
     & = \sum_{i=1}^{n} \frac{c_i\left(\beta^i-\alpha^i\right)}{i}
\end{align*}
which is indeed a linear function of the coefficients.

This can be represented as $a^Tc$, where
\[a_i = \frac{\beta^i - \alpha^i}{i}\]

\subsection{Part B}

Taking the derivative at $\alpha$ gives us
\begin{align*}
    \frac{d}{dx} p(\alpha)
     & = \sum_{i=1}^{n-1} c_{i+1}i\alpha^{i-1} \\
     & = \sum_{i=2}^{n} c_i (i-1)\alpha^{i-2}
\end{align*}
which is also a linear function of the coefficients.

This can be written as $b^Tc$, where
\[b_i = \begin{cases}
        0                 & i=1   \\
        (i-1)\alpha^{i-2} & i > 1
    \end{cases}\]

\section{Exercise T3.9}

Notice that
\begin{align*}
    \norm{x-c}^2-\norm{x-d}^2
     & = \sum_{i=1}^{n} (x_i-c_i)^2 - (x_i-d_i)^2                     \\
     & = \sum_{i=1}^{n} 2x_id_i - 2x_ic_i + c_i^2 - d_i^2             \\
     & = \sum_{i=1}^{n} x_i(2d_i-2c_i) + \sum_{i=1}^{n} c_i^2 - d_i^2 \\
     & = 2(d-c)^Tx + \norm{c}^2 - \norm{d}^2
\end{align*}
so this is indeed an affine transformation of $x$.

Here,
\begin{gather*}
    a = 2(d-c)^T \\
    b = \norm{c}^2 - \norm{d}^2
\end{gather*}

\pagebreak

\section{Exercise A1.2}

Let $a$ and $b$ be vectors defined by:
\begin{align*}
    a_i=\sqrt{x_i} &  & b_i=\frac{1}{\sqrt{x_i}}
\end{align*}

Then by CS we've:
\begin{align*}
               & |x^Ty| \le \norm{x}\norm{y}                                                                \\
    \implies{} & \left|\sum_{i=1}^{n} \sqrt{x_i} \cdot \frac{1}{\sqrt{x_i}}\right|
    \le \sqrt{\sum_{i=1}^{n} x_i}\sqrt{\sum_{i=1}^{n} \frac{1}{x_i}}                                        \\
    \implies{} & n \le \sqrt{\sum_{i=1}^{n} x_i \cdot \sum_{i=1}^{n} \frac{1}{x_i}}                         \\
    \implies{} & n^2 \le \sum_{i=1}^{n} x_i \cdot \sum_{i=1}^{n} \frac{1}{x_i}                              \\
    \implies{} & \frac{n}{\sum_{i=1}^{n} \frac{1}{x_i}} \le \frac{1}{n}\sum_{i=1}^{n} x_i                   \\
    \implies{} & \left(\sum_{i=1}^{n} \frac{1}{n}\right)^{-1} \le \frac{1}{n}\sum_{i=1}^{n} x_i\quad\square
\end{align*}

\pagebreak

\section{Exercise A1.3}

OK THIS PROBLEM WASN'T ON THE HW I THOUGHT THIS WAS 1.2 HAHAHAHAHAHAAHHA

Following the hint, let $\gamma = \arg\left(b^Ha\right)$, $\alpha=\norm{a}$, and $\beta=\norm{b}$.

This way, $b^Ha = \left|b^Ha\right|e^{j\gamma}$.

Then,
\begin{align*}
    0
     & \le \norm{\beta a - \alpha e^{j\gamma} b}^2                                                                                                                                   \\
     & = \left(\beta a - \alpha e^{j\gamma} b\right)^H\left(\beta a - \alpha e^{j\gamma} b\right)                                                                                    \\
     & = \norm{\beta a}^2+\norm{\alpha e^{j\gamma} b}^2 - (\beta a)^H(\alpha e^{j\gamma} b) - (\alpha e^{j\gamma} b)^H(\beta a)                                                      \\
     & = \beta^2\norm{a}^2+\alpha^2\norm{b}^2 - \alpha\beta e^{j\gamma}\left(a^Hb + b^Ha\right)                                 & \text{since $\norm{e^{j\gamma}} = 1$}              \\
     & = 2\norm{a}^2\norm{b}^2 - 2\norm{a}\norm{b}e^{j\gamma}\left|b^Ha\right|                                                  & \text{since $\left|b^Ha\right|=\left|a^Hb\right|$} \\
     & = 2\norm{a}^2\norm{b}^2 - 2\norm{a}\norm{b}\left|b^Ha\right|
\end{align*}

Moving the second term to the left side gets us
\[2\norm{a}\norm{b}\left|b^Ha\right| \le 2\norm{a}^2\norm{b}^2
    \implies \left|b^Ha\right| \le \norm{a}\norm{b}\]

Equality holds true when $\beta a - \alpha e^{j\gamma} b=0$,
or in other words when $\beta a = \alpha e^{j\gamma} b$.

This equation trivially holds when at least one is the zero vector.

When both are nonzero, this occurs when $a$ and $b$ have the same "shape".
One has to be some scalar multiple of the other, but here the scalar can be a complex number.

\pagebreak

\section{Exercise T3.25}

\subsection{Part A}

The mean is
\begin{align*}
    \avg(p)
     & = \frac{1}{n}\sum_{i=1}^{n} \left(\theta r + (1-\theta) \mu^{\text{rf}}\mathbf{1}\right)_i \\
     & = \frac{\theta \mathbf{1}^Tr + n(1-\theta)\mu^{\text{rf}}}{n}                              \\
     & = \theta\mu + (1-\theta)\mu^{\text{rf}}
\end{align*}
and the STD is
\begin{align*}
    \std(p)
     & = \frac{1}{\sqrt{n}} \sqrt{\sum_{i=1}^{n} (p_i - \avg(p))^2}          \\
     & = \frac{1}{\sqrt{n}} \sqrt{\sum_{i=1}^{n} (\theta r_i - \theta\mu)^2} \\
     & = \frac{|\theta|}{\sqrt{n}} \sqrt{\sum_{i=1}^{n} (r_i-\mu)^2}         \\
     & = |\theta|\sigma
\end{align*}

At $\theta=0$, the mean is $\mu^{\text{rf}}$ and the STD is $0$, which makes sense
since you know 100\% that all your holdings are still gonna be there.

At $\theta=1$, the mean is $\mu$ and the STD is $\sigma$, which also makes sense
as you're basically completely at the whims of the market.

\subsection{Part B}

If we wanted a target risk level of $\sigma^{\text{tar}}$,
then we'd simply choose $\theta=\pm \frac{\sigma^{\text{tar}}}{\sigma}$.
Since $\theta$ is inside an absolute value, there's two solutions.

\subsection{Part C}

Leverage happens when $\theta > 1$, or when $\sigma^{\text{tar}} > \sigma$.

Shorting happens when $\theta < 0$, or simply whenever we choose to take the negative $\theta$.

Hedging happens in all the other cases.

\end{document}
