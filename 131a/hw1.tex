\documentclass[12pt]{article}

% a template that a friend gave, it's worked well enough for me
% i have added some packages and stuff that have proved useful

\usepackage{fancyhdr}
\usepackage{tipa}
\usepackage{fontspec}
\usepackage{amsfonts}
\usepackage{enumitem}
\usepackage[margin=1in]{geometry}
\usepackage{graphicx}
\usepackage{float}
\usepackage{amsmath}
\usepackage{braket}
\usepackage{amssymb}
\usepackage{booktabs}
\usepackage{hyperref}
\usepackage{mathtools}
\usepackage{xcolor}
\usepackage{float}
\usepackage{algpseudocodex}
\usepackage{titlesec}
\usepackage{bbm}

\pagestyle{fancy}
\fancyhf{} % sets both header and footer to nothing
\lhead{Kevin Sheng}
\setmainfont{Comic Neue}
\renewcommand{\headrulewidth}{1pt}
\setlength{\headheight}{0.75in}
\setlength{\oddsidemargin}{0in}
\setlength{\evensidemargin}{0in}
\setlength{\voffset}{-.5in}
\setlength{\headsep}{10pt}
\setlength{\textwidth}{6.5in}
\setlength{\headwidth}{6.5in}
\setlength{\textheight}{8in}
\renewcommand{\headrulewidth}{0.5pt}
\renewcommand{\footrulewidth}{0.3pt}
\setlength{\textwidth}{6.5in}
\usepackage{setspace}
\usepackage{multicol}
\usepackage{float}
\setlength{\columnsep}{1cm}
\setlength\parindent{24pt}
\usepackage [english]{babel}
\usepackage [autostyle, english = american]{csquotes}
\MakeOuterQuote{"}

\setlength{\parskip}{6pt}
\setlength{\parindent}{0pt}

\titlespacing\section{0pt}{12pt plus 4pt minus 2pt}{0pt plus 2pt minus 2pt}
\titlespacing\subsection{0pt}{12pt plus 4pt minus 2pt}{0pt plus 2pt minus 2pt}
\titlespacing\subsubsection{0pt}{12pt plus 4pt minus 2pt}{0pt plus 2pt minus 2pt}

\hypersetup{colorlinks=true, urlcolor=blue}

\newcommand{\correction}[1]{\textcolor{red}{#1}}


\rhead{Math 131A}

\begin{document}

\begin{enumerate}
      \item[1.] We WTS $\forall c \in C\ \exists a \in A: g(f(a))=c$.

            $g$ is surjective, so $\exists b \in B: g(b)=c$.
            $f$ is surjective as well, so $\exists a \in A: f(a)=b$.
            Replacing $b$ in one before the previous equation gets us $g(f(a))=c$. $\square$
      \item[4.] \[\exists x > 0: \forall y \in \mathbb{R}\ \exists a \in [x, y]: f(a) > 2\]
      \item[5.] Nothing is said about the continuity of said function, so let's define
            \[f(x)=\begin{cases}
                        0       & \exists n \in \mathbb{Z}: x=\frac{\pi}{2}+n\pi \\
                        \tan(x) & \text{otherwise}
                  \end{cases}\]
            The casework is just to fill in the holes of the original domain of $\tan(x)$.
            For any $y \in \mathbb{R}$, we can get two values of $x$ that output it:
            \[\tan(\arctan(y))=\tan(\arctan(y)+\pi)=y\]
\end{enumerate}
\end{document}
