\documentclass[12pt]{article}

% a template that a friend gave, it's worked well enough for me
% i have added some packages and stuff that have proved useful

\usepackage{fancyhdr}
\usepackage{tipa}
\usepackage{fontspec}
\usepackage{amsfonts}
\usepackage{enumitem}
\usepackage[margin=1in]{geometry}
\usepackage{graphicx}
\usepackage{float}
\usepackage{amsmath}
\usepackage{braket}
\usepackage{amssymb}
\usepackage{booktabs}
\usepackage{hyperref}
\usepackage{mathtools}
\usepackage{xcolor}
\usepackage{float}
\usepackage{algpseudocodex}
\usepackage{titlesec}
\usepackage{bbm}
\usepackage{pythonhighlight}

\pagestyle{fancy}
\fancyhf{} % sets both header and footer to nothing
\lhead{Kevin Sheng}
\setmainfont{Comic Neue}
\renewcommand{\headrulewidth}{1pt}
\setlength{\headheight}{0.75in}
\setlength{\oddsidemargin}{0in}
\setlength{\evensidemargin}{0in}
\setlength{\voffset}{-.5in}
\setlength{\headsep}{10pt}
\setlength{\textwidth}{6.5in}
\setlength{\headwidth}{6.5in}
\setlength{\textheight}{8in}
\renewcommand{\headrulewidth}{0.5pt}
\renewcommand{\footrulewidth}{0.3pt}
\setlength{\textwidth}{6.5in}
\usepackage{setspace}
\usepackage{multicol}
\usepackage{float}
\setlength{\columnsep}{1cm}
\setlength\parindent{24pt}
\usepackage [english]{babel}
\usepackage [autostyle, english = american]{csquotes}
\MakeOuterQuote{"}

\setlength{\parskip}{6pt}
\setlength{\parindent}{0pt}

\titlespacing\section{0pt}{12pt plus 4pt minus 2pt}{0pt plus 2pt minus 2pt}
\titlespacing\subsection{0pt}{12pt plus 4pt minus 2pt}{0pt plus 2pt minus 2pt}
\titlespacing\subsubsection{0pt}{12pt plus 4pt minus 2pt}{0pt plus 2pt minus 2pt}

\hypersetup{colorlinks=true, urlcolor=blue}

\newcommand{\correction}[1]{\textcolor{red}{#1}}


\begin{document}
\begin{enumerate}
      \item \begin{enumerate}
                  \item Since $\beta$ is a basis, we can write vectors $a, b \in V$ in the following form:
                        \begin{align*}
                              a=\sum_{i=1}^{n} a_i \beta_i &  & b=\sum_{i=1}^{n} b_i \beta_i
                        \end{align*}
                        Given this, we know that $a+\lambda b$ can be expressed as follows:
                        \[a+\lambda b=\sum_{i=1}^{n} (a_i+\lambda b_i) \beta_i\]
                        Then we proceed as normal with proving that $T$ is linear.
                        \begin{align*}
                              T(a+\lambda b) & =[a+\lambda b]_\beta            \\
                                             & =\begin{bmatrix}
                                                      a_1+\lambda b_1 \\
                                                      \vdots          \\
                                                      a_n+\lambda b_n
                                                \end{bmatrix}                \\
                                             & = \begin{bmatrix}
                                                       a_1    \\
                                                       \vdots \\
                                                       a_n
                                                 \end{bmatrix} +\lambda
                              \begin{bmatrix}
                                    b_1    \\
                                    \vdots \\
                                    b_n
                              \end{bmatrix}                                   \\
                                             & = [a]_\beta+\lambda [b]_\beta   \\
                                             & = T(a)+\lambda T(b)\quad\square
                        \end{align*}
                  \item We know that if $a \ne b$, then their linear combinations in terms of a basis $\beta$ are also distinct.
                        In particular, if $a =\sum_{i=1}^{n} a_i \beta_i$ and $b=\sum_{i=1}^{n} b_i \beta_i$, then $\exists i: a_i \ne b_i$.
                        Since $f_\beta$ just takes these coefficients, we know that it is injective.

                        For surjectivity, we can construct any $v \in V$ with a column vector like so:
                        \[f^{-1}_\beta\left(\begin{bmatrix}
                                          v_1    \\
                                          \vdots \\
                                          v_n
                                    \end{bmatrix}\right)=\sum_{i=1}^{n} v_i \beta_i \in V\]
                        Plugging this linear combiantion back into $f_\beta$ gives us the same column vector, so $f_\beta$ is surjective as well. $\square$
                  \item Because $f_\beta$ is a bijection, we can convert any coordinate in $F^n$ back to a vector in $v$ by applying its inverse.
                        It's important that this function is linear because then we can add vectors in $F^n$ as we please and get the same
                        value when converting back when compared to if we had done the operations to the vectors in $V$ initially.
            \end{enumerate}
            \pagebreak
      \item \begin{enumerate}
                  \item We know the $j$th column of $[T]^\gamma_\beta$ is equivalent to $[T(v_j)]_\gamma$.
                        Looking at the summation in $Q$, it's clear that it performs a summation over all elements in the $j$th column as well.
                        
                        Also, $[T(v_j)]_\gamma$ is equivalent to the coefficients of the elements in $\gamma$
                        s.t. they make a linear combination that sums to $T(v_j)$ when combined with $\gamma$.
                        
                        This is precisely what the elements of the column vector are doing in the summation, so
                        the proposition is true.

                  \item \begin{enumerate}
                              \item $T(0,1)=(0,0)=\vec{0}$ \\
                                    $T(1,0)=(1,0)=1 \cdot (1, 0)$ \\
                                    Combining these two results, we get
                                    \[[T]^\beta_\beta=\begin{bmatrix}
                                                0 & 0 \\
                                                0 & 1
                                          \end{bmatrix}\]
                              \item $T(0,1)=(0,0)=\vec{0}$ (again) \\
                                    $T(1,0)=(1,0)=1 \cdot (1, 1) - 1 \cdot (0, 1)$ \\
                                    With these two, we get that
                                    \[[T]^\gamma_\beta=\begin{bmatrix}
                                                0 & 1  \\
                                                0 & -1
                                          \end{bmatrix}\]
                        \end{enumerate}
                  \item \begin{enumerate}
                              \item Applying $T$ to each of the elements in $\beta$, we get
                                    \begin{align*}
                                          T(E_{11})=E_{11} &  & T(E_{12})=E_{21} \\
                                          T(E_{21})=E_{12} &  & T(E_{22})=E_{22}
                                    \end{align*}
                                    The result of each is simply another element in the basis, so
                                    we compute how the result has been shuffled and obtain
                                    \[[T]^\beta_\beta=\begin{bmatrix}
                                                1 & 0 & 0 & 0 \\
                                                0 & 0 & 1 & 0 \\
                                                0 & 1 & 0 & 0 \\
                                                0 & 0 & 0 & 1
                                          \end{bmatrix}\]
                              \item Now we have to express the results from the linear transformations in terms of a new basis, $\gamma$.
                                    This isn't too hard, since $\gamma$ only slightly differs from $\beta$.
                                    \begin{align*}
                                          E_{11} & =(E_{11}+E_{12})-E_{12} &  & E_{21}=E_{21} \\
                                          E_{12} & =E_{12}                 &  & E_{22}=E_{22}
                                    \end{align*}
                                    With this, we get
                                    \[[T]^\gamma_\beta=\begin{bmatrix}
                                                1  & 0 & 0 & 0 \\
                                                -1 & 0 & 1 & 0 \\
                                                0  & 1 & 0 & 0 \\
                                                0  & 0 & 0 & 1
                                          \end{bmatrix}\]
                        \end{enumerate}
            \end{enumerate}
\end{enumerate}
\end{document}
