\documentclass[12pt]{article}

% a template that a friend gave, it's worked well enough for me
% i have added some packages and stuff that have proved useful

\usepackage{fancyhdr}
\usepackage{tipa}
\usepackage{fontspec}
\usepackage{amsfonts}
\usepackage{enumitem}
\usepackage[margin=1in]{geometry}
\usepackage{graphicx}
\usepackage{float}
\usepackage{amsmath}
\usepackage{braket}
\usepackage{amssymb}
\usepackage{booktabs}
\usepackage{hyperref}
\usepackage{mathtools}
\usepackage{xcolor}
\usepackage{float}
\usepackage{algpseudocodex}
\usepackage{titlesec}
\usepackage{bbm}

\pagestyle{fancy}
\fancyhf{} % sets both header and footer to nothing
\lhead{Kevin Sheng}
\setmainfont{Comic Neue}
\renewcommand{\headrulewidth}{1pt}
\setlength{\headheight}{0.75in}
\setlength{\oddsidemargin}{0in}
\setlength{\evensidemargin}{0in}
\setlength{\voffset}{-.5in}
\setlength{\headsep}{10pt}
\setlength{\textwidth}{6.5in}
\setlength{\headwidth}{6.5in}
\setlength{\textheight}{8in}
\renewcommand{\headrulewidth}{0.5pt}
\renewcommand{\footrulewidth}{0.3pt}
\setlength{\textwidth}{6.5in}
\usepackage{setspace}
\usepackage{multicol}
\usepackage{float}
\setlength{\columnsep}{1cm}
\setlength\parindent{24pt}
\usepackage [english]{babel}
\usepackage [autostyle, english = american]{csquotes}
\MakeOuterQuote{"}

\setlength{\parskip}{6pt}
\setlength{\parindent}{0pt}

\titlespacing\section{0pt}{12pt plus 4pt minus 2pt}{0pt plus 2pt minus 2pt}
\titlespacing\subsection{0pt}{12pt plus 4pt minus 2pt}{0pt plus 2pt minus 2pt}
\titlespacing\subsubsection{0pt}{12pt plus 4pt minus 2pt}{0pt plus 2pt minus 2pt}

\hypersetup{colorlinks=true, urlcolor=blue}

\newcommand{\correction}[1]{\textcolor{red}{#1}}


\begin{document}
\begin{enumerate}
      \item For all these parts, let $v=a+bi$.
            \begin{enumerate}
                  \item \begin{align*}
                              ||cv|| & = \braket{cv, cv}^{1/2}                                   \\
                                     & = \left(c(a+bi) \cdot \overline{c(a+bi)}\right)^{1/2}     \\
                                     & = \left|(ca+cbi)(ca-cbi)\right|^{1/2}                     \\
                                     & = \left(c^2a^2+c^2b^2\right)^{1/2}                        \\
                                     & = \left(a^2+b^2\right)^{1/2} \cdot \left(c^2\right)^{1/2} \\
                                     & = ||v|| \cdot |c|
                        \end{align*}
                  \item \textbf{Forward Direction:} \\
                        If $||v||=0$, then $\braket{v, v}^{1/2}=0$ and by extension $\braket{v, v}=v\overline{v}=0$ as well.
                        Then, $v\overline{v}=a^2+b^2$, and the only way this can be true is if $a=b=0$.
                        Thus, $v=\vec{0}$.

                        \textbf{Backward Direction:} \\
                        If $v=\vec{0}$, then $\braket{v, v}=(0+0i)(0+0i)=0$ and $||v||=0$.
                  \item $||v||=\sqrt{a^2+b^2}$.
                        This gives us that $\frac{v}{||v||}=\frac{a+bi}{\sqrt{a^2+b^2}}$.
                        \begin{align*}
                              \left|\left|\frac{v}{||v||}\right|\right| & = \left(\frac{a+bi}{\sqrt{a^2+b^2}} \cdot \frac{a-bi}{\sqrt{a^2+b^2}}\right)^{1/2} \\
                                                                        & = \left(\frac{a^2+b^2}{a^2+b^2}\right)^{1/2}                                       \\
                                                                        & = 1^{1/2}                                                                          \\
                                                                        & = 1
                        \end{align*}
                        Since its norm is $1$, $\frac{v}{||v||}$ is a normal vector.
            \end{enumerate}

            \pagebreak

      \item \begin{enumerate}
                  \item BWOC let $u$ and $v$ be LD vectors.
                        This means that $\exists c: u=cv$.
                        Taking their product, we have
                        \[\braket{u, v} = u\overline{v} = cv\overline{v}\]
                        In WS8, we proved that this is zero iff $v=\vec{0}$.
                        However, $v \ne \vec{0}$ by the premise of the proof.
                        Contradiction.
                        Thus, $\{u, v\}$ must be an LI set of vectors.
                  \item \textbf{Lemma:} If $\braket{x, a_1}=\braket{x, a_2}=\cdots =\braket{x, a_n}=0$,
                        Then $\braket{x, \sum_{i=1}^{c_i a_i}}=0$ for all possible coefficients.
                        This follows from the following properties of inner products:
                        \begin{gather*}
                              \braket{x, y}+\braket{x, z}=\braket{x,y+z} \\
                              \braket{x, cy}=\overline{c}\braket{x, y}
                        \end{gather*}

                        \textbf{Actual Proof}: \\
                        We prove this by induction on the number of vectors in the set.
                        The base case for two vectors was already proved, and so we assume
                        that if $\{v_1, \cdots, v_n\}$ is an orthogonal set of vectors, then they must be LI as well.

                        It remains to prove this implies the same proposition
                        for $\{v_1, \cdots, v_n, v_{n+1}\}$ as well.

                        Taking away $v_{n+1}$ from the set, we have a set of $n$ orthogonal vectors
                        that by the hypothesis we know are LI.
                        Now, BWOC suppose that the set was LD.
                        This would mean that there exists coefficients $c_1, \cdots, c_n$ s.t.
                        \[\sum_{i=1}^{n} c_i v_i = v_{n+1}\]
                        Since this is an orthogonal set of vectors, we also have by the lemma that
                        \[\braket{\sum_{i=1}^{n} c_i v_i, v_{n+1}}=0 \rightarrow \braket{v_{n+1}, v_{n+1}}=0\]
                        Once again, by the result in WS8, this forces $v_{n+1}$ to be $\vec{0}$,
                        which we already established wasn't possible.

                        Thus, $\{v_1, \cdots, v_n, v_{n+1}\}$ has to be an LI set of vectors as well,
                        and the inductive step is complete. $\square$
            \end{enumerate}
\end{enumerate}
\end{document}
