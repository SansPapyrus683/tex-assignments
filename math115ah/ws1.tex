\documentclass[12pt]{article}

% a template that a friend gave, it's worked well enough for me
% i have added some packages and stuff that have proved useful

\usepackage{fancyhdr}
\usepackage{tipa}
\usepackage{fontspec}
\usepackage{amsfonts}
\usepackage{enumitem}
\usepackage[margin=1in]{geometry}
\usepackage{graphicx}
\usepackage{float}
\usepackage{amsmath}
\usepackage{braket}
\usepackage{amssymb}
\usepackage{booktabs}
\usepackage{hyperref}
\usepackage{mathtools}
\usepackage{xcolor}
\usepackage{float}
\usepackage{algpseudocodex}
\usepackage{titlesec}
\usepackage{bbm}
\usepackage{pythonhighlight}

\pagestyle{fancy}
\fancyhf{} % sets both header and footer to nothing
\lhead{Kevin Sheng}
\setmainfont{Comic Neue}
\renewcommand{\headrulewidth}{1pt}
\setlength{\headheight}{0.75in}
\setlength{\oddsidemargin}{0in}
\setlength{\evensidemargin}{0in}
\setlength{\voffset}{-.5in}
\setlength{\headsep}{10pt}
\setlength{\textwidth}{6.5in}
\setlength{\headwidth}{6.5in}
\setlength{\textheight}{8in}
\renewcommand{\headrulewidth}{0.5pt}
\renewcommand{\footrulewidth}{0.3pt}
\setlength{\textwidth}{6.5in}
\usepackage{setspace}
\usepackage{multicol}
\usepackage{float}
\setlength{\columnsep}{1cm}
\setlength\parindent{24pt}
\usepackage [english]{babel}
\usepackage [autostyle, english = american]{csquotes}
\MakeOuterQuote{"}

\setlength{\parskip}{6pt}
\setlength{\parindent}{0pt}

\titlespacing\section{0pt}{12pt plus 4pt minus 2pt}{0pt plus 2pt minus 2pt}
\titlespacing\subsection{0pt}{12pt plus 4pt minus 2pt}{0pt plus 2pt minus 2pt}
\titlespacing\subsubsection{0pt}{12pt plus 4pt minus 2pt}{0pt plus 2pt minus 2pt}

\hypersetup{colorlinks=true, urlcolor=blue}

\newcommand{\correction}[1]{\textcolor{red}{#1}}


\begin{document}
    \begin{enumerate}
        \item \begin{enumerate}
                  \item \begin{enumerate}
                            \item Since equivalence relations are reflexive, we know that $a \sim b$ and thus $a \in [a]$.
                            \item We have to prove this statement in both directions. \\
                            \textbf{If $a \sim b$, then $[a]=[b]$:}
                            To prove $[a] \subseteq [b]$, notice that for every element $c$ in $[a]$, $c \sim a$ and by the transitive property $c \sim b$ as well. Thus, $c \in [a] \rightarrow c \in [b]$.
                            By symmetry, $[b] \subseteq [a]$ as well. \\
                            \textbf{Vice versa:} Assume $[a] = [b]$ and $a \nsim b$ for sake of contradiction.
                            By the reflexivity property, we know that $a \in [a]$, and by extension $a \in [b]$.
                            Since $a \in [b] \rightarrow a \sim b$, we have a contradiction. $\square$
                  \end{enumerate}
                  \item \begin{enumerate}
                            \item We just have to show that $[a] \in P(S)\,\forall\,a \in S$. \\
                            By definition, $[a] \subseteq S \rightarrow [a] \in P(S)$. $\square$
                            \item $S/R$ pertains to 1(a)(ii) above in that duplicate equivalence classes aren't included in the set.
                            \item We have to show that each $r$ between $0$ and $n-1$ is in its own equivalence class.
                            Take two instances of $r$ and call them $i$ and $j$ respectively.
                            WLOG assume $i<j$.
                            Since $0 \le i < j < n$, we know that $-n < i-j < 0 \therefore n \nmid i-j$, and by extension $i$ and $j$ are not related.
                            Because they aren't related, $i$ and $j$ must also be in separate equivalence classes.

                            An equivalence class in $\mathbb{Z}/n\mathbb{Z}$ can't have zero values of $r$.
                            This is because for any $x \in \mathbb{Z}$, there is some $0 \le y < n$ s.t. $n | x-y$.
                            We can always find such a $y$ by reducing $x \mod n$.

                            Thus, any equivalence class in this quotient can't have zero or more than two values of $r$, so each must have their own distinct value of $r$. $\square$
                  \end{enumerate}
        \end{enumerate}

        \item \begin{enumerate}
                  \item \begin{enumerate}
                            \item Multiplication in $\mathbb{Z}/3\mathbb{Z}$ \\
                            \begin{tabular}{llll}
                                & {[}0{]} & {[}1{]} & {[}2{]} \\
                                {[}0{]} & 0       & 0       & 0       \\
                                {[}1{]} & 0       & 1       & 2       \\
                                {[}2{]} & 0       & 2       & 1
                            \end{tabular}
                            \item Addition in $\mathbb{Z}/3\mathbb{Z}$ \\
                            \begin{tabular}{llll}
                                & {[}0{]} & {[}1{]} & {[}2{]} \\
                                {[}0{]} & 0       & 1       & 2      \\
                                {[}1{]} & 1       & 2       & 0       \\
                                {[}2{]} & 2       & 0       & 1
                            \end{tabular}
                            \item Multiplication in $\mathbb{Z}/4\mathbb{Z}$ \\
                            \begin{tabular}{lllll}
                                & {[}0{]} & {[}1{]} & {[}2{]} & {[}3{]} \\
                                {[}0{]} & 0       & 0       & 0    & 0  \\
                                {[}1{]} & 0       & 1       & 2    & 3   \\
                                {[}2{]} & 0       & 2       & 0    & 2   \\
                                {[}3{]} & 0       & 3       & 2    & 1   \\
                            \end{tabular}
                  \end{enumerate}

                  \pagebreak

                  \item \begin{enumerate}
                            \item $3$ is prime because its only divisors are $1$ and itself.
                            $4=2 \cdot 2$, so it's composite.
                            \item The table is symmetric along its diagonal, so commutativity seems to hold. \\
                            $0_F$ and $1_F$ are $[0]$ and $[1]$ respectively. \\
                            Upon inspection, the opposite of each number $x$ is $n-x$, while the inverse is $x^{-1} \mod n$.
                            \item $\mathbb{Z}/4\mathbb{Z}$ isn't a field because $2$ doesn't have an inverse.
                            Since $x^{-1} \mod n$ only exists if $x$ and $n$ are coprime, composite modulo are bound to have some numbers less than them without inverses.
                  \end{enumerate}
        \end{enumerate}
    \end{enumerate}

\end{document}
