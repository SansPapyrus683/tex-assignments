\documentclass[12pt]{article}

% a template that a friend gave, it's worked well enough for me
% i have added some packages and stuff that have proved useful

\usepackage{fancyhdr}
\usepackage{tipa}
\usepackage{fontspec}
\usepackage{amsfonts}
\usepackage{enumitem}
\usepackage[margin=1in]{geometry}
\usepackage{graphicx}
\usepackage{float}
\usepackage{amsmath}
\usepackage{braket}
\usepackage{amssymb}
\usepackage{booktabs}
\usepackage{hyperref}
\usepackage{mathtools}
\usepackage{xcolor}
\usepackage{float}
\usepackage{algpseudocodex}
\usepackage{titlesec}
\usepackage{bbm}

\pagestyle{fancy}
\fancyhf{} % sets both header and footer to nothing
\lhead{Kevin Sheng}
\setmainfont{Comic Neue}
\renewcommand{\headrulewidth}{1pt}
\setlength{\headheight}{0.75in}
\setlength{\oddsidemargin}{0in}
\setlength{\evensidemargin}{0in}
\setlength{\voffset}{-.5in}
\setlength{\headsep}{10pt}
\setlength{\textwidth}{6.5in}
\setlength{\headwidth}{6.5in}
\setlength{\textheight}{8in}
\renewcommand{\headrulewidth}{0.5pt}
\renewcommand{\footrulewidth}{0.3pt}
\setlength{\textwidth}{6.5in}
\usepackage{setspace}
\usepackage{multicol}
\usepackage{float}
\setlength{\columnsep}{1cm}
\setlength\parindent{24pt}
\usepackage [english]{babel}
\usepackage [autostyle, english = american]{csquotes}
\MakeOuterQuote{"}

\setlength{\parskip}{6pt}
\setlength{\parindent}{0pt}

\titlespacing\section{0pt}{12pt plus 4pt minus 2pt}{0pt plus 2pt minus 2pt}
\titlespacing\subsection{0pt}{12pt plus 4pt minus 2pt}{0pt plus 2pt minus 2pt}
\titlespacing\subsubsection{0pt}{12pt plus 4pt minus 2pt}{0pt plus 2pt minus 2pt}

\hypersetup{colorlinks=true, urlcolor=blue}

\newcommand{\correction}[1]{\textcolor{red}{#1}}


\begin{document}
\begin{enumerate}
    \item \begin{enumerate}
              \item First, let's prove that any $T(v)$ can be expressed as a linear combination of $T(s)$ for the elements in $S$.
                    We know that $\text{span}(S)=V$, so any $v$ is expressable in terms of the formula $\sum_{i=1}^n a_i s_i$.

                    Thus, \[T(v)=T\left(\sum_{i=1}^n a_i s_i\right)=\sum_{i=1}^n T(a_i s_i)=\sum_{i=1}^n a_i T(s_i)\]

                    We also have to prove that any linear combination of $T(s)$ can be expressed as $T(v)$.
                    The proof for this goes much the same way:
                    \[\sum_{i=1}^n a_i T(s_i)=\sum_{i=1}^n T(a_i s_i)=T\left(\sum_{i=1}^n a_i s_i\right)\]
                    We end up right back where we started. $\square$
              \item \begin{enumerate}
                        \item The basis of $P_n(F)$ is $\{1, x, x^2, \cdots x^n\}$.
                              Passing these through the transformation gives us a set of all $\vec{0}$ except for $1$ and $x^n$,
                              so the image is $\boxed{\text{span}(\{1, x^n\})}$.

                        \item Passing the standard basis into the transformation gives us the following four vectors:
                              \[\begin{bmatrix}
                                      1 \\ 4 \\ 3
                                  \end{bmatrix}, \begin{bmatrix}
                                      0 \\ 0 \\ 3
                                  \end{bmatrix}, \begin{bmatrix}
                                      1 \\ 0 \\ -1
                                  \end{bmatrix}, \begin{bmatrix}
                                      2 \\ 0 \\ 0
                                  \end{bmatrix}\]
                              By our previous result, the image is simply the span of these four columns.
                              Notice that these four vectors happen to form the columns of the given matrix used to transform the tuple.

                        \item Using the standard basis for all $2 \times 2$ matrices, we have
                              \[\begin{bmatrix}
                                      1 & 0 \\ 0 & 0
                                  \end{bmatrix}, \begin{bmatrix}
                                      0 & 0 \\ 0 & 1
                                  \end{bmatrix}, \begin{bmatrix}
                                      1 & 0 \\ 0 & 0
                                  \end{bmatrix}, \begin{bmatrix}
                                      0 & 0 \\ 0 & 1
                                  \end{bmatrix}\]
                              The image is the span of these four matrices, but since there's dupes,
                              we can cut them out and just say that the image is
                              \[\boxed{\text{span}\left(\begin{bmatrix}
                                              1 & 0 \\ 0 & 0
                                          \end{bmatrix}, \begin{bmatrix}
                                              0 & 0 \\ 0 & 1
                                          \end{bmatrix}\right)}\]

                        \item The kernel for the polynomial transformation is $\{x^i: i \in \mathbb{Z}, 1<i<n\}$.

                              As for the first matrix transformation, we first row reduce the matrix to
                              \[\begin{bmatrix}
                                      1 & 0 & 0  & 0 \\
                                      0 & 3 & -1 & 0 \\
                                      0 & 0 & 1  & 2
                                  \end{bmatrix}\]
                              From this, we can see that the kernel of this transformation is
                              \[\text{span}\left(\begin{bmatrix}
                                          0 \\ 1 \\ -\frac{1}{3} \\ \frac{1}{6}
                                      \end{bmatrix}\right)\]

                              Finally, the kernel of the second matrix transformation is
                              \[\text{span}\left(\begin{bmatrix}
                                          1  & 0 \\
                                          -1 & 0
                                      \end{bmatrix}, \begin{bmatrix}
                                          0 & 1  \\
                                          0 & -1
                                      \end{bmatrix}\right)\]
                    \end{enumerate}
          \end{enumerate}
    \item \begin{enumerate}
              \item If $T$ is one-to-one, then there can only be one $v$ s.t.
                    $T(v)=\vec{0} \therefore \text{Ker}(T)=\{\vec{0}\}$.

                    The other direction is a bit tricker to prove.
                    Assume for the sake of contradiction that $T$ wasn't one-to-one.
                    Then, we have two different vectors $a$ and $b$ s.t. $T(a)=T(b)$.

                    This means that $T(a)-T(b)=0$ and by extension $T(a-b)=0$.
                    However, $a-b \ne \vec{0}$, which contradicts our premise that only $\vec{0}$ is in $\text{Ker}(T)$.
                    Thus, $T$ must be one-to-one if its kernel only has $\vec{0}$. $\square$
              \item If $\text{Im}(T)=W$, then $T$ is onto by definition.
                    We can create any vector in $W$ by transforming a vector in $V$.

                    As for the other direction, we have to prove that $\text{Im}(T) \subseteq W$ and vice versa.
                    The first direction is true by how we defined $T$- it only maps to elements in $W$ and nothing more.
                    As for the second direction, since $T$ is onto, we know that any $w \in W$ has a corresponding $v \in V$ s.t. $T(v) = w$.
                    Thus, anything in $W$ must also be in $\text{Im}(T)$. $\square$
              \item \begin{enumerate}
                        \item For $T$ to be one-to-one, its nullity must be $0$.
                              Then this means $0+\text{rank } T=\dim V \rightarrow \text{rank } T=\dim V$.

                              However, the rank of a transformation cannot be greater than the dimensionality
                              of its codomain, so $T$ cannot be one-to-one if $\dim V > \dim W$. $\square$
                        \item We know that the nullity plus the rank of $T$ has to equal the dimensionality of $V$.
                              For $T$ to be onto, its rank has to be equal to $\dim W$.

                              However, this would imply that the nullity is $\dim V - \dim W < 0$, which can't be possible.
                              Thus, $T$ cannot be onto if $\dim V < \dim W$. $\square$
                        \item If $T$ is one-to-one, then its nullity is $0$.
                              This means that its rank must be equal to the dimension of $W$, which then implies that $T$ is also onto.

                              The proof for the other direction goes much the same way.
                              $T$ being onto means that its rank is equal to its dimension,
                              which means that its nullity is $0$ and that it's also one-to-one. $\square$
                    \end{enumerate}
          \end{enumerate}
\end{enumerate}
\end{document}
