\documentclass[12pt]{article}

% a template that a friend gave, it's worked well enough for me
% i have added some packages and stuff that have proved useful

\usepackage{fancyhdr}
\usepackage{tipa}
\usepackage{fontspec}
\usepackage{amsfonts}
\usepackage{enumitem}
\usepackage[margin=1in]{geometry}
\usepackage{graphicx}
\usepackage{float}
\usepackage{amsmath}
\usepackage{braket}
\usepackage{amssymb}
\usepackage{booktabs}
\usepackage{hyperref}
\usepackage{mathtools}
\usepackage{xcolor}
\usepackage{float}
\usepackage{algpseudocodex}
\usepackage{titlesec}
\usepackage{bbm}

\pagestyle{fancy}
\fancyhf{} % sets both header and footer to nothing
\lhead{Kevin Sheng}
\setmainfont{Comic Neue}
\renewcommand{\headrulewidth}{1pt}
\setlength{\headheight}{0.75in}
\setlength{\oddsidemargin}{0in}
\setlength{\evensidemargin}{0in}
\setlength{\voffset}{-.5in}
\setlength{\headsep}{10pt}
\setlength{\textwidth}{6.5in}
\setlength{\headwidth}{6.5in}
\setlength{\textheight}{8in}
\renewcommand{\headrulewidth}{0.5pt}
\renewcommand{\footrulewidth}{0.3pt}
\setlength{\textwidth}{6.5in}
\usepackage{setspace}
\usepackage{multicol}
\usepackage{float}
\setlength{\columnsep}{1cm}
\setlength\parindent{24pt}
\usepackage [english]{babel}
\usepackage [autostyle, english = american]{csquotes}
\MakeOuterQuote{"}

\setlength{\parskip}{6pt}
\setlength{\parindent}{0pt}

\titlespacing\section{0pt}{12pt plus 4pt minus 2pt}{0pt plus 2pt minus 2pt}
\titlespacing\subsection{0pt}{12pt plus 4pt minus 2pt}{0pt plus 2pt minus 2pt}
\titlespacing\subsubsection{0pt}{12pt plus 4pt minus 2pt}{0pt plus 2pt minus 2pt}

\hypersetup{colorlinks=true, urlcolor=blue}

\newcommand{\correction}[1]{\textcolor{red}{#1}}


\begin{document}
\begin{enumerate}
      \item To prove that any function $T$ is linear, we just have to prove that
            $T(\lambda a+b)=\lambda T(a)+b$ for any matrices $a$ and $b$.
            We know that $\lambda a+b=[\lambda a_{ij}+b_{ij}]$.
            Thus, the $ij$-th element of $T(\lambda a+b)$ is defined by $\lambda a_{ji}+b_{ji}$.

            As for the RHS, the $ij$-th element of $T(a)$ is equal to $a_{ji}$.
            Performing the other operations gives us a result of $ca_{ji}+\lambda$,
            which is equal to the LHS.
            Thus, $T$ is a linear transformation.
      \item \begin{enumerate}
                  \item $\mathbb{Z}/p\mathbb{Z}$ has $p$ elements.
                        $(\mathbb{Z}/p\mathbb{Z})^2$ has $p^2$ elements, and $(\mathbb{Z}/p\mathbb{Z})^3$ has $p^3$ elements.
                  \item We can form any ordered tuple of $i+1$ elements by tacking on another element to the end of an $i$-tuple.
                  \item \textbf{Claim:} $|(\mathbb{Z}/p\mathbb{Z})^n|=p^n$ \\
                        The proof is by induction.

                        \textbf{Base Case:} \\
                        That $|\mathbb{Z}/p\mathbb{Z}|=p$ has already been established.
                        We claim $P_i$, that is, $|(\mathbb{Z}/p\mathbb{Z})^i|=p^i$.

                        \textbf{Inductive step:} \\
                        It remains to prove that $P_i \rightarrow P_{i+1}$.
                        This is simple as $(\mathbb{Z}/p\mathbb{Z})^{i+1}=(\mathbb{Z}/p\mathbb{Z})^i \times \mathbb{Z}/p\mathbb{Z}$,
                        so the size of the LHS is the same as the product of the sizes of the RHS.
                        This gives us $|(\mathbb{Z}/p\mathbb{Z})^{i+1}|=p^i \cdot p=p^{i+1}$,
                        completing the inductive step. $\square$
            \end{enumerate}
      \item Let $V=W=\mathbb{R}^2$, $T(v)=5v$, and $U(v)=3v$.
            While these have the same kernel (just $\vec{0}$) and image (all of $W$),
            they clearly output different results for any non-zero input.
      \item \begin{enumerate}
                  \item Let's start with $(S+T)(\lambda a+b)$ and try to work our way towards $\lambda(S+T)(a)+(S+T)(b)$.
                        \begin{align*}
                              (S+T)(\lambda a+b) & = S(\lambda a+b)+T(\lambda a+b)        \\
                                                 & = \lambda S(a)+S(b)+\lambda T(a)+T(b)  \\
                                                 & =\lambda (S(a)+T(a))+S(b)+T(b)         \\
                                                 & =\lambda (S+T)(a)+(S+T)(b)\quad\square
                        \end{align*}
                  \item \hfill$
                              \begin{aligned}[t]
                                    (\lambda T)(ca+b) & = \lambda \cdot T(ca+b)                     \\
                                                      & = \lambda (cT(a)+T(b))                      \\
                                                      & =c\lambda T(a)+\lambda T(b)                 \\
                                                      & =c(\lambda T)(a)+(\lambda T)(b)\quad\square
                              \end{aligned}$\hfill\null
                  \item \hfill$
                              \begin{aligned}[t]
                                    \vec{0}(\lambda a+b) & = \vec{0}_W                                 \\
                                                         & = \lambda \vec{0}_W+\vec{0}_W               \\
                                                         & = \lambda \vec{0}(a)+\vec{0}(b)\quad\square
                              \end{aligned}$\hfill\null
                  \item \textbf{Commutativity}
                        \begin{align*}
                              (S+T)(x) & = S(x)+T(x) \\
                                       & =T(x)+S(x)  \\
                                       & =(T+S)(x)
                        \end{align*}

                        \textbf{Associativity}
                        \begin{align*}
                              ((S+T)+U)(x) & = (S(x)+T(x))+U(x) \\
                                           & =S(x)+(T(x)+U(x))  \\
                                           & =(S+(T+U))(x)
                        \end{align*}

                        \textbf{Additive Identity} \\
                        Consider $I(x)=\vec{0}$.
                        \begin{align*}
                              (T+I)(x) & = T(x)+I(x)    \\
                                       & = T(x)+\vec{0} \\
                                       & = T(x)
                        \end{align*}

                        \textbf{Additive Inverse} \\
                        For any linear transformation $T(x)$, consider $U(x)=-T(x)$.
                        The negative operation is defined since the codomain of $T$ is a vector space and thus all element in there have additive inverses.
                        \begin{align*}
                              (T+U)(x) & =T(x)+U(x)    \\
                                       & =T(x)+(-T(x)) \\
                                       & =\vec{0}      \\
                                       & = I(x)
                        \end{align*}

                        \textbf{Scalar Identity} \\
                        Consider the element $1 \in F$.
                        \begin{align*}
                              (1 \cdot T)(x) & = 1 \cdot T(x) \\
                                             & =T(x)
                        \end{align*}

                        \textbf{Associativity w.r.t. Scalar Multiplication}
                        \begin{align*}
                              ((ab) \cdot T)(x) & = (ab) \cdot T(x)        \\
                                                & =a \cdot b \cdot T(x)    \\
                                                & = a \cdot (b \cdot T(x)) \\
                                                & =(a(b \cdot T))(x)
                        \end{align*}

                        \textbf{Distributivity P1}
                        \begin{align*}
                              (\lambda \cdot (T+U))(x) & = \lambda \cdot (T+U)(x)        \\
                                                       & = \lambda (T(x)+U(x))           \\
                                                       & = \lambda T(x)+\lambda U(x)     \\
                                                       & = (\lambda T)(x)+(\lambda U)(x)
                        \end{align*}

                        \textbf{Distributivity P2}
                        \begin{align*}
                              ((a+b)T)(x) & = (a+b) \cdot T(x)              \\
                                          & = a \cdot T(x)+b \cdot T(x)     \\
                                          & = (a \cdot T)(x)+(b \cdot T)(x)
                        \end{align*}

                        All the axioms are satisfied. $\square$
            \end{enumerate}
      \item \textbf{Problem 2:} \\
            The null space of $T$ is all ordered triples that have the form $(x,x,0)$.
            Thus, a valid basis for it is $\{(1,1,0)\}$, and the nullity is $1$.

            As for the image of $T$, notice that any ordered pair $(a,b)$ can be expressed as $T((a,0,b))$.
            This means that the rank is the same as the dimension of the codomain, which is $2$.
            Thus, the standard basis $\{(0, 1), (1, 0)\}$ works as basis for $\text{Im}(T)$.

            $1+2=3=\dim \mathbb{R}^3$, so the dimension theorem holds true as it should.

            Although $T$ is not one-to-one ($T((1,1,0))=T((2,2,0))$), it is onto,
            as the basis of the image of $T$ is also a basis for its codomain.

            \textbf{Problem 6:} \\
            For matrix to have a trace of zero, the sum the elements must be the zero element.
            The basis for this null space first consists of all elementary matrices that don't have their element along the diagonal.

            It also has $n-1$ matrices that can be defined as follows:
            \begin{center}
                  The $i$th matrix in this list is all zeros except for a $1$ in $a_{ii}$ and a $-1$ in $a_{nn}$
            \end{center}
            Thus, the nullity of $T$ is $n^2-1$, since we have a matrix for every element except for the very last one in the bottom right corner.

            The image of $T$ is a single element in $F$, so a basis for that is just $\{1\}$ and its rank is $1$.

            $n^2-1+1=n^2$, which is the dimension of the set of all $n \times n$ matrices, so the dimension theorem holds.

            The trace function is not one-to-one.
            To prove this, consider these two matrices:
            \[\begin{bmatrix} 1 & 0 \\ 0 & -1 \end{bmatrix},
                  \begin{bmatrix} 0 & 0 \\ 0 & 0 \end{bmatrix}\]
            Both of these matrices have a trace of $0$.
            However, it is onto, as the rank matches the dimension of the codomain.
      \item \begin{enumerate}
                  \item We know that the nullity plus the rank of $T$ has to equal the dimensionality of $V$.
                        For $T$ to be onto, its rank has to be equal to $\dim W$.
                        However, this would imply that the nullity is $\dim V - \dim W < 0$, which can't be possible.
                        Thus, $T$ cannot be onto if $\dim V < \dim W$. $\square$
                  \item For $T$ to be one-to-one, its nullity must be $0$.
                        Then this means $0+\text{rank } T=\dim V \rightarrow \text{rank } T=\dim V$.
                        However, the rank of a transformation cannot be greater than the dimensionality
                        of its codomain, so $T$ cannot be one-to-one. $\square$
            \end{enumerate}
      \item \begin{enumerate}
                  \item We just have to prove that $T(\lambda a+b)=\lambda T(a)+T(b)$, and likewise for $U$.
                        \begin{align*}
                              T(\lambda a+b) & =T(\lambda \cdot (a_1, a_2, \cdots) + (b_1, b_2, \cdots)) \\
                                             & =(\lambda a_2+b_2, \lambda a_3+b_3, \cdots)               \\
                                             & =(\lambda a_2, \lambda a_3, \cdots)+(b_2, b_3, \cdots)    \\
                                             & =\lambda (a_2, a_3, \cdots)+(b_2, b_3, \cdots)            \\
                                             & =\lambda T(a)+T(b)
                        \end{align*}
                        \begin{align*}
                              U(\lambda a+b) & =U(\lambda \cdot (0, a_1, a_2, \cdots) + (0, b_1, b_2, \cdots)) \\
                                             & =(0, \lambda a_1+b_1, \lambda a_2+b_2, \cdots)                  \\
                                             & =(0, \lambda a_1, \lambda a_2, \cdots)+(0, b_1, b_2, \cdots)    \\
                                             & =\lambda (0, a_1, a_2, \cdots)+(0, b_1, b_2, \cdots)            \\
                                             & =\lambda U(a)+U(b)
                        \end{align*}
                  \item $T$ is onto.
                        We can create any sequence $(a_1, a_2, \cdots)$ by calling $T(0, a_1, a_2, \cdots)$.

                        However, $T$ is not one-to-one.
                        Consider the sequences $(2,1,1, \cdots)$ and \\ $(1,1,1, \cdots)$.
                        When left-shifted, they both result in an infinite sequence of $1$s.

                  \item $U$ is one-to-one.
                        If two infinite sequences differ at some position $n$, then the right-shifted versions
                        of them also differ at the position $n+1$.

                        However, it's not onto.
                        No sequence that starts with a nonzero number can come out of $U$,
                        as its first element will always be $0$.
            \end{enumerate}
      \item We first have to prove that all elements $n$ in the kernel of $T$ satisfy $T(n+s)=b$.
            This part is pretty straightfoward, as we have
            \[T(s+n)=T(s)+T(n)=b+\vec{0}=b\]

            Next, we have to prove that if $T(x)=b$, then $x=s+n$ for some $n \in \text{Ker}(T)$.
            That means that $n=x-s \notin \text{Ker}(T)$.
            This would mean that
            \begin{gather*}
                  T(x)=T(s+n) \\
                  T(x)=T(s)+T(n)
            \end{gather*}
            However, since $T(n) \ne \vec{0}$, this would mean that $T(x) \ne T(s)$, which is a contradiction.

            Thus, we know that $K=\{s\}+\text{Ker}(T)$. $\square$
      \item \begin{enumerate}
                  \item Since $\mathcal{B}$ is a basis, any vector $v \in V$ can be written as the following:
                        \[v=\sum_{i=1}^n a_iv_i \rightarrow T(v)=\sum_{i=1}^n a_i T(v_i)\]
                        \textbf{One-to-one} \\
                        If two vectors are different, then they must come out of different linear combinations of $\mathcal{B}$.
                        This means that among their coefficients $(a_1, a_2, \cdots a_n)$ and $(b_1, b_2, \cdots b_n)$,
                        $\exists i: a_i \ne b_i$.

                        When transformed, we have the two following linear combinations:
                        \begin{align*}
                               & \sum_{i=1}^n a_i w_i &  & \sum_{i=1}^{n} b_i w_i
                        \end{align*}
                        Since we didn't change the coefficient at all (just the vectors they're attached to),
                        these two linear combinations are also different.
                        Given that $\mathcal{G}$ is a basis, any $w \in W$ can be written \textit{uniquely}
                        as a linear combination of elements in the basis.

                        From this and that the two combinations have different coefficients, we can now safely
                        say that two different vectors stay different after being transformed by $T$
                        and by extension that $T$ is one-to-one.

                        \textbf{Onto} \\
                        Any vector in $W$ can be written using elements from $\mathcal{G}$,
                        which can then be turned into elements from $\mathcal{B}$ using $T$:
                        \[w=\sum_{i=1}^n a_iw_i =\sum_{i=1}^n a_i T(v_i)=T(v)\]

                  \item \[T\left(a_3x^3+a_2x^2+a_1x+a_0\right)=\begin{bmatrix}
                                    a_3 & a_2 \\
                                    a_1 & a_0
                              \end{bmatrix}\]
            \end{enumerate}
\end{enumerate}
\end{document}
