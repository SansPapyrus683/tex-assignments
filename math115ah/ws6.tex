\documentclass[12pt]{article}

% a template that a friend gave, it's worked well enough for me
% i have added some packages and stuff that have proved useful

\usepackage{fancyhdr}
\usepackage{tipa}
\usepackage{fontspec}
\usepackage{amsfonts}
\usepackage{enumitem}
\usepackage[margin=1in]{geometry}
\usepackage{graphicx}
\usepackage{float}
\usepackage{amsmath}
\usepackage{braket}
\usepackage{amssymb}
\usepackage{booktabs}
\usepackage{hyperref}
\usepackage{mathtools}
\usepackage{xcolor}
\usepackage{float}
\usepackage{algpseudocodex}
\usepackage{titlesec}
\usepackage{bbm}

\pagestyle{fancy}
\fancyhf{} % sets both header and footer to nothing
\lhead{Kevin Sheng}
\setmainfont{Comic Neue}
\renewcommand{\headrulewidth}{1pt}
\setlength{\headheight}{0.75in}
\setlength{\oddsidemargin}{0in}
\setlength{\evensidemargin}{0in}
\setlength{\voffset}{-.5in}
\setlength{\headsep}{10pt}
\setlength{\textwidth}{6.5in}
\setlength{\headwidth}{6.5in}
\setlength{\textheight}{8in}
\renewcommand{\headrulewidth}{0.5pt}
\renewcommand{\footrulewidth}{0.3pt}
\setlength{\textwidth}{6.5in}
\usepackage{setspace}
\usepackage{multicol}
\usepackage{float}
\setlength{\columnsep}{1cm}
\setlength\parindent{24pt}
\usepackage [english]{babel}
\usepackage [autostyle, english = american]{csquotes}
\MakeOuterQuote{"}

\setlength{\parskip}{6pt}
\setlength{\parindent}{0pt}

\titlespacing\section{0pt}{12pt plus 4pt minus 2pt}{0pt plus 2pt minus 2pt}
\titlespacing\subsection{0pt}{12pt plus 4pt minus 2pt}{0pt plus 2pt minus 2pt}
\titlespacing\subsubsection{0pt}{12pt plus 4pt minus 2pt}{0pt plus 2pt minus 2pt}

\hypersetup{colorlinks=true, urlcolor=blue}

\newcommand{\correction}[1]{\textcolor{red}{#1}}


\begin{document}
\begin{enumerate}
      \item \begin{enumerate}
                  \item We're proving two matrices are equal.
                        To do this, we have to show that all the elements of the matrices are equal.
                  \item $S \circ T$ is a function from $U$ to $W$, and thus it has $l$ columns, each of length $n$.
                        Following the same logic, $[S]^\delta_\gamma$ is an $n \times m$ matrix and $[T]^\gamma_\beta$ is a $m \times l$ matrix.
                        Multiplying an $n \times m$ matrix by an $m \times l$ one results in an $n \times l$ matrix, which matches up.
                  \item Two column vectors are equal iff all of their elements are equal.
                        If the column vectors of two matrices are equal, this means that their elements are equal as well.
                  \item We start with the RHS and use the formulas to work our way towards the RHS.
                        \begin{align*}
                              \left([S]^\delta_\gamma \cdot [T]^\gamma_\beta\right)[u]_\beta & = [S]^\delta_\gamma\left([T]^\gamma_\beta[u]_\beta\right) \\
                                                                                             & = [S]^\delta_\gamma [T(u)]_\gamma                         \\
                                                                                             & = [S(T(u))]_\delta                                        \\
                                                                                             & = [(S \circ T)(u)]_\delta                                 \\
                                                                                             & = [S \circ T]^\delta_\beta [u]_\beta \quad\square
                        \end{align*}
                  \item The dimensions of $Ae_y$ are $n \times 1$, which matches up with the dimensions for a column.
                        Let the result of the multiplication be $c$. Then
                        \[c_{ij}=\sum_{x=1}^l A_{ix}(e_y)_{xj}\]
                        Notice that $j=1$ always and thus $(e_y)_{xj}$ is $1$ if $x=y$ and $0$ otherwise.

                        We can simplify this equation to $c_{ij}=A_{iy}$,
                        which by inspection is the same as taking the $y$th column of $A$. $\square$
                  \item The linear combination for $u_i$ is $\sum_{j=1}^{l} a_j u_j$, where $a_j$ is $0$ if $j=i$ and $0$ otherwise.
                        When we turn this tuple of $a_j$s into a column vector, $e_i$ is exactly what we get.
                  \item Take any $u_i \in \beta$. We know that $[u_i]_\beta=e_i$, which when plugged into our previously derived equation gives us
                        \[[S \circ T]^\delta_\beta e_i = \left([S]^\delta_\gamma \cdot [T]^\gamma_\beta\right)e_i\]
                        Multiplying any matrix by $e_i$ gives us the $i$th column, so from this equality
                        we can deduce that the $i$th columns of the LHS matrix and RHS matrix are the same.
                        This equality is true for any $i$, so all the columns and by extension the matrices themselves must be equal.
                        From this, we finally have that
                        \[[S \circ T]^\delta_\beta = [S]^\delta_\gamma \cdot [T]^\gamma_\beta\quad\square\]
            \end{enumerate}
            \pagebreak
      \item \begin{enumerate}
                  \item \textbf{Linear:} \begin{align*}
                              1_V(a+\lambda b) & =a+\lambda b           \\
                                               & =1_V(a)+\lambda 1_V(b)
                        \end{align*}

                        \textbf{Invertible:} \\
                        The inverse of $1_V$ is itself, as $1_V(1_V(a))=a$.
                  \item $1_V(v_i)=v_i$.
                        To write this uniquely as a linear combination of $\beta$, we can do this:
                        \[v_i=\sum_{j=1}^{n} \mathrm{1}_{j=i}v_j\]
                        Where $\mathrm{1}_{j=i}$ is an indicator function that returns $1$ if $j=i$ and $0$ otherwise.

                        Concatenating the coefficients of these linear combos into a matrix, we get $I_n$.
                  \item $T$ being invertible means that there's a $T^{-1}$ s.t. $T \circ T^{-1}=1_V$ and $T^{-1} \circ T=1_W$.
                  By extension, we can say that
                  \[L_{[T]^\gamma_\beta} \circ L_{[T^{-1}]^\beta_\gamma}=1_V=L_{I_n}\]
                  Since $L_a \circ L_b=L_{ab}$, $[T]^\gamma_\beta \cdot [T^{-1}]^\beta_\gamma=I_n$ and thus $[T]^\gamma_\beta$ is invertible.
            \end{enumerate}
\end{enumerate}
\end{document}
