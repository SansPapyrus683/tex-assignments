\documentclass[12pt]{article}

% a template that a friend gave, it's worked well enough for me
% i have added some packages and stuff that have proved useful

\usepackage{fancyhdr}
\usepackage{tipa}
\usepackage{fontspec}
\usepackage{amsfonts}
\usepackage{enumitem}
\usepackage[margin=1in]{geometry}
\usepackage{graphicx}
\usepackage{float}
\usepackage{amsmath}
\usepackage{braket}
\usepackage{amssymb}
\usepackage{booktabs}
\usepackage{hyperref}
\usepackage{mathtools}
\usepackage{xcolor}
\usepackage{float}
\usepackage{algpseudocodex}
\usepackage{titlesec}
\usepackage{bbm}

\pagestyle{fancy}
\fancyhf{} % sets both header and footer to nothing
\lhead{Kevin Sheng}
\setmainfont{Comic Neue}
\renewcommand{\headrulewidth}{1pt}
\setlength{\headheight}{0.75in}
\setlength{\oddsidemargin}{0in}
\setlength{\evensidemargin}{0in}
\setlength{\voffset}{-.5in}
\setlength{\headsep}{10pt}
\setlength{\textwidth}{6.5in}
\setlength{\headwidth}{6.5in}
\setlength{\textheight}{8in}
\renewcommand{\headrulewidth}{0.5pt}
\renewcommand{\footrulewidth}{0.3pt}
\setlength{\textwidth}{6.5in}
\usepackage{setspace}
\usepackage{multicol}
\usepackage{float}
\setlength{\columnsep}{1cm}
\setlength\parindent{24pt}
\usepackage [english]{babel}
\usepackage [autostyle, english = american]{csquotes}
\MakeOuterQuote{"}

\setlength{\parskip}{6pt}
\setlength{\parindent}{0pt}

\titlespacing\section{0pt}{12pt plus 4pt minus 2pt}{0pt plus 2pt minus 2pt}
\titlespacing\subsection{0pt}{12pt plus 4pt minus 2pt}{0pt plus 2pt minus 2pt}
\titlespacing\subsubsection{0pt}{12pt plus 4pt minus 2pt}{0pt plus 2pt minus 2pt}

\hypersetup{colorlinks=true, urlcolor=blue}

\newcommand{\correction}[1]{\textcolor{red}{#1}}


\begin{document}
\begin{enumerate}
      \item \begin{enumerate}
                  \item These polynomials correspond to the vectors $[1,2,1]$, $[3,0,2]$, and $[0,1,1]$.
                        \begin{align*}
                              \begin{bmatrix}
                                    1 & 3 & 0 & 0 \\
                                    2 & 0 & 1 & 0 \\
                                    1 & 2 & 1 & 0
                              \end{bmatrix}
                               & \rightarrow \begin{bmatrix}
                                                   2 & 0 & 1 & 0 \\
                                                   1 & 3 & 0 & 0 \\
                                                   1 & 2 & 1 & 0
                                             \end{bmatrix}           \\
                               & \rightarrow \begin{bmatrix}
                                                   2 & 0 & 1           & 0 \\
                                                   1 & 3 & 0           & 0 \\
                                                   0 & 0 & \frac{5}{6} & 0
                                             \end{bmatrix}  \\
                               & \rightarrow \begin{bmatrix}
                                                   2 & 0 & 1            & 0 \\
                                                   0 & 3 & -\frac{1}{2} & 0 \\
                                                   0 & 0 & \frac{5}{6}  & 0
                                             \end{bmatrix}
                        \end{align*}
                        As the resulting matrix has $3$ pivots, these polynomials are linearly independent.
                  \item $F^2$:
                        \begin{enumerate}
                              \item $(0, 1)$ and $(1, 0)$
                              \item $(1, 2)$ and $(2, 1)$
                              \item $(0, 7)$ and $(1, 5)$
                        \end{enumerate}
                        $M_{2 \times 2}(F)$:
                        \begin{enumerate}
                              \item $\begin{bmatrix}1 & 0 \\ 0 & 0\end{bmatrix}$,
                                    $\begin{bmatrix}0 & 1 \\ 0 & 0\end{bmatrix}$,
                                    $\begin{bmatrix}0 & 0 \\ 1 & 0\end{bmatrix}$, and
                                    $\begin{bmatrix}0 & 0 \\ 0 & 1\end{bmatrix}$
                              \item $\begin{bmatrix}1 & 1 \\ 0 & 0\end{bmatrix}$,
                                    $\begin{bmatrix}1 & 0 \\ 1 & 0\end{bmatrix}$,
                                    $\begin{bmatrix}1 & 0 \\ 0 & 1\end{bmatrix}$, and
                                    $\begin{bmatrix}1 & 0 \\ 0 & 0\end{bmatrix}$
                              \item $\begin{bmatrix}1 & 1 \\ 1 & 0\end{bmatrix}$,
                                    $\begin{bmatrix}1 & 1 \\ 0 & 1\end{bmatrix}$,
                                    $\begin{bmatrix}1 & 0 \\ 1 & 1\end{bmatrix}$, and
                                    $\begin{bmatrix}0 & 1 \\ 1 & 1\end{bmatrix}$
                        \end{enumerate}
                  \item One basis for all upper triangular matrices is the set of all matrices
                        such that they have exactly one nonzero element, and the element's position $ij$ satisfies $i \le j$.
            \end{enumerate}
      \item Let $u_2'=u_2-\lambda u_1$ for convenience.
            \begin{enumerate}
                  \item It suffices to prove a bijection between the solutions of
                        $\sum_{i=1}^n a_i u_i=\vec{0}$ and the solutions of that equation but with $S'$.
                        For any solution to the equation in the old set, we can transform it into a solution for the new
                        set by replacing $a_1$ with $a_1+a_2\lambda$.

                        This works because \[a_1u_1+a_2u_2=(a_1+a_2\lambda)u_1+a_2u_2'\]

                        It's clear that this transformation is injective, so we just have to prove surjectivity.
                        Notice that we can transform any solution to the equation for $S'$ back to a solution for $S$
                        as well by replacing $a_1$ with $a_1-a_2\lambda$, which results in a similar equality:
                        \[a_1u_1+a_2u_2'=(a_1-a_2\lambda)u_1+a_2u_2\]

                        If $S$ is linearly independent, the only set of solutions to the equation is $a_i=0$.
                        Transforming this set of solutions also yields $a_i=0$.
                        Thus, the only set of solutions to $\sum_{i=1}^n a_i u_i$ for $S'$ is $a_i=0$ as well. $\square$
                  \item Say we have a linear combination of $S$ as $\sum_{i=1}^n a_i u_i$. \\
                        % TODO: prove inclusion on both sides
                        Notice that $a_2u_2=a_2u_2'+a_1\lambda u_1$, so any linear combo of $S$ can be expressed in terms of $S'$.
                        Thus, the two sets have the same span. $\square$
            \end{enumerate}
      \item If the set is linearly independent, then we know that it forms a basis for $F^n$ since it has $n$ elements.

            Notice that this set, when put in a matrix, becomes a matrix with just $1$s along the diagonal.
            This matrix is already in reduced row echelon form and has $n$ pivots, meaning that the set
            is linearly independent and by extension forms a basis for $F^n$. $\square$
      \item \begin{enumerate}
                  \item First we have to establish that $U_1 \cap U_2=\{\vec{0}\}$.
                        This can be done through noticing that any element in the intersection has to meet the condition $a_1=a_2=a_3=0$,
                        and the only element that fulfills this condition is $\vec{0}$.

                        Next we must demonstrate that any $(x,y,z) \in F^3$ can be expressed as a sum of an element in $U_1$ and another in $U_2$.
                        This can be done by considering $(x,0,0) \in U_1$ and $(0,y,z) \in U_2$, as they meet the requirements. $\square$
                  \item $A \rightarrow B$: \\
                        We're given that every vector in $V$ can be written uniquely as a sum of a vector from $U_1$ and $U_2$.
                        Thus, we know that $V=U_1+U_2$.
                        It remains to prove that $U_1 \cap U_2=\{\vec{0}\}$. \\
                        To prove this, suppose that their intersection had something other than $\vec{0}$.
                        Call this vector $x$.
                        Then, we can express $x \in V$ with $u_1=\vec{0}, u_2=x$ and $u_1=x, u_2=\vec{0}$.
                        This contradicts the uniqueness condition.
                        Thus, $V=U_1 \oplus U_2$.

                        $B \rightarrow A$: \\
                        If $V=U_1 \oplus U_2$, we know that every $v \in V$ has at least one pair of vectors in $u_1$ and $u_2$ that sum to it.
                        What we have to prove is that this pair is always unique. \\
                        Say we have $u_1, u_1' \in U_1$ and $u_2, u_2' \in U_2$ s.t. $u_1+u_2=u_1'+u_2'=v$.
                        We then get that $u_1-u_1'=u_2-u_2'$.
                        Since vector spaces are closed under addition, we get that
                        whatever $u_1-u_1'$ and $u_2-u_2'$ are equal to has to be in both $U_1$ and $U_2$.
                        However, the only element that's in both sets is $\vec{0}$, and so $u_1=u_1'$
                        and $u_2=u_2'$. $\square$
            \end{enumerate}
      \item \begin{enumerate}
                  \item First, let's prove that any $T(v)$ can be expressed as a linear combination of $T(s)$ for the elements in $S$.
                        We know that $\text{span}(S)=V$, so any $v$ is expressable in terms of the formula $\sum_{i=1}^n a_i s_i$.

                        Thus, \[T(v)=T\left(\sum_{i=1}^n a_i s_i\right)=\sum_{i=1}^n T(a_i s_i)=\sum_{i=1}^n a_i T(s_i)\]

                        We also have to prove that any linear combination of $T(s)$ can be expressed as $T(v)$.
                        The proof for this goes much the same way:
                        \[\sum_{i=1}^n a_i T(s_i)=\sum_{i=1}^n T(a_i s_i)=T\left(\sum_{i=1}^n a_i s_i\right)\]
                        We end up right back where we started. $\square$
                  \item We know that the span of any subset is automatically a subspace, and we just proved that $\text{Im}(T)$ is equivalent to a span.
                        Thus, $\text{Im}(T)$ is a subspace of $W$.
            \end{enumerate}
      \item \begin{enumerate}
                  \item \textbf{Base case $n=1$:} \[\sum_{j=1}^1 j^3=1\]
                        Given this, our inductive hypothesis is \[\sum_{j=1}^n=\frac{n^2(n+1)^2}{4}\]
                        and we have to prove that \[\sum_{j=1}^{n+1}=\frac{(n+1)^2(n+2)^2}{4}\]

                        Adding $(n+1)^3$ to both sides, we have \begin{gather*}
                              \sum_{j=1}^n j^3+(n+1)^3 =\frac{n^2(n+1)^2}{4}+(n+1)^3 \\
                              \sum_{j=1}^{n+1} j^3=(n+1)^2\left(\frac{n^2}{4}+(n+1)\right) \\
                              \sum_{j=1}^{n+1} j^3=(n+1)^2 \cdot \frac{n^2+4n+4}{4} \\
                              \text{''}=(n+1)^2 \cdot \frac{(n+2)^2}{4}\quad\square
                        \end{gather*}
                  \item \[\left(\sum_{j=1}^n j\right)^2=\left(\frac{n(n+1)}{2}\right)^2=\sum_{j=1}^{n} j^3\]
            \end{enumerate}
      \item \begin{enumerate}
                  \item Since $\dim V=n$, we know that all bases for $V$ have $n$ elements.
                        Also, since these bases are all linearly independent, by the replacement theorem
                        any spanning set must have at least $n$ elements. $\square$
                  \item Suppose for a contradiction that $S$ wasn't a basis.
                        This implies that $S$ is LD, which furthermore implies that there is some element $v \in S$
                        that can be expressed as a linear combination of the other elements in $S$.

                        Then, we know that $S \\ \{v\}$ spans $V$.
                        However, the size of this set is $n-1$, which contradicts that $\dim V=n$.
                        Thus, $S$ must be a basis. $\square$ \label{list:n-elem}
                  \item Let us show that we can construct a basis from any spanning subset $G$.

                        First off, we know that any set with greater than $n$ elements has to be LD
                        by the previous problem.
                        When we add another vector to an $n$-element set, either the set is already LD
                        or the set is LI and we can write this vector in terms of a linear combination of elements from the set,
                        which makes the new set LD.

                        Thus, while $G$ has greater than $n$ elements, we know that it must be linearly dependent in some manner.
                        This means that the equation $\sum_{i=1}^n a_i v_i=\vec{0}$ must have some non-trivial set of solutions,
                        which also implies that one of the vectors in $G$ can be expressed with a linear combination of the other elements.

                        We can then remove this vector without impacting the span of $G$ in any way.
                        We do this until $G$ has $n$ elements, which makes it a basis by the previous problem.
                        \label{list:construct}
            \end{enumerate}
      \item This follows naturally from the replacement theorem, since we know
            that all spanning sets are at least as large as any linearly independent subset.
            As long as one infinitely large LI subset exists, every spanning subset must be infinitely large as well.
      \item \begin{enumerate}
                  \item If the basis for $U$ was hypothetically larger than that of $V$,
                        then we would have an LI set of vectors of at least size $\dim V +1$ that are in $U$ and by extension $V$.
                        However, no set of size greater than $\dim V$ can be LI (see \ref{list:construct}).
                        Contradiction. Thus, $\dim U \le \dim V$. \label{list:dim}
                  \item A basis for $U$ has $\dim U=\dim V$ elements.
                        Since all elements in this basis are also in $V$, and these vectors are LI,
                        then by the result in \ref{list:n-elem} we know that this must also form a basis for $V$.

                        Now we know that any basis $B$ of $U$ is a basis for $V$ if $\dim U=\dim V$.
                        We know that $U$ is a subset of $V$, so now it remains to prove that $V$ is a subset of $U$ for equality.
                        Any vector $v \in V$ can be expressed by a combination of elements in $B$,
                        and since $U=\text{span}(B)$, this means $v \in U$ as well. $\square$
            \end{enumerate}
      \item \begin{enumerate}
                  \item $W_1 \cap W_2$ is a subspace of $W_2$, and since $\dim W_2=n$, by the result in problem \ref{list:dim} we know that $\dim(W_1 \cap W_2) \le n$. $\square$
                  \item Take any vector $v=w_1+w_2$, where $w_1 \in W_1$ and $w_2 \in W_2$.
                        Any $w_1$ can be expressed as the linear combination of no more than $m$ elements from $W_1$
                        and any $w_2$ can be expressed as the combo of no more than $n$ elements from $W_2$.

                        Having this, we know that any vector $v \in W_1+W_2$ can be expressed as the combo
                        of no more than $n+m$ elements from $W_1+W_2$ (this is because $W_1, W_2 \subseteq W_1+W_2$).
                        By extension, any basis for $W_1+W_2$ cannot have more than $n+m$ elements, so $\dim(W_1+W_2) \le m+n$. $\square$
            \end{enumerate}
      \item All elements in $\beta_1$ have to be in $W_1$, and the same holds for $\beta_2$.
            Since $W_1 \cap W_2=\{\vec{0}\}$ and no basis can have the zero vector, $\beta_1 \cap \beta_2=\varnothing$.

            % TODO: you forgot to show that shit's LI
            Since $V=W_1 \oplus W_2$, any $v \in V$ can be written as $w_1+w_2$,
            which can then be expressed as the sum of two linear combinations of the elements of $\beta_1$ and $\beta_2$ respectively.
            Addition of two linear combinations makes another combo, so we know that $\beta_1 \cup \beta_2$ makes a basis for $V$.
\end{enumerate}
\end{document}
