\documentclass[12pt]{article}

% a template that a friend gave, it's worked well enough for me
% i have added some packages and stuff that have proved useful

\usepackage{fancyhdr}
\usepackage{tipa}
\usepackage{fontspec}
\usepackage{amsfonts}
\usepackage{enumitem}
\usepackage[margin=1in]{geometry}
\usepackage{graphicx}
\usepackage{float}
\usepackage{amsmath}
\usepackage{braket}
\usepackage{amssymb}
\usepackage{booktabs}
\usepackage{hyperref}
\usepackage{mathtools}
\usepackage{xcolor}
\usepackage{float}
\usepackage{algpseudocodex}
\usepackage{titlesec}
\usepackage{bbm}

\pagestyle{fancy}
\fancyhf{} % sets both header and footer to nothing
\lhead{Kevin Sheng}
\setmainfont{Comic Neue}
\renewcommand{\headrulewidth}{1pt}
\setlength{\headheight}{0.75in}
\setlength{\oddsidemargin}{0in}
\setlength{\evensidemargin}{0in}
\setlength{\voffset}{-.5in}
\setlength{\headsep}{10pt}
\setlength{\textwidth}{6.5in}
\setlength{\headwidth}{6.5in}
\setlength{\textheight}{8in}
\renewcommand{\headrulewidth}{0.5pt}
\renewcommand{\footrulewidth}{0.3pt}
\setlength{\textwidth}{6.5in}
\usepackage{setspace}
\usepackage{multicol}
\usepackage{float}
\setlength{\columnsep}{1cm}
\setlength\parindent{24pt}
\usepackage [english]{babel}
\usepackage [autostyle, english = american]{csquotes}
\MakeOuterQuote{"}

\setlength{\parskip}{6pt}
\setlength{\parindent}{0pt}

\titlespacing\section{0pt}{12pt plus 4pt minus 2pt}{0pt plus 2pt minus 2pt}
\titlespacing\subsection{0pt}{12pt plus 4pt minus 2pt}{0pt plus 2pt minus 2pt}
\titlespacing\subsubsection{0pt}{12pt plus 4pt minus 2pt}{0pt plus 2pt minus 2pt}

\hypersetup{colorlinks=true, urlcolor=blue}

\newcommand{\correction}[1]{\textcolor{red}{#1}}


\begin{document}
\begin{enumerate}
    \item \begin{enumerate}
              \item Since $z \in V$, we know that there is an additive inverse $-z$ s.t. $z+(-z)=0$.
                    Given this, we can add $-z$ to both sides like so:
                    \begin{align*}
                        x+z+(-z)  & =y+z+(-z)      \\
                        x+\vec{0} & =y+\vec{0}     \\
                        x         & =y\quad\square
                    \end{align*}
              \item \begin{itemize}
                        \item Say we have a $\vec{0}$ and $\vec{0}'$ s.t. $x+\vec{0}=x$ and $x+\vec{0}'=x$.
                              Then $x+\vec{0}=x+\vec{0}'$ and by the cancellation law for vector addition $\vec{0}=\vec{0}'$. $\square$
                        \item Assume for sake of contradiction that $0 \cdot v \ne \vec{0}$.
                              Then $(0+c)v$ would evaluate to either $cv$ or $0v+cv$, depending on if we evaluated the addition
                              or multiplication operation first, which is a contradiction.
                              Thus, $0 \cdot v = \vec{0}$. $\square$
                        \item \hfill$\begin{gathered}[t]
                                      (1-1)v=0\cdot v=\vec{0} \\
                                      1 \cdot v + (-1) \cdot v = (1-1) \cdot v=\vec{0}
                                  \end{gathered}$\hfill\null \\
                              By the definition of the additive inverse, $-v=(-1) \cdot v$. $\square$
                    \end{itemize}
          \end{enumerate}
    \item \begin{enumerate}
              \item \hfill$\begin{gathered}[t]
                            [v]=\{x \in V\ |\ x-v \in U\} \\
                            V/U=\{\{b \in V\ |\ a-b \in U\}\ |\ a \in V\}
                        \end{gathered}$\hfill\null
              \item For any $(x,y)$ we know that $(x,0) \sim (x,y)$ since their difference $(0,0) \in U$.

                    To prove uniqueness, suppose there's an $a$ and $a'$ s.t.
                    $(x,y) \sim_R (a,0)$ and $~{(xy) \sim_R (a',0)}$.
                    Then, by the transitive property, $(a,0) \sim_R (a',0)$.
                    For this to be true, $a-a'=0 \therefore a=a'$. $\square$
              \item \begin{enumerate}[label=\arabic*]
                        \item \textbf{Commutativity}
                              \begin{align*}
                                  [a]+[b] & =[a +_V b] \\
                                          & =[b +_V a] \\
                                          & =[b]+[a]
                              \end{align*}
                        \item \textbf{Associativity}
                              \begin{align*}
                                  ([a]+[b])+[c] & =([a +_V b])+[c]     \\
                                                & =([(a +_V b) +_V c]) \\
                                                & =([a +_V (b +_V c)]) \\
                                                & =[a]+([b +_V c])     \\
                                                & =[a]+([b]+[c])
                              \end{align*}
                        \item \textbf{Additive Identity}
                              \[[a]+[\vec{0}] = [a +_V \vec{0}] = [a]\]
                        \item \textbf{Additive Inverse} \\
                              For any $a \in V$, let $-a$ denote the additive inverse of $a$ in $V$.
                              \[[a]+[-a] = [a +_V (-a)]=[\vec{0}]\]
                        \item \textbf{Scalar Identity}
                              \[1 \cdot [b] = [1 \cdot_V b] = [b]\]
                        \item \textbf{Associativity w.r.t. Scalar Multiplication}
                              \begin{align*}
                                  (xy)[a] & = [(xy) \cdot_V a]     \\
                                          & =[x(y \cdot_V a)]      \\
                                          & =x \cdot [y \cdot_V a] \\
                                          & =x(y[a])
                              \end{align*}
                        \item \textbf{Distributivity P1}
                              \begin{align*}
                                  x([a]+[b]) & =x \cdot [a +_V b] \\
                                             & =[x(a +_V b)]      \\
                                             & =[xa +_V xb]       \\
                                             & =x[a]+x[b]
                              \end{align*}
                        \item \textbf{Distributivity P2}
                              \begin{align*}
                                  (x+y)[a] & =[(x+y) \cdot_V a] \\
                                           & =[xa +_V ya]       \\
                                           & =x[a]+y[b]
                              \end{align*}
                    \end{enumerate}
          \end{enumerate}
\end{enumerate}
\end{document}
