\documentclass[12pt]{article}

% a template that a friend gave, it's worked well enough for me
% i have added some packages and stuff that have proved useful

\usepackage{fancyhdr}
\usepackage{tipa}
\usepackage{fontspec}
\usepackage{amsfonts}
\usepackage{enumitem}
\usepackage[margin=1in]{geometry}
\usepackage{graphicx}
\usepackage{float}
\usepackage{amsmath}
\usepackage{braket}
\usepackage{amssymb}
\usepackage{booktabs}
\usepackage{hyperref}
\usepackage{mathtools}
\usepackage{xcolor}
\usepackage{float}
\usepackage{algpseudocodex}
\usepackage{titlesec}
\usepackage{bbm}

\pagestyle{fancy}
\fancyhf{} % sets both header and footer to nothing
\lhead{Kevin Sheng}
\setmainfont{Comic Neue}
\renewcommand{\headrulewidth}{1pt}
\setlength{\headheight}{0.75in}
\setlength{\oddsidemargin}{0in}
\setlength{\evensidemargin}{0in}
\setlength{\voffset}{-.5in}
\setlength{\headsep}{10pt}
\setlength{\textwidth}{6.5in}
\setlength{\headwidth}{6.5in}
\setlength{\textheight}{8in}
\renewcommand{\headrulewidth}{0.5pt}
\renewcommand{\footrulewidth}{0.3pt}
\setlength{\textwidth}{6.5in}
\usepackage{setspace}
\usepackage{multicol}
\usepackage{float}
\setlength{\columnsep}{1cm}
\setlength\parindent{24pt}
\usepackage [english]{babel}
\usepackage [autostyle, english = american]{csquotes}
\MakeOuterQuote{"}

\setlength{\parskip}{6pt}
\setlength{\parindent}{0pt}

\titlespacing\section{0pt}{12pt plus 4pt minus 2pt}{0pt plus 2pt minus 2pt}
\titlespacing\subsection{0pt}{12pt plus 4pt minus 2pt}{0pt plus 2pt minus 2pt}
\titlespacing\subsubsection{0pt}{12pt plus 4pt minus 2pt}{0pt plus 2pt minus 2pt}

\hypersetup{colorlinks=true, urlcolor=blue}

\newcommand{\correction}[1]{\textcolor{red}{#1}}


\begin{document}
\begin{enumerate}
      \item For all of these problems, we'll let $\beta$ be the standard basis for simplicity.
            \begin{enumerate}
                  \item \[[T]_\beta=\begin{bmatrix}
                                    T((1, 0)) & T((0, 1))
                              \end{bmatrix}=\begin{bmatrix}
                                    1 & 0 \\
                                    1 & 0
                              \end{bmatrix}\]
                        and we get our characteristic polynomial from
                        \begin{align*}
                              \det \left(\begin{bmatrix}
                                               t & 0 \\
                                               0 & t
                                         \end{bmatrix}-\begin{bmatrix}
                                                             1 & 0 \\
                                                             1 & 0
                                                       \end{bmatrix}\right)
                               & = \det \begin{bmatrix}
                                              t-1 & 0 \\
                                              -1  & t
                                        \end{bmatrix} \\
                               & = t^2-t               \\
                               & = t(t-1)
                        \end{align*}
                        The roots of this polynomial are $t=1$ and $t=0$, so those are our eigenvalues.

                        The result when we multiply the matrix by an arbitrary vector is
                        \[\begin{bmatrix}
                                    t-1 & 0 \\
                                    -1  & t
                              \end{bmatrix} \cdot \begin{bmatrix}
                                    x \\ y
                              \end{bmatrix}=\begin{bmatrix}
                                    x(t-1) \\ -x+ty
                              \end{bmatrix}\]

                        From this, we can calculate that
                        $E_0=\text{span}(\{(0, 1)\})$ and $E_1=\text{span}(\{(1,1)\})$.
                        Furthermore, the algebraic and geometric multiplicities
                        of both these eigenvalues are all $1$ by coincidence. \label{list:1a}
                  \item The steps are basically the same as in \ref{list:1a}.
                        \begin{gather*}
                              [T]_\beta=\begin{bmatrix}
                                    T((1, 0)) & T((0, 1))
                              \end{bmatrix}=\begin{bmatrix}
                                    0 & 1 \\
                                    2 & 0
                              \end{bmatrix} \\
                              \begin{aligned}
                                    \det \left(\begin{bmatrix}
                                                     t & 0 \\
                                                     0 & t
                                               \end{bmatrix}-\begin{bmatrix}
                                                                   0 & 1 \\
                                                                   2 & 0
                                                             \end{bmatrix}\right)
                                     & = \det \begin{bmatrix}
                                                    t  & -1 \\
                                                    -2 & t
                                              \end{bmatrix} \\
                                     & = t^2-2
                              \end{aligned}
                        \end{gather*}
                        Although this polynomial does become $0$ when $t=\sqrt{2}$,
                        it isn't in the field we defined.
                        Thus, this setup has no eigenvalues or corresponding eigenspaces.
                  \item The setup is the exact same except now we can take $t=\pm\sqrt{2}$ as an eigenvalue.
                        The characteristic polynomial then becomes
                        \[t^2-2=(t-\sqrt{2})(t+\sqrt{2})\]
                        Plugging $t=\pm \sqrt{2}$ back into the linear transformation, we get
                        \begin{gather*}
                              E_{\sqrt{2}}=\text{span}(\{(1, \sqrt{2})\}) \\
                              E_{-\sqrt{2}}=\text{span}(\{(1, -\sqrt{2})\})
                        \end{gather*}
                        As in \ref{list:1a}, all multiplicities are $1$.
            \end{enumerate}

            \pagebreak

      \item \begin{enumerate}
                  \item Since $\lambda_1$ is a root of the characteristic polynomial,
                        $\det\left(\lambda_1 I_n - [T]_\beta\right)=0$ and the matrix is non-invertible.
                        This also means that the linear transformation $\lambda_1 \mathbf{1}_V - T$ is non-invertible.

                        If a transformation isn't invertible, it isn't bijective either.
                        Since the domain and codomain of this transformation are the same,
                        by the dimension theorem the nullity of this transformation must be nonzero.

                        With that,  $\lambda_1 \mathbf{1}_V - T$ must have nonzero nullity
                        and there must be at least one element in its kernel.
                        Thus, $\dim E_{\lambda_1} \ge 1$. $\square$
                  \item Since all vectors from $v_1$ through $v_k$ are eigenvectors,
                        $T(v_i)=\lambda_1 v_i$ for these vectors.
                        Thus, the first $k$ columns of $[T]_\beta$ are the first $k$
                        columns of $\lambda_1 \cdot I_n$.
                  \item \begin{enumerate}
                              \item We've already proved that the first $k$ columns of $[T]_\beta$
                                    are equal to the first $k$ columns of $\lambda_1 I_n$.
                                    If we take only the first $k$ rows as well, we get $\lambda_1 I_k$,
                                    which is exactly what we wanted.
                              \item \begin{align*}
                                          f_T(t) & = \det (tI_n-[T]_\beta)                  \\
                                                 & = \det \left[\begin{array}{c|c}
                                                                            tI_n - \lambda_1 I_k & B    \\
                                                                            \midrule
                                                                            0                    & D(t) \\
                                                                      \end{array}\right] \\
                                                 & = \det (tI_n - \lambda_1 I_k) \det D(t)  \\
                                                 & = (t-\lambda_1)^k \det D(t)\quad\square
                                    \end{align*}
                        \end{enumerate}
            \end{enumerate}

      \item From the premise, the largest power that $t-\lambda_1$ can be raised to is $n_1$.
            Thus, the largest $k$ can be is $n_1$ as a contradiction would arise if it were any larger.
            As we have defined $k$ to be equal to $\dim E_{\lambda_1}$, this means that $E_{\lambda_1} \le n_1$.
\end{enumerate}
\end{document}
