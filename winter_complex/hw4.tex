\documentclass[12pt]{article}

% a template that a friend gave, it's worked well enough for me
% i have added some packages and stuff that have proved useful

\usepackage{fancyhdr}
\usepackage{tipa}
\usepackage{fontspec}
\usepackage{amsfonts}
\usepackage{enumitem}
\usepackage[margin=1in]{geometry}
\usepackage{graphicx}
\usepackage{float}
\usepackage{amsmath}
\usepackage{braket}
\usepackage{amssymb}
\usepackage{booktabs}
\usepackage{hyperref}
\usepackage{mathtools}
\usepackage{xcolor}
\usepackage{float}
\usepackage{algpseudocodex}
\usepackage{titlesec}
\usepackage{bbm}
\usepackage{pythonhighlight}

\pagestyle{fancy}
\fancyhf{} % sets both header and footer to nothing
\lhead{Kevin Sheng}
\setmainfont{Comic Neue}
\renewcommand{\headrulewidth}{1pt}
\setlength{\headheight}{0.75in}
\setlength{\oddsidemargin}{0in}
\setlength{\evensidemargin}{0in}
\setlength{\voffset}{-.5in}
\setlength{\headsep}{10pt}
\setlength{\textwidth}{6.5in}
\setlength{\headwidth}{6.5in}
\setlength{\textheight}{8in}
\renewcommand{\headrulewidth}{0.5pt}
\renewcommand{\footrulewidth}{0.3pt}
\setlength{\textwidth}{6.5in}
\usepackage{setspace}
\usepackage{multicol}
\usepackage{float}
\setlength{\columnsep}{1cm}
\setlength\parindent{24pt}
\usepackage [english]{babel}
\usepackage [autostyle, english = american]{csquotes}
\MakeOuterQuote{"}

\setlength{\parskip}{6pt}
\setlength{\parindent}{0pt}

\titlespacing\section{0pt}{12pt plus 4pt minus 2pt}{0pt plus 2pt minus 2pt}
\titlespacing\subsection{0pt}{12pt plus 4pt minus 2pt}{0pt plus 2pt minus 2pt}
\titlespacing\subsubsection{0pt}{12pt plus 4pt minus 2pt}{0pt plus 2pt minus 2pt}

\hypersetup{colorlinks=true, urlcolor=blue}

\newcommand{\correction}[1]{\textcolor{red}{#1}}


\rhead{winter complex pain}

\DeclareMathOperator{\cis}{cis}
\DeclareMathOperator{\Log}{Log}
\newcommand{\lra}{\xLeftrightarrow}
\newcommand{\ra}{\xRightarrow}

\begin{document}

\section{Chapter 4.4}

\subsection{Exercise 1}

\begin{enumerate}[label=(\alph*)]
      \item Yeah, this is just pinching $\Gamma$ really tight around three points.
      \item No, the point at $z=4$ blocks the deformation.
      \item Yeah, it encircles the three banned points just like $\Gamma$.
      \item No, since $z=2i$ is outside of the contour \& blocks the deformation.
\end{enumerate}

\subsection{Exercise 13}

We first do PFD on the term inside the integral:
\begin{align*}
      \frac{1}{1+z^2}
       & = \frac{1}{(z+i)(z-i)}             \\
       & = \frac{A_1}{z+i}+\frac{A_2}{z-i}  \\
       & = \frac{i/2}{z+i}+\frac{-i/2}{z-i}
\end{align*}
so
\[\int_\Gamma \frac{1}{1+z^2}\,dz
      =\frac{i}{2}\int_\Gamma \frac{1}{z+i}\,dz-\frac{i}{2}\int_\Gamma \frac{i}{z-i}\,dz\]

\subsubsection{Part A}

We can deform this into a circle around $i$.

The first addend is $0$ since it's analytic at $i$ and we can deform it into a point there.

The second is $-\frac{i}{2} \cdot 2\pi i = \pi$ by that formula in the textbook.

Thus the total value is \boxed{\pi}.

\subsubsection{Part B}

This big loop can be turned into a combination of $\Gamma_1$ and $\Gamma_3$,
and from the answers in those two parts we see that the integral here is \boxed{0}.

\subsubsection{Part C}

This one is a circle around $-i$.

By similar logic as in part A, the total value here is \boxed{-\pi}.

\subsection{Exercise 15}

Doing PFD again gets us
\[\frac{z}{(z+2)(z-1)}=\frac{2/3}{z+2}+\frac{1/3}{z-1}\]
so
\[\int_\Gamma \frac{z}{(z+1)(z-1)}\,dz=
      \frac{2}{3}\int_\Gamma \frac{1}{z+2}\,dz+\frac{1}{3}\int_\Gamma \frac{1}{z-1}\,dz\]

In both cases, by that stupid textbook formula the integrals come out to $-4\pi i$
(since it's traversed twice clockwise), so our final answer is
\[\frac{2}{3}(-4\pi i) + \frac{1}{3}(-4 \pi i) = \boxed{-4\pi i}\]

\subsection{Exercise 16}

i mean this is lowk self-evident icl

like $\int_\Gamma \frac{A_k}{z^k}\,dz=0$ for all $k > 1$ and $A_1 \cdot 2\pi i$ for $k=1$,
and $g$ is analytic on \& inside the unit circle so $\int_\Gamma g(z)\,dz=0$ as well

not even gonna put a square this is just a corollary

\subsection{Exercise 18}

\begin{enumerate}[label=(\alph*)]
      \item The only banned points are $z=0$ and $z=1$.
            The circle $|z|=2$ lies outside of that, so we can deform it into any larger circle.

      \item Notice that
            \[\left|z^2(z-1)^3\right| \ge R^2(R-1)^3\]
            because $|z|=R$ and $|z-1| \ge R-1$ by the Reverse Triangle Inequality.

            This UB times the circumference of $2\pi R$ gives the UB in the problem.

      \item The numerator stays constant and the denominator $R(R-1)^3$ grows arbitrarily large,
            so the fraction as a whole goes to $0$.

      \item If $I \ne 0$, then the limit would break.
            Simple as that. $\square$
\end{enumerate}

\end{document}
