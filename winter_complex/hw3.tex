\documentclass[12pt]{article}

% a template that a friend gave, it's worked well enough for me
% i have added some packages and stuff that have proved useful

\usepackage{fancyhdr}
\usepackage{tipa}
\usepackage{fontspec}
\usepackage{amsfonts}
\usepackage{enumitem}
\usepackage[margin=1in]{geometry}
\usepackage{graphicx}
\usepackage{float}
\usepackage{amsmath}
\usepackage{braket}
\usepackage{amssymb}
\usepackage{booktabs}
\usepackage{hyperref}
\usepackage{mathtools}
\usepackage{xcolor}
\usepackage{float}
\usepackage{algpseudocodex}
\usepackage{titlesec}
\usepackage{bbm}
\usepackage{pythonhighlight}

\pagestyle{fancy}
\fancyhf{} % sets both header and footer to nothing
\lhead{Kevin Sheng}
\setmainfont{Comic Neue}
\renewcommand{\headrulewidth}{1pt}
\setlength{\headheight}{0.75in}
\setlength{\oddsidemargin}{0in}
\setlength{\evensidemargin}{0in}
\setlength{\voffset}{-.5in}
\setlength{\headsep}{10pt}
\setlength{\textwidth}{6.5in}
\setlength{\headwidth}{6.5in}
\setlength{\textheight}{8in}
\renewcommand{\headrulewidth}{0.5pt}
\renewcommand{\footrulewidth}{0.3pt}
\setlength{\textwidth}{6.5in}
\usepackage{setspace}
\usepackage{multicol}
\usepackage{float}
\setlength{\columnsep}{1cm}
\setlength\parindent{24pt}
\usepackage [english]{babel}
\usepackage [autostyle, english = american]{csquotes}
\MakeOuterQuote{"}

\setlength{\parskip}{6pt}
\setlength{\parindent}{0pt}

\titlespacing\section{0pt}{12pt plus 4pt minus 2pt}{0pt plus 2pt minus 2pt}
\titlespacing\subsection{0pt}{12pt plus 4pt minus 2pt}{0pt plus 2pt minus 2pt}
\titlespacing\subsubsection{0pt}{12pt plus 4pt minus 2pt}{0pt plus 2pt minus 2pt}

\hypersetup{colorlinks=true, urlcolor=blue}

\newcommand{\correction}[1]{\textcolor{red}{#1}}


\rhead{winter complex pain}

\DeclareMathOperator{\cis}{cis}
\DeclareMathOperator{\Log}{Log}
\newcommand{\lra}{\xLeftrightarrow}
\newcommand{\ra}{\xRightarrow}

\begin{document}

\section{Chapter 3.3}

\subsection{Exercise 12}

Do we just need to find a branch that makes $\log 1$ $2\pi i$?

If so, then $\mathcal{L}_\pi(z)$ should work. :clueless:

\subsection{Exercise 13}

\begin{enumerate}[label=(\alph*)]
      \item Principal branch $\mathcal{L}(z)$ should work.
      \item Maybe $\mathcal{L}_0(z)$?
      \item And here it's, uh, $\mathcal{L}_{\pi/2}(z)$?
\end{enumerate}

\subsection{Exercise 14}

BWOC let such an $F$ exist.

Now consider $g(z)=F(z)-\Log z$.
$g'(z)=F'(z)-\frac{1}{z}=0$, which by a previous theorem implies that $g(z)=c$
for some constant $c$.
This then must imply that $F(z)=\Log z + c$ for some constant $c$ as well.

We know that no matter what branch we choose, there's some argument $\theta$
which always makes the logarithm non-analytic.
This contradicts the premise, so no such $F$ can exist. $\square$

\section{Chapter 3.5}

\subsection{Exercise 15}

\begin{enumerate}
      \item Following the hint, our branch is
            \begin{align*}
                  (z-1)^{1/2}(z+1)^{1/2}
                   & = (-1(1-z))^{1/2}(z+1)^{1/2}               \\
                   & = i(1-z)^{1/2}(z+1)^{1/2}                  \\
                   & = i\exp\left(\frac{1}{2}\Log(1-z^2)\right)
            \end{align*}
            which is nonanalytic only when $z \in \R$ and $1-z \le 0$ or $z+1 \le 0$.
            In other words, this is only when $|z| \ge 1$ and is outside of the unit disk.

      \item ok i looked at the solution guide and it says $z\left(\frac{4}{z^2}+1\right)^{1/2}$
            with the principal log works but it literally isn't analytic at $z=\pm 2i$

      \item this is literally just the example: $z^2\exp\left(\frac{1}{2}\Log\left(1-\frac{1}{z^4}\right)\right)$

      \item is this not the same thing??? $z\exp\left(\frac{1}{3}\Log\left(1-\frac{1}{z^3}\right)\right)$
\end{enumerate}

\subsection{Exercise 16}

wha

i mean like if we choose $\Log c$, then $e^{z\Log c}$ is just
a composition of two analytic functions, so we get the derivative as
$\Log c \cdot e^{z \Log c} = \Log c \cdot c^z$?

\section{Chapter 4.1}

\subsection{Exercise 9}

The first circle is just $-2+\frac{1}{2}e^{it}$, where $0 \le t \le 2\pi$.

The second line can be $-\frac{7}{2}+7t$, where $0 \le t \le 1$.

The final circle starts in the negative real axis,
so we make it $2-\frac{1}{2}e^{it}$, where $0 \le t \le 2\pi$ again.

Just put all these together and do some rescaling to get the true parameterization.

\subsection{Exercise 11}

This is just a circle of radius $5$ with another semicircle laid on top of it.

Using the high school formula for circumference gets us a length of \boxed{15\pi}.

\section{Chapter 4.2}

\subsection{Exercise 5}

We can break this down by the addition signs to get the following integrals:
\begin{itemize}
      \item $\int_C 6(z-i)^{-2}\,dz=0$
      \item $\int_C 2(z-i)^{-1}\,dz=2 \cdot 2\pi i = 4\pi i$
      \item $\int_C (z-i)^{0}\,dz=0$
      \item $\int_C -3(z-i)^2\,dz=0$
\end{itemize}
So the overall sum is just \boxed{4\pi i}.

\subsection{Exercise 8}

I'm pretty sure we can make $t \in [0, 1]$ for all components
since they're integrated separately.

But yeah, here's the parameterization and integrations for each part.
\begin{enumerate}
      \item $z(t)=t$, $z'(t)=1$
            \[\int_{0}^{1} \exp(z(t))z'(t)\,dt=\int_{0}^{1} e^t\,dt = e-1\]
      \item $z(t)=1+ti$, $z'(t)=i$
            \begin{align*}
                  \int_{0}^{1} \exp(1+ti) \cdot i\,dt
                   & = ei\int_{0}^{1} \exp(ti)\,dt                 \\
                   & = ei\intval{\frac{1}{i}\exp(ti)}^1_0          \\
                   & = ei\left(\frac{1}{i}e^i - \frac{1}{i}\right) \\
                   & = e^{1+i}-e
            \end{align*}
      \item $z(t)=1-t+i$, $z'(t)=-1$
            \begin{align*}
                  \int_{0}^{1} \exp(1-t+i) \cdot -1\,dt
                   & = -e^{1+i}\int_{0}^{1} e^{-t}\,dt  \\
                   & = -e^{1+i}\intval{-e^{-t}}^1_0     \\
                   & = -e^{1+i}\left(-e^{-1} + 1\right) \\
                   & = e^{i}-e^{1+i}
            \end{align*}
      \item $z(t)=i(1-t)$, $z'(t)=-i$
            \begin{align*}
                  \int_{0}^{1} \exp(i-it) \cdot -i\,dt
                   & = -ie^{i} \int_{0}^{1} e^{-it}\,dt                   \\
                   & = -ie^{i} \intval{-\frac{1}{i}e^{-it}}^1_0           \\
                   & = -ie^{i}\left(-\frac{1}{i}e^{-i}+\frac{1}{i}\right) \\
                   & = 1 - e^i
            \end{align*}
\end{enumerate}
Adding all these together and cancelling terms indeed gives a value of $0$. $\square$

\subsection{Exercise 10}

oh my god bruh

Well, same contours as in the previous problem, let's do this:
\begin{enumerate}
      \item $z(t)=t$
            \begin{align*}
                  \int_{0}^{1} \overline{t}^2 \cdot 1\,dt
                   & = \int_{0}^{1} t^2\,dt \\
                   & = \frac{1}{3}
            \end{align*}
      \item $z(t)=1+ti$
            \begin{align*}
                  \int_{0}^{1} (1-ti)^2 \cdot i\,dt
                   & = i\intval{-\frac{1}{3i}(1-ti)^3}^1_0 \\
                   & = \intval{-\frac{1}{3}(1-ti)^3}^1_0   \\
                   & =-\frac{1}{3}(1-i)^3+\frac{1}{3}      \\
                   & = 1+\frac{2i}{3}
            \end{align*}
      \item $z(t)=1-t+i$
            \begin{align*}
                  \int_{0}^{1} (1-t-i)^2 \cdot -1\,dt
                   & = -\int_{0}^{1} (1-t-i)^2\,dt         \\
                   & = -\intval{-\frac{1}{3}(1-t-i)^3}^1_0 \\
                   & = \intval{\frac{1}{3}(1-t-i)^3}^1_0   \\
                   & = \frac{2}{3}+i
            \end{align*}
      \item $z(t)=i(1-t)$
            \begin{align*}
                  \int_{0}^{1} (-i(1-t))^2 \cdot -i\,dt
                   & = i\int_{0}^{1} (1-t)^2\,dt \\
                   & = \frac{i}{3}
            \end{align*}
\end{enumerate}
Adding these together gives \boxed{2(1+i)}.

\subsection{Exercise 14}

\begin{enumerate}[label=(\alph*)]
      \item The curve length is $2\pi \cdot 3 = 6\pi$, so it STP the following:
            \[|z|=3 \implies |z^2-i| \ge 8 \implies \left|\frac{1}{z^2-i}\right| \le \frac{1}{8}\]
            The first implication is true since $|z^2|=9$ and $|i|=1$,
            and the second is self-evident. $\square$

      \item The length is $2\pi$, so it STP that along the line
            $\left|\frac{e^{3z}}{1+e^z}\right| \le \frac{e^{3R}}{e^R-1}$.

            If we express $z=R+ti$ where $t \in [0, 2\pi]$, then
            \begin{align*}
                  \left|\frac{e^{3z}}{1+e^z}\right|
                   & = \left|\frac{e^{3R}e^{3ti}}{1+e^R e^{ti}}\right| \\
                   & = e^{3R}\left|\frac{e^{3ti}}{1+e^Re^{ti}}\right|  \\
                   & \le e^{3R}\left|\frac{1}{1+e^Re^{ti}}\right|      \\
                   & = \frac{e^{3R}}{\left|1+e^Re^{ti}\right|}         \\
                   & \le \frac{e^{3R}}{\left|1-e^R\right|}             \\
                   & = \frac{e^{3R}}{e^R - 1}\quad\square
            \end{align*}
            where the final couple of equalities come from that $R > 0$.

      \item The length is $\frac{\pi}{2}$, so we NTS that when $z$ is
            in the first quadrant and $|z|=1$, $|\Log z| \le \frac{\pi}{2}$.

            Thankfully,
            \begin{align*}
                  |\Log z|
                   & = |\log |z| + i\arg z|      \\
                   & = |i\arg z|                 \\
                   & = \arg z                    \\
                   & < \frac{\pi}{2}\quad\square
            \end{align*}

            \pagebreak

      \item The length is $1$, so we need $|\exp(\sin z)| \le 1$ too.

            Since $z=ai$ where $a \in [0, 1]$,
            \begin{align*}
                  |\exp(\sin z)|
                   & = \left|\exp\left(\frac{e^{iz}-e^{-iz}}{2i}\right)\right|                        \\
                   & = \left|\exp\left(\frac{e^{-a}-e^{a}}{2i}\right)\right|                          \\
                   & = \left|\frac{\exp(e^{-a}/2i)}{\exp(e^a/2i)}\right|                              \\
                   & = \left|\frac{\exp(-i \cdot e^{-a}/2)}{\exp(-i \cdot e^a/2)}\right|              \\
                   & = \frac{\left|\exp(-i \cdot e^{-a}/2)\right|}{\left|\exp(-i \cdot e^a/2)\right|} \\
                   & = \frac{\exp(e^{-a}/2) \cdot \exp(-i)}{\exp(e^a/2) \cdot \exp(-i)}               \\
                   & = \frac{\exp(e^{-a}/2)}{\exp(e^a/2)}                                             \\
                   & \le 1\quad\square
            \end{align*}
\end{enumerate}

\pagebreak

\section{Chapter 4.3}

\subsection{Exercise 2}

All polynomials are continuous and have an antiderivative by basic calculus,
so they fulfill the conditions of this chapter's theorem.

\subsection{Exercise 3}

\subsubsection{Part A}

It STP that $\left|\int_{\Gamma} f(z)\,dz\right| \le \epsilon\ \forall \epsilon > 0$.

First, fix an $\epsilon > 0$.
By our premise, $\exists P(z): |f(z)-P(z)| < \epsilon\ \forall z \in \Gamma$.

Then, if we let $L$ be the length of $\Gamma$,
\begin{align*}
      L\epsilon
      &\ge \left|\int_\Gamma f(z)-P(z)\,dz\right| \\
      &= \left|\int_\Gamma f(z)\,dz - \int_\Gamma P(z)\,dz\right| \\
      &= \left|\int_\Gamma f(z)\,dz\right|
\end{align*}
so we're off by a constant factor, so we're basically done. $\square$

\subsubsection{Part B}

Well, given part A, we know that $f$ has an antiderivative in $D$.

This then necessarily implies that $D$ is entire...yeah? what?

\subsection{Exercise 4}

ok i had to look at the solution for this icl

but turns out the answer's false with $\int_{|z|=1} \frac{1}{z}\,dz=2\pi i$ as a counterexample

\pagebreak

\section{Riemann Zeta Function}

\subsection{Part A}

I'm not sure if these are sufficient conditions, but uh,
\begin{align*}
      \left|\frac{1}{n^s}\right|
      &= \frac{1}{|\exp(s \log n)|} \\
      &= \frac{1}{\left|\exp(s)^{\log n}\right|} \\
      &= \frac{1}{\left|\left(e^{|s|} \cdot e^{i \arg s}\right)^{\log n}\right|} \\
      &= \frac{1}{\left|\exp(|s| \log n) \cdot \exp(i \arg s)\right|} \\
      &= \frac{1}{\left|\exp(|s| \log n)\right|} \\
      &= \frac{1}{n^{|s|}}
\end{align*}
As long as $|s| > 1$, it should be fine, right...?

\subsection{Part B}

We can still use all the standard differentation rules, right?
\begin{align*}
      \frac{d}{ds} \frac{1}{n^s}
      &= \frac{0-\frac{d}{ds} n^s \cdot 1}{n^{2s}} \\
      &= \frac{-\log n n^s}{n^{2s}} \\
      &= -\frac{\log n}{n^s}
\end{align*}
So we see that $\zeta'(s)$ takes the form $\sum_{s=1}^{n} \frac{a_n}{n^s}$ with $a_n=-\log n$.

\pagebreak

\subsection{Part C}

$\zeta(2)=\sum_{n=1}^{\infty} \frac{1}{n^2}$, oh wow...

But yeah, multiplying the terms out on the RHS of the given formula
gives $z^3$'s coefficient as
\[\pi\left(-\frac{1}{1^2}-\frac{1}{2^2}-\frac{1}{3^2}\cdots\right) = -\pi \cdot \zeta(2)\]
since we can take at most one $z^2$ term from the coefficients.

However, in the Maclaurin series of $\sin(\pi x)$, the coefficient of $z^3$ is
\begin{align*}
      \frac{f^{(3)}(0)}{3!}
      &= \frac{-\pi^3 \cos(\pi \cdot 0)}{3!} \\
      &= -\frac{\pi^3}{6}
\end{align*}
Since both these polynomial approximations are equivalent, I guess they have
to have the same coefficient for $z^3$, so
\[-\pi \cdot \zeta(2) = -\frac{\pi^3}{6} \implies \zeta(2)=\frac{\pi^2}{6}\quad\square\]
I'm not even sure if I can say this.
I didn't take algebra so I wouldn't know.

\end{document}
