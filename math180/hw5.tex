\documentclass[12pt]{article}

% a template that a friend gave, it's worked well enough for me
% i have added some packages and stuff that have proved useful

\usepackage{fancyhdr}
\usepackage{tipa}
\usepackage{fontspec}
\usepackage{amsfonts}
\usepackage{enumitem}
\usepackage[margin=1in]{geometry}
\usepackage{graphicx}
\usepackage{float}
\usepackage{amsmath}
\usepackage{braket}
\usepackage{amssymb}
\usepackage{booktabs}
\usepackage{hyperref}
\usepackage{mathtools}
\usepackage{xcolor}
\usepackage{float}
\usepackage{algpseudocodex}
\usepackage{titlesec}
\usepackage{bbm}

\pagestyle{fancy}
\fancyhf{} % sets both header and footer to nothing
\lhead{Kevin Sheng}
\setmainfont{Comic Neue}
\renewcommand{\headrulewidth}{1pt}
\setlength{\headheight}{0.75in}
\setlength{\oddsidemargin}{0in}
\setlength{\evensidemargin}{0in}
\setlength{\voffset}{-.5in}
\setlength{\headsep}{10pt}
\setlength{\textwidth}{6.5in}
\setlength{\headwidth}{6.5in}
\setlength{\textheight}{8in}
\renewcommand{\headrulewidth}{0.5pt}
\renewcommand{\footrulewidth}{0.3pt}
\setlength{\textwidth}{6.5in}
\usepackage{setspace}
\usepackage{multicol}
\usepackage{float}
\setlength{\columnsep}{1cm}
\setlength\parindent{24pt}
\usepackage [english]{babel}
\usepackage [autostyle, english = american]{csquotes}
\MakeOuterQuote{"}

\setlength{\parskip}{6pt}
\setlength{\parindent}{0pt}

\titlespacing\section{0pt}{12pt plus 4pt minus 2pt}{0pt plus 2pt minus 2pt}
\titlespacing\subsection{0pt}{12pt plus 4pt minus 2pt}{0pt plus 2pt minus 2pt}
\titlespacing\subsubsection{0pt}{12pt plus 4pt minus 2pt}{0pt plus 2pt minus 2pt}

\hypersetup{colorlinks=true, urlcolor=blue}

\newcommand{\correction}[1]{\textcolor{red}{#1}}


\rhead{Math 180}

\makeatletter
\def\@seccntformat#1{%
  \expandafter\ifx\csname c@#1\endcsname\c@section\else
  \csname the#1\endcsname\quad
  \fi}
\makeatother

\begin{document}

\section{Chapter 7.1}

\begin{enumerate}
    % probably redundant, idc
    \item[1] Construct an auxiliary graph $G$ similar to the one in lecture,
        where each face in the OG graph maps to a vertex and edges are
        drawn across adjacent faces if the vertices over the OG edge are assigned different labels.

        A face will have all three labels iff its corresponding vertex in $G$
        has odd degree.
        This is because a face with one and two distinct labels will have a node degree of $0$
        and $2$ degree respectively, while if it has three labels it'll have a node degree of $3$.

        All graphs have an even number of odd degee vertices,
        so there must be an even number of faces whose vertices get all labels. $\square$
\end{enumerate}

\section{Chapter 7.2}

\begin{enumerate}
    \item[1] \begin{enumerate}
            \item As in lecture, we count the size of the set
                  \[X=\{(N, C) \mid N \in \mathcal{N}, \text{$C$ is a maximal chain, and } N \in C\}\]
                  A semi-independent set has at most two elements from any chain, so $|X| \le 2n!$.

                  Now we directly count and bound $|X|$ like so:
                  \begin{align*}
                                   & |X| = \sum_{N \in \mathcal{N}} |N|!(n-|N|)!                              \\
                      \therefore{} & \sum_{N \in \mathcal{N}} |N|!(n-|N|)! \le 2n!                            \\
                      \therefore{} & \sum_{N \in \mathcal{N}} \frac{|N|!(n-|N|)!}{2n!} \le 1                  \\
                      \therefore{} & \sum_{N \in \mathcal{N}} \frac{1}{2\binom{n}{|N|}} \le 1                 \\
                      \therefore{} & \sum_{N \in \mathcal{N}} \frac{1}{2\binom{n}{\lfloor n/2 \rfloor}} \le 1 \\
                      \therefore{} & |N| \le 2\binom{n}{\lfloor n/2 \rfloor}\quad\square
                  \end{align*}

                  \pagebreak

            \item For any $X$ with odd size we can construct a semi-independent set
                  which meets the UB exactly like so:
                  \[\mathcal{N}=\left\{\text{all subsets of size $\left\lfloor \frac{n}{2} \right\rfloor$}\right\}
                      \cup \left\{\text{all subsets of size $\left\lceil \frac{n}{2} \right\rceil$}\right\}\]
                  For a subset in $\mathcal{N}$ to be a subset, by
                  construction they must be of different sizes.
                  There's only two different sizes in this $\mathcal{N}$,
                  so no three sets form a chain.
        \end{enumerate}

    \item[6] WLOG assume $a_i>0$.
        $\epsilon_i$ is what determines the sign, anyways.

        For convenience, let $a$ denote the vector of all $a_i$s and $\epsilon$ denote
        the vector of all $\epsilon_i$s.
        \begin{enumerate}
            \item Consider the set of sets
                  \[\{\{i \mid \epsilon_i=1\} \mid \text{all possible $\epsilon$}\}\]
                  I propose that if a subset of this only has $\epsilon$ where $-1 < a \cdot \epsilon < 1$,
                  then it must be independent as well.

                  BWOC say $\exists \epsilon_1, \epsilon_2$ s.t.
                  both are valid choise and that $\epsilon_1$'s associated set
                  was a strict subset of the set of $\epsilon_2$.
                  At least one $a_i$ must be converted from a negative to a positive,
                  and since $a_i > 1$, so
                  \[a \cdot \epsilon_2 > a \cdot \epsilon_1 + 2 \therefore a \cdot \epsilon_2 \therefore a \cdot \epsilon_2 > 1\]
                  which is a contradiction.

                  With this, we can apply Sperner's Theorem.
                  The largest independent subset of the $\epsilon$ set defined above
                  has size $\binom{n}{\lfloor n/2 \rfloor}$, and since
                  a set with only valid $\epsilon$ must be independent,
                  this binomial expression must be an upper bound on the size
                  of the set with all valid $\epsilon$s. $\square$

            \item If $n$ is even, we can let $a_i=1\ \forall i$.
                  Then, any $\epsilon$ with exactly $\frac{n}{2}$ $1$s and $-1$s
                  will have a dot product of $0$ with $a$.
                  Clearly, there's $\binom{n}{n/2}$ such $\epsilon$s.
        \end{enumerate}

    \item[7] If $n$ isn't divisible by the square of any other number,
        its prime factorization has at most one (1) power of any prime.

        A divisor of $n$ can then be represented by a subset of all the primes that divide $n$.
        To get the actual divisor, we just multiply together all the elements.

        If a divisor $x$'s set is a subset of another divosor's set $y$,
        then by construction $x \mid y$ and the two can't both be in $M$.

        We see that we just want the largest independent system of subsets
        from the set of prime factors of $n$.
        Sperner's Theorem gets us $\boxed{|M| \le \binom{p}{\lfloor p/2 \rfloor}}$,
        where $p$ is the number of prime factors of $n$.
\end{enumerate}

\section{Chapter 7.3}

\begin{enumerate}
    \item[5b] A function is convex if and only if its second derivative is nonnegative.
        The second derivatives of $0$ and $\frac{x(x-1)}{2}$ are $0$ and $1$ respectively,
        so we know the individual pieces are piecewise convex.

        To handle the case wehre $x \le 1$ and $y > 1$, notice that
        \begin{align*}
            f(\lambda x + (1-\lambda)y)
             & \le f(\lambda \cdot 1 + (1-\lambda) y) &  & \text{$f$ is monotonic}                \\
             & \le \lambda f(1) + (1-\lambda)f(y)     &  & \text{the right half of $f$ is convex} \\
             & = \lambda f(x) + (1-\lambda)f(y)       &  & \text{$f$ is constant before $1$}      \\
        \end{align*}

    \item[5c] We start again at the summation step:
        \begin{align*}
            \binom{n}{2}
             & \ge \sum_{i=1}^{n} \binom{d_i}{2}                                                                   \\
             & = \sum_{i=1}^{n} f(d_i)                                        &  & \text{since $d_i \ge 1$}        \\
             & = n \cdot \sum_{i=1}^{n} \frac{1}{n} \cdot f(d_i)                                                   \\
             & \ge n \cdot f\left(\frac{1}{n} \cdot \sum_{i=1}^{n} d_i\right) &  & \text{by Jensen's Inequality}   \\
             & = n \cdot f\left(\frac{2m}{n}\right)                           &  & \text{by the Handshaking Lemma}
        \end{align*}

        Then simplify the inequality:
        \begin{align*}
                         & n \cdot f\left(\frac{m}{2n}\right) \le \binom{n}{2}              \\
            \therefore{} & \frac{2m}{n} \cdot \left(\frac{2m}{n} - 1\right) \le n-1         \\
            \therefore{} & 4m^2-2mn \le n^3-n^2                                             \\
            \therefore{} & \left(2m-\frac{1}{2}n\right) \le n^3-\frac{3}{4}n^2              \\
            \therefore{} & 2m-\frac{1}{2}n \le n\sqrt{n-\frac{3}{4}}                        \\
            \therefore{} & m \le \frac{1}{2}\left(n\sqrt{n-\frac{3}{4}}+\frac{1}{2}n\right) \\
            \therefore{} & m \le \frac{1}{4}\left(n\sqrt{4n-3} + n\right)
        \end{align*}
        which we know is less than $\frac{1}{2}\left(n^{3/2}+n\right)$ because
        \begin{align*}
                   & \frac{1}{4}\left(n\sqrt{4n-3}+n\right) \le \frac{1}{2}\left(n^{3/2}+n\right)                                                      \\
            \iff{} & n\sqrt{4n-3} \le 2n^{3/2}+n                                                  &  & \text{multiply by $4$ and subtract $n$}         \\
            \iff{} & \sqrt{4n-3} \le 2\sqrt{n}+1                                                  &  & \text{can divide since $n > 0$}                 \\
            \iff{} & 4n-3 \le 4n+1+4\sqrt{n}                                                      &  & \text{can square since both sides are positive} \\
            \iff{} & -4 \le \sqrt{n}
        \end{align*}
        so we've actually wound up proving a lower UB! $\square$ \label{prob:5c}

        \pagebreak

    \item[6] Like with proving graphs with no $K_{2,2}$, we use the set
        \[\{(v, \{a, b, c\}) \mid (v, a), (v, b), (v, c) \in E\}\]
        No unordered triple $\{a, b, c\}$ can appear in the set more than twice,
        so the size of this set is at most $2\binom{n}{3}$.

        Now counting from the PoV of the $v$s, we get the inequality
        \[\sum_{i=1}^{n} \binom{d_i}{3} \le 2\binom{n}{3}\]
        Notice that $f(x)=\binom{x}{3}$ is convex for $x \ge 2$
        (its second derivative is $x-1$), so we do a similar inequality chain as in \hyperref[prob:5c]{5c}:
        \begin{align*}
            2\binom{n}{3}
             & \ge \sum_{i=1}^{n} f(d_i)                              \\
             & = n \sum_{i=1}^{n} \frac{1}{n} f(d_i)                  \\
             & \le n f\left(\frac{1}{n} \cdot \sum_{i=1}^{d_i}\right) \\
             & = n f\left(\frac{2m}{n}\right)
        \end{align*}
        and then simplify the inequality:
        \begin{align*}
                         & 2\binom{n}{3} \ge n\binom{\frac{2m}{n}}{3}                                         \\
            \therefore{} & 2(n-1)(n-2) \ge \frac{2m}{n}\left(\frac{2m}{n}-1\right)\left(\frac{2m}{n}-2\right) \\
            \therefore{} & n(n-1)(n-2) \ge m\left(\frac{2m}{n}-1\right)\left(\frac{2m}{n}-2\right)            \\
            \therefore{} & O\left(n^3\right) \ge \frac{2m^3}{n^2} - \frac{6m^2}{n} + 2                        \\
            \therefore{} & O\left(n^3\right) \ge \frac{m^3}{n^2}-\frac{3m^2}{n}                               \\
            \therefore{} & O\left(n^4\right) \ge \frac{m^3}{n}-3m^2
        \end{align*}
        BWOC suppose $m > O\left(n^{5/3}\right)$, then
        \begin{align*}
            \frac{m^3}{n}-3m^2
             & > \frac{O\left(n^{5/3}\right)^3}{n}-3O\left(n^{5/3}\right)^2 \\
             & = \frac{\left(n^5\right)}{n}-O\left(n^{10/3}\right)          \\
             & = O\left(n^4\right)
        \end{align*}
        which contradicts our initial inequality, so $m \le O\left(n^{5/3}\right)$. $\square$
\end{enumerate}

\end{document}
