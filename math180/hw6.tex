\documentclass[12pt]{article}

% a template that a friend gave, it's worked well enough for me
% i have added some packages and stuff that have proved useful

\usepackage{fancyhdr}
\usepackage{tipa}
\usepackage{fontspec}
\usepackage{amsfonts}
\usepackage{enumitem}
\usepackage[margin=1in]{geometry}
\usepackage{graphicx}
\usepackage{float}
\usepackage{amsmath}
\usepackage{braket}
\usepackage{amssymb}
\usepackage{booktabs}
\usepackage{hyperref}
\usepackage{mathtools}
\usepackage{xcolor}
\usepackage{float}
\usepackage{algpseudocodex}
\usepackage{titlesec}
\usepackage{bbm}

\pagestyle{fancy}
\fancyhf{} % sets both header and footer to nothing
\lhead{Kevin Sheng}
\setmainfont{Comic Neue}
\renewcommand{\headrulewidth}{1pt}
\setlength{\headheight}{0.75in}
\setlength{\oddsidemargin}{0in}
\setlength{\evensidemargin}{0in}
\setlength{\voffset}{-.5in}
\setlength{\headsep}{10pt}
\setlength{\textwidth}{6.5in}
\setlength{\headwidth}{6.5in}
\setlength{\textheight}{8in}
\renewcommand{\headrulewidth}{0.5pt}
\renewcommand{\footrulewidth}{0.3pt}
\setlength{\textwidth}{6.5in}
\usepackage{setspace}
\usepackage{multicol}
\usepackage{float}
\setlength{\columnsep}{1cm}
\setlength\parindent{24pt}
\usepackage [english]{babel}
\usepackage [autostyle, english = american]{csquotes}
\MakeOuterQuote{"}

\setlength{\parskip}{6pt}
\setlength{\parindent}{0pt}

\titlespacing\section{0pt}{12pt plus 4pt minus 2pt}{0pt plus 2pt minus 2pt}
\titlespacing\subsection{0pt}{12pt plus 4pt minus 2pt}{0pt plus 2pt minus 2pt}
\titlespacing\subsubsection{0pt}{12pt plus 4pt minus 2pt}{0pt plus 2pt minus 2pt}

\hypersetup{colorlinks=true, urlcolor=blue}

\newcommand{\correction}[1]{\textcolor{red}{#1}}


\rhead{Math 180}

\makeatletter
\def\@seccntformat#1{%
  \expandafter\ifx\csname c@#1\endcsname\c@section\else
  \csname the#1\endcsname\quad
  \fi}
\makeatother

\DeclareMathOperator{\E}{E}

\begin{document}

\section{Chapter 10.1}

\begin{enumerate}
    \item[2a] We do casework on the size of $|X|$.

        \textbf{Case 1: $|X| \le 14$} \\
        Like in lecture, we add enough dummy points to get $|X|=14$.


        Consider a random coloring where we pick some $7$ points to be black and the rest to be white.
        For a specific $4$-tuple, there's $2 \cdot \binom{10}{7}$ configs where
        it's monochromatic, so the chance this is the case is $\frac{2 \cdot \binom{10}{7}}{\binom{14}{7}}=\frac{240}{3432}$.

        Then we use the union bound to see that
        \begin{align*}
            P(\text{any tuple is MC})
             & \le \sum_{\text{all tuples}} P(\text{specific tuple is MC}) \\
             & = 14 \cdot \frac{240}{3432}                                 \\
             & = \frac{3360}{3432}                                         \\
             & < 1
        \end{align*}
        so there does indeed exist a 2-coloring that satisfies all tuples.

        \textbf{Case 2: $|X| > 14$} \\
        If there's at least $15$ elements, we can always find a disconnected pair.
        This is because each tuple can contribute $\binom{4}{2}=6$ pairings,
        while there's at least $\binom{15}{2}=105$ in total.

        We only have $14$ tuples, so the input can cover at most $6 \cdot 14=84$ of them.
        By the Pigeonhole Principle, then, there must exist some disconnected pair
        of elements which we can then reduce to a single element
        by a process identical to the one used in lecture.

    \item[4] Before we start, let a "move" be defined as moving a specific car from
        A to I, I to II, and so on and so forth.
        There's only 5 relevant tracks: A, I, II, III, and B.

        Each car can be moved at most $4$ times.
        Moving it any more would imply it going on I, II, or III more than once.
        Thus, any train order must be achieved in at most $4n$ moves.

        Also, given any track config, there's only $10$ moves we can make,
        since we can only touch the "top" of tracks.
        Cars on track A can be moved to 4 others,
        those on track I have 3 options, you get the idea.

        Point is, there's at most $10^{4n}$ move sequences we can make.
        However, there are $n!$ total car orders, and since
        factorial dominates exponential, $\exists n: n! > 10^{4n}$ and at that point
        there literally aren't enough move sequences to cover all orders,
        again by the Pigeonhole Principle. $\square$
\end{enumerate}

\section{Chapter 10.2}

\begin{enumerate}
    \item[3] Consider any three vertices.
        Each edge has an independent $\frac{1}{2}$ chance of being there or not,
        so the chance that these three edges \textit{don't} form a triangle is $\frac{7}{8}$.

        Any graph has $\left\lfloor \frac{n}{3} \right\rfloor$ disjoint
        sets of $3$ vertices, and since they're disjoint the event that
        any one of them form a triangle is independent from the others.

        Then we have
        \begin{align*}
            P(\text{no triangles})
             & \le P(\text{no triangles within the selected disjoint sets})                              \\
             & = P(\text{three vertices don't form a triangle})^{\left\lfloor \frac{n}{3} \right\rfloor} \\
             & = \left(\frac{7}{8}\right)\left\lfloor \frac{n}{3} \right\rfloor
        \end{align*}
        which pretty clearly goes to $0$ as $n \to \infty$. $\square$

    \item[9] Don't you love it when textbooks assign problems
        \href{https://en.wikipedia.org/wiki/Boy_or_girl_paradox}{with their own Wikipedia pages}?

        Anyways, we have
        \begin{align*}
            P(\text{two boys} \mid \text{at least one boy})
             & = \frac{P(\text{two boys} \cap \text{at least one boy})}{P(\text{at least one boy})} \\
             & = \frac{\frac{1}{4}}{1-P(\text{no boys})}                                            \\
             & = \frac{\frac{1}{4}}{1-\frac{1}{4}}                                                  \\
             & = \boxed{\frac{1}{3}}
        \end{align*}
\end{enumerate}

\pagebreak

\section{Chapter 10.3}

\begin{enumerate}
    \item[1] It's counterexample time!

        \textbf{$\E \left[f^2\right] = \left(E f\right)^2$:} \\
        Note that disproving this one implies  $\E [fg] \ne \E f \cdot \E g$.

        Let $f$ take on values $0$ and $1$, each with probability $\frac{1}{2}$.
        $\E \left[f^2\right]=1$, while $\left(E f\right)^2=\left(\frac{1}{2}\right)^2=\frac{1}{4}$,
        so this property doesn't hold.

        \textbf{$\E [1/f] = \frac{1}{\E f}$:} \\
        Let $f$ take on values $1$ and $2$, each with probability $\frac{1}{2}$.
        While $\E [1/f]=\frac{3}{4}$, $\frac{1}{\E f}=\frac{2}{3}$,
        so this property doesn't hold either.

    \item[3] We'll compute $\sum_{i=1}^{n} \E \mathbbm{1}[\pi_i=i]$ which is
        equivalent to what we want by linearity of expectation.

        Among all permutations, $(n-1)!$ of them map $i$ to itself, so
        \[P(\pi_i=i)=\frac{(n-1)!}{n!}=\frac{1}{n} \implies \E \mathbbm{1}[\pi_i=i]=\frac{1}{n}\]
        Summing over the $n$ $i$s gets us that $\E f=\boxed{1}$.
\end{enumerate}

\end{document}
