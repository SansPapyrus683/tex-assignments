\documentclass[12pt]{article}

% a template that a friend gave, it's worked well enough for me
% i have added some packages and stuff that have proved useful

\usepackage{fancyhdr}
\usepackage{tipa}
\usepackage{fontspec}
\usepackage{amsfonts}
\usepackage{enumitem}
\usepackage[margin=1in]{geometry}
\usepackage{graphicx}
\usepackage{float}
\usepackage{amsmath}
\usepackage{braket}
\usepackage{amssymb}
\usepackage{booktabs}
\usepackage{hyperref}
\usepackage{mathtools}
\usepackage{xcolor}
\usepackage{float}
\usepackage{algpseudocodex}
\usepackage{titlesec}
\usepackage{bbm}
\usepackage{pythonhighlight}

\pagestyle{fancy}
\fancyhf{} % sets both header and footer to nothing
\lhead{Kevin Sheng}
\setmainfont{Comic Neue}
\renewcommand{\headrulewidth}{1pt}
\setlength{\headheight}{0.75in}
\setlength{\oddsidemargin}{0in}
\setlength{\evensidemargin}{0in}
\setlength{\voffset}{-.5in}
\setlength{\headsep}{10pt}
\setlength{\textwidth}{6.5in}
\setlength{\headwidth}{6.5in}
\setlength{\textheight}{8in}
\renewcommand{\headrulewidth}{0.5pt}
\renewcommand{\footrulewidth}{0.3pt}
\setlength{\textwidth}{6.5in}
\usepackage{setspace}
\usepackage{multicol}
\usepackage{float}
\setlength{\columnsep}{1cm}
\setlength\parindent{24pt}
\usepackage [english]{babel}
\usepackage [autostyle, english = american]{csquotes}
\MakeOuterQuote{"}

\setlength{\parskip}{6pt}
\setlength{\parindent}{0pt}

\titlespacing\section{0pt}{12pt plus 4pt minus 2pt}{0pt plus 2pt minus 2pt}
\titlespacing\subsection{0pt}{12pt plus 4pt minus 2pt}{0pt plus 2pt minus 2pt}
\titlespacing\subsubsection{0pt}{12pt plus 4pt minus 2pt}{0pt plus 2pt minus 2pt}

\hypersetup{colorlinks=true, urlcolor=blue}

\newcommand{\correction}[1]{\textcolor{red}{#1}}


\allowdisplaybreaks

\begin{document}
\section{Textbook}\label{sec:textbook}

\subsection*{8.1}\label{sec:8.1}

\begin{enumerate}
    \item[25] Each of the $m$ vertices on one side has to be connected to all $n$ on the other side.
          Thus, $|E|=\boxed{mn}$.
    \item[30] A plausible model would represent all mathematicians (including Erdos himself) as nodes.
          An edge is drawn between two mathematicians if they ever published a paper together.
          The Erdos number of a mathematicians would then be the smallest distance between
          Erdos's node and the node of the aformentioned mathematician.
\end{enumerate}

\subsection*{8.2}\label{sec:8.2}

\begin{enumerate}
    \item[41] If teams are nodes and games are edges,
          then this proposal would result in a graph with $11$ nodes, each of degree $9$.
          It is impossible to have an odd-number of odd-degree nodes,
          so this proposal is nonsense. $\square$
    \item[49] Removing a single edge from any tree disconnects it.
    \item[64] For a path of length exactly $k \ge 1$, there are $\binom{n}{k+1}$ sets of nodes we can visit.
          Each of these sets of nodes can then be visited in $(k+1)!$ ways.
          Thus, our final answer is
          \[\sum_{k=1}^{n-1} \binom{n}{k+1}(k+1)! = \sum_{k=1}^{n-1} \perm{n}{k+1}\]
    \item[67] Paths from $v$ to $w$ can range from length $k=1$ to $k=n-1$.
          For a path of length $k$, we can choose $\binom{n-2}{k-1}$ intermediate nodes between $v$ and $w$
          and then order them in $(k-1)!$ ways.
          Notice that this differs from the result in the previous question because unlike in that one,
          the position of the start and end are already defined.

          Thus, the number of simple paths from $v$ to $w$ is
          \begin{align*}
              \sum_{k=1}^{n-1} \binom{n-2}{k-1}(k-1)! & = \sum_{k=0}^{n-2} \binom{n-2}{k}k!               \\
                                                      & = \sum_{k=0}^{n-2} \frac{(n-2)!}{k!(n-k-2)!}k!    \\
                                                      & = (n-2)!\sum_{k=0}^{n-2} \frac{1}{(n-k-2)!}       \\
                                                      & = (n-2)!\sum_{k=0}^{n-2} \frac{1}{k!}\quad\square
          \end{align*}
    \item[78] Notice that choosing an independent set on a graph that's just $n$ vertices in a single line
          is equivalent to counting the number of binary strings such that no two $\texttt{1}$s are adjacent.

          It was proven in a previous homework that the result of the latter is equivalent to the Fibonacci
          sequence, so this one must be too. $\square$
\end{enumerate}

\subsection*{8.3}\label{sec:8.3}

\begin{enumerate}
    \item[10] A graph that's $n$ nodes in a loop has a Euler cycle that's also a Hamiltonian cycle.
    \item[14] Let us proceed by induction.

          \textbf{Base case $n=3$:}
          $K_3$ is a triangle.
          By inspection, this graph has a cycle that goes from $v_1$ to $v_2$, then to $v_3$, and back to $v_1$.
          We now assume $K_n$ with $n \ge 3$ has a Hamiltonian cycle.

          \textbf{Inductive step:}
          Take $K_n$ and its Hamiltonian cycle $v_1, v_2, \cdots, v_n, v_1$.
          To construct $K_{n+1}$ from $K_n$, we can add an additional vertex $v_{n+1}$ along
          with edges to it from every other preexisting vertex.

          Notice that inserting $v_{n+1}$ between $v_n$ and $v_1$ will create another cycle.
          No edges are repeated, since $K_n$ can't have already had any edges that go to or from $v_{n+1}$.
          The new sequence of edges starts and ends at the same place and consists of $n+1$ distinct vertices.

          Thus, we have constructed a Hamiltonian cycle for $K_{n+1}$, and the inductive hypothesis is complete.
    \item[15] $K_{m,n}$ contains a Hamiltonian cycle when $m,n > 1$ and $m=n$.
    \item[20] All edges in $G$ either go from a vertex in $V_1$ to a vertex in $V_2$ or vice versa.
    Thus, a Hamiltonian cycle must alternate between vertices in $V_1$ and those in $V_2$.

    This cycle must also have $|V_1|+|V_2|$ vertices.
    If it starts at a vertex in $V_1$, then the final vertex visited must be in $V_2$
    so that the cycle can be completed with an edge back to that vertex in $V_1$.
    
    As we can see, these conditions force $|V_1|=|V_2|$ as otherwise
    we would either have two adjacent vertices from the same set or a cycle that isn't complete. $\square$
\end{enumerate}

\pagebreak

\section{Handout}\label{sec:handout}

\begin{itemize}
    \item[A1] Values for the degree of a vertex can range from $0$ to $n-1$ inclusive.
          Since there's $n$ vertices, and $n$ possible values, it may seem like there's no contradiction here.

          However, notice that we cannot have a vertex of degree $n-1$ and another of degree $0$ at the same time.
          This is because the vertex of degree $n-1$ must be connected to every other node in the graph,
          \textit{including the vertex of degree $0$}.
          This would contradict that the vertex of degree $0$, is, well, of degree $0$.

          Thus, there can only really be $n-1$ possible values to spread among $n$ vertices,
          since two of those values cannot coexist in the same graph.
          By the pigeonhole principle, two vertices must then share the same degree. $\square$
\end{itemize}

\end{document}
