\documentclass[12pt]{article}

% a template that a friend gave, it's worked well enough for me
% i have added some packages and stuff that have proved useful

\usepackage{fancyhdr}
\usepackage{tipa}
\usepackage{fontspec}
\usepackage{amsfonts}
\usepackage{enumitem}
\usepackage[margin=1in]{geometry}
\usepackage{graphicx}
\usepackage{float}
\usepackage{amsmath}
\usepackage{braket}
\usepackage{amssymb}
\usepackage{booktabs}
\usepackage{hyperref}
\usepackage{mathtools}
\usepackage{xcolor}
\usepackage{float}
\usepackage{algpseudocodex}
\usepackage{titlesec}
\usepackage{bbm}

\pagestyle{fancy}
\fancyhf{} % sets both header and footer to nothing
\lhead{Kevin Sheng}
\setmainfont{Comic Neue}
\renewcommand{\headrulewidth}{1pt}
\setlength{\headheight}{0.75in}
\setlength{\oddsidemargin}{0in}
\setlength{\evensidemargin}{0in}
\setlength{\voffset}{-.5in}
\setlength{\headsep}{10pt}
\setlength{\textwidth}{6.5in}
\setlength{\headwidth}{6.5in}
\setlength{\textheight}{8in}
\renewcommand{\headrulewidth}{0.5pt}
\renewcommand{\footrulewidth}{0.3pt}
\setlength{\textwidth}{6.5in}
\usepackage{setspace}
\usepackage{multicol}
\usepackage{float}
\setlength{\columnsep}{1cm}
\setlength\parindent{24pt}
\usepackage [english]{babel}
\usepackage [autostyle, english = american]{csquotes}
\MakeOuterQuote{"}

\setlength{\parskip}{6pt}
\setlength{\parindent}{0pt}

\titlespacing\section{0pt}{12pt plus 4pt minus 2pt}{0pt plus 2pt minus 2pt}
\titlespacing\subsection{0pt}{12pt plus 4pt minus 2pt}{0pt plus 2pt minus 2pt}
\titlespacing\subsubsection{0pt}{12pt plus 4pt minus 2pt}{0pt plus 2pt minus 2pt}

\hypersetup{colorlinks=true, urlcolor=blue}

\newcommand{\correction}[1]{\textcolor{red}{#1}}


\begin{document}
\section{Textbook}\label{sec:textbook}

\subsection*{6.2}\label{sec:6.2}

\begin{enumerate}
    \item[12] We can think of the substrings $DB$ and $AE$ as just their own letters,
        Our units become $DB$, $AE$, and $C$, which gives us $3! = \boxed{6}$ possible strings.
    \item[14] Notice that in half the permutations, $A$ comes before $D$, and in the other half the opposite is true.
        Thus, our answer is $\frac{5!}{2}=\boxed{60}$ possible strings.
    \item[35] Let's use complementary counting.
        There's $\binom{6}{4}=15$ possible committees that consist of only men, and $\binom{13}{4}=715$ possible committees in total.
        Subtracting the former from the latter gives $\boxed{700}$ possible committees that have at least one woman.
    \item[37] Take the answer from the previous question and also subtract the $\binom{7}{4}=35$ that consist of only women
        to get $\boxed{665}$ committees that have both a man and woman.
\end{enumerate}

\subsection*{6.3}\label{sec:6.3}

\begin{enumerate}
    \item[6] $\frac{7!}{3! \cdot 2!}=\boxed{420}$
    \item[8] Let's start by choosing the nubmer of positions for the $S$s.
        There's $8$ possible dividers, $3$ of which must be used to separate the $S$s.
        Thus, we have $\binom{5+4}{4}=126$ ways to place the $S$s.

        After placing the $S$s, there's $\frac{8!}{2!}=20160$ ways to arrange the other letters,
        giving us a total of $126 \cdot 20160 = \boxed{2540160}$ arrangements.

    \item[34] We have six digits, each of which can go from $0$ to $9$, and thus
        this problem reduces to finding the number of possible solutions to the equation
        \[\sum_{i=1}^6 d_i=15 \quad 0 \le d_i \le 9\]
        Without the upper bound on $d_i$, there's $\binom{15+6-1}{15}$ solutions to the equation.

        Taking into account the upper bound, notice that the only way it can be violated
        is if \textit{exactly one} $d_i > 9$.
        Our equation then becomes $\sum_{i=1}^6 d_i=5$, which has $\binom{5+6-1}{5}$ solutions.

        Since there's $6$ different $d_i$s, our final answer is $\boxed{\binom{20}{15}-6 \cdot \binom{10}{5}}$.
        \label{list:count-digits}
\end{enumerate}

\pagebreak

\section{Handout}\label{sec:handout}
\begin{itemize}
    \item[A1] Notice that we can turn this into an equality by placing the "unused" amount into a separate nonnegative variable $x_8$.
        This turns the equation into
        \[\sum_{i=1}^8 x_i=47 \quad x_i \in \mathbb{Z}_{\ge 0}\]
        Using stars and bars, our final answer is $\binom{47+8-1}{8-1}=\boxed{\binom{54}{7}}$
    \item[A2] \begin{enumerate}[label=\alph*]
            \item We must count the number of solutions to
                  \[\sum_{i=1}^3 d_i=12 \quad 1 \le d_i \le 6\]
                  Notice that this is equivalent to if $0 \le d_i \le 5$ and they had to sum to $9$.

                  Like in \ref*{list:count-digits}, only one $d_i$ can violate the constraints.
                  With a violated constraint, we have $\sum_{i=1}^{3} d_i=3$ with $\binom{3+3-1}{3}$ solutions.

                  Thus, the answer to this part is $\boxed{\binom{11}{9}-3 \cdot \binom{5}{3}}$.
            \item Let's say all dice started at $6$, and want to count the number of ways to subtract values from them such that their sum was $15$.
            Notice that this becomes the number of solutions to
            \[\sum_{i=1}^3 d_i=3 \quad 0 \le d_i \le 5\]
            which is much easier to handle because the upper bound never comes into effect.
            Using stars and bars, we get $\boxed{\binom{5}{3}}$.
        \end{enumerate}
    \item[A3] First, notice that there's $7$ possible values for $4x_1+4x_2$.
        We first solve this part of the equation, noticing that there's $x+1$ solutions to $4x_1+4x_2=4x$.

        After this, we have $x_3+x_4=24-x$, which has $24-4x+1$ solutions.

        Summing over all values of $x$ gives us
        \[\sum_{x=0}^6 (x+1)(24-4x+1)=\boxed{252}\]
    \item[A4] This is basic stars and bars.
        \[\binom{5+4-1}{4-1} \cdot \binom{3+4-1}{4-1}=\boxed{1120}\]
\end{itemize}
\end{document}
