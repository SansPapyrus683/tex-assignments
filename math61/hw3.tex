\documentclass[12pt]{article}

% a template that a friend gave, it's worked well enough for me
% i have added some packages and stuff that have proved useful

\usepackage{fancyhdr}
\usepackage{tipa}
\usepackage{fontspec}
\usepackage{amsfonts}
\usepackage{enumitem}
\usepackage[margin=1in]{geometry}
\usepackage{graphicx}
\usepackage{float}
\usepackage{amsmath}
\usepackage{braket}
\usepackage{amssymb}
\usepackage{booktabs}
\usepackage{hyperref}
\usepackage{mathtools}
\usepackage{xcolor}
\usepackage{float}
\usepackage{algpseudocodex}
\usepackage{titlesec}
\usepackage{bbm}

\pagestyle{fancy}
\fancyhf{} % sets both header and footer to nothing
\lhead{Kevin Sheng}
\setmainfont{Comic Neue}
\renewcommand{\headrulewidth}{1pt}
\setlength{\headheight}{0.75in}
\setlength{\oddsidemargin}{0in}
\setlength{\evensidemargin}{0in}
\setlength{\voffset}{-.5in}
\setlength{\headsep}{10pt}
\setlength{\textwidth}{6.5in}
\setlength{\headwidth}{6.5in}
\setlength{\textheight}{8in}
\renewcommand{\headrulewidth}{0.5pt}
\renewcommand{\footrulewidth}{0.3pt}
\setlength{\textwidth}{6.5in}
\usepackage{setspace}
\usepackage{multicol}
\usepackage{float}
\setlength{\columnsep}{1cm}
\setlength\parindent{24pt}
\usepackage [english]{babel}
\usepackage [autostyle, english = american]{csquotes}
\MakeOuterQuote{"}

\setlength{\parskip}{6pt}
\setlength{\parindent}{0pt}

\titlespacing\section{0pt}{12pt plus 4pt minus 2pt}{0pt plus 2pt minus 2pt}
\titlespacing\subsection{0pt}{12pt plus 4pt minus 2pt}{0pt plus 2pt minus 2pt}
\titlespacing\subsubsection{0pt}{12pt plus 4pt minus 2pt}{0pt plus 2pt minus 2pt}

\hypersetup{colorlinks=true, urlcolor=blue}

\newcommand{\correction}[1]{\textcolor{red}{#1}}


\begin{document}
\section{Textbook}\label{sec:textbook}

All problems are from section 6.1 of the textbook.

\begin{enumerate}
      \item[5] There's $2^{2 \cdot 3}-1=\boxed{63}$ distinct Braille characters.
      \item[9] If repetitions are allowed, we have $26^3 \cdot 10^2=\boxed{1757600}$ possible license plates.
            If we can't repeat a character, then this number decreases to
            $(26 \cdot 25 \cdot 24) \cdot (10 \cdot 9)=\boxed{1404000}$.
      \item[28] 6 outcomes give $7$, while 2 outcomes give $11$.
      \item[39] There's a total of $2^8$ eight-bit strings, $2^6$ of which
            have both the second and the fourth bit set to $0$.
            Excluding those, our final answer is $\boxed{2^8-2^6}$.
      \item[45] Excluding B and F, we have four remaining people to choose three positions four.
            Thus, our answer is $4 \cdot 3 \cdot 2=\boxed{24}$.
      \item[46] There's $3 \cdot 2=6$ ways to choose positions for B and F, and $4$ people that can fill
            whatever the remaining position is.
            This gives us our final answer of $6 \cdot 4 = \boxed{24}$.
      \item[78] Since all the terms in each polynomial are distinct, we can just multiply
            the lengths of all the terms together to get
            \[2 \cdot 3 \cdot 3 \cdot 2=\boxed{36}\]
\end{enumerate}

\pagebreak

\section{Handout}\label{sec:handout}
\begin{itemize}
      \item[A1] Each function has to map all $m$ elements to one of $n$ possible outputs,
            which means that there's $\boxed{n^m}$ possible functions from $X$ to $Y$.
      \item[A2] Say we line up the elements of $X$ and $Y$.
            Notice that we can form any bijection by drawing lines from
            an element in $X$ to the one in $Y$ right across from it.

            Keeping the positions of the elements in $X$ constant, we have $n!$ ways to arrange
            the elements of $Y$.
            Since there is a bijection between all these arrangements of $Y$
            and the possible bijections one can form between $X$ and $Y$ (bijection-ception?),
            we know that there's $n!$ bijections between $X$ and $Y$.
      \item[A3] For the first value of $X$, there's $n$ possible values of $Y$ we can map it to.
            As for the second, there's $n-1$ possible values.
            Extending this all the way to the $k$-th element, we get that there's
            a total of $\prod_{i=1}^k n-i+1=\frac{n!}{(n-k)!}$ injective functions. $\square$
      \item[A4] \textbf{Edge case $k=0$:} \\
            There's only one subset of size 0, and that's $\varnothing$.
            Plugging $k=0$ into the formula also yields $\frac{n!}{0!(n-0)!}=1$, so this proposition holds for $k=0$.

            \textbf{General $k>0$:} \\
            Define $[k]=\{i \in \mathbb{Z}: 1 \le i \le k\}$.
            By the result in A3 we know that there's $\frac{n!}{(n-k)!}$ ways to map each
            number in $[k]$ injectively to elements in $X$.
            However, we also know that there's $k!$ ways to form a bijection between $[k]$
            and any certain subset of size $k$ in $X$.
            Thus, we have to divide the former by the latter, giving us our result of $\frac{n!}{k!(n-k)!}$. $\square$

      \item[A5] We know that $\binom{n}{k}$ is equivalent to the number of all subsets of size $k$.
            We also know that $2^n$ is the total number of subsets without regard to size.

            Thus, summing up all the subsets of every size is clearly equal to the ultimate number of subsets. $\square$
\end{itemize}
\end{document}
