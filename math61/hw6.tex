\documentclass[12pt]{article}

% a template that a friend gave, it's worked well enough for me
% i have added some packages and stuff that have proved useful

\usepackage{fancyhdr}
\usepackage{tipa}
\usepackage{fontspec}
\usepackage{amsfonts}
\usepackage{enumitem}
\usepackage[margin=1in]{geometry}
\usepackage{graphicx}
\usepackage{float}
\usepackage{amsmath}
\usepackage{braket}
\usepackage{amssymb}
\usepackage{booktabs}
\usepackage{hyperref}
\usepackage{mathtools}
\usepackage{xcolor}
\usepackage{float}
\usepackage{algpseudocodex}
\usepackage{titlesec}
\usepackage{bbm}
\usepackage{pythonhighlight}

\pagestyle{fancy}
\fancyhf{} % sets both header and footer to nothing
\lhead{Kevin Sheng}
\setmainfont{Comic Neue}
\renewcommand{\headrulewidth}{1pt}
\setlength{\headheight}{0.75in}
\setlength{\oddsidemargin}{0in}
\setlength{\evensidemargin}{0in}
\setlength{\voffset}{-.5in}
\setlength{\headsep}{10pt}
\setlength{\textwidth}{6.5in}
\setlength{\headwidth}{6.5in}
\setlength{\textheight}{8in}
\renewcommand{\headrulewidth}{0.5pt}
\renewcommand{\footrulewidth}{0.3pt}
\setlength{\textwidth}{6.5in}
\usepackage{setspace}
\usepackage{multicol}
\usepackage{float}
\setlength{\columnsep}{1cm}
\setlength\parindent{24pt}
\usepackage [english]{babel}
\usepackage [autostyle, english = american]{csquotes}
\MakeOuterQuote{"}

\setlength{\parskip}{6pt}
\setlength{\parindent}{0pt}

\titlespacing\section{0pt}{12pt plus 4pt minus 2pt}{0pt plus 2pt minus 2pt}
\titlespacing\subsection{0pt}{12pt plus 4pt minus 2pt}{0pt plus 2pt minus 2pt}
\titlespacing\subsubsection{0pt}{12pt plus 4pt minus 2pt}{0pt plus 2pt minus 2pt}

\hypersetup{colorlinks=true, urlcolor=blue}

\newcommand{\correction}[1]{\textcolor{red}{#1}}


\begin{document}
\section{Handout}\label{sec:handout}
\begin{enumerate}
    \item \begin{enumerate}
              \item $\Omega=\{R, B\}^3$.
                    The corresponding probabilities are as follows:
                    \begin{align*}
                        P((R, R, R)) = \frac{2^3}{3^3} &  & P((R, R, B)) = \frac{2^2}{3^3} \\
                        P((R, B, R)) = \frac{2^2}{3^3} &  & P((R, B, B)) = \frac{2}{3^3}   \\
                        P((B, R, R)) = \frac{2^2}{3^3} &  & P((B, R, B)) = \frac{2}{3^3}   \\
                        P((B, B, R)) = \frac{2}{3^3}   &  & P((B, B, B)) = \frac{1}{3^3}
                    \end{align*} \label{list:1a}
              \item $\Omega=\{(a, b, c) \in \{R, B\}^3\ |\ \text{Only one is B}\}$. \\
                    Probabilities:
                    \[P((R, R, B))=P((R, B, R))=P((B, R, R))=\frac{1}{3}\]
              \item The $\Omega$ and probability function are exactly the same as in \ref{list:1a}.
                    Just replace the letters $R$ and $B$ with $T$ and $H$ respectively and you're done.
          \end{enumerate}
    \item Let's order the robberies from $1$ to $6$.
          The chance that robbery $i$ didn't happen in a city that was already "taken" by a previous robbery is $\frac{6-i+1}{6}$.
          Thus, the chance of all robberies occurring in different cities is
          \[\prod_{i=1}^{6} \frac{6-i+1}{6}=\frac{6!}{6^6}=\frac{5}{324}\]

          To get our actual answer, we take the complement of this, that is, $\boxed{\frac{319}{324}}$. \label{list:2}
    \item This problem is equivalent to \ref{list:2}. \label{list:3}
    \item Let's do this by counting the sequences of valid dice rolls.
          We can form a sequence of nondecreasing dice rolls by having die $i$ be represented by a value $\delta_i$,
          which indicates how much greater their value is than the previous one.

          Each $\delta_i$ must be nonnegative, and their sum must be no greater than $5$.
          The number of possible values of $\delta_i$ is found to be $\binom{6+5-1}{5}$ using stars and bars.
          Putting this over the total number of possible dice rolls, we get $\boxed{\frac{\binom{10}{5}}{6^6}}$.
    \item We can solve this by forming a recurrence relation, where $P_i$ indicates the number of length-$i$ coin tosses that don't have two heads in a row.
          The base cases are $P_0=1$ and $P_1=2$, and our recurrence is $P_i=P_{i-1}+P_{i-2}$
          by a result in HW5.

          $P_5=13$, and there's a total of $2^5=32$ outcomes from $5$ tosses, so our answer is $\boxed{\frac{13}{32}}$.
    \item To count the number of injective functions, we first note that there are $\binom{n}{m}$
          possible subsets that $X$ can map to if each input is to have a distinct output.
          Then, there's $m!$ possible ways that each subset can be mapped to the values of $X$, since order matters.

          We also have $n^m$ total possible functions that are possible, since
          each of the $m$ inputs can be mapped to $n$ possible outputs.

          Thus, our final answer is $\boxed{\frac{\binom{n}{m}m!}{n^m}}$.
          \ref{list:2} and \ref{list:3} relate to this in that they're a special case when $n=m$.
    \item \begin{enumerate}
              \item \boxed{\frac{r}{r+b}}
              \item \boxed{\frac{r}{r+b}} (yes, it's the same answer)
          \end{enumerate}
    \item Elmer should choose the father-champion-father set, since he has a higher chance of beating his father.
    \item \begin{enumerate}
              \item Not independent.
                    \begin{enumerate}
                        \item If I know I got exactly 3 heads, then I know I got exactly 4 tails as well.
                        \item $P(E_1)=\frac{\binom{7}{3}}{2^7}$ and $P(E_2)=\frac{\binom{7}{4}}{2^7}$, while $P(E_1 \cap E_2)=P(E_1)=P(E_2)$.
                              Given this, it's clear that $P(E_1 \cap E_2) \ne P(E_1)P(E_2)$.
                    \end{enumerate}
              \item Independent.
                    \begin{enumerate}
                        \item If we know that $f(1)=1$, then we know that $2$ isn't taken by $f(1)$ which makes
                              it ever so slightly more likely that $f(2)=2$.
                        \item $P(E_1)=\frac{1}{5}$ since it's $f(1)$ is equally likely to be assigned any number,
                              and $P(E_2)=\frac{1}{5}$ by the same logic.
                              \[P(E_1 \cap E_2)=\frac{3!}{5!}=\frac{1}{20} \ne \frac{1}{5} \cdot \frac{1}{5}\]
                    \end{enumerate}
          \end{enumerate}
    \item If we set $P_i$ as the number of toss sequences with an even number of heads,
          we'll find that $P_i=P_{i-1}+(2^{i-1}-P_{i-1})=2^{i-1}$.
          Then, $\frac{2^{i-1}}{2^i}=\boxed{\frac{1}{2}}$.
\end{enumerate}
\end{document}
