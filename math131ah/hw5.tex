\documentclass[12pt]{article}

% a template that a friend gave, it's worked well enough for me
% i have added some packages and stuff that have proved useful

\usepackage{fancyhdr}
\usepackage{tipa}
\usepackage{fontspec}
\usepackage{amsfonts}
\usepackage{enumitem}
\usepackage[margin=1in]{geometry}
\usepackage{graphicx}
\usepackage{float}
\usepackage{amsmath}
\usepackage{braket}
\usepackage{amssymb}
\usepackage{booktabs}
\usepackage{hyperref}
\usepackage{mathtools}
\usepackage{xcolor}
\usepackage{float}
\usepackage{algpseudocodex}
\usepackage{titlesec}
\usepackage{bbm}
\usepackage{pythonhighlight}

\pagestyle{fancy}
\fancyhf{} % sets both header and footer to nothing
\lhead{Kevin Sheng}
\setmainfont{Comic Neue}
\renewcommand{\headrulewidth}{1pt}
\setlength{\headheight}{0.75in}
\setlength{\oddsidemargin}{0in}
\setlength{\evensidemargin}{0in}
\setlength{\voffset}{-.5in}
\setlength{\headsep}{10pt}
\setlength{\textwidth}{6.5in}
\setlength{\headwidth}{6.5in}
\setlength{\textheight}{8in}
\renewcommand{\headrulewidth}{0.5pt}
\renewcommand{\footrulewidth}{0.3pt}
\setlength{\textwidth}{6.5in}
\usepackage{setspace}
\usepackage{multicol}
\usepackage{float}
\setlength{\columnsep}{1cm}
\setlength\parindent{24pt}
\usepackage [english]{babel}
\usepackage [autostyle, english = american]{csquotes}
\MakeOuterQuote{"}

\setlength{\parskip}{6pt}
\setlength{\parindent}{0pt}

\titlespacing\section{0pt}{12pt plus 4pt minus 2pt}{0pt plus 2pt minus 2pt}
\titlespacing\subsection{0pt}{12pt plus 4pt minus 2pt}{0pt plus 2pt minus 2pt}
\titlespacing\subsubsection{0pt}{12pt plus 4pt minus 2pt}{0pt plus 2pt minus 2pt}

\hypersetup{colorlinks=true, urlcolor=blue}

\newcommand{\correction}[1]{\textcolor{red}{#1}}


\rhead{Math 131AH}

\makeatletter
\def\@seccntformat#1{%
  \expandafter\ifx\csname c@#1\endcsname\c@section\else
  \csname the#1\endcsname\quad
  \fi}
\makeatother

\newcommand{\lra}{\xLeftrightarrow}
\newcommand{\ra}{\xRightarrow}

\begin{document}

\section{Problem 1}

To see what this actually converges to, notice that for $n > 1$
\[a_n = \prod_{i=1}^{n-1} \left(1-\frac{1}{(i+1)^2}\right)\]
which is true by a straightforward induction proof.

Now, notice
\begin{align*}
  a_n
  =\prod_{i=1}^{n-1} \left(1-\frac{1}{(i+1)^2}\right)
   & = \prod_{i=2}^{n} \frac{i^2-1}{i^2}            \\
   & = \prod_{i=2}^{n} \frac{(i-1)(i+1)}{i^2}       \\
   & = \frac{1}{2} \cdot \frac{n+1}{n}              \\
   & = \frac{1}{2} \cdot \left(1+\frac{1}{n}\right)
\end{align*}
by telescoping products.

From lecture, $\lim_{n \to \infty} \frac{1}{n}=0$ so the limit of this sequence is $\boxed{\frac{1}{2}}$.
\section{Problem 2}

For convenience let $L=\sup A$.

Take any $a_1 \in A$ as our starting element.
$L \notin A$, so let $\delta = L-a_1 > 0$.

Now we define our sequence:
\[a_{n+1} = \text{any element in $A$ s.t. } a_{n+1} > \max\left(a_n, L - \frac{\delta}{n}\right)\]
with the exception of $a_1$, which we defined by hand above.

Such an element is guaranteed to exist.
If $a_n \ge L-\frac{\delta}{n}$, then using that $L \notin A$
we know that there's always a bigger element in $A$.
OTOH, if the fraction is bigger, there's still a guranteed element
since $\forall \epsilon > 0\ \exists a \in A: a > \sup A - \epsilon$.

To show that this converges, consider any $\epsilon > 0$.
$\exists n_\epsilon \in \N: \frac{\delta}{n} < \epsilon$,
so let's use that as our breakpoint.
$L - \epsilon < L-\frac{\delta}{n_\epsilon} < a_{n_\epsilon} < L$,
and since $a_n$ is increasing and bounded above by $L$ (everything in the sequence is in $A$ as well),
we know this is true for all $n$ past $n_\epsilon$ as well. $\square$

\pagebreak

\section{Problem 3}

% OH MY GOD WHAT THE ACTUAL FUCK IS THIS PROBLEM HOLY JESUS FUCK MENZ

I'll use things like $\mathbf{0}$ and $\mathbf{1}$ to denote constant sequences.

\subsection{Equivalence Relation}

\subsubsection{Reflexive}

$\{a_n-a_n\}_{n \ge 1}=\mathbf{0}$, which obviously converges to zero.

\subsubsection{Symmetric}

If $\lim_{n \to \infty} \{a_n-b_n\}_{n \ge 1} = 0$,
\begin{align*}
  \lim_{n \to \infty} \{b_n-a_n\}_{n \ge 1}
   & = \lim_{n \to \infty} -\{a_n-b_n\}_{n \ge 1} \\
   & = -\lim_{n \to \infty} \{a_n-b_n\}_{n \ge 1} \\
   & = 0 \quad\square
\end{align*}

\subsubsection{Transitive}

We're given $\lim_{n \to \infty} \{a_n-b_n\}_{n \ge 1}=\lim_{n \to \infty} \{c_n-b_n\}_{n \ge 1}=0$.
With that,
\begin{align*}
  \lim_{n \to \infty} \{a_n-c_n\}_{n \ge 1}
   & = \lim_{n \to \infty} \{a_n-b_n+b_n-c_n\}_{n \ge 1}                                     \\
   & = \lim_{n \to \infty} \{a_n-b_n\}_{n \ge 1} + \lim_{n \to \infty} \{b_n-c_n\}_{n \ge 1} \\
   & = 0 + 0                                                                                 \\
   & = 0 \quad\square
\end{align*}

\subsection{Closure}

I'll omit the $n \ge 1$ subscript for brevity.
Say we have two Cauchy sequences $\{a_n\}$ and $\{b_n\}$.

\subsubsection{Addition}

To show that $\{a_n+b_n\}$ is Cauchy, notice that for any $\epsilon$
we can pick an $n_1$ s.t. $\forall n, m \ge n_1$ we have $|a_n-a_m| < \frac{\epsilon}{2}$
and a similar $n_2$ for $b_n$.

Taking $n_\epsilon = \max(n_1, n_2)$,
we now have that $\forall n, m \ge n_\epsilon$
\begin{align*}
  |a_n+b_n - a_m-b_m|
   & \le |a_n-a_m| + |b_n-b_m|                   \\
   & \le \frac{\epsilon}{2} + \frac{\epsilon}{2} \\
   & = \epsilon\quad\square
\end{align*}

\subsubsection{Multiplication}

All Cauchy sequences are bounded, so let $|a_n|, |b_n| < M$ for all $n$.

For any $\epsilon$, pick an $n_1$ s.t. $\forall n, m \ge n_1$
we have $|a_n-a_m| < \frac{\epsilon}{2M}$ and a similar $n_2$ for $b_n$.

Taking $n_\epsilon = \max(n_1, n_2)$,
we now have that $\forall n, m \ge n_\epsilon$
\begin{align*}
  |a_nb_n - a_mb_m|
   & = |a_nb_n - a_nb_m + a_nb_m - a_mb_m|                       \\
   & \le |a_nb_n - a_nb_m| + |a_nb_m - a_mb_m|                   \\
   & = |a_n||b_n - b_m| + |b_m||a_n - a_m|                       \\
   & < M \cdot \frac{\epsilon}{2M} + M \cdot \frac{\epsilon}{2M} \\
   & = \epsilon\quad\square
\end{align*}

\subsection{Valid Operation Definitions}

Let $\{a_n\} \sim \{c_n\}$ and $\{b_n\} \sim \{d_n\}$.

\subsubsection{Addition}

For addition, we NTS $\{a_n + b_n\} \sim \{c_n + d_n\}$:
\begin{align*}
  \lim_{n \to \infty} \{a_n+b_n-c_n-d_n\}
   & = \lim_{n \to \infty} \{a_n-c_n\} + \lim_{n \to \infty} \{b_n-d_n\} \\
   & = 0+0                                                               \\
   & = 0
\end{align*}

\subsubsection{Multiplication}

For this, we NTS $\lim_{n \to \infty} \{a_nb_n-c_nd_n\} = 0$.
First, take $M \in \Q$ s.t. it's greater than the absolute values of all the elements of all the sequences,
since we know they're all Cauchy and therefore bounded.

Then for any $\epsilon > 0$, since $\{a_n\} \sim \{c_n\}$ and $\{b_n\} \sim \{d_n\}$,
we can choose $n_1 \in \N: |a_n-c_n| < \frac{\epsilon}{2M}\ \forall n > n_1$
and a similar $n_2$ for $|b_n-d_n|$.

Taking $n_\epsilon = \max(n_1, n_2)$, we have
\begin{align*}
  |a_nb_n-c_nd_n|
   & = |a_nb_n-b_nc_n+b_nc_n-c_nd_n|                             \\
   & \le |a_nb_n-b_nc_n| + |b_nc_n-c_nd_n|                       \\
   & = |b_n||a_n - c_n| + |c_n||b_n-d_n|                         \\
   & < M \cdot \frac{\epsilon}{2M} + M \cdot \frac{\epsilon}{2M} \\
   & = \epsilon\quad\square
\end{align*}

\subsection{Field Axioms}

\subsubsection{Commutativity, Associativity, and Distributivity}

All of these come for free since these properties are true for $\Q$.
Here we're just applying them elementwise across the sequences,
and no matter what order we do the operations in the resulting sequences
will always be the same.

\subsubsection{Identities}

$\mathbf{0}$ is the additive identity.
Adding $0$ to each element of a sequence doesn't do anything to it.

For the multiplicative identity, we have $\mathbf{1}$,
since multiplying each element of a sequence by $1$ doesn't change anything either.

\subsubsection{Additive Inverse}

The additive inverse for $\{a_n\}$ is $\{-a_n\}$.
This is because $\forall n \in \N$
\[a_n + (-a_n) = 0\]
which gives us the resulting sequence of $\mathbf{0}$, our additive identity.

\subsubsection{Multiplicative Inverse}

If $\{a_n\}$ doesn't converge to $0$,
$\exists \epsilon > 0: \forall n_\epsilon \in \N\ \exists n \ge n_\epsilon: |a_n| > \epsilon$.

$\{a_n\}$ is also Cauchy, so $\exists x \in \N: |a_n-a_m| < \epsilon\ \forall n, m \ge x$.

Take the $n_t$ that's at least $x$ where $|a_{n_t}| > \epsilon$.
Then for all $i \ge n_t$ we have $a_i \ne 0$,
because if it \textit{was} $0$, $|a_i-a_{n_t}| = |a_{n_t}| < \epsilon$, which is a contradiction.

Now $\{a_n\}^{-1}$ is defined like so:
\[a_n = \begin{cases}
    a_n           & n < n_t          \\
    \frac{1}{a_n} & \text{otherwise}
  \end{cases}\]
We can take the reciprocal of anything after $n_t$ because it's all nonzero past it.

Then $\{a_n\} \cdot \{a_n\}^{-1}$ is going to be all $1$s past a certain point,
so we know it has to converge to $1$ (which is what $\mathbf{1}$ also converges to).

\pagebreak

\subsection{Order Relations}

\subsubsection{Well-Defined}

We NTS that if $\{a_n\} \sim \{c_n\}$, $\{b_n\} \sim \{d_n\}$,
and $\{a_n\} < \{b_n\}$, then $\{c_n\} < \{d_n\}$.

By definition, $\exists N \in \N: a_n < b_n\ \forall n \ge N$.

The inequality is strict, so take $\epsilon=\frac{b_n-a_n}{2}>0$ so $a_n + \epsilon \le b_n - \epsilon\ \forall n > N$ as well.

Choose $n_1$ s.t. $|a_n-c_n| < \epsilon\ \forall n \ge n_1$,
and $n_2$ s.t. $|b_n-d_n| < \epsilon\ \forall n \ge n_2$.

I propose that $b_n < d_n\ \forall n \ge N' = \max(N, n_1, n_2)$.
This is because for any $n \ge N'$ we have
\[|a_n-c_n| < \epsilon \therefore -\epsilon < a_n-c_n < \epsilon \therefore c_n < a_n +\epsilon\]
and
\[|b_n-d_n| < \epsilon \therefore -\epsilon < b_n-d_n < \epsilon \therefore b_n - \epsilon < d_n\]
so we have the inequality chain
\[c_n < a_n + \epsilon < b_n - \epsilon < d_n\quad\square\]

\subsubsection{Trichotomy}

We first prove that at least one is true by showing
$[a_n] \nless [\mathbf{0}] \land [a_n] \ngtr [\mathbf{0}] \ra{} [a_n] = [\mathbf{0}]$.

It STP $\lim_{n \to \infty} a_n = 0$, so we need an $n_\epsilon$ for any $\epsilon > 0$ and we're free.

By our assumptions, past any $N$ we can always find a nonpositive and nonnegative elemnt.
Since $\{a_n\}$ is Cauchy, take $M: |a_n-a_m| < \epsilon\ \forall n, m \ge M$.
There must exist some $a_x \le 0$ past $M$, so $|a_n-a_x| < \epsilon \therefore a_n < \epsilon\ \forall n \ge M$. 

Similarly, we can find an $a_y \ge 0$ past $M$, so $|a_y-a_n| < \epsilon \therefore a_n > -\epsilon\ \forall n \ge M$.
This inequality combined with the one above shows that $|a_n| < \epsilon\ \forall n \ge M$,
so I guess our $n_\epsilon$ is just $M$.

Now, to show that these categories are mutually exclusive,
notice that to actually be greater or less than $[\mathbf{0}]$
a sequence has to by definition not converge to $0$, so that's a given.
Also, if a sequence is always strictly positive past a certain point,
it definitely can't be strictly negative past any other point. $\square$

\pagebreak

\subsubsection{Addition Closure}

Suppose we have $[a_n] > [\mathbf{0}]$ and $[b_n] > [\mathbf{0}]$.

By definition, we have $N_a \in \N: a_n > 0\ \forall n \ge N_a$ and a similar $N_b$ for $\{b_n\}$.
We also have $\epsilon_a > 0: \forall n_{\epsilon_a} \in \N\ \exists n_a \ge n_{\epsilon_a}: |a_n| \ge \epsilon_a$.
and a similar $\epsilon_b > 0$.

For addition, we have to prove $[a_n+b_n]$ is positive past an $N \in \N$ and doesn't converge to $0$.

The first part is quite simple: we take $N=\max(N_a, N_b)$.
Then we have $a_n+b_n > 0\ \forall n \ge N$, since two positives add to another positive.

Take $\epsilon = \epsilon_a$.
We NTS $\forall n_\epsilon \in \N\ \exists n \ge n_\epsilon: |a_n+b_n| \ge \epsilon$.

By how we defined $N$ and that $a_n$ doesn't converge to $0$,
for any $n_\epsilon$ we can find an $n$ after it s.t. $a_n, b_n > 0$ and $|a_n| \ge \epsilon_a$.
This then gives us
\[|a_n+b_n|=a_n+b_n \ge \epsilon + b_n > \epsilon\]
so for any $n_\epsilon$ we can always find an $n$ after where $|a_n+b_n| \ge \epsilon$.

\subsubsection{Multiplication Closure}

We now prove the same things for multiplication.

Again, past $N = \max(N_a, N_b)$, $a_nb_n > 0$
since two a positive times a positive is still positive.

Now I'll show that if $\epsilon=\frac{\epsilon_a\epsilon_b}{2}$,
we can always find an $n$ past any $n_\epsilon$ s.t. $|a_nb_n-0|=|a_nb_n| \ge \epsilon$

Since $b_n$ is Cauchy, we can also find an $M \in \N: |b_x-b_y| < \frac{\epsilon_b}{2}\ \forall, x, y \ge M$.

Now, past anything greater than $\max(N, M)$, there must exist an $n$ and $m$ where
\begin{itemize}[nolistsep]
  \item $a_n, b_m, b_n > 0$
  \item $a_n \ge \epsilon_a$
  \item $b_m \ge \epsilon_b$
  \item $|b_n-b_m| < \frac{\epsilon_b}{2} \ra{} b_n > b_m - \frac{\epsilon_b}{2} \ge \frac{\epsilon_b}{2}$
\end{itemize}
so we can write
\[|a_nb_n|=a_nb_n \ge \frac{\epsilon_a\epsilon_b}{2}=\epsilon\]
which is exactly what we wanted. $\square$

With all these tedious properties and definitions finally proven,
we can conclue that these cursed Cauchy sequences do indeed form a godforsaken Ordered Field\textsuperscript{TM}.

\pagebreak

\section{Problem 4}\label{sec:p4}

For any $n \ge \N$, consider the three sets
\begin{align*}
  S_n=\{a_k+b_k \mid k \ge n\}
   &  & A_n=\{a_k \mid k \ge n\}
   &  & B_n=\{b_k \mid k \ge n\}
\end{align*}
This way, we can now denote the inequality we WTS as
\[\lim_{n \to \infty} \sup S_n \le \lim_{n \to \infty} \sup A_n + \lim_{n \to \infty} \sup B_n\]

It STP $\sup S_n \le \sup A_n + \sup B_n\ \forall n \in \N$.
I'll omit the subscripts for convenience.

Recall from HW 3 that
\[\sup \{x+y \mid x \in A, y \in B\} = \sup A + \sup B\]
since both sequences are bounded.
Denoting the LHS set as $A+B$, notice that $S \subseteq A+B$,
so $\sup S \le \sup A+B = \sup A + \sup B$. $\square$

\section{Problem 5}

Let $A_n$ and $B_n$ be as they were in \ref{sec:p4} and $S_n=\{a_kb_k \mid k \ge n\}$.

Again from HW 3, we know that if we let $A \cdot B = \{a \cdot b \mid a \in A, b \in B\}$
where $A$ and $B$ are non-negative, $\sup A \cdot B = \sup A \cdot \sup B$.
(look i know the problem said \textit{positive} but allowing $0$ in the sets literally doesn't change any part of the proof)

Applying the same general logic as we did in \ref{sec:p4},
\[S \subseteq \sup A \cdot B \therefore \sup S \le \sup A \cdot B = \sup A \cdot \sup B\quad\square\]

\pagebreak

\section{Problem 6}

By definition,
\[\limsup_{n \to \infty} |a_n|=\infty \iff \text{$|a_n|$ is unbounded above}\]

Assuming that $\{a_n\}_{n \ge 1}$ is bounded, $\exists M \in \R: |a_n| < M\ \forall n \in \N$,
which then means $|a_n|$ is bounded from above, thus making $\limsup_{n \to \infty} |a_n| < \infty$.

OTOH, if we assume $\limsup{n \to \infty} |a_n| < \infty$, this means $|a_n|$ is bounded above,
or that $\exists M \in \R: |a_n| \in \R\ \forall n \in \N$.
This makes $a_n$ itself bounded as well by definition. $\square$

\section{Problem 7}

Let $B=\lim_{n \to \infty} b_n$.

First assume $B \in \R$.
For all elements $b_n$, there's a subsequence in $a_n$ that converges to them.
Let the subsequence corresponding to $b_x$ be $k^{(x)}_n$.
Since $a_{k^{(x)}_n}$ converges to $b_x$,
\[\forall \epsilon > 0\ \exists n_\epsilon \in \N: \left|a_{k^{(x)}_{n_\epsilon}}-b_x\right| < \epsilon\ \forall n \ge n_\epsilon\]

Now let's define the subsequence $i_n$.
If $\epsilon=|b_n - B|$, $i_n$ is the first term in $k_n$
that's greater than both $k_{n_\epsilon}$ and $i_{n-1}$.
If we're talking about the first term it just has to be greater than $k_{n_\epsilon}$.

By construction, $i_n$ is indeed a valid subsequence and $|a_{i_n} - b_n| < |b_n - B|\ \forall n \in \N$.

Now take any $\epsilon > 0$ (the old $\epsilon$ i used for sequence construction doesn't apply anymore).
Since $b_n$ converges, $\exists N \in \N: |b_n-B| < \frac{\epsilon}{2}\ \forall n \ge N$.
This gives us
\begin{align*}
  |a_{i_n}-B|
   & = |a_{i_n} - b_n + b_n - B|               \\
   & \le |a_{i_n} - b_n| + |b_n - B|           \\
   & < \frac{\epsilon}{2} + \frac{\epsilon}{2} \\
   & = \epsilon
\end{align*}
so we can see that $a_{i_n}$ itself also converges to $B$.

For the cases where $B=\pm \infty$, this just means that $A$ is either
unbounded above or below, which implies that the corresponding infinity is in $A$ as well. $\square$

\pagebreak

\section{Problem 8}

The middle inequality comes for free since from lecture we know that the
liminf of a seqeuence is always less than or equal to its limsup.

\subsection{Right Inequality}

\subsubsection{Real Numbers}

First let's handle the case where the limsup is a real number.

I'll show that $\limsup_{n \to \infty} s_n \le M\ \forall M > \limsup_{n \to \infty} a_n$.

For this $M$, since successive supremums are nonincreasing,
$\exists N \in \N: \sup \{a_n \mid n > N\} < M$,
so taking $\epsilon = M - \sup \{a_n \mid n > N\} > 0$, we have $a_n + \epsilon < M\ \forall n > N$.

With this, we can show that
\begin{align*}
  Mn - \sum_{i=N+1}^{N+n} a_i
   & = \sum_{i=N+1}^{N+n} (M - a_i) \\
   & > \sum_{i=N+1}^{N+n} \epsilon  \\
   & = n\epsilon
\end{align*}

By the Archimedian Property, we can take $T \in \N: t\epsilon > \sum_{i=1}^{N} a_i - MN\ \forall t \ge T$.
We can say this because the term on the RHS is constant.
This gives us
\begin{gather*}
  t\epsilon > \sum_{i=1}^{N} a_i - MN \\
  Mt - \sum_{i=N+1}^{N+t} a_i > \sum_{i=1}^{N} a_i - MN \\
  M(N+t) > \sum_{i=1}^{N+t} a_i \\
  M > \frac{\sum_{i=1}^{N+t} a_i}{N+t} = s_{N+t}\ \forall t \ge T
\end{gather*}
So $\sup \{s_n \mid n \ge N+T\} \le M$.

Now for any $n \ge N+T$, we have
$\sup \{s_{n'} \mid n' \ge n\} \le M \therefore \limsup_{n \to \infty} s_n \le M$. $\square$

\subsubsection{Edge Cases}

Suppose $a_n$ wasn't BA, making $\limsup_{n \to \infty} a_n = +\infty$.
Then no matter what $\limsup_{n \to \infty} s_n$ is it's always going to be less than $\infty$.

OTOH, if $\limsup_{n \to \infty} a_n = -\infty$, we NTS $\limsup_{n \to \infty} s_n = -\infty$ as well.
With that limsup, for any $M < 0$ we can find $N \in \N: \sup \{a_n \mid n > N\} < M$.

Then with a proof that goes exactly the same way as it did in the previous section
with this $M$ that's some $\epsilon$ away from all $a_n$,
we can show that $\limsup_{n \to \infty} s_n \le M$.
Since this is true for all negative $M$, the only way for this to be true is if the limsup is $s_n$.

\subsection{Left Inequality}

The left inequality goes nearly the exact same way as the right one.
I don't think it's worth my time to write all that again here.

\pagebreak

\section{Problem 9}

$\{a_n\}_{n \ge 1}$ is BA, so we can assume $L \in \R$ (thank god).

\subsection{Part 1}\label{sec:prob9p1}

BWOC say $\exists \epsilon > 0$ s.t. there's infinitely many $a_n$ that are greater than $L+\epsilon$.
Then for any $n \in \N$, $\{a_{n'} \mid n' \ge n\}$ always has a number strictly greater than $L+\epsilon$.
If this isn't the case then there's at most $n$ numbers greater than $L+\epsilon$ which is a contradiction.

With this,
\begin{align*}
  \limsup_{n \to \infty} a_n
   & = \lim_{n \to \infty} \sup \{a_{n'} \mid n' \ge n\} \\
   & > \lim_{n \to \infty} L+\epsilon                    \\
   & = L+\epsilon                                        \\
   & > L
\end{align*}
which contradicts that $L$ is the limsup of the sequence. $\square$

\subsection{Part 2}\label{sec:prob9p2}

BWOC suppose $\exists \epsilon > 0$ s.t. there's only finitely many $a_n > L-\epsilon$.
This means $\exists n \in \N$ s.t. $a_{n'} \le L-\epsilon\ \forall n' \ge n$.

This then means $\sup \{a_{n'} \mid n' \ge n\} \le L-\epsilon$,
since the above implies that $L-\epsilon$ is a UB.

From lecture, we know tail supremums form a nonincreasing sequence, so we have
\[L \le \sup \{a_{n'} \mid n' \ge n\} \le L-\epsilon\]
which is a contradiction. $\square$

\pagebreak

\section{Problem 10}

\subsection{Not Bounded Above}

If $\{a_n\}_{n \ge 1}$ isn't BA, then no real number $L$ exists.

This is because for any real number $L$ and any positive $\epsilon$,
\[\exists N \in \N: a_n > L+\epsilon\ \forall n \ge N\]
by definition of converging to $\infty$, which means there's infinitely many $a_n$ greater than $L=\epsilon$.

\subsection{Assuming BA}

If the sequence \textit{is} BA, then only $L=\limsup_{n \to \infty} a_n$ satisfies this property.

Suppose there $\exists l > L$ that worked.
Take $0 < \delta < l-L$.
The second requirement needs infinitely many $a_n$ that are greater than $l-\delta$.
But $l-\delta > L$, so we can choose $\epsilon=l-\delta-L > 0$
to see that by \ref{sec:prob9p1} there's only finitely many $a_n$ that are greater than $l-\delta$,
which breaks the second constraint.

OTOH, suppose $\exists l < L$ that worked.
Then take $0 < \delta < L-l$ and $\epsilon = L-l-\delta > 0$.
The first requirement needs finitely many $a_n$ that are less than $l+\delta$,
but by \ref{sec:prob9p2} there's infinitely many elements greater than $L-\epsilon = l+\delta$,
so any $l < L$ breaks the first constraint.

Eliminating these two options means that only $L$ can meet the requirements described. $\square$

\end{document}
