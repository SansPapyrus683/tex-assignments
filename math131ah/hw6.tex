\documentclass[12pt]{article}

% a template that a friend gave, it's worked well enough for me
% i have added some packages and stuff that have proved useful

\usepackage{fancyhdr}
\usepackage{tipa}
\usepackage{fontspec}
\usepackage{amsfonts}
\usepackage{enumitem}
\usepackage[margin=1in]{geometry}
\usepackage{graphicx}
\usepackage{float}
\usepackage{amsmath}
\usepackage{braket}
\usepackage{amssymb}
\usepackage{booktabs}
\usepackage{hyperref}
\usepackage{mathtools}
\usepackage{xcolor}
\usepackage{float}
\usepackage{algpseudocodex}
\usepackage{titlesec}
\usepackage{bbm}

\pagestyle{fancy}
\fancyhf{} % sets both header and footer to nothing
\lhead{Kevin Sheng}
\setmainfont{Comic Neue}
\renewcommand{\headrulewidth}{1pt}
\setlength{\headheight}{0.75in}
\setlength{\oddsidemargin}{0in}
\setlength{\evensidemargin}{0in}
\setlength{\voffset}{-.5in}
\setlength{\headsep}{10pt}
\setlength{\textwidth}{6.5in}
\setlength{\headwidth}{6.5in}
\setlength{\textheight}{8in}
\renewcommand{\headrulewidth}{0.5pt}
\renewcommand{\footrulewidth}{0.3pt}
\setlength{\textwidth}{6.5in}
\usepackage{setspace}
\usepackage{multicol}
\usepackage{float}
\setlength{\columnsep}{1cm}
\setlength\parindent{24pt}
\usepackage [english]{babel}
\usepackage [autostyle, english = american]{csquotes}
\MakeOuterQuote{"}

\setlength{\parskip}{6pt}
\setlength{\parindent}{0pt}

\titlespacing\section{0pt}{12pt plus 4pt minus 2pt}{0pt plus 2pt minus 2pt}
\titlespacing\subsection{0pt}{12pt plus 4pt minus 2pt}{0pt plus 2pt minus 2pt}
\titlespacing\subsubsection{0pt}{12pt plus 4pt minus 2pt}{0pt plus 2pt minus 2pt}

\hypersetup{colorlinks=true, urlcolor=blue}

\newcommand{\correction}[1]{\textcolor{red}{#1}}


\rhead{Math 131AH}

\makeatletter
\def\@seccntformat#1{%
  \expandafter\ifx\csname c@#1\endcsname\c@section\else
  \csname the#1\endcsname\quad
  \fi}
\makeatother

\newcommand{\lra}{\xLeftrightarrow}
\newcommand{\ra}{\xRightarrow}
\newcommand{\N}{\mathbb{N}}
\newcommand{\R}{\mathbb{R}}
\newcommand{\Q}{\mathbb{Q}}

\begin{document}

\section{Problem 1}

Since $|a_n| \ge 0$ and liminfs are nondecreasing,
this must mean $\inf_{n \ge N} |a_n|=0\ \forall N \in \N$.

So for any $N$ and $\epsilon > 0$, we can find an $a_n$ s.t. $n \ge N$ and $|a_n| < \epsilon$.

Let's define the subsequence element by element.
$k_n$ is going to be any index $i$ s.t. $|a_i| < \frac{1}{2^n}$ and $n \ge k_{n-1}$.
The second condition doesn't apply if we're talking about $k_1$.
By what was said above, we can always find such a $k_n$.

This gives us a subsequence $a_{k_n}$ whose absolute value is always
less than $\frac{1}{2^n}$, a geometric series that definitely converges.
By the Comparison Test, the series of this subsequence must
absolutely converge, which implies convergence. $\square$

\section{Problem 2}

\subsection{Series 1}

Let's use the Ratio Test.
\begin{align*}
  \limsup_{n \to \infty} \left|\frac{\frac{(n+1)^4}{2^{n+1}}}{\frac{n^4}{2^n}}\right|
   & = \limsup_{n \to \infty} \left|\frac{\frac{(n+1)^4}{n^4}}{2}\right| \\
   & = \frac{\limsup_{n \to \infty} \frac{(n+1)^4}{n^4}}{2}              \\
   & = \frac{1}{2}
\end{align*}
I was able to take out the absolute value since both $(n+1)^4$ and $n^4$ are positive.

$\frac{1}{2}<1$, so the series converges.

\subsection{Series 2}

Let's use the Ratio Test again.
\begin{align*}
  \limsup_{n \to \infty} \left|\frac{\frac{2^{n+1}}{(n+1)!}}{\frac{2^n}{n!}}\right|
   & = \limsup_{n \to \infty} \left|\frac{2}{n+1}\right| \\
   & = 0
\end{align*}
$0 < 1$, so the series converges.

\subsection{Series 3 \& 4}

$(-1)^n$ and $\sin \left(\frac{n\pi}{3}\right)$ don't even converge to $0$; no way their series converges.

\pagebreak

\section{Problem 3}

\subsection{Series 1}

The numerator and denominator are both positive, so the sequence
only has positive elements and we can rearrange them without consequences.

Listing out the terms, we get
\[\frac{1}{(2+1)^2}, \frac{1}{(3-1)^2}, \frac{1}{(4+1)^2}, \frac{1}{(5-1)^2}, \frac{1}{(6+1)^2}, \cdots\]
Notice that this sequence is just $\{a_n\}_{n \ge 2}$ with $a_n=\frac{1}{n^2}$ and adjacent terms swapped.

From lecture we know the latter converges, so the OG series must as well.

\subsection{Series 2}

The $n$th partial sum is
\begin{align*}
  \sum_{i=1}^{n} \sqrt{i+1}-\sqrt{i}
   & = \sum_{i=2}^{n+1} \sqrt{i} - \sum_{i=1}^{n} \sqrt{i} \\
   & = \sqrt{n+1}-\sqrt{1}
\end{align*}
The partial sums pretty clearly diverge, so the series itself should too.

\subsection{Series 3}

Using the Ratio Test, we get that
\begin{align*}
  \frac{\frac{(n+1)!}{(n+1)^{n+1}}}{\frac{n!}{n^n}}
   & = \frac{n+1}{\frac{(n+1)^(n+1)}{n^n}}            \\
   & = \frac{n^n}{(n+1)^n}                            \\
   & = \left(\frac{n}{n+1}\right)^n                   \\
   & = \left(\left(1+\frac{1}{n}\right)^n\right)^{-1}
\end{align*}
The portion inside the inverse converges to $e$ as $n \to \infty$,
so the ratios as a whole converge to $\frac{1}{e}$ and the series converges.

\section{Problem 4}

\subsection{Series 1}

We use the Dyadic criterion to reduce the proof of convergence to showing that
\begin{align*}
  \sum_{n \ge 2} \frac{2^n \cdot \left(2^n\right)^{n \ln 2}}{(n \ln 2)^{2^n}}
\end{align*}
converges.

By the Comparison Test, we can chain inequalities to get
\begin{align*}
  \frac{2^n \cdot \left(2^n\right)^{n \ln 2}}{(n \ln 2)^{2^n}}
   & = \frac{\left(2^n\right)^{n \ln 2+1}}{(n \ln 2)^{2^n}} \\
   & \le \frac{\left(2^n\right)^{n \ln 2+1}}{2^{2^n}}       \\
   & = 2^{n^2\ln 2 + n - 2^n}                               \\
   & \le 2^{n^2 \cdot 2 + n^2 - 2^n}                        \\
   & = 2^{3n^2+2^n}
\end{align*}
To prove that the series defined by the formula above converges,
we use the Ratio Test:
\begin{align*}
  \frac{2^{3(n+1)^2-2^{n+1}}}{2^{3n^2-2^n}}
   & = 2^{6n+3-2^{n+1}+2^n} \\
   & = 2^{6n+3-2^n}
\end{align*}
This converges to $0$ iff $6n+3-2^n$ goes to $-\infty$.

To show this, notice that past $n=3$ the expression decreases by at least $1$.
The proof is by induction,
where the base case is $n=3$ and $n=4$:
\begin{gather*}
  6 \cdot 3 + 3 - 2^3 = 13 \\
  6 \cdot 4 + 3 - 2^4 = 11
\end{gather*}
and the inductive step goes like so:
\begin{align*}
  \left(6x+3-2^x\right) - \left(6(x+1)+3+2^{x+1}\right)
  &= 2^{x+1}-2^x - 6 \\
  &= 2^x - 6 \\
  &> 1
\end{align*}
since $x \ge 3$ and $2^x \ge 8$.
This means that the ratio goes to $0$, the series we used as a UB converges,
and the OG series converges too. $\square$

\subsection{Series 2}

Using the Dyadic criterion again gets us
\[\sum_{n \ge 2} \frac{1}{(\ln n)^{\ln n}}\text{ converges} \iff \sum_{n \ge 2} \frac{2^n}{(\ln 2^n)^{\ln 2^n}}\text{ converges}\]

We simplify the terms of the new series:
\[\frac{2^n}{(\ln 2^n)^{\ln 2^n}} = \frac{2^n}{(n \ln 2)^{n \ln 2}}\]
and use the Root Test:
\[\left(\frac{2^n}{(n \ln 2)^{n \ln 2}}\right)^{\frac{1}{n}} = \frac{2}{(n \ln 2)^{\ln 2}}\]
The numerator is constant and the bottom is increasing, so this converges to $0$.
$0 < 1$, so our new series and by extension the OG series both converge.

\subsection{Series 3}

The sequence itself doesn't converge. 'Nuff said.

\section{Problem 5}

\subsection{Divergent Then Convergent}

$\sum_{n \ge 1} \frac{1}{n}$ diverges, but $\sum_{n \ge 1} \frac{1}{n^2}$ converges.

\subsection{Convergent Both Times}

Define $b_n=|a_n|$ so $b_n \ge 0$.
Squaring makes things positive anyways so $\sum a_n^2=\sum b_n^2$.

If $b_n$ is convergent, then it must tend to $0$
and $\exists n_\epsilon: 0 \le b_n < 1\ \forall n \ge n_\epsilon$.

Since $0 \le b_n < 1\ \forall n \ge n_\epsilon$, $b_n^2 < b_n$ as well and we have
\begin{align*}
  \sum_{n=1}^{\infty} b_n^2
   & = \sum_{n=1}^{n_\epsilon-1} b_n^2 + \sum_{n=n_\epsilon}^{\infty} b_n^2 \\
   & < \sum_{n=1}^{n_\epsilon-1} b_n^2 + \sum_{n=n_\epsilon}^{\infty} b_n
\end{align*}
The first summand is finite, while the second
must converge since it's just the tail of an already converging series.

Thus, $\sum b_n^2$ must converge. $\square$

\subsection{Convergent Then Divergent}

$\sum_{n \ge 1} \frac{(-1)^n}{\sqrt{n}}$ converges since it's
alternating and its absolute value tends to $0$.
However, it's square is the harmonic series, which diverges.

\section{Problem 6}

We use telescoping sums to get the $n$th partial sum:
\begin{align*}
  \sum_{i=1}^{n} \frac{1}{i}-\frac{1}{i+1}
   & = \sum_{i=1}^{n} \frac{1}{I} - \frac{i=2}{n+1} \frac{1}{i} \\
   & = 1-\frac{1}{n+1}
\end{align*}
This clearly converges to $\boxed{1}$ as $n$ (and $n+1$) go to $\infty$.

\section{Problem 7}

\subsection{Auxiliary Series}

We again get the $n$th partial sum with telescoping sums:
\begin{align*}
  \sum_{i=1}^{n} \frac{i-1}{2^{i+1}}
   & = \sum_{i=1}^{n} \frac{i}{2^i} - \frac{i+1}{2^{i+1}}            \\
   & = \sum_{i=1}^{n} \frac{n}{2^n} - \sum_{i=2}^{n+1} \frac{n}{2^n} \\
   & = \frac{1}{2^1}-\frac{n+1}{2^{n+1}}
\end{align*}
and see that the subband tends to $0$, so the overall series converges to $\boxed{\frac{1}{2}}$.

\subsection{Actual Result}

We can multiply the initial series by $2^2$ to have
\begin{align*}
  4 \cdot \sum_{n \ge 1} \frac{n-1}{2^{n+1}}
   & = \sum_{n \ge 1} \frac{n-1}{2^{n-1}}           \\
   & = \frac{0}{2^0} + \sum_{n \ge 1} \frac{n}{2^n} \\
   & = \sum_{n \ge 1} \frac{n}{2^n}
\end{align*}
The LHS is equal to $2$, so the series on the RHS must be $\boxed{2}$.

\section{Problem 8}

\textbf{I'm omitting absolute values since we're only dealing with nonnegatives.}

\subsection{Sequence Over Itself}

If $\lim_{n \to \infty} a_n = a > 0$,
\[\lim_{n \to \infty} \frac{a_n}{a_n+1}=\frac{a}{a+1} > 0\]
and summing this to infinity will definitely produce a divergent series.

Now we handle the nontrivial case of $a=0$, which implies an $M: a_n < M$.
$a_n$ diverges, so it violates the Cauchy Criterion:
\[\exists \epsilon: \forall n_\epsilon \in \N\ \exists n \ge n_\epsilon, k \ge 0: \sum_{i=n}^{n+k} a_n \ge \epsilon\]

Notice that $\forall n, k \ge 0$,
\begin{align*}
  \sum_{i=n}^{n+k} \frac{a_n}{a_n+1}
   & > \sum_{i=n}^{n+k} \frac{a_n}{M+1} \\
   & = \frac{\sum_{i=n}^{n+k} a_n}{M+1} \\
   & \ge \frac{\epsilon}{M+1}
\end{align*}
thus proving that $\frac{\epsilon}{M+1}$ is the divergent $\epsilon$ for our new series. $\square$

\subsection{Sequence Over Partial Sums}

Since $a_n$ is nonnegative, $s_n$ monotonically increases and must diverge to $\infty$.

We have
\begin{align*}
  \sum_{k=1}^{n} \frac{a_{N+k}}{s_{N+k}}
   & \le \sum_{k=1}^{n} \frac{a_{N+k}}{s_{N+n}} \\
   & = \frac{\sum_{k=1}^{n} a_{N+k}}{s_{N+n}}   \\
   & = \frac{s_{N+n} - s_N}{s_{N+n}}            \\
   & = 1-\frac{s_N}{s_{N+n}}\quad\square
\end{align*}

Now we can arbitrarily take $\epsilon=\frac{1}{2}$ for the negation of convergence.

For any $n_\epsilon$, by the inequality we've proven we
can take $n=n_\epsilon+1$ and $k$ big enough s.t.
\[\sum_{i=n}^{n+k} \frac{a_i}{s_i} \ge 1-\frac{s_n}{s_{n+k}} \ge \frac{1}{2}\]
since $s_n$ is constant and we can make $s_{n+k}$ arbitrarily large.

Thus, even after dividing by partial sums, we still have a divergent series.

\subsection{Squaring the Sums}

We have
\begin{align*}
  \frac{1}{s_{n-1}} - \frac{1}{s_n}
   & = \frac{s_n-s_{n-1}}{s_ns_{n-1}}  \\
   & = \frac{a_n}{s_ns_{n-1}}          \\
   & \ge \frac{a_n}{s_n^2}\quad\square
\end{align*}

Now to show the convergence of $\sum_{n \ge 1} \frac{a_n}{s_n^2}$, we use the Comparison Test on $\sum_{n \ge 2} \frac{a_n}{s_n^2}$.

Notice that $\sum_{n \ge 2} \frac{1}{s_{n-1}}-\frac{1}{s_n}$ converges.
Taking $t_n$ as the $n$th partial sum, we get
\begin{align*}
  t_n
   & =\sum_{i=2}^{n} \frac{1}{s_{i-1}}-\frac{1}{s_i}                \\
   & =\sum_{i=1}^{n-1} \frac{1}{s_i} - \sum_{i=2}^{n} \frac{1}{s_i} \\
   & =\frac{1}{s_1} - \frac{1}{s_i}
\end{align*}
which converges to $\frac{1}{s_1}$ since $\lim_{n \to \infty} \frac{1}{s_n}=0$.

By the inequality above and the convergence of $\sum_{n \ge 2} \frac{1}{s_{n-1}}-\frac{1}{s_n}$,
$\sum_{n \ge 2} \frac{a_n}{s_n^2}$ converges.
This means that the OG series must converge too since it only has one more term.

\section{Problem 9}

Since $\sum a_n$ converges, the partial sums have to be Cauchy.
This means that $\forall \epsilon > 0\ \exists n_\epsilon \in \N$
s.t. all range sums which start past that point are strictly less than $\epsilon$.

Notice that for all $n \in \N$,
\[\sum_{i=n+1}^{2n} a_i \ge na_{2n}\]
For any $\epsilon > 0$, we can find an $n_\epsilon$ and take $n=n_\epsilon$
in the above expression to get $na_{2n} < \frac{\epsilon}{n}$.
This gives us $\lim_{n \to \infty} na_{2n}=0$.

Now, also notice that
\[\sum_{i=n+2}^{2n+1} a_i \ge na_{2n+1}\]
and we can use the same chain of logic as we did above to see that $\lim_{n \to \infty} na_{2n+1}=0$.

Now we use limit properties and that $\lim_{n \to \infty} a_n=0$ to see
\begin{gather*}
  \lim_{n \to \infty} na_{2n} = \lim_{n \to \infty} 2na_{2n}=0 \\
  \lim_{n \to \infty} na_{2n+1}= \lim_{n \to \infty} 2na_{2n+1}+a_{2n+1}= \lim_{n \to \infty} (2n+1)a_{2n+1}=0
\end{gather*}
The odd and even subsequences both converge to $0$,
so the overall subsequence $na_n$ must do the same. $\square$

\end{document}
