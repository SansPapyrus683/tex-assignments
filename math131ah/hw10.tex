\documentclass[12pt]{article}

% a template that a friend gave, it's worked well enough for me
% i have added some packages and stuff that have proved useful

\usepackage{fancyhdr}
\usepackage{tipa}
\usepackage{fontspec}
\usepackage{amsfonts}
\usepackage{enumitem}
\usepackage[margin=1in]{geometry}
\usepackage{graphicx}
\usepackage{float}
\usepackage{amsmath}
\usepackage{braket}
\usepackage{amssymb}
\usepackage{booktabs}
\usepackage{hyperref}
\usepackage{mathtools}
\usepackage{xcolor}
\usepackage{float}
\usepackage{algpseudocodex}
\usepackage{titlesec}
\usepackage{bbm}

\pagestyle{fancy}
\fancyhf{} % sets both header and footer to nothing
\lhead{Kevin Sheng}
\setmainfont{Comic Neue}
\renewcommand{\headrulewidth}{1pt}
\setlength{\headheight}{0.75in}
\setlength{\oddsidemargin}{0in}
\setlength{\evensidemargin}{0in}
\setlength{\voffset}{-.5in}
\setlength{\headsep}{10pt}
\setlength{\textwidth}{6.5in}
\setlength{\headwidth}{6.5in}
\setlength{\textheight}{8in}
\renewcommand{\headrulewidth}{0.5pt}
\renewcommand{\footrulewidth}{0.3pt}
\setlength{\textwidth}{6.5in}
\usepackage{setspace}
\usepackage{multicol}
\usepackage{float}
\setlength{\columnsep}{1cm}
\setlength\parindent{24pt}
\usepackage [english]{babel}
\usepackage [autostyle, english = american]{csquotes}
\MakeOuterQuote{"}

\setlength{\parskip}{6pt}
\setlength{\parindent}{0pt}

\titlespacing\section{0pt}{12pt plus 4pt minus 2pt}{0pt plus 2pt minus 2pt}
\titlespacing\subsection{0pt}{12pt plus 4pt minus 2pt}{0pt plus 2pt minus 2pt}
\titlespacing\subsubsection{0pt}{12pt plus 4pt minus 2pt}{0pt plus 2pt minus 2pt}

\hypersetup{colorlinks=true, urlcolor=blue}

\newcommand{\correction}[1]{\textcolor{red}{#1}}


\rhead{Math 131AH}

\makeatletter
\def\@seccntformat#1{%
  \expandafter\ifx\csname c@#1\endcsname\c@section\else
  \csname the#1\endcsname\quad
  \fi}
\makeatother

\DeclareMathOperator{\Fr}{Fr}
\newcommand{\lra}{\xLeftrightarrow}
\newcommand{\ra}{\xRightarrow}
\newcommand{\N}{\mathbb{N}}
\newcommand{\R}{\mathbb{R}}
\newcommand{\Z}{\mathbb{Z}}
\newcommand{\Q}{\mathbb{Q}}
\newcommand{\norm}[1]{\left\lVert#1\right\rVert}

\begin{document}

\section{Problem 1}

If $A \subseteq X$ is complete, then any Cauchy sequence in $A$ also converges in $A$.
Convergent sequences are Cauchy, so they must also converge in $A$. $\square$

\section{Problem 2}

If we have a Cauchy sequence in $F$, it must converge to something in $X$.
However, since $F$ is closed, the limit of all convergent sequences in it are in $F$,
so all Cauchy sequences must converge to some point in $F$. $\square$

\section{Problem 3}

\subsection{Subspace 1}

$\Q$ is neither open nor closed, since you can't go any radius $r$
out from a rational without hitting an irrational, and $\overline{\Q}=\R$.

By what we'll show in \ref{sec:p4}, it's disconnected too.

$A'=\R$, since the rationals are dense.

\subsection{Subspace 2}

$\Q \cap [0, 1]$ isn't open nor closed for the same reasons $\Q$ isn't.

It's also disconnected.
To show this, we can consider the cut at $\frac{\sqrt{2}}{2}$ instead of $\sqrt{2}$.

$A'=[0, 1]$ since the rationals are dense in this interval regardless.

\subsection{Subspace 3}

The set isn't open since $-2 \in A$ and there aren't really any "adjacent" elements to allow a ball for it.
$-1$ and $1$ are in $\overline{A}$, but they aren't in the actual sequence,
which prevents $A$ from being closed too.

This set is pretty clearly disconnected: take $A=(-\infty, 0)$ and $B=(0, \infty)$ and you're done.

$A'=\{-1, 1\}$, since every other point in $\overline{A}$ is isolated.
For these two points, no matter what $r$ you take, $A$ will have either
$-1-\frac{1}{2n+1}$ or $1+\frac{1}{2n}$ s.t. the fractional part is less than $r$.

\subsection{Subspace 4}

The complement of this set can be written as an infinite union of open intervals:
\[A^C=(-\infty, 0) \cup \bigcup_{n \in \N} \left(n+\frac{1}{n}, n + 1\right)\]
which is open, so this set must be closed.
It's disconnected though- just take $A=\left(-\infty, \frac{5}{2}\right)$ and $B=\left(\frac{5}{2}, \infty\right)$.

$A'=A$ since it's just a union of closed intervals.

\subsection{Subspace 5}

This union is the interval $\left(0, \frac{1}{2}\right]$, so it's neither open nor closed.
However, it \textit{is} connected.

Since it's an interval, its set of accumulation points is just its closure, i.e. $\left[0, \frac{1}{2}\right]$.

\section{Problem 4}\label{sec:p4}

The cut at $\sqrt{2}$ is both open and closed in $\Q$.
For any rational number less than $\sqrt{2}$, we can always find
a greater than that's still less than it, so the set is open.

This set's complement, every rational \textit{greater} than $\sqrt{2}$,
is also open, since given a rational greater than $\sqrt{2}$
we can always find another smaller rational that's still greater than $\sqrt{2}$.

This set and its complement are both open, so the set has to be both open and closed.

\section{Problem 5}\label{sec:p5}

We prove that they're separated straight from the definition:
\begin{gather*}
  \overline{A} \cap (B \cup C) = \left(\overline{A} \cap B\right) \cup \left(\overline{A} \cap C\right) = \varnothing \\
  A \cap \overline{(B \cup C)} \subseteq A \cap \left(\overline{B} \cup \overline{C}\right) = \left(A \cap \overline{B}\right) \cup \left(A \cap \overline{C}\right) = \varnothing
  \quad\square
\end{gather*}

\pagebreak

\section{Problem 6}

\subsection{Connectedness of Unions}

I'll just show that $A \cup B$ is connected- by symmetry, $A \cup C$ must be too.

BWOC say $A \cup B$ is disconnected, so we can write $A \cup B = C \cup D$ where $C$ and $D$ are separated.

$A$ has to be a subset of either $C$ or $D$- let's just say $A \subseteq C$.
This means $D \subseteq B$, and so $D$ and $C$ are separated.
$D$ and $A$ also have to be separated.

But then by \ref{sec:p4} $D$ and $C \cup C$ are separated while
\[D \cup (C \cup C)=(C \cup D) \cup C = A \cup B \cup C = C\]
so we've split $X$, a connected subspace, into two separated sets.
Contradiction. $\square$

\subsection{Closedness of Unions}

Again I'll just show it for $A \cup B$; it STP $\overline{A \cup B} \subseteq A \cup B$.

Since $B$ and $C$ are separated, $\overline{B} \cap C = \varnothing$
and anything that's in $\overline{B} \setminus B$ must be in $A$.
Thus,
\begin{align*}
  & A \cup \overline{B} \subseteq A \cup B \\
  \implies{} & \overline{A} \cup \overline{B} \subseteq A \cup B \\
  \implies{} & \overline{A \cup B} \subseteq \overline{A} \cup \overline{B} \subseteq A \cup B\quad\square
\end{align*}

\section{Problem 7}

BWOC say $A=C \cup D$ where $C$ and $D$ are separated.
By the notes, we know $C$ and $D$ have to both be closed.

Then $(C \cup D) \cap B$ must be connected.
Since it's a subset of $C \cup D$, this means $(C \cup D) \cap B$ is a subset of either $C$ or $D$.
It doesn't really matter which one we pick; say it's $C$.

This then means $D \cap B =\varnothing$.
Both are closed, so the two must be separated.
But then we get that
\[A \cup B = D \cup (C \cup B)\]
Since $D$ and $C \cup B$ are separated by \ref{sec:p5}, this means that $A \cup B$
is disconnected, which is a contradiction. $\square$

\end{document}
