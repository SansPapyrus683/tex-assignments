\documentclass[12pt]{article}

% a template that a friend gave, it's worked well enough for me
% i have added some packages and stuff that have proved useful

\usepackage{fancyhdr}
\usepackage{tipa}
\usepackage{fontspec}
\usepackage{amsfonts}
\usepackage{enumitem}
\usepackage[margin=1in]{geometry}
\usepackage{graphicx}
\usepackage{float}
\usepackage{amsmath}
\usepackage{braket}
\usepackage{amssymb}
\usepackage{booktabs}
\usepackage{hyperref}
\usepackage{mathtools}
\usepackage{xcolor}
\usepackage{float}
\usepackage{algpseudocodex}
\usepackage{titlesec}
\usepackage{bbm}

\pagestyle{fancy}
\fancyhf{} % sets both header and footer to nothing
\lhead{Kevin Sheng}
\setmainfont{Comic Neue}
\renewcommand{\headrulewidth}{1pt}
\setlength{\headheight}{0.75in}
\setlength{\oddsidemargin}{0in}
\setlength{\evensidemargin}{0in}
\setlength{\voffset}{-.5in}
\setlength{\headsep}{10pt}
\setlength{\textwidth}{6.5in}
\setlength{\headwidth}{6.5in}
\setlength{\textheight}{8in}
\renewcommand{\headrulewidth}{0.5pt}
\renewcommand{\footrulewidth}{0.3pt}
\setlength{\textwidth}{6.5in}
\usepackage{setspace}
\usepackage{multicol}
\usepackage{float}
\setlength{\columnsep}{1cm}
\setlength\parindent{24pt}
\usepackage [english]{babel}
\usepackage [autostyle, english = american]{csquotes}
\MakeOuterQuote{"}

\setlength{\parskip}{6pt}
\setlength{\parindent}{0pt}

\titlespacing\section{0pt}{12pt plus 4pt minus 2pt}{0pt plus 2pt minus 2pt}
\titlespacing\subsection{0pt}{12pt plus 4pt minus 2pt}{0pt plus 2pt minus 2pt}
\titlespacing\subsubsection{0pt}{12pt plus 4pt minus 2pt}{0pt plus 2pt minus 2pt}

\hypersetup{colorlinks=true, urlcolor=blue}

\newcommand{\correction}[1]{\textcolor{red}{#1}}


\rhead{Math 131AH}

\makeatletter
\def\@seccntformat#1{%
  \expandafter\ifx\csname c@#1\endcsname\c@section\else
  \csname the#1\endcsname\quad
  \fi}
\makeatother

\DeclareMathOperator{\Fr}{Fr}
\newcommand{\lra}{\xLeftrightarrow}
\newcommand{\ra}{\xRightarrow}
\newcommand{\N}{\mathbb{N}}
\newcommand{\R}{\mathbb{R}}
\newcommand{\Z}{\mathbb{Z}}
\newcommand{\Q}{\mathbb{Q}}
\newcommand{\norm}[1]{\left\lVert#1\right\rVert}

\begin{document}

\section{Problem 1}

Every set is open (so everything's closed too).
For any $A \subseteq X$ and $a \in A$, we find that
$a \in \mathring{A}$ since $B_{0.5}(a)=\{a\} \subseteq A\ \forall a$.

\section{Problem 2}\label{sec:p2}

\subsection{Forwards}

If $A$'s open, for all $x \in X\ \exists r > 0: B_r(x) \subseteq A$.

By construction, $y \in B_r(x) \implies d_1(x, y) < r$ and $\frac{d_1(x, y)}{1+d_1(x, y)} < \frac{r}{1+r}$,
as $f(x)=\frac{x}{1+x}$ is monotonically increasing as long as $x \ge 0$.

Thus, we can choose $r'=\frac{r}{1+r}>0$ so that $B_{r'}(x) \subseteq B_r(x) \subseteq A$ under $d_2$.

\subsection{Backwards}

We do a similar sort of radius transformation from $r'$ back to $r$.
This way, if $B_{r'}(x) \subseteq A$ under $d_2$, $B_r(x) \subseteq A$ under $d_1$.

If $d_2(x,y) < r'$, we can set $r=\frac{r'}{1-r'}$.
$0 \le r' < 1$ so this expression doesn't become invalid.

Setting this, we the following for any $y \in B_r(x)$ (i.e. $d_1(x, y) < r$):
\begin{align*}
  d_2(x,y)
   & = \frac{d_1(x,y)}{1+d_1(x,y)}               \\
   & < \frac{\frac{r'}{1-r'}}{1+\frac{r'}{1-r'}} \\
   & = \frac{r'}{1-r'+r'}                        \\
   & = r'
\end{align*}
so $B_r(x) \subseteq B_{r'}(x) \subseteq A$. $\square$

\pagebreak

\section{Problem 3}

Like in \ref{sec:p2}, we try to construct another ball under the new metric
such that anything in the new ball is in the old ball.

In all the following subsections, we'll pick some $x \in A$ and an $r > 0$
s.t. $B_r(x) \subseteq A$, i.e. $\norm{y-x}_p < r \implies y \in A$.
Then it suffices to give an $r'$ s.t. $\norm{y-x}_q < r' \implies \norm{y-x}_p < r$.

\subsection{\texorpdfstring{$p < \infty$}{p < inf} and \texorpdfstring{$q = \infty$}{q = inf}}

Consider $r'=\frac{r}{\sqrt[p]{n}}$ under the new metric.

Now for any $y \in B_{r'}(x)$ under $l_\infty$, $\norm{x-y}_\infty < \frac{r}{\sqrt[p]{n}}$,
or in other words $\max |x_i-y_i| < \frac{r}{\sqrt[p]{n}}$.

This means
\begin{align*}
  &\sum_{i=1}^{n} |x_i-y_i|^p < \left(\frac{r}{\sqrt[p]{n}}\right)^p \cdot n \\
  \implies{} & \sum_{i=1}^{n} |x_i-y_i|^q < r^p \\
  \implies{} & \norm{x-y}_p < r
\end{align*}

\subsection{\texorpdfstring{$p = \infty$}{p = inf} and \texorpdfstring{$q < \infty$}{q < inf}}

For the other way round, we can actually keep the radius the same.

If $\norm{x-y}_q < r$,
\begin{align*}
  & \sum_{i=1}^{n} |x_i-y_i|^q < r^q \\
  \implies{} & \max_{i=1, \cdots, n} |x_i-y_i|^q < r^q \\
  \implies{} & \max_{i=1, \cdots, n} |x_i-y_i| < r 
\end{align*}
so anything in $B_r(x)$ under $l_\infty$ will also be in the same ball under $l_p$,
which we know is a subset or $A$.

\pagebreak

\subsection{Both Finite}

Consider $r'=\sqrt[p]{\frac{r}{n}}$.
Then, if $\norm{x-y}_q < r$, we have
\begin{align*}
  & \sum_{i=1}^{n} |x_i-y_i|^q < \left(\frac{r}{n}\right)^{q/p} \\
  \implies{} & |x_i-y_i|^q < \left(\frac{r}{n}\right)^{q/p}\ \forall i=1, \cdots, n \\
  \implies{} & |x_i-y_i| < \sqrt[p]{\frac{r}{n}} \\
  \implies{} & |x_i-y_i|^p < \frac{r}{n} \\
  \implies{} & \norm{x-y}_p < r\quad\square
\end{align*}

\section{Problem 4}

We know from the notes that the union of open sets is still open and that all balls are open,
so the reverse direction is free; the hard part is the forward direction.

For the forwards direction, if we know a set $A$ is open, we can write it like so:
\[A=\bigcup_{a \in A} B_{r_a}(a)\]
where $r_a$ is chosen for each $a$ s.t. $B_{r_a}(a) \subseteq A$.

By definition, the LHS is a subset of the RHS.
OTOH, if we have some $x \in A$, we know its ball will be somewhere in the union by construction,
so $x$ is certainly in the LHS as well. $\square$

\pagebreak

\section{Problem 5}

$A \subseteq \overline{A}$, so $\delta(A) \le \delta(\overline{A})$
and it suffices to just show the inequality in the other direction.

Let $d$ and $\overline{d}$ be the distance sets for $A$ and $\overline{A}$ respectively.
I'll show $\forall \epsilon > 0, x \in \overline{d}\ \exists y \in d: x \le y + \epsilon$.

$x \in \overline{d}$ implies an $a, b \in \overline{A}: d(a, b)=x$.
Take a ball of radius $\frac{\epsilon}{2}$ around both $a$ and $b$.
Both must intersect with $A$ at some point since they're in the closure of $A$,
so $\exists c, d \in A: d(a, c) < \frac{\epsilon}{2}, d(b, d) < \frac{\epsilon}{2}$.

By the triangle inequality, we have
\begin{align*}
    d(c, d)
    &\le d(a, c) + d(c, d) + d(d, b) \\
    &< \frac{\epsilon}{2} + d(c, d) + \frac{\epsilon}{2} \\
    &= \epsilon + d(c, d)
\end{align*}
so there does exist a $d(c, d) \in A$ s.t. $x \le y+\epsilon$. $\square$

\section{Problem 6}

By the assumptions given, we can take $x \in A: d(x, a) < r$.
Then for any $b \in A$,
\begin{align*}
    d(b, a)
    &\le d(b, x) + d(x, a) \\
    &\le \delta(A) + r \\
    &\le 2r
\end{align*}
so anything in $A$ is at most $2r$ away from $a$. $\square$

\pagebreak

\section{Problem 7}

Any set is contained in its closure, so proving the second implies the first.

Also, $O \cap A \subseteq O \cap \overline{A}$, so $\overline{O \cap A} \subseteq \overline{O \cap \overline{A}}$.
We just have to prove the other direction for things to work out.

Say $x \in \overline{O \cap \overline{A}}$,
or in other words $\forall r > 0\ \exists a: d(a, x) < r \land x \in O, \overline{A}$.
$O$ is open, so $\exists R: B_R(x) \subseteq O$.
Any radius less than $R$ will also work, so take $r'=\min\left(R, r\right)$.

Not only is $B_{r'}(x) \subseteq O$, $B_{r'}(x) \cap A \ne \varnothing$ since $x \in \overline{A}$.
This means $\exists y \in A: d(x, y) < r$.
So for all $r$, we've shown the existence of a $y \in A \cap O$ that's less than $r$ away from $x$,
which implies $x \in \overline{A \cap O}$. $\square$

If $O \cap A = \varnothing$, $\overline{O \cap A}=\varnothing$ too and by the equality
we just proved $\overline{O \cap A}=\varnothing$ as well.

\section{Problem 8}

The forward direction's pretty easy.
If $A$'s closed, $A=\overline{A}$ and $\overline{A} \cap \overline{A^C}=A \cap \overline{A^C} \subseteq A$.

For the backward direction, BWOC let there be an $a \in \overline{A}: a \notin A$.
Since $a \notin A$, $a \in \overline{A^C}$ and combined with our earlier condition this makes $a \in \Fr(A)$.
This shows the existence of something in $\Fr(A)$ that's not in $A$, which is a contradiction. $\square$

\section{Problem 9}

Again we have a simple forward direction:
\[A=\mathring{A}
\implies A \cap \overline{A} \cap \overline{A^C}
=\mathring{A} \cap \overline{\mathring{A}} \cap \left(\mathring{A}\right)^C
=\varnothing\]

For the backwards direction, again for a contradiction assume $\exists a \in A: a \notin \mathring{A}$.
$a$ is obviously both in $A$ and $\overline{A}$, and since it's not in the interior, $a \in \left(\mathring{A}\right)^C$ as well.
This means the intersection of $A$ and its frontier is nonempty, which is a contradiction. $\square$

\pagebreak

\section{Problem 10}

\subsection{General Subset}

If something's in $\Fr(A \cup B)=\overline{A \cup B} \cap \overline{(A \cup B)^C}$,
for any $x$ in the set and positive radius $r$:
\begin{enumerate}
    \item $B_r(x) \cap (A \cup B) \ne \varnothing$- something's in $A$ or $B$.
    \item $B_r(x) \cap (A \cup B)^C \ne \varnothing$- something's in neither $A$ nor $B$.
\end{enumerate}

Take any $x \in \Fr(A \cup B)$.
By the second statement, we know $x \in \overline{B^C}, \overline{A^C}$.

Now we just NTS $x$ is in at least one of $\overline{A}$ and $\overline{B}$.
FSOC say $x \notin \overline{A} \land x \notin \overline{B}$, which means
\begin{enumerate}
    \item $\exists r_a: B_{r_a} \cap A = \varnothing$
    \item $\exists r_b: B_{r_b} \cap B = \varnothing$
\end{enumerate}
If we assume WLOG $r_a < r_b$, this means $B_{r_a} \cap B = \varnothing$ as well,
which gets us
\[B_{r_a}(x) \cap B \cup B_{r_a}(x) \cap A = B_{r_a}(x) \cap (A \cup B) = \varnothing\]
which is a contradiction.

So if $x \in \Fr(A \cup B)$, it's in $\overline{A^C}$, $\overline{B^C}$,
and at least one of $\overline{A}$ and $\overline{B}$.

This automatically puts it in
\[\Fr(A) \cup \Fr(B)=\left(\overline{A} \cap \overline{A^C}\right) \cup \left(\overline{B} \cap \overline{B^C}\right)\quad\square\]

\subsection{Disjoint Case}

Since $\overline{A}$ and $\overline{B}$ are disjoint,
$\overline{A} \cap \overline{A^C}$ and $\overline{B} \cap \overline{B^C}$ are disjoint as well.

Take any $x \in \overline{A} \cap \overline{A^C}$.
To prove that this is in the LHS, we need to find within any distance $r$
an element that's in either $A$ or $B$ as well as one that's in neither.

Because of the set we chose $x$ from, any ball will certainly have an element from $A$.
As $x \notin \overline{B}$, $\exists r_b: B_{r_b}(x) \cap B = \varnothing$.
For any $r \le r_b$, we can take the element in $B_r(x) \cap A^C$.
Since $B_r(x) \subseteq B_{r_b}(x)$ has nothing in common with $B$,
this element is guaranteed to be in $B^C$ as well.

Also, if $r > r_b$, then we can just use the old element we already found,
since expanding the ball only increases our options.
With that, we've found our two needed elements and proven $x \in \Fr(A \cup B)$.

The argument for it $x \in \overline{B} \cap \overline{B^C}$ goes
exactly the same way; I don't think it needs to be mentioned. $\square$

\section{Problem 11}

I'm assuming in this problem $d(x, y)=|x-y|$.

If $F$ is closed, $\overline{F}=F$.
For any $\epsilon$, $\exists f \in F: f > \sup F - \epsilon$,
which means $B_\epsilon(\sup F) \cap F \ne \varnothing$
(as we've just found an $f$ in there).

This is precisely the condition that needs to be met for $\sup F \in \overline{F}=F$,
so the maximum of the set does exist. $\square$

\section{Problem 12}

I'll denote let $X$ denote $\R^n$ for convenience.

\subsection{Part 1}

This is equivalent to asking if $B^S_r(x)$ is open,
since if that's open the set in the question is the complement of that
so it's closed by definition.

Since $B^S_r(x)=B^X_r(x) \cap S$ and $B^X_r(x)$ is open,
by that proposition we proved in lecture $B^S_r(x)$ is open as well.

\subsection{Part 2}

I don't think so.

If $n=2$, consider the subset $S=\{(a, b) \mid a \in \{0, 2\}\}$, $x=(0, 0)$, and $r=2$.

The set $\{y \in S \mid d(x, y) \ge 2\}$ has all points
on the y-axis where the absolute value of the y-coordinate is at least $2$,
but it also has $(2, 0)$, which is also $2$ away from $x$.

However, nothing in $B_1((2, 0))$ is within the set $\{y \in S \mid d(x, y) > 2\}$,
so the first set isn't necessarily within the closure of the second.

\end{document}
