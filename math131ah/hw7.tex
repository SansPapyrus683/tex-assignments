\documentclass[12pt]{article}

% a template that a friend gave, it's worked well enough for me
% i have added some packages and stuff that have proved useful

\usepackage{fancyhdr}
\usepackage{tipa}
\usepackage{fontspec}
\usepackage{amsfonts}
\usepackage{enumitem}
\usepackage[margin=1in]{geometry}
\usepackage{graphicx}
\usepackage{float}
\usepackage{amsmath}
\usepackage{braket}
\usepackage{amssymb}
\usepackage{booktabs}
\usepackage{hyperref}
\usepackage{mathtools}
\usepackage{xcolor}
\usepackage{float}
\usepackage{algpseudocodex}
\usepackage{titlesec}
\usepackage{bbm}

\pagestyle{fancy}
\fancyhf{} % sets both header and footer to nothing
\lhead{Kevin Sheng}
\setmainfont{Comic Neue}
\renewcommand{\headrulewidth}{1pt}
\setlength{\headheight}{0.75in}
\setlength{\oddsidemargin}{0in}
\setlength{\evensidemargin}{0in}
\setlength{\voffset}{-.5in}
\setlength{\headsep}{10pt}
\setlength{\textwidth}{6.5in}
\setlength{\headwidth}{6.5in}
\setlength{\textheight}{8in}
\renewcommand{\headrulewidth}{0.5pt}
\renewcommand{\footrulewidth}{0.3pt}
\setlength{\textwidth}{6.5in}
\usepackage{setspace}
\usepackage{multicol}
\usepackage{float}
\setlength{\columnsep}{1cm}
\setlength\parindent{24pt}
\usepackage [english]{babel}
\usepackage [autostyle, english = american]{csquotes}
\MakeOuterQuote{"}

\setlength{\parskip}{6pt}
\setlength{\parindent}{0pt}

\titlespacing\section{0pt}{12pt plus 4pt minus 2pt}{0pt plus 2pt minus 2pt}
\titlespacing\subsection{0pt}{12pt plus 4pt minus 2pt}{0pt plus 2pt minus 2pt}
\titlespacing\subsubsection{0pt}{12pt plus 4pt minus 2pt}{0pt plus 2pt minus 2pt}

\hypersetup{colorlinks=true, urlcolor=blue}

\newcommand{\correction}[1]{\textcolor{red}{#1}}


\rhead{Math 131AH}

\makeatletter
\def\@seccntformat#1{%
  \expandafter\ifx\csname c@#1\endcsname\c@section\else
  \csname the#1\endcsname\quad
  \fi}
\makeatother

\newcommand{\lra}{\xLeftrightarrow}
\newcommand{\ra}{\xRightarrow}
\newcommand{\N}{\mathbb{N}}
\newcommand{\R}{\mathbb{R}}
\newcommand{\Z}{\mathbb{Z}}
\newcommand{\Q}{\mathbb{Q}}

\begin{document}

\section{Problem 1}

The proof is by induction on the size of $A$.

The base case is trivial; if we have to map one element to $m$
other elements, there's only one possible function.
Now we assume that there's $m^n$ functions mapping $A$ to $B$
with $|A|=n$ and $|B|=m$, and prove that it works for $A \cup \{a\}$ with $a \notin A$.

We first show that $\left|B^{A \cup \{a\}}\right|=\left|B^A \times B^{\{a\}}\right|$.
Given a function $f$ from $A \cup \{a\}$ to $B$, we can make $f_A: A \to B$ and $f_a: \{a\} \to B$, where
\begin{align*}
  f_A(x)=f(x) && f_a(x)=f(x)
\end{align*}

This mapping is clearly injective, since if two $f$s give different outputs
for some $x$ the change will be reflected in either $f_A$ or $f_a$.
$x \in A \cup \{a\}$, so not being in the domain of either function isn't an option.

It's also surjective.
For any $f_A$ and $f_a$, we can get the original $f$ like so:
\[f(x)=\begin{cases}
  f_A(x) & x \in A \\
  f_a(x) & x \in \{a\}
\end{cases}\]
By construction, our mapping does map this back to the $f_A$ and $f_a$ we desire.

With cardinality equality shown, we can say:
\begin{align*}
  \left|B^{A \cup \{a\}}\right|
   & =\left|B^A \times B^{\{a\}}\right| \\
   & =\left|B^A\right| \cdot \left|B^{\{a\}}\right|    \\
   & = m^n \cdot m                      \\
   & = m^{n+1}\quad\square
\end{align*}
\pagebreak

\section{Problem 2}

\subsection{Equivalence Relation}

\subsubsection{Reflexity}

$x-x=0 \in \Z\ \forall x$. Yeah.

\subsubsection{Symmetricity}

$x-y=z \in \Z \therefore y-x=-z \in \Z$, so symmetricity holds.

\subsubsection{Transitivity}

If $x-y=a \in \Z$ and $y-z=b \in \Z$, we have
\[x-z=(x-y)+(y-z)=a+b \in \Z\quad\square\]

\subsection{Representatives}

\subsubsection{Injectivity}

For this, we NTS $\forall x, y \in [0, 1)\ x-y \in \Z \implies x=y$,
or in other words $x \ne y \implies x-y \notin \Z$.

BWOC say $x-y \in \Z$.
WLOG assume $x > y$.
Then $x-y>0$, and the smallest it can be is $1$,
so this really means $x-y \ge 1$.
Since $x \in [0, 1)$,
\begin{align*}
               & x < 1     \\
  \therefore{} & x-y < 1-y \\
  \therefore{} & 1 < 1-y   \\
  \therefore{} & y < 0
\end{align*}
which contradicts that $y$ is also in $[0, 1)$.

\subsubsection{Surjectivity}

It suffices to show that every real number is equivalent to some $x \in [0, 1)$.

We previously proved by the Archimedean Principle that any $y \in \R$.
can be placed in $[n, n+1)$ where $n \in \Z$.
I'll show that $x=y-n$ is both similar to $y$ and in $[0, 1)$.
\begin{gather*}
  n \le y \therefore n-n=0 \le y-n=x \\
  y < n+1 \therefore y-n=x < n+1-n=1 \\
  x-y=n \in \Z\text{ by construction}\quad\square
\end{gather*}

\section{Problem 3}

\subsection{Injectivity}

Suppose $f(n, m)=f(a, b)$.

We first show $n+m=a+b$ by contradiction.
WLOG let $n+m > a+b$, and let's assign $x=a+b$ and $y=n+m-x$ for convenience.
Then we do some omega-long, omega-nasty algebraic manipulation to find that
\begin{align*}
             & \frac{(y+x-1)(y+x-2)}{2}+n=\frac{(x-1)(x-2)}{2}+a \\
  \implies{} & \frac{1}{2}((y+x-1)(y+x-2)-(x-1)(x-2))=a-n        \\
  \implies{} & \frac{y}{2}(2x+y-3)=a-n
\end{align*}
Then we do some inequalities to have
\begin{align*}
  a-n
   & = \frac{y}{2}(2x+y-3)           \\
   & = xy+\frac{y^2}{2}-\frac{3y}{2} \\
   & \ge x-1                         \\
   & = a+b-1
\end{align*}
$a-n \ge a+b-1 \implies 1 \ge b+n$, which is a contradiction since $b, n \ge 1$.

So with that out of the way, the fractions must be equal
(since they only use $n+m$ and $a+b$ respectively) and $a=n$ as well.
Now we have
\begin{align*}
             & (n+m-1)(n+m-2)=(a+b-1)(a+b-2)   \\
  \implies{} & (n+m)^2-3(n+m)=(a+b)^2-3a-3b    \\
  \implies{} & (n+m)^2-3(n+m)-(a+b)^2+3(a+b)=0 \\
  \implies{} & (n+m-a-b)(n+m+a+b)-3(n+m-a-b)=0 \\
  \implies{} & (n+m-a-b)(n+m+a+b-3)=0
\end{align*}
Since $n, m, a, b \ge 1$, the second coefficient can't be $0$.
This forces the first coefficient to be $0$, so $n+m=a+b$ and by extension $m=b$.

So with all this, we've shown $f(n, m)=f(a, b) \implies (n, m)=(a, b)$. $\square$

\pagebreak

\subsection{Surjectivity}

We proceed by induction.
$f(1, 1)=1$, so that's the base case.

We now assume $\exists n, m \in \N: f(n, m)=k$
and try to find another pair that goes to $k+1$.
For this, we have
\begin{align*}
  k+1
   & = f(n, m)+1                                      \\
   & = \frac{(n+m-2)(n+m-1)}{2}+n+1                   \\
   & = \frac{((n+1)+(m-1)-2)((n+1)+(m-1)-1)}{2}+(n+1) \\
   & = f(n+1, m-1)
\end{align*}
so as long as $m > 1$, we can get $k+1=f(n+1, m-1)$.

For the case where $m=1$ and can't be subtracted from, we have
\begin{align*}
  k+1
   & = f(n, 1)+1            \\
   & = \frac{n(n-1)}{2}+n+1 \\
   & = \frac{n(n+1)}{2}+1   \\
   & = f(1, n+1)
\end{align*}
The last part works if you plug it into $f$, I swear.

But anyways, that's both parts shown, so we have a bijection! $\square$

\pagebreak

\section{Problem 4}

BWOC say there existed a bijection $f: A \to B$.

Since $B \subsetneq A$, $\exists a_0 \in A: a_0 \notin B$.
Let this be the starting element for a sequence where $a_i=f(a_{i-1})$,
so $a_1=f(a_0)$, $a_2=f(a_1)$, so on and so forth.

Notice that $a_i \ne a_j\ \forall j < i$.
We prove this by induction.
$a_1 \ne a_0$ since $a_1 \in B$ and $a_0 \notin B$.
Again, $a_i \ne a_0$ since one's in $B$ while the other isn't.
Now for any $a_i$, $a_{i-1} \ne a_j\ \forall j < i-1$,
so $a_i=f(a_{i-1}) \ne f(a_j)=a_{j+1}\ \forall j < i-1$,
which is exactly what we wanted to prove.
This implies that all elements in the sequence of $a_n$ are distcint.

We can apply $f$ however many times we want to get a countably
infinite number of elements that are all in $A$,
which contradicts that $A$ is a finite set. $\square$

\section{Problem 5}

Any infinite set $A$ minus a single element is still infinite.
In fact, we can take $A \setminus B$ and it'll still be infinite
as long as $B$ is finite.

Thus, we can construct the following sequence:
\begin{enumerate}[nolistsep]
  \item $a_1$ is any element in $A$
  \item $a_n$ is any element in $A \setminus \{a_1, \cdots, a_{n-1}\}$
\end{enumerate}
By construction, there's countably many elements in this sequence,
so we can indeed find some countable subset within $A$.

\end{document}
