\documentclass[12pt]{article}

% a template that a friend gave, it's worked well enough for me
% i have added some packages and stuff that have proved useful

\usepackage{fancyhdr}
\usepackage{tipa}
\usepackage{fontspec}
\usepackage{amsfonts}
\usepackage{enumitem}
\usepackage[margin=1in]{geometry}
\usepackage{graphicx}
\usepackage{float}
\usepackage{amsmath}
\usepackage{braket}
\usepackage{amssymb}
\usepackage{booktabs}
\usepackage{hyperref}
\usepackage{mathtools}
\usepackage{xcolor}
\usepackage{float}
\usepackage{algpseudocodex}
\usepackage{titlesec}
\usepackage{bbm}
\usepackage{pythonhighlight}

\pagestyle{fancy}
\fancyhf{} % sets both header and footer to nothing
\lhead{Kevin Sheng}
\setmainfont{Comic Neue}
\renewcommand{\headrulewidth}{1pt}
\setlength{\headheight}{0.75in}
\setlength{\oddsidemargin}{0in}
\setlength{\evensidemargin}{0in}
\setlength{\voffset}{-.5in}
\setlength{\headsep}{10pt}
\setlength{\textwidth}{6.5in}
\setlength{\headwidth}{6.5in}
\setlength{\textheight}{8in}
\renewcommand{\headrulewidth}{0.5pt}
\renewcommand{\footrulewidth}{0.3pt}
\setlength{\textwidth}{6.5in}
\usepackage{setspace}
\usepackage{multicol}
\usepackage{float}
\setlength{\columnsep}{1cm}
\setlength\parindent{24pt}
\usepackage [english]{babel}
\usepackage [autostyle, english = american]{csquotes}
\MakeOuterQuote{"}

\setlength{\parskip}{6pt}
\setlength{\parindent}{0pt}

\titlespacing\section{0pt}{12pt plus 4pt minus 2pt}{0pt plus 2pt minus 2pt}
\titlespacing\subsection{0pt}{12pt plus 4pt minus 2pt}{0pt plus 2pt minus 2pt}
\titlespacing\subsubsection{0pt}{12pt plus 4pt minus 2pt}{0pt plus 2pt minus 2pt}

\hypersetup{colorlinks=true, urlcolor=blue}

\newcommand{\correction}[1]{\textcolor{red}{#1}}


\rhead{Math 131AH}

\makeatletter
\def\@seccntformat#1{%
  \expandafter\ifx\csname c@#1\endcsname\c@section\else
  \csname the#1\endcsname\quad
  \fi}
\makeatother

\newcommand{\lra}{\xLeftrightarrow}
\newcommand{\ra}{\xRightarrow}
\newcommand{\N}{\mathbb{N}}
\newcommand{\R}{\mathbb{R}}
\newcommand{\Z}{\mathbb{Z}}
\newcommand{\Q}{\mathbb{Q}}

\begin{document}

\section{Problem 1}

We proceed by induction on $n$.
The base case $n=1$ is free, since $A_1$ is known to be countable.

Now, assuming $A_1 \times \cdots \times A_n$ is countable,
we prove that $A_1 \times \cdots \times A_{n+1}$ is also countable.
Notice that everything in that product can be uniquely represented
by an $n$-tuple taking elements from $A_1$ to $A_n$ along with
another element from $A_{n+1}$.
To get to $A_1 \times \cdots \times A_{n+1}$, we just
tack on the lone element to the end of the tuple and call it a day.

By our inductive hypothesis and the premise,
both the $n$-tuple and the element can be bijectively mapped to $\N$.
This allows us to construct the following bijection:
\begin{center}
  \begin{tabular}{c|cccc}
               & \textbf{1} & \textbf{2} & \textbf{3} & \textbf{4} \\ \hline
    \textbf{1} & 1          & 2          & 4          & 7          \\
    \textbf{2} & 3          & 5          & 8          & \ldots     \\
    \textbf{3} & 6          & 9          & \ldots     & \ldots     \\
    \textbf{4} & 10         & \ldots     & \ldots     & \ldots     \\
  \end{tabular}
\end{center}
Each row represents an element in $A_1 \times \cdots \times A_n$
and each column something in $A_{n+1}$, but that doesn't really matter.

We go along each top-right to bottom-left diagonal, assigning the next
natural number to each new element and creating a bijection. $\square$

\pagebreak

\section{Problem 2}

$A \sim B \implies \exists$ a bijection $f: A \to B$.
I propose the bijection $g: \mathcal{P}(A) \to \mathcal{P}(B)$ where
\[g(s)=\{f(a) \mid a \in s\}\]

\subsection{Injective}

If $X, Y \subseteq A$, $\exists a \in X: a \notin Y$.
$a \ne a'\ \forall a' \in Y$, and since $f$ is injective, $f(a) \ne f(a')\ \forall a' \in Y$.

By how we defined $g$, this means $f(a) \in g(X)$ while $f(a) \notin g(Y)$, so $g(X) \ne g(Y)$.

\subsection{Surjective}

Consider any $X \subseteq B$.

I'm not sure if this edge case is needed, but if $X \subseteq B$ is the null set we have $f(\varnothing)=X$.

Regardless, if it isn't, we'll have that
\[Y=\left\{f^{-1}(b) \mid b \in X\right\}\]
maps to $X$.
This is because $f$ is invertible, so $f^{-1}$ exists where $f\left(f^{-1}(b)\right)=b\ \forall b \in B$.

Anyways, that's both bijection properties proven. $\square$

\section{Problem 3}

I'll make the bijection from $2^\N$ to $\mathcal{P}(\N)$, since notation is a little easier:
\[g(f)=\{x \in \N: f(x)=1\}\]
This just maps $f$ to a set of all natural numbers that make it return $1$.

\subsection{Injectivity}

Suppose $\exists n \in \N: f(n) \ne f'(n)$.
WLOG assume $f(n)=1$, so $n \in g(f)$.
Since the range is $\{0, 1\}$, $f'(n)=0$ and $n \notin g(f')$.
This by definition implies $g(f) \ne  g(f')$.

\subsection{Surjectivity}

For any subset $N \subseteq \N$, we can define a function $f: \N \to \{0, 1\}$:
\[f(x)=\begin{cases}
    1 & x \in N          \\
    0 & \text{otherwise}
  \end{cases}\]
$f$ returns $1$ iff $x$ is in $N$, so $g(f)=N$. $\square$

\pagebreak

\section{Problem 4}

By Schroder-Bernstein it suffices to show an injective function in both directions.

\subsection{\texorpdfstring{$\N^\N \to 2^\N$}{N\^N to 2\^N}}

Turn each $a_n$ into its unary representation of $a_n$ $1$s in a row.
Then, we can concatenate all of them together with $0$s separating adjacent numbers.

To show that this is injective, let $i$ be the lowest number s.t. $a_i \ne a'_i$.
Before that, the two are equal, so by construction the two are at the same index before $i$.

WLOG let $a_i > a'_i$.
We add $a_i$ $1$s onto the first sequence.
However, since $a'_i < a_i$, when we finish adding all the $1$s for $a'_i$ onto
the second sequence the $0$ we put on at the end is going to be different
from the $1$ on the first sequence.

\subsection{\texorpdfstring{$2^\N \to \N^\N$}{2\^N to N\^N}}

This one is pretty simple- we just have the new sequence $b_n$ be defined like
\[b_n=a_n+1\]
$0$ turns to $1$ and $1$ turns to $2$, so everything's a natural number no problem.
If two sequences are different, i.e. $\exists i, j: a_i \ne a'_i$,
adding $1$ to both sides won't change that they're different. $\square$

\pagebreak

\section{Problem 5}

We again do an injective thing both ways.

\subsection{Naturals to Roots}

The ID function works, since any natural number $n$ is the root of the polynomial $x-n$.

\subsection{Roots to Naturals}

Each polynomial can be represented by a finite sequence of natural numbers;
using the bijection between $\N$ and $\Z$ given in the notes,
we can just list the coefficients from lowest power to highest.

All polynomials have a finite number of roots, so for each
polynomial we can map their roots to $\{1, \cdots, n\}$ for some $n \in \N$.
With this, we're now able to map the $i$-th real root of some polynomial to some finite sequence.
As for the order, sorting the roots suffices.

Consider $x^2-5x+6$, for example.
To map its first root to a natural number sequence, we can have the sequence
\[1, f(6), f(-5), f(1)\]
where $f$ is a bijective mapping from $\Z$ to $\N$.

If a number is the root of multiple polynomials, we take the lexicograhpically lowest one.

It was previously proven that all finite natural number sequences can be
injectively mapped to $\N$ by raising the $n$th prime number to the $n$th
element in the sequence and multiplying them all together,
so we can indeed injectively map all roots to the naturals. $\square$

\pagebreak

\section{Problem 6}

Another injective mapping both ways; yippee.

\subsection{Naturals to Subsets}

Each $i \in \N$ can be mapped to the following subset:
\[\{i + j \mid 0 \le j < n\}\]
If $a \ne b$, assuming WLOG $a < b$, by construction $a$ isn't in $b$'s set.
This is because $b$'s set only has elements equal to or greater than $b$.

\subsection{Subsets to Naturals}

We can use the following function:
\[f(S)=\sum_{i \in S} 2^{i-1}\]
The key reason this works is because all subsets are finite, so this sum will always exist.

This function is equivalent to a binary number representing $S$,
where the $i$-th least significant digit encodes whether or not $i \in S$.
Thus, if two sets are different, they would have different binary representations,
and $f$ would output different numbers. $\square$

\section{Problem 7}

Since $(0, 1) \sim 2^{\aleph_0}$, it suffices to show
a bijective mapping between $(0, 1)$ and $\R$.

Let the mapping be
\[f(x)=\tan\left(\pi x - \frac{\pi}{2}\right)\]
From trig, we know the OG tangent function provides a bijection from
$\left(-\frac{\pi}{2}, \frac{\pi}{2}\right)$ to all of $\R$.

Given this, we shift by $\frac{\pi}{2}$ then scale by $\pi$ to get that
this interval becomes
\[\left(-\frac{\pi}{2}, \frac{\pi}{2}\right) \to (0, \pi) \to \left(0, 1\right)\]
so this new $f$ does bijectively map $(0, 1)$ to $\R$. $\square$

\pagebreak

\section{Problem 8}

Injection both ways; you know how it is.

\subsection{Irrationals to Reals}

The identity function works.

\subsection{Reals to Irrationals}

The function is as follows:
\[f(x)=\begin{cases}
    a+\sqrt{2}(b+1) & \text{if $\exists a \in \Q, b \in \N_0: x=a+b\sqrt{2}$} \\
    x               & \text{otherwise}
  \end{cases}\]
This keeps most irrationals as is.
However, it adds $\sqrt{2}$ to all rationals and moves
all irrationsl of the form $a+\sqrt{2}b$ "up" one step to make room.

\subsubsection{Well-Defined}

To make sure this function is well-defined, we first have to make
sure there's only one way to represent $x$ as $a+b\sqrt{2}$.
Consider $a$, $b$, $a'$, and $b'$ s.t.
\[a+b\sqrt{2}=a'+b'\sqrt{2}=x\]
Then $a-a'=\sqrt{2}(b-b')$.
$a-a' \in \Q$, while $\sqrt{2}$ times any nonzero integer is still irrational.
Thus, the only way for this equality to hold is if $b-b'=a-a'=0$,
which also means $a=a'$ and $b=b'$.

Also, the range of this function is indeed all irrationals,
since we add $\sqrt{2}$ to every rational and anything in $\Q$ plus $\sqrt{2}$
is irrational.

\subsubsection{Actual Injectivity}

Consider $x, y \in \R$.

If $x$ falls in the first case and the $y$ in the second case,
adding $\sqrt{2}$ to $x$ won't make change its representability by $a+\sqrt{2}b$,
while $y$ doesn't change, so the two remain unequal.

If both are in the first or second case, adding $\sqrt{2}$ or $0$
to both numbers also preserves inequality. $\square$

(actually i'm p sure this is a bijection but too lazy to prove surjectivity lol)

\end{document}
