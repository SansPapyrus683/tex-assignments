\documentclass[12pt]{article}

% a template that a friend gave, it's worked well enough for me
% i have added some packages and stuff that have proved useful

\usepackage{fancyhdr}
\usepackage{tipa}
\usepackage{fontspec}
\usepackage{amsfonts}
\usepackage{enumitem}
\usepackage[margin=1in]{geometry}
\usepackage{graphicx}
\usepackage{float}
\usepackage{amsmath}
\usepackage{braket}
\usepackage{amssymb}
\usepackage{booktabs}
\usepackage{hyperref}
\usepackage{mathtools}
\usepackage{xcolor}
\usepackage{float}
\usepackage{algpseudocodex}
\usepackage{titlesec}
\usepackage{bbm}

\pagestyle{fancy}
\fancyhf{} % sets both header and footer to nothing
\lhead{Kevin Sheng}
\setmainfont{Comic Neue}
\renewcommand{\headrulewidth}{1pt}
\setlength{\headheight}{0.75in}
\setlength{\oddsidemargin}{0in}
\setlength{\evensidemargin}{0in}
\setlength{\voffset}{-.5in}
\setlength{\headsep}{10pt}
\setlength{\textwidth}{6.5in}
\setlength{\headwidth}{6.5in}
\setlength{\textheight}{8in}
\renewcommand{\headrulewidth}{0.5pt}
\renewcommand{\footrulewidth}{0.3pt}
\setlength{\textwidth}{6.5in}
\usepackage{setspace}
\usepackage{multicol}
\usepackage{float}
\setlength{\columnsep}{1cm}
\setlength\parindent{24pt}
\usepackage [english]{babel}
\usepackage [autostyle, english = american]{csquotes}
\MakeOuterQuote{"}

\setlength{\parskip}{6pt}
\setlength{\parindent}{0pt}

\titlespacing\section{0pt}{12pt plus 4pt minus 2pt}{0pt plus 2pt minus 2pt}
\titlespacing\subsection{0pt}{12pt plus 4pt minus 2pt}{0pt plus 2pt minus 2pt}
\titlespacing\subsubsection{0pt}{12pt plus 4pt minus 2pt}{0pt plus 2pt minus 2pt}

\hypersetup{colorlinks=true, urlcolor=blue}

\newcommand{\correction}[1]{\textcolor{red}{#1}}


\begin{document}
\section{Textbook}\label{sec:textbook}
\begin{itemize}
      \item[3.4.27] If an equivalence relation $R \subseteq S \times S$ has only one equivalence class, then $R=S \times S$.
            This is because only one class implies that all elements are related to each other.
      \item[3.4.41] \begin{itemize}
                  \item \textbf{Symmetry:}
                        If $x \sim y$, then $f(x)=f(y) \therefore f(y)=f(x)$ and $y \sim x$.
                  \item \textbf{Reflexivity:}
                        $f(x)=f(x) \therefore x \sim x$
                  \item \textbf{Transitivity:} If $x \sim y$ and $y \sim z$, then $f(x)=f(y)=f(z) \therefore f(x)=f(z) \therefore x \sim z$.
            \end{itemize}
            With all three properties proven, we know that $R$ is an equivalence relation on $X$.
      \item[3.4.43] Let's first show that all elments in $X$ are in at least one set of $\mathcal{S}$.

            Suppose for the sake of contradiction that $\exists x \in X$ s.t. it isn't in any of the sets specified.
            Letting $f(x)=y \in Y$, we know that $x \in f^{-1}(y)$ by the definition of the inverse,
            which means that $x$ must be in one of the sets somewhere.
            Contradiction.

            We also know that $x \in X$ can't be in more than two elements of $\mathcal{S}$
            because that would mean we would have
            \[y_1, y_2 \in Y: y_1 \ne y_2 \land x \in f^{-1}(y_1) \land x \in f^{-1}(y_2)\]
            We know this can't be true by the definition of a function.

            Since every element of $X$ is in exactly one set in $\mathcal{S}$, the specified set is a partition. $\square$
      \item[3.4.44] $f(x)=f(y)$ when $[x]=[y]$.
            In other words, the two are equal when $x \sim y$.
\end{itemize}

\section{Handout}\label{sec:handout}
\begin{itemize}
      \item[A1] \begin{enumerate}[label=\alph*]
                  \item We can prove this by showing $\bigcup_{y \in Y} f^{-1}(y)=X$.


                        Thus, $\bigcup_{y \in Y} f^{-1}(y)=X$ and by extension $\sum_{y \in Y} |f^{-1}(y)|=|X|$. $\square$
                  \item Since $f$ is injective, we know $x \ne y \rightarrow f(x) \ne f(y)$.
                        Because of this, we know that $|X|=|\text{Im}(X)|$
                        Since $\text{Im}(x) \subseteq Y$, we have $|X|=|\text{Im}(x)|\le |Y|$. $\square$
                  \item Since $f$ is surjective, we know $\forall y \in Y\ \exists x \in x: f(x)=y$
                        We've proven that $\sum_{y \in Y} |f^{-1}(y)|=|X|$, and since $|f^{-1}(y)| \ge 1\ \forall y \in Y$,
                        we have \[|Y| \le \sum_{y \in Y} |f^{-1}(y)|=|X|\quad\square\]
                  \item This follows naturally from the previous two statements, as
                        \[|X| \le |Y| \land |X| \ge |Y| \rightarrow |X|=|Y|\]
            \end{enumerate}
      \item[A2] \begin{enumerate}[label=\alph*]
                  \item For each ordered tuple $(c_1, c_2, \cdots , c_n)$, where $c_i \in X$,
                        we can form a string by simply listing all the characters specified in the tuple in order.

                        \textbf{Injectivity:}
                        If $(a_1, a_2, \cdots , a_n) \ne (b_1, b_2, \cdots , b_n)$, then there is
                        some position $i$ s.t. $a_i \ne b_i$.
                        By extension, the strings $a$ and $b$ would also differ at that position $i$.
                        Thus, $f$ is injective.

                        \textbf{Surjectivity:}
                        For any string $s$ of length $n$, we can decompose it into an $n$-tuple
                        that consists of the characters of $s$ in their respective order.
                        For example, $abc$ would decompose into the $3$-tuple $(a,b,c)$.
                        Thus, $f$ is surjective.
                  \item Let us define our bijection function as $g$.
                        For a function $f \in \text{Fun}(\{1, \cdots , n\}, X)$,
                        $g(f)$ is an $n$-tuple $(a_1, a_2, \cdots , a_n)$ such that $a_i=f(i)$.

                        \textbf{Injectivity:}
                        Say we have two different functions $f_1$ and $f_2$.
                        Since they're different, $\exists i \in \{1, \cdots , n\}: f_1(i) \ne f_2(i)$.
                        If we have $g(f_1)=(a_1, a_2, \cdots , a_n)$ and $~{g(f_2)=(b_1, b_2, \cdots , b_n)}$,
                        we can deduce $a_i=f_1(i) \ne f_2(i)=b_i$.
                        Thus, $g$ produces distinct outputs for distinct inputs and is injective by definition.

                        \textbf{Surjectivity:}
                        For any $n$-tuple $(a_1, a_2, \cdots , a_n)$,
                        we have a function $f$ that can be defined as $f(i)=a_i$.
                        Plugging this back into $g$ does yield the desired tuple, so $g$ is also surjective.
            \end{enumerate}
      \item[A3] $n=2$: $f$ is injective and surjective since $f(0)=0$ and $f(1)=1$

            $n=5$: $f$ is neither injective nor surjective since $f(i)=0\ \forall r \in \mathbb{Z}/5\mathbb{Z}$.

            $n=7$: Since $5$ and $7$ are coprime, $f$ is both injective and surjective.
            This can also be verified by going through all values of $r$.
\end{itemize}
\end{document}
