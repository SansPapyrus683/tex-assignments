\documentclass[12pt]{article}

% a template that a friend gave, it's worked well enough for me
% i have added some packages and stuff that have proved useful

\usepackage{fancyhdr}
\usepackage{tipa}
\usepackage{fontspec}
\usepackage{amsfonts}
\usepackage{enumitem}
\usepackage[margin=1in]{geometry}
\usepackage{graphicx}
\usepackage{float}
\usepackage{amsmath}
\usepackage{braket}
\usepackage{amssymb}
\usepackage{booktabs}
\usepackage{hyperref}
\usepackage{mathtools}
\usepackage{xcolor}
\usepackage{float}
\usepackage{algpseudocodex}
\usepackage{titlesec}
\usepackage{bbm}

\pagestyle{fancy}
\fancyhf{} % sets both header and footer to nothing
\lhead{Kevin Sheng}
\setmainfont{Comic Neue}
\renewcommand{\headrulewidth}{1pt}
\setlength{\headheight}{0.75in}
\setlength{\oddsidemargin}{0in}
\setlength{\evensidemargin}{0in}
\setlength{\voffset}{-.5in}
\setlength{\headsep}{10pt}
\setlength{\textwidth}{6.5in}
\setlength{\headwidth}{6.5in}
\setlength{\textheight}{8in}
\renewcommand{\headrulewidth}{0.5pt}
\renewcommand{\footrulewidth}{0.3pt}
\setlength{\textwidth}{6.5in}
\usepackage{setspace}
\usepackage{multicol}
\usepackage{float}
\setlength{\columnsep}{1cm}
\setlength\parindent{24pt}
\usepackage [english]{babel}
\usepackage [autostyle, english = american]{csquotes}
\MakeOuterQuote{"}

\setlength{\parskip}{6pt}
\setlength{\parindent}{0pt}

\titlespacing\section{0pt}{12pt plus 4pt minus 2pt}{0pt plus 2pt minus 2pt}
\titlespacing\subsection{0pt}{12pt plus 4pt minus 2pt}{0pt plus 2pt minus 2pt}
\titlespacing\subsubsection{0pt}{12pt plus 4pt minus 2pt}{0pt plus 2pt minus 2pt}

\hypersetup{colorlinks=true, urlcolor=blue}

\newcommand{\correction}[1]{\textcolor{red}{#1}}


\begin{document}
\section{Textbook}\label{sec:textbook}

\begin{enumerate}
    \item[6.7.16] We choose $1$ and $-1$ as our values for $a$ and $b$.
        \begin{align*}
            (1+(-1))^n & =\sum_{k=0}^{n} \binom{n}{k}(1)^{n-k}(-1)^k      \\
                       & = \sum_{k=0}^{n} (-1)^k \binom{n}{k}\quad\square
        \end{align*}
    \item[6.8.9] It is possible for a month to cover $5$ Fridays.
        In some month, the professor must be paid on the first of these Fridays.
        Then, they must also be paid on the third and fifth as well.
        Thus, there does exist a month in which Prof. Euclid is paid thrice. $\square$
    \item[6.8.11] Let's assign each available item a number corresponding to its slot in the inventory.
        It suffices to prove that when we take the set $\{i_1, i_2, \cdots i_{60}\}$ and add $4$ to each of them,
        there must be some number in the new set that must be in the old set as well.

        Consider the set $\{i_1, i_2, \cdots i_{60}\} \cup \{i_1+4, i_2, \cdots i_{60}\}$.
        Values in this set can take on the values from $1$ to $119$.
        However, since the sum of the sizes of the two sets is $60+60=120$,
        there must be some element that occurs in both sets. $\square$
    \item[7.1.19] \label{list:7.1.19} For some base cases, let $S_0=1$ and $S_1=2$.
        From then on, for $S_n$, there's two ways we can build a binary string of length $n$ from previous cases:
        \begin{enumerate}[label=\arabic*]
            \item Add a $\texttt{1}$ to the strings of $S_{n-1}$.
            \item Add a $\texttt{10}$ to the strings of $S_{n-2}$.
                  Notice that this won't overlap with the previous case because
                  all of these ended with $1$, while these end with $0$.
        \end{enumerate}
        Thus, our recurrence relation is $S_n=S_{n-1}+S_{n-2}$.
    \item[7.1.21] Let us use a combinatorial argument.
        It has been established that $f_{n+2}=S_n$,
        where $S_n$ is the number of $n$-bit strings without the pattern $\texttt{00}$ in them.

        Let us define another way to count the number of these strings by
        placing fixed amounts of $\texttt{0}$s around the string and setting the rest to $\texttt{1}$.

        By the Pigeonhole Principle, we can place at most $\left\lfloor\frac{n+1}{2}\right\rfloor$ $\texttt{0}$s
        around the string before two must be adjacent.
        Now, given that we are to place $i$ $\texttt{0}$s on the string, there's $\binom{n-(i-1)}{i}$
        possible strings such that no two $\texttt{0}$s are adjacent by stars and bars.

        Summing over all possible values of $i$, we have that there are
        \[\sum_{i=0}^{\left\lfloor\frac{n+1}{2}\right\rfloor} \binom{n-(i-1)}{i}\]
        strings with no two adjacent $\texttt{0}$s.
        This counts the same thing as $f_{n+2}$ does, so we've proved equality. $\square$
    \item[7.1.57] Our base case is $R_1=2$.
        After that, each additional line must cross paths with all previous lines, creating an additional
        region with an extra one for each line crossed.

        Given this, we can derive our recurrence relation $R_n=R_{n-1}+n$.
    \item[7.1.61] Our bases cases are $S_0=1$, $S_1=2$, and $S_2=4$.
        As in the other problem, let's try to construct a string of length $n$ from previous ones
        while meeting the constraints.
        There's multiple ways of doing so, many of which are familiar:
        \begin{enumerate}[label=\arabic*]
            \item Add a $\texttt{1}$ to the strings of $S_{n-1}$.
            \item Add a $\texttt{10}$ to the strings of $S_{n-2}$.
            \item Add a $\texttt{100}$ to the strings of $S_{n-3}$.
        \end{enumerate}
        This gives us our recurrence relation $S_n=S_{n-1}+S_{n-2}+S_{n-3}$.
    \item[7.2.2] $a_n=2a_{n-2}-a_{n-1}$ is not linear and homogeneous, as $a_{n-2}$ has a coefficient of $n$.
\end{enumerate}

\pagebreak

\section{Handout}\label{sec:handout}
\begin{itemize}
    \item[A1] Consider a group of $n$ people that wants to choose a committee of size $k$.
        A way we can count the number of possible committees is $\binom{n}{k}$.

        However, let us consider another way.
        We can split this group into a group of size $n-1$ and one lone person.
        When choosing a committee, we can either choose $k$ from the larger group,
        or we can choose $k-1$ from the larger group and include the lone person as well.
        Combining these two gives us a total of $\binom{n}{k-1}+\binom{n-1}{k-1}$.

        These both are valid ways of counting the number of possible committees. $\square$
    \item[A2] This time, consider a group of $n+1$ people $\{p_1, p_2, \cdots p_{n+1}\}$
        that wants to choose a committee of size $k+1$.
        The most straightforward way to do this is to use a binomial
        to see that there are $\binom{n+1}{k+1}$ possible committees.

        Another way to do this is to set a person with the maximum number and select the remaining $k$ people.
        For example, if $k=4$, we'd choose $p_5$ as the person with the maximum number and proceed
        to choose $k-1=3$ people among the set $\{p_1, p_2, p_3, p_4\}$.

        We start at $p_{k+1}$, as before that there aren't enough people with smaller numbers
        that are able to fill the committee.
        From then on, everyone until $p_{n+1}$ can be a maximum person.
        Thus, we would have a total of
        \[\sum_{i=k+1}^{n+1} \binom{i-1}{k}=\binom{k}{k}+\binom{k+1}{k}+\binom{k+2}{k} \cdots + \binom{n}{k}\]
        possible committees.

        Again, we see that these are both valid ways of counting the same thing. $\square$

        (or just look up "hockey stick binomial" and call it a day haha)
\end{itemize}
\end{document}
