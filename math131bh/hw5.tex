\documentclass[12pt]{article}

% a template that a friend gave, it's worked well enough for me
% i have added some packages and stuff that have proved useful

\usepackage{fancyhdr}
\usepackage{tipa}
\usepackage{fontspec}
\usepackage{amsfonts}
\usepackage{enumitem}
\usepackage[margin=1in]{geometry}
\usepackage{graphicx}
\usepackage{float}
\usepackage{amsmath}
\usepackage{braket}
\usepackage{amssymb}
\usepackage{booktabs}
\usepackage{hyperref}
\usepackage{mathtools}
\usepackage{xcolor}
\usepackage{float}
\usepackage{algpseudocodex}
\usepackage{titlesec}
\usepackage{bbm}

\pagestyle{fancy}
\fancyhf{} % sets both header and footer to nothing
\lhead{Kevin Sheng}
\setmainfont{Comic Neue}
\renewcommand{\headrulewidth}{1pt}
\setlength{\headheight}{0.75in}
\setlength{\oddsidemargin}{0in}
\setlength{\evensidemargin}{0in}
\setlength{\voffset}{-.5in}
\setlength{\headsep}{10pt}
\setlength{\textwidth}{6.5in}
\setlength{\headwidth}{6.5in}
\setlength{\textheight}{8in}
\renewcommand{\headrulewidth}{0.5pt}
\renewcommand{\footrulewidth}{0.3pt}
\setlength{\textwidth}{6.5in}
\usepackage{setspace}
\usepackage{multicol}
\usepackage{float}
\setlength{\columnsep}{1cm}
\setlength\parindent{24pt}
\usepackage [english]{babel}
\usepackage [autostyle, english = american]{csquotes}
\MakeOuterQuote{"}

\setlength{\parskip}{6pt}
\setlength{\parindent}{0pt}

\titlespacing\section{0pt}{12pt plus 4pt minus 2pt}{0pt plus 2pt minus 2pt}
\titlespacing\subsection{0pt}{12pt plus 4pt minus 2pt}{0pt plus 2pt minus 2pt}
\titlespacing\subsubsection{0pt}{12pt plus 4pt minus 2pt}{0pt plus 2pt minus 2pt}

\hypersetup{colorlinks=true, urlcolor=blue}

\newcommand{\correction}[1]{\textcolor{red}{#1}}


\rhead{Math 131BH}

\makeatletter
\def\@seccntformat#1{%
  \expandafter\ifx\csname c@#1\endcsname\c@section\else
  \csname the#1\endcsname\quad
  \fi}
\makeatother

\DeclareMathOperator{\Fr}{Fr}
\newcommand{\lra}{\xLeftrightarrow}
\newcommand{\ra}{\xRightarrow}
\newcommand{\N}{\mathbb{N}}
\newcommand{\R}{\mathbb{R}}
\newcommand{\Z}{\mathbb{Z}}
\newcommand{\Q}{\mathbb{Q}}
\newcommand{\norm}[1]{\left\lVert#1\right\rVert}

\begin{document}

\section{Problem 1}

BWOC say $f$ wasn't continuous at $0$,
so $\exists \epsilon$ s.t. any $(-\delta, \delta)$ always has some violation.

Among any partition, there's always some interval $[x_i, x_{i+1}]$ that contains $0$.
In this interval, $\Delta \alpha_i$ is guaranteed to equal $1$.

By what we've just said,
\[\sup_{[x_i, x_{i+1}]} f(x)-\inf_{[x_i, x_{i+1}]} f(x) \ge \epsilon\]
since we can always find an element that's at least $\epsilon$ away from $f(0)$,
which the interval is guaranteed to contain.

This means $U(P, f) - L(P, f) \ge \epsilon$ as well, which contradicts that $f \in \mathcal{R}(\alpha)$. $\square$

\section{Problem 2}

$\R \setminus \mathcal{C}$ should be dense in $\mathcal{C}$, so $L(P, f)=0$ for any partition.

Now we just have find a $P$ for any $\epsilon$ s.t. $U(P, f) < \epsilon$.

To start, we can have $P_0=\left\{0, \frac{1}{3}, \frac{2}{3}, 1\right\}$.
I'll find "sub-partitions" for each third, and now we just need:
\begin{itemize}
  \item The left and right halves to be less than $\frac{3\epsilon}{7}$
  \item The middle half to be less than $\frac{\epsilon}{1}$
\end{itemize}

While we can't really say anything about the left and right halves, $f$ on the middle half is just
\[f(x)=\begin{cases}
    0 & \frac{1}{3} < x < \frac{2}{3} \\
    1 & \text{otherwise}
  \end{cases}\]
so the sub-partition $\left\{\frac{1}{3}, \frac{1+0.1\epsilon}{3}, \frac{2-0.1\epsilon}{3}, \frac{2}{3}\right\}$
has an upper sum of at most $\frac{\epsilon}{7}$.

This leaves the left and right halves.
Notice that both are equivalent to $f$ shrunk horizontally by a factor of $3$, so we can do recursion, basically.

To show that this process will terminate, notice that while $f$ shrinks by a factor of $\frac{1}{3}$ each time,
$\epsilon$ shrinks by a factor of $\frac{3}{7} > \frac{1}{3}$.
Eventually the length of the entire interval we're considering will be less than the current $\epsilon$,
and our subpartition will just be the entire interval.

Recursively applying this process should yield a partition whose upper sum is less than $\epsilon$. $\square$

\pagebreak

\section{Problem 3}

\subsection{Part B}\label{sec:p3pb}

It STP $\int_{a}^{b} (fg)(x)\,dx < 1+\epsilon\ \forall \epsilon > 0$.

Take a partition $P_f$ s.t. $U(P_f, f, \alpha) < 1+\epsilon$.
Take a similar $P_g$ and let $P=P_f \cup P_g$.

Let $FP_i$ and $GQ_i$ indicate the range sups of $f^p$ and $g^q$ respectively.
If two-letter variable names work for physics, who's to say they won't work for math?

Given that we have
\begin{align*}
  \sum_{i=1}^{n} FP_i \Delta \alpha_i < 1+\epsilon &  &
  \sum_{i=1}^{n} GQ_i \Delta \alpha_i < 1+\epsilon
\end{align*}
we can add scalar multiples of the two to get
\[\sum_{i=1}^{n} \left(\frac{FP_i}{p}+\frac{GQ_i}{q}\right)\Delta\alpha_i < \frac{1+\epsilon}{p}+\frac{1+\epsilon}{q}=1+\epsilon\]

Also, since by what was proven in part A last week,
\[f(x) \cdot g(x) \le \frac{f(x)^p}{p}+\frac{g(x)^q}{q}\]
$f$ and $g$ are both nonnegative, so the range sup of their product is
equal to the product of their sups.
This allows us to write
\begin{align*}
  \sum_{i=1}^{n} (FG_i) \Delta\alpha_i
  &=  \sum_{i=1}^{n} (F_i \cdot G_i) \Delta\alpha_i \\
  &\le\sum_{i=1}^{n} \left(\frac{FP_i}{p}+\frac{GQ_i}{q}\right)\Delta\alpha_i  \\
  &< 1+\epsilon\quad\square
\end{align*}

\pagebreak

\subsection{Part C}

WLOG assume $f$ and $g$ are nonnegative; everything's wrapped up in absolute values anyways.

Let $A_f=\int_{a}^{b} f(x)^p\,dx$ and $A_g$ be defined similarly but with $q$.

Consider the functions $F(x)=\frac{f(x)}{A_f^{1/p}}$ and $G(x)=\frac{g(x)}{A_g^{1/q}}$
Notice that
\[\int_{a}^{b} F(x)^p\,dx=\int_{a}^{b} \frac{f(x)^p}{A_f}\,dx=1\]
and the same holds true for $G$, so we can apply the results in \ref{sec:p3pb}:
\begin{align*}
  & \int_{a}^{b} (FG)(x)\,dx \le 1 \\
  \implies{} & \int_{a}^{b} \frac{f(x)g(x)}{A_f^{1/p}A_g^{1/q}} \le 1 \\
  \implies{} & \int_{a}^{b} (fg)(x) \le A_f^{1/p}A_g^{1/q} \\
  \implies{} & \int_{a}^{b} (fg)(x) \le \left(\int_{a}^{b} f(x)^p\,dx\right)^{1/p}\left(\int_{a}^{b} g(x)^q\,dx\right)^{1/q}
\end{align*}
We already assumed everything was nonnegative, so the $|\cdot|$s don't do anything. $\square$

\pagebreak

\section{Problem 4}

\subsection{What the Heck}

Lemme pull \textit{another} deus ex machina and write $g(x)=xf(x)^2$.
Its derivative is
\[g'(x)=f(x)^2+2xf(x)f'(x)\]
so by the FTC we have
\begin{align*}
  & \int_{a}^{b} f(x)^2+2xf(x)f'(x)\,dx = g(b)-g(a) \\
  \implies{} & \int_{a}^{b} f(x)^2\,dx+2\int_{a}^{b} xf(x)f'(x)\,dx = 0 \\
  \implies{} & \int_{a}^{b} xf(x)f'(x)\,dx=-\frac{\int_{a}^{b} f(x)^2\,dx}{2} \\
  \implies{} & \int_{a}^{b} xf(x)f'(x)\,dx=-\frac{1}{2}\quad\square
\end{align*}

\subsection{WHAT THE HE - Actually Hold On}

Apply Holder's with $xf(x)$, $f'(x)$, and $p=q=2$:
\begin{align*}
  & \left|\int_{b}^{a} xf(x) \cdot f'(x)\,dx\right| \le \sqrt{\int_{b}^{a} (xf(x))^2\,dx}\sqrt{\int_{b}^{a} f'(x)^2\,dx} \\
  \implies{} & \frac{1}{2} \le \sqrt{\int_{b}^{a} (xf(x))^2\,dx \cdot \int_{b}^{a} f'(x)^2\,dx} \\
  \implies{} & \frac{1}{4} \le \int_{b}^{a} (xf(x))^2\,dx \cdot \int_{b}^{a} f'(x)^2\,dx\quad\square
\end{align*}

\end{document}
