\documentclass[12pt]{article}

% a template that a friend gave, it's worked well enough for me
% i have added some packages and stuff that have proved useful

\usepackage{fancyhdr}
\usepackage{tipa}
\usepackage{fontspec}
\usepackage{amsfonts}
\usepackage{enumitem}
\usepackage[margin=1in]{geometry}
\usepackage{graphicx}
\usepackage{float}
\usepackage{amsmath}
\usepackage{braket}
\usepackage{amssymb}
\usepackage{booktabs}
\usepackage{hyperref}
\usepackage{mathtools}
\usepackage{xcolor}
\usepackage{float}
\usepackage{algpseudocodex}
\usepackage{titlesec}
\usepackage{bbm}
\usepackage{pythonhighlight}

\pagestyle{fancy}
\fancyhf{} % sets both header and footer to nothing
\lhead{Kevin Sheng}
\setmainfont{Comic Neue}
\renewcommand{\headrulewidth}{1pt}
\setlength{\headheight}{0.75in}
\setlength{\oddsidemargin}{0in}
\setlength{\evensidemargin}{0in}
\setlength{\voffset}{-.5in}
\setlength{\headsep}{10pt}
\setlength{\textwidth}{6.5in}
\setlength{\headwidth}{6.5in}
\setlength{\textheight}{8in}
\renewcommand{\headrulewidth}{0.5pt}
\renewcommand{\footrulewidth}{0.3pt}
\setlength{\textwidth}{6.5in}
\usepackage{setspace}
\usepackage{multicol}
\usepackage{float}
\setlength{\columnsep}{1cm}
\setlength\parindent{24pt}
\usepackage [english]{babel}
\usepackage [autostyle, english = american]{csquotes}
\MakeOuterQuote{"}

\setlength{\parskip}{6pt}
\setlength{\parindent}{0pt}

\titlespacing\section{0pt}{12pt plus 4pt minus 2pt}{0pt plus 2pt minus 2pt}
\titlespacing\subsection{0pt}{12pt plus 4pt minus 2pt}{0pt plus 2pt minus 2pt}
\titlespacing\subsubsection{0pt}{12pt plus 4pt minus 2pt}{0pt plus 2pt minus 2pt}

\hypersetup{colorlinks=true, urlcolor=blue}

\newcommand{\correction}[1]{\textcolor{red}{#1}}


\rhead{Math 131BH}

\makeatletter
\def\@seccntformat#1{%
  \expandafter\ifx\csname c@#1\endcsname\c@section\else
  \csname the#1\endcsname\quad
  \fi}
\makeatother

\DeclareMathOperator{\Fr}{Fr}
\newcommand{\lra}{\xLeftrightarrow}
\newcommand{\ra}{\xRightarrow}
\newcommand{\N}{\mathbb{N}}
\newcommand{\R}{\mathbb{R}}
\newcommand{\Z}{\mathbb{Z}}
\newcommand{\Q}{\mathbb{Q}}
\newcommand{\norm}[1]{\left\lVert#1\right\rVert}

\begin{document}

\section{Problem 1}

Consider these two sequences on $(0, 1)$:
\begin{gather*}
  f_n(x)=x\left(1+\frac{1}{n}\right) \\
  g_n(x)=\frac{1}{x^2}
\end{gather*}

$f_n$ converges to $f(x)=x$, since $\left|x-\left(x\left(1+\frac{1}{n}\right)\right)\right|=x\left|\frac{1}{n}\right|$ on the domain,
and the largest this can get is $\frac{1}{n}$.
$g_n$ is constant so there's not much to say there.

However, $h_n(x)=g_n(x) \cdot f_n(x)=\frac{1+\frac{1}{n}}{x}$ \textit{doesn't} uniformly converge,
even though it pointwise converges to $h(x)=\frac{1}{x}$.

To show this, let $\epsilon=1$.
As cross all $n \in \N$, we have that as long as $x < \frac{1}{n}$,
\begin{align*}
  |h_n(x)-h(x)|
   & = \left|\frac{1+\frac{1}{n}}{x}-\frac{1}{x}\right| \\
   & = \frac{1}{nx}                                     \\
   & > 1
\end{align*}
so $d_\infty(h_n, h) > 1$ and thus the product of the two uniformly convergent
sequences is not uniformly convergent itself.

\pagebreak

\section{Problem 2}

I propose that $f_n$ pointwise converges to $f(x)=0$.

Fix an $x$.
If $x \le 0$ or $x > 1$, $f_n(x)=0$ across all $n$.
For the case where $0 < x \le 1$, pick $N$ s.t. $\frac{1}{N} < x$.
Since $n \ge N \implies \frac{1}{n} \le \frac{1}{N} < x$, every $f_n$
past $N$ gives $0$.

However, across all $f_n$ $\exists x: f_n(x) = 1$.
If we let $x=\frac{1}{n+\frac{1}{2}}$, then $\frac{1}{n+1} \le \frac{1}{n+\frac{1}{2}} \le \frac{1}{n}$ and
\begin{align*}
  f_n(x)
   & = \sin^2 \frac{\pi}{\frac{1}{n+\frac{1}{2}}}        \\
   & = \sin^2 \left(\pi\left(n+\frac{1}{2}\right)\right) \\
   & = (\pm 1)^2                                         \\
   & = 1
\end{align*}
so any $f_n$ will always have a point that's $1$ away from $f$.

It should be noted that the series $F(x)=\sum_{n=1}^{\infty} f_n(x)$ absolutely converges as well.

This is because across all $x$, as we've seen above $\exists N: f_n(x)=0\ \forall n \ge N$ and so
\begin{align*}
  F(x)
   & = \sum_{n=1}^{N} f_n(x) \\
   & < \infty
\end{align*}
as a finite sum of finite terms will never do anything bad.

\pagebreak

\section{Problem 3}

\subsection{Uniform Convergence}

If $F_n(x)=\sum_{n=1}^{n} c_n I(x-x_n)$, we need to show that this sequence
of functions uniformly converges.
I guess a way we can do that is to prove that it's Cauchy, so fix an $\epsilon$.

By the convergence of $\sum |c_n|$,
$\exists N: \sum_{i=n}^{m} |c_n| < \epsilon\ \forall N \le n \le m$

This then implies that
\begin{align*}
  d_\infty(F_n, F_m)
  &= \sup \left|\sum_{i=1}^{n} c_i I(x'-x_i) - \sum_{i=1}^{m} c_iI(x'-x_i)\right| \\
  &= \sup \left|\sum_{i=n+1}^{m} c_i I(x'-x_i)\right| \\
  &\le \sup \sum_{i=n+1}^{m} |c_i I(x'-x_i)| \\
  &\le \sum_{i=n+1}^{m} |c_i| \\
  &< \epsilon\quad\square
\end{align*}

\subsection{Continuity}

Fix an $x \notin \{x_n\}$ and an $\epsilon > 0$.

Since $\sum c_n$ absolutely converges, $\exists N: \sum_{n=N}^{\infty} < \epsilon$.

Take $\delta=\min_{i=1, \cdots, N-1} |x-x_i|$, which is guaranteed to be positive.

Then, for any $x' \in (x-\delta, x+\delta)$,
\begin{align*}
  |f(x')-f(x)|
   & = \left|\sum_{n=1}^{\infty} c_nI(x'-x_n) - \sum_{n=1}^{\infty} c_nI(x'-x_n)\right| \\
   & = \left|\sum_{n=N}^{\infty} c_n(I(x'-x_n) - I(x-x_n))\right|                       \\
   & = \sum_{n=N}^{\infty} |c_n|                                                        \\
   & < \epsilon\quad\square
\end{align*}

\pagebreak

\section{Problem 4}

The sequence converges uniformly, so the resulting function
\[g(x)=\lim_{n \to \infty} f_n(x)\]
is continuous, with $g(0)=0$ and $g(1)=1$.

Thus we have
\begin{align*}
  g(x)
   & = \lim_{n \to \infty} f_n(x)                                                                         \\
   & = \lim_{n \to \infty} f(f_{n-1}(x))                          & \text{let $f_0(x)=x$ for convenience} \\
   & = f\left(\lim_{n \to \infty} f_{n-1}(x)\right) & \text{since $f$ is continuous} \\
   &= f(g(x))
\end{align*}

By the IVT, $\forall y \in [0, 1]\ \exists x: g(x)=y$,
and so $g(x)=f(g(x)) \implies y=f(y)$. $\square$

\pagebreak

\section{Problem 5}

Consider the Dirichlet function on $[0, 1]$.

$\Q$ is countably infinite, so define enumerate them like $\{q_1, q_2, \cdots\}$ and let $f_n$ be
\[f_n(x)=\begin{cases}
    0 & x \in \{q_1, q_2, \cdots, q_n\} \cup \R \\
    1 & \text{otherwise}
  \end{cases}\]
Every $x \in [0, 1]$ will become $0$ eventually, so $f_n$ pointwise converges to $f(x)=0$.

However, $|f_n(x)-f(x)|^2=f_n(x)^2$ isn't even integrable no matter what $n$ is,
so such a limit wouldn't even make sense.

\section{Problem 6}

For convenience I'll let $F_i$, $G_i$, and $FG_i$ denote the range sups of $f_n$, $g$, and $f_n \cdot g$.
Surely it'll be clear what I'm talking about when messing with partitions :clueless:.

Pick an $\epsilon$.
We need an $N$ s.t. the expression is in $g(0) \pm \epsilon$ for everything past $N$.

\subsection{Middle Part}

WLOG assume $g(0) > \epsilon$.
$g$ is continuous, so take $c$ s.t. $|x| < c \implies |g(x)-g(0)| < \epsilon$.

For any partition $P$ of $[-c, c]$, notice that since everything in the
interal can be at most $\epsilon$ away from $g(0)$ and that $f$ is nonnegative,
\[\sup_{x \in [x_i, x_{i+1}]} F_i(g(0)-\epsilon) \le \sup_{x \in [x_i, x_{i+1}]} FG_i \le \sup_{x \in [x_i, x_{i+1}]} F_i(g(0)+\epsilon)\]
giving us
\begin{align*}
             & \sum_{i=1}^{n} F_i(g(0)-\epsilon)\Delta x_i \le \sum_{i=1}^{n} FG_i \Delta x_i \le \sum_{i=1}^{n} F_i(g(0)+\epsilon)\Delta x_i \\
  \implies{} & (g(0)-\epsilon) \int_{-c}^{c} f(x)\,dx \le \int_{-c}^{c} f(x)g(x)\,dx \le (g(0)+\epsilon) \int_{-c}^{c} f(x)\,dx
\end{align*}

\pagebreak

\subsection{Ending Parts}

Now we can only mess with $n$.
Let $G=\sup_{x \in [-1, 1]} |g(x)|$ and choose $N$ s.t.
\[|f_n(x)| < \min\left(\frac{\epsilon}{2G}, \frac{\epsilon}{2}\right)\ \forall n \ge N, x \in [-1, -c] \cup [c, 1]\]
Taking the integral of $f_n \cdot g$ from $c$ to $1$, we see
\begin{align*}
  \left|\int_{c}^{1} f_n(x)g(x)\,dx\right|
   & \le \int_{c}^{1} |f_n(x)g(x)|\,dx                                      \\
   & = \inf_P \sum_{i=1}^{n} |FG_i| \Delta x_i                              \\
   & \le \inf_P \sum_{i=1}^{n} |F_i||G_i| \Delta x_i                        \\
   & \le \inf_P \sum_{i=1}^{n} \frac{\epsilon}{2G} \cdot G \cdot \Delta x_i \\
   & \le \frac{\epsilon}{2}
\end{align*}
and by similar logic $\left|\int_{-1}^{-c} f_n(x)g(x)\,dx\right| < \frac{\epsilon}{2}$.

Also, since $f_n$ is nonnegative, $\int_{-c}^{c} f(x)\,dx \le 1$ and
\begin{align*}
  \int_{-c}^{c} f(x)\,dx
   & = \int_{-1}^{1} f(x)\,dx - \left(\int_{-1}^{-c} f(x)\,dx + \int_{c}^{1} f(x)\,dx\right) \\
   & \ge 1 - 2 \cdot \frac{\epsilon}{2}                                                      \\
   & = 1-\epsilon
\end{align*}

\pagebreak

\subsection{Putting it Together}

With that, we can actually have a good bound on the middle term.
On the left,
\begin{align*}
  \int_{-c}^{c} f_n(x)g(x)\,dx
   & \ge (g(0)-\epsilon)\int_{-c}^{c} f_n(x)\,dx \\
   & \ge (g(0)-\epsilon)(1-\epsilon)             \\
   & = g(0) - \epsilon(\epsilon+g(0)+1)
\end{align*}
and on the right,
\begin{align*}
  \int_{-c}^{c} f_n(x)g(x)\,dx
   & \le (g(0)+\epsilon)\int_{-c}^{c} f_n(x)\,dx \\
   & \le g(0)+\epsilon
\end{align*}
so the middle term can't be too far away from $g(0)$.

As we've shown previously,
\[\left|\int_{-c}^{1} f_n(x)g(x)\,dx + \int_{c}^{1} f_n(x)g(x)\,dx \right|\le \epsilon\]
and thus
\[g(0)-\epsilon(\epsilon+g(0)+2) \le \int_{-1}^{1} f_n(x)g(x) \le g(0)+2\epsilon\]
and uh, oops, this is not the interval we (or at least I) was looking for,
but it's close enough.
$\epsilon$ going to $0$ makes these bounds squeeze the middle to $g(0)$
and that's what were were trying to show regardless. $\square$

\pagebreak

\section{Problem 7}

Take a convergent sequence $f_n$ in $\mathcal{F}$, so
\[\exists f: \forall \epsilon > 0\ \exists N: d_\infty(f_n, f) < \epsilon\ \forall n \ge N\]
It STP $f \in \mathcal{F}$ as well.

To do that, fix an $\epsilon$.
Take the big $N$ in the above statement corresponding to $\frac{\epsilon}{2}$.

$f_N \in \mathcal{F}$, so $\exists R: |x| > R \implies |f_N(x)| < \frac{\epsilon}{2}$.
$d_\infty(f_N, f) < \frac{\epsilon}{2}$, so
\[|f(x)| \le |f(x)-f_N(x)|+|f_N(x)-0| \le \epsilon\]
and just like that, we see that this $R$ works for $f$ itself as well. $\square$

\end{document}
