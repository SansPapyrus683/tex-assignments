\documentclass[12pt]{article}

% a template that a friend gave, it's worked well enough for me
% i have added some packages and stuff that have proved useful

\usepackage{fancyhdr}
\usepackage{tipa}
\usepackage{fontspec}
\usepackage{amsfonts}
\usepackage{enumitem}
\usepackage[margin=1in]{geometry}
\usepackage{graphicx}
\usepackage{float}
\usepackage{amsmath}
\usepackage{braket}
\usepackage{amssymb}
\usepackage{booktabs}
\usepackage{hyperref}
\usepackage{mathtools}
\usepackage{xcolor}
\usepackage{float}
\usepackage{algpseudocodex}
\usepackage{titlesec}
\usepackage{bbm}
\usepackage{pythonhighlight}

\pagestyle{fancy}
\fancyhf{} % sets both header and footer to nothing
\lhead{Kevin Sheng}
\setmainfont{Comic Neue}
\renewcommand{\headrulewidth}{1pt}
\setlength{\headheight}{0.75in}
\setlength{\oddsidemargin}{0in}
\setlength{\evensidemargin}{0in}
\setlength{\voffset}{-.5in}
\setlength{\headsep}{10pt}
\setlength{\textwidth}{6.5in}
\setlength{\headwidth}{6.5in}
\setlength{\textheight}{8in}
\renewcommand{\headrulewidth}{0.5pt}
\renewcommand{\footrulewidth}{0.3pt}
\setlength{\textwidth}{6.5in}
\usepackage{setspace}
\usepackage{multicol}
\usepackage{float}
\setlength{\columnsep}{1cm}
\setlength\parindent{24pt}
\usepackage [english]{babel}
\usepackage [autostyle, english = american]{csquotes}
\MakeOuterQuote{"}

\setlength{\parskip}{6pt}
\setlength{\parindent}{0pt}

\titlespacing\section{0pt}{12pt plus 4pt minus 2pt}{0pt plus 2pt minus 2pt}
\titlespacing\subsection{0pt}{12pt plus 4pt minus 2pt}{0pt plus 2pt minus 2pt}
\titlespacing\subsubsection{0pt}{12pt plus 4pt minus 2pt}{0pt plus 2pt minus 2pt}

\hypersetup{colorlinks=true, urlcolor=blue}

\newcommand{\correction}[1]{\textcolor{red}{#1}}


\rhead{Math 131BH}

\makeatletter
\def\@seccntformat#1{%
  \expandafter\ifx\csname c@#1\endcsname\c@section\else
  \csname the#1\endcsname\quad
  \fi}
\makeatother

\DeclareMathOperator{\Fr}{Fr}
\newcommand{\lra}{\xLeftrightarrow}
\newcommand{\ra}{\xRightarrow}

\begin{document}

\section{Problem 1}

> hint \\
> look inside \\
> sketch of entire proof \\
> mfw

\subsection{Continuous}

Lemme first just prove that $x(t)$ is continuous; fix a $t$ and $\epsilon > 0$.

Since $x(t)$ is bounded by a geometric series,
\[\exists N \in \N: \sum_{n=N+1}^{\infty} 2^{-n}f\left(3^{2n-1}t\right) < \epsilon\]
Now let's only consider the first $N$ terms of the series.

$f$ is continuous, so pick $\delta'$ s.t. $|x-y| < \delta' \implies |f(x)-f(y)| < \frac{\epsilon}{N}$.

Now let our actual $\delta=\frac{\delta'}{3^{2N-1}}$.
Now, for any $1 \le n \le N$ and $t' \in B_\delta'(t)$,
\begin{align*}
  \left|3^{2n-1}t' - 3^{2n-1}t\right|
   & = 3^{2n-1}\left|t'-t\right|               \\
   & < 3^{2n-1} \cdot \frac{\delta'}{3^{2N-1}} \\
   & < \delta
\end{align*}
which means $\left|f\left(3^{2n-1}t'\right)-f\left(3^{2n-1}t\right)\right| < \frac{\epsilon}{N}$ and
\begin{align*}
        & |x(t)-x(t')|                                                                                           \\
  ={}   & \left|\left(\sum_{n=1}^{N} 2^{-n}\left(f\left(3^{2n-1}t\right)-f\left(3^{2n-1}t'\right)\right)\right)
  +\left(\sum_{n=N+1}^{\infty} 2^{-n}\left(f\left(3^{2n-1}t\right)-f\left(3^{2n-1}t'\right)\right)\right)\right| \\
  ={}   & \left|\sum_{n=1}^{N} 2^{-n}\left(f\left(3^{2n-1}t\right)-f\left(3^{2n-1}t'\right)\right)\right|
  +\left|\sum_{n=N+1}^{\infty} 2^{-n}\left(f\left(3^{2n-1}t\right)-f\left(3^{2n-1}t'\right)\right)\right|        \\
  \le{} & \sum_{n=1}^{N} 2^{-n}\frac{\epsilon}{N} + 2\epsilon                                                    \\
  \le{} & 3\epsilon\quad\square
\end{align*}

\pagebreak

\subsection{Onto}

Fix $x_0$ and $y_0$; we'll try to find a $t_0$ s.t. $x(t_0)=x_0$ and $y(t_0)=y_0$.

Since all numbers have a binary expansion, any $x \in [0, 1]$ can be written as
\[x=\sum_{n=1}^{\infty} 2^{-n}x_n\]
If we let $x_n$ be the sequence for $x_0$ and $y_n$ be the same for $y_0$, then
we can interleave the two sequences into a new sequence $a_n$ like so:
\[a_n=\begin{cases}
    x_{(n+1)/2} & \text{$n$ is odd}  \\
    y_{n/2}     & \text{$n$ is even}
  \end{cases}\]
and we can then write $x_0$ and $y_0$ like so:
\begin{align*}
  x_0=\sum_{n=1}^{\infty} 2^{-n}a_{2n-1} &  &
  y_0=\sum_{n=1}^{\infty} 2^{-n}a_{2n}
\end{align*}

Rudin told me that letting $t_0=\sum_{t=1}^{\infty} 3^{-t-1}(2a_t)$, should work, so let's prove it.

First off, I should say that $t_0$ is actually in the Cantor set,
since it's ternary expansion has only $0$s and $2$s.
By proving this we're also proving that $f$ maps both this set and $[0, 1]$ onto $[0, 1]^2$.

Notice that if $k \in \N$, then
\begin{align*}
  f\left(3^kt_0\right)
   & = f\left(3^k\sum_{t=1}^{\infty} 3^{-t-1}(2a_t)\right)                              \\
   & = f\left(\sum_{t=1}^{\infty} 3^{k-t-1}(2a_t)\right)                                \\
   & = f\left(2\sum_{t=1}^{k-1} 3^{k-t-1}a_t+\sum_{t=k}^{\infty} 3^{k-t-1}(2a_t)\right) \\
   & = f\left(\sum_{t=k}^{\infty} 3^{k-t-1}(2a_t)\right)                                \\
   & = f\left(\frac{2a_k}{3}+\sum_{t=k+1}^{\infty} 3^{k-t-1}(2a_t)\right)               \\
   & = a_k
\end{align*}
We can remove the first $k-1$ terms of the summation because
those add integer multiples of $2$ to the argument and $f$ is $2$-periodic.

Also, since the rest of the summation can't exceed $\frac{\frac{1}{9}}{1-\frac{1}{3}}=\frac{1}{6}$,
the only part that matters is $\frac{2a_k}{3}$, which is $0$ if $a_k=0$ and $\frac{2}{3}$ otherwise.

With this identity, we have for $x$
\begin{align*}
  x(t_0)
   & = \sum_{n=1}^{\infty} 2^{-n}f\left(3^{2n-1}t_0\right) \\
   & = \sum_{n=1}^{\infty} 2^{-n}a_{2n-1}                  \\
   & = x_0
\end{align*}
and for $y$
\begin{align*}
  y(t_0)
   & = \sum_{n=1}^{\infty} 2^{-n}f\left(3^{2n}t_0\right) \\
   & = \sum_{n=1}^{\infty} 2^{-n}a_{2n}                  \\
   & = y_0\quad\square
\end{align*}

\pagebreak

\section{Problem 2}

\subsection{Bound on \texorpdfstring{$f_p$}{f\_p}}

Since everything's the output of $d$, all numbers here are nonnegative.

Anyways, if $d(x, p) \ge d(x, a)$ then $|f_p(x)|=d(x, p)-d(x, a)$ and
$d(x, p)-d(x, a) \le d(a, p)$ by the triangle inequality.

Similarly, if $d(x, p) < d(x, a)$ then $|f_p(x)|=d(x, a)-d(x, p)$ and
we get the same thing with $d(x,a)-d(x,p) \le d(a, p)$ by the triangle inequality.

\subsection{Continuity}

Fix an $\epsilon$ and $x \in X$.
If we force $x' \in B_{\frac{\epsilon}{2}}(x)$, then
\begin{align*}
  |f_p(x)-f_p(x')|
   & = |d(x, p) - d(x, a) - d(x', p) + d(x', a)|     \\
   & \le |d(x, p) - d(x', p)| + |d(x', a) - d(x, a)| \\
   & \le d(x, x') + d(x, x')                         \\
   & \le \epsilon
\end{align*}

\subsection{Distance Between Functions}

We have for all $x \in X$
\begin{align*}
  |f_p(x) - f_q(x)|
   & = |d(x, p) - d(x, a) - d(x, q) + d(x, a)| \\
   & = |d(x, p) - d(x, q)|                     \\
   & \le d(p, q)
\end{align*}
so $\norm{f_p-f_q} \le d(p, q)$.
Equality can be achieved if we choose $x=p$ or $x=q$,
so this means $\norm{f_p-f_q} = d(p, q)$ and not an $\epsilon$ greater.

I'm not really sure why we need to show that $Y$ is complete,
since it's closed in a complete metric space.

\pagebreak

\section{Problem 3}

\subsection{LTs are Pointwise Bounded}

I'm NGL there's probably a more elegant way to do this with bases or whatever,
but no one said I was an elegant person.

Choose the standard basis and let $A$ be the matrix associated with the basis and LT.

$A$ is finite, so $\exists M: |A_{ij}| < M\ \forall i, j$.
Also, let $X=\max_{i=1, \cdots, n} x_i$.

For each $1 \le i \le m$, we've
\begin{align*}
  |(Ax)_i|
   & = \left|\sum_{j=1}^{n} A_{ij}x_j\right| \\
   & < M\sum_{j=1}^{n} |x_j|                 \\
   & \le MXn
\end{align*}

Then to actually bound the magnitude we have
\begin{align*}
  \norm{Ax}
   & = \sqrt{\sum_{i=1}^{m} (Ax)_i^2}  \\
   & < \sqrt{\sum_{i=1}^{m} M^2X^2n^2} \\
   & = Mn\sqrt{mX^2}                   \\
   & = Mn\sqrt{m} \cdot X              \\
   & \le Mn\sqrt{m} \cdot \norm{x}
\end{align*}
so our actual bound is \boxed{Mn\sqrt{m}}. $\square$

\pagebreak

\subsection{Differentiability Implies Continuity}

Really quick tangent- I'm not sure if this needs to be proven,
but for any vectors $x$ and $y$ we have by the triangle inequality
\[\norm{x} \le \norm{x-y}+\norm{-y} \implies \norm{x}-\norm{y} \le \norm{x-y}\]

But anyways, for continuity, for any $\epsilon > 0$ we need a $\delta$ s.t.
\[\norm{\vec{h}} < \delta \implies \norm{f\left(x_0+\vec{h}\right)-f(x_0)} < \epsilon\]

Since $f$'s differentiable at $x_0$, $\exists A \in \R^{m \times n}$
s.t. $\exists \delta'$ s.t.
\[\norm{\vec{h}} < \delta' \implies
  \frac{\norm{f\left(x_0+\vec{h}\right)-f(x_0)-A\vec{h}}}{\norm{\vec{h}}} < \epsilon\]

By what we just proved in the earlier part, $\exists M: \norm{A\vec{x}} < M\norm{\vec{x}}$.

Taking $\delta = \min(\delta', \epsilon)$ (kinda cursed but it is what it is),
we have for all $\norm{\vec{h}} < \delta$
\begin{align*}
             & \norm{f\left(x_0+\vec{h}\right)-f(x_0)-A\vec{h}} < \epsilon\norm{\vec{h}}        \\
  \implies{} & \norm{f\left(x_0+\vec{h}\right)-f(x_0)}-\norm{A\vec{h}} < \epsilon\norm{\vec{h}}
\end{align*}
so we can then bound the term we want like so:
\begin{align*}
  \norm{f\left(x_0+\vec{h}\right)-f(x_0)}
   & < \epsilon\norm{\vec{h}} + \norm{A\vec{h}} \\
   & < \epsilon\delta + M\delta                 \\
   & \le \epsilon^2 + M\epsilon\quad\square
\end{align*}

\pagebreak

\section{Problem 4}

\subsection{Not Partially Differentiable}

If there were a partial derivative, it'd have to be
\[\begin{bmatrix}\frac{\partial f}{\partial x} & \frac{\partial f}{\partial y}\end{bmatrix}\]
To calculate this, we take the limit of the difference quotient first holding $y$ constant:
\begin{align*}
  \lim_{h \to 0} \frac{f(0 + h, 0)-f(0, 0)}{h}
   & = \lim_{h \to 0} \frac{f(h, 0)}{h}         \\
   & = \lim_{h \to 0} \frac{\frac{h^3}{h}}{h^2} \\
   & = 1
\end{align*}
and then do the same but hold $x$ constant instead:
\begin{align*}
  \lim_{h \to 0} \frac{f(0, h)}{h}
   & = \lim_{h \to 0} \frac{\frac{0}{h^2}}{h} \\
   & = 0
\end{align*}
so if $Df$ existed it'd have to be $\begin{bmatrix}1 & 0\end{bmatrix}$.

However, if we let $\vec{h}=\braket{d, d}$ and send that to $0$,
\begin{align*}
  \lim_{d \to 0} \frac{\norm{f(d, d)-f(0, 0)-(1 \cdot d + 0 \cdot d)}}{\norm{d}}
   & = \lim_{d \to 0} \frac{\norm{\frac{d^3}{2d^2}-d}}{\sqrt{d^2 + d^2}} \\
   & = \lim_{d \to 0} \frac{\left|\frac{d}{2}\right|}{\sqrt{2}|d|}       \\
   & = \frac{1}{2\sqrt{2}}                                               \\
   & \ne 0
\end{align*}
which contradicts that the given matrix specified a valid linear transformation. $\square$

\pagebreak

\subsection{But OK in Any One Direction}

If we let $\hat{h}=\braket{h_1, h_2}$ be a unit vector,
we just have to take directional derivatives like so:
\begin{align*}
  \lim_{d \to 0} \frac{\braket{0, 0} + f(d\hat{h})-f(\braket{0, 0})}{d}
   & = \lim_{d \to 0} \frac{1}{d} \cdot \frac{d^3h_1^3}{d^2h_1^2+d^2h_2^2} \\
   & = \lim_{d \to 0} \frac{d^2h_1^3}{d^2h_1^2+d^2h_2^2}                   \\
   & = \frac{h_1^3}{h_1^2+h_2^2}
\end{align*}
I mean yeah, the limit exists, not much else to say here.

\pagebreak

\section{Problem 5}

\subsection{Differentiable}

I propose that the derivative is $\begin{bmatrix}0 & 0\end{bmatrix}$ at $(0, 0)$.

So if we go from the definition, we need to show that
\begin{align*}
  ={} & \lim_{\norm{h} \to 0} \frac{f(h_1 + h_2)-f(0, 0) - (0 \cdot h_1 + 0 \cdot h_2)}{\norm{h}} \\
  ={} & \lim_{\norm{h} \to 0} \frac{\frac{h_1^3h_2}{h_1^2+h_2^2}}{\sqrt{h_1^2+h_2^2}}                        \\
  ={} & \lim_{\norm{h} \to 0} \frac{h_1^3h_2}{\left(h_1^2+h_2^2\right)^{3/2}}
\end{align*}
is $0$.

I'll actually first prove that the limit of the square is $0$, or in other words
\[\lim_{\norm{h} \to 0} \frac{h_1^6h_2^2}{h_1^2+h_2^2}=0\]

Fix an $\epsilon$.
If we choose $\delta=\min(\epsilon, 1)$, then for any $h$ s.t. $\norm{h} < \delta$,
$h_1, h_2 < \delta$ as well and
\begin{align*}
  \frac{h_1^6h_2^2}{h_1^2+h_2^2}
   & \le \frac{h_1^6h_2^2}{h_1^2h_2^2+h_2^2h_1^2} \\
   & = \frac{h_1^4}{2}                            \\
   & \le \frac{\epsilon^4}{2}
\end{align*}
so as $\epsilon \to 0$, the bound we've achieved goes to $0$ as well
since it's bounded from above and clearly nonnegative.

Since the square goes to $0$ the original limit must go to $0$ too. $\square$

\pagebreak

\subsection{Order Matters}

Since we just showed that $f'(0, 0)=\begin{bmatrix}0 & 0\end{bmatrix}$,
this implies that $\frac{\partial f}{\partial x}(0, 0) = \frac{\partial f}{\partial y}(0, 0) = 0$.

By basic calculus
\begin{align*}
  \frac{\partial f}{\partial x}
   & = \frac{3x^2y\left(x^2+y^2\right)-2x \cdot x^3y}{\left(x^2+y^2\right)^2} \\
   & = \frac{x^4y+3x^2y^3}{\left(x^2+y^2\right)^2}
\end{align*}
Then we use the limit definition to calculate the partial of this wrt $y$:
\begin{align*}
  \lim_{h \to 0} \frac{\frac{\partial f}{\partial x}(0, h)-\frac{\partial f}{\partial x}(0, 0)}{h}
   & = \lim_{h \to 0} \frac{\frac{0}{h^4}}{h} \\
   & = 0
\end{align*}

OTOH, if we started with $y$, we get
\begin{align*}
  \frac{\partial f}{\partial y}
   & = \frac{x^3\left(x^2+y^2\right)-2y \cdot x^3y}{\left(x^2+y^2\right)^2} \\
   & = \frac{x^5-2x^3y^2}{\left(x^2+y^2\right)^2}
\end{align*}
and then using the limit definition again to get the partial wrt $x$ we've
\begin{align*}
  \lim_{h \to 0} \frac{\frac{\partial f}{\partial y}(h, 0)-\frac{\partial f}{\partial y}(0, 0)}{h}
   & = \lim_{h \to 0} \frac{\frac{h^5}{h^4}}{h} \\
   & = 1
\end{align*}
giving us a different result from above.

\pagebreak

\section{Problem 6}

All the partials are bounded, so
$\left|\frac{\partial f}{\partial x_i}\right| < M\ \forall x \in E, 1 \le i \le n$.

Let $\vec{h}=\sum_{i=1}^{n} c_i\vec{e}_i$.
I'll denote $s(x)$ where x is an integer in $[0, n]$ as $\sum_{i=1}^{x} c_i\vec{e}_i$.
For convenience let $s(0)=\vec{0}$.
In other words, it represents the first $x$ elements of $h$.

To get a bound on $|f(x+h)-f(x)|$, let's rewrite it as a telescoping sum:
\begin{align*}
  |f(x+\vec{h})-f(x)|
   & = |f(x+s(n))-f(x+s(0))|                                                                        \\
   & = \left|\sum_{i=1}^{n} f(x+s(i))-f(x+s(i-1))\right|                                            \\
   & = \sum_{i=1}^{n} \left|\frac{\partial f}{\partial x_i}(x+s(i-1)+k_i\vec{e}_i) \cdot c_i\right| \\
   & \le \sum_{i=1}^{n} M|c_i|
\end{align*}
where $0 < k_i < c_i$ by the MVT.

As $\norm{h}$ tends to $0$, all the $c_i$s do as well.
Besides that, everything else is finite, so
\[\lim_{n \to 0} |f(x+\vec{h})-f(x)| = 0\quad\square\]

\pagebreak

\section{Problem 7}

\subsection{Weaker Condition and Proof}

The weaker condition is as follows.
For any $x$ and $y$, we only need the convexity property to hold if
$x_i=y_i$ for all $i$ from $2$ to $n$. ($E$ is still has to be open.)

BWOC $f(x)$ depended on $x_1$, so $\exists \vec{x}, c$ s.t. $f(\vec{x}) \ne f(\vec{x}+c\vec{e}_1)$.

By the MVT, $\exists 0 < m < c$ s.t.
\[\frac{\partial f}{\partial x_1}(x+m\vec{e}_1)=\frac{f(x+c\vec{e}_1)-f(x)}{c}\]
Since $x+c\vec{e}_1$ and $x$ only differ in the first term,
the linear combination of them at which we are evaluating the partial must be $0$.
However, the numerator and denominator on the RHS are both nonzero,
which is a contradiction. $\square$

\subsection{Needed BTW}

Nothing was stated about continuity of $f$ or connectedness of $E$,
so here's a really troll example with the following $f: \R \to \R$:
\[f(x)=\begin{cases}
    1 & x \ge 0          \\
    0 & \text{otherwise}
  \end{cases}\]

Then, let $E=(-2, -1) \cup (1, 2)$.
Though $f'(x)$ exists and is $0$ on both these intervals,
$f$ clearly depends on $x$.

\pagebreak

\section{Problem 8}

\subsection{Product Rule}

I suppose there's no other choice but to bash out the limits.
Lemme first do some stuff with the difference, though:
\begin{align*}
      & f(x+h)g(x+h)-f(x)g(x)-(f(x)g'(x)+g(x)f'(x)) \cdot h         \\
  ={} & f(x+h)(g(x+h)-g(x))+g(x)(f(x+h)-f(x))-f(x)g'(x)h-g(x)f'(x)h \\
  ={} & f(x+h)(g(x+h)-g(x))-f(x)g'(x)h+g(x)(f(x+h)-f(x)-f'(x)h)
\end{align*}

Then we actually do the limit.
I've removed the subscripts, as all of them are as $\norm{h} \to 0$.
\begin{align*}
      & \lim \frac{\text{that expression above}}{\norm{h}}                       \\
  ={} & \lim \frac{f(x+h)(g(x+h)-g(x))-f(x)g'(x)h}{\norm{h}}
  + g(x)\lim \frac{f(x+h)-f(x)-f'(x)h}{\norm{h}}                                 \\
  ={} & \lim \frac{f(x+h)(g(x+h)-g(x))-f(x)g'(x)h}{\norm{h}}                     \\
  ={} & \lim \frac{f(x+h)(g(x+h)-g(x)-g'(x)h)+f(x+h)g'(x)h-f(x)g'(x)h}{\norm{h}} \\
  ={} & \lim f(x+h) \cdot \lim \frac{g(x+h)-g(x)-g'(x)h}{\norm{h}}
  + \lim \frac{g'(x)h \cdot (f(x+h)-f(x))}{\norm{h}}                             \\
  ={} & \lim g'(x)\frac{h}{\norm{h}} \cdot \lim f(x+h)-f(x)               \\
  ={} & 0\quad\square
\end{align*}

\subsection{Inverse Thing}

We can apply the chain rule.
Let $f: \R^n \to \R$ be as stated and $h: \R \setminus \{0\} \to \R$ be $h(x)=\frac{1}{x}$.

Then, yeah, we just have this:
\begin{align*}
  D(h \circ f)(x_0)
   & = Dh(f(x_0))Df(x_0)                      \\
   & = -\frac{1}{f(x_0)^2}Df(x_0)\quad\square
\end{align*}

\end{document}
