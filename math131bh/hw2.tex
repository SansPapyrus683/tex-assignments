\documentclass[12pt]{article}

% a template that a friend gave, it's worked well enough for me
% i have added some packages and stuff that have proved useful

\usepackage{fancyhdr}
\usepackage{tipa}
\usepackage{fontspec}
\usepackage{amsfonts}
\usepackage{enumitem}
\usepackage[margin=1in]{geometry}
\usepackage{graphicx}
\usepackage{float}
\usepackage{amsmath}
\usepackage{braket}
\usepackage{amssymb}
\usepackage{booktabs}
\usepackage{hyperref}
\usepackage{mathtools}
\usepackage{xcolor}
\usepackage{float}
\usepackage{algpseudocodex}
\usepackage{titlesec}
\usepackage{bbm}
\usepackage{pythonhighlight}

\pagestyle{fancy}
\fancyhf{} % sets both header and footer to nothing
\lhead{Kevin Sheng}
\setmainfont{Comic Neue}
\renewcommand{\headrulewidth}{1pt}
\setlength{\headheight}{0.75in}
\setlength{\oddsidemargin}{0in}
\setlength{\evensidemargin}{0in}
\setlength{\voffset}{-.5in}
\setlength{\headsep}{10pt}
\setlength{\textwidth}{6.5in}
\setlength{\headwidth}{6.5in}
\setlength{\textheight}{8in}
\renewcommand{\headrulewidth}{0.5pt}
\renewcommand{\footrulewidth}{0.3pt}
\setlength{\textwidth}{6.5in}
\usepackage{setspace}
\usepackage{multicol}
\usepackage{float}
\setlength{\columnsep}{1cm}
\setlength\parindent{24pt}
\usepackage [english]{babel}
\usepackage [autostyle, english = american]{csquotes}
\MakeOuterQuote{"}

\setlength{\parskip}{6pt}
\setlength{\parindent}{0pt}

\titlespacing\section{0pt}{12pt plus 4pt minus 2pt}{0pt plus 2pt minus 2pt}
\titlespacing\subsection{0pt}{12pt plus 4pt minus 2pt}{0pt plus 2pt minus 2pt}
\titlespacing\subsubsection{0pt}{12pt plus 4pt minus 2pt}{0pt plus 2pt minus 2pt}

\hypersetup{colorlinks=true, urlcolor=blue}

\newcommand{\correction}[1]{\textcolor{red}{#1}}


\rhead{Math 131BH}

\makeatletter
\def\@seccntformat#1{%
  \expandafter\ifx\csname c@#1\endcsname\c@section\else
  \csname the#1\endcsname\quad
  \fi}
\makeatother

\DeclareMathOperator{\Fr}{Fr}
\newcommand{\lra}{\xLeftrightarrow}
\newcommand{\ra}{\xRightarrow}
\newcommand{\N}{\mathbb{N}}
\newcommand{\R}{\mathbb{R}}
\newcommand{\Z}{\mathbb{Z}}
\newcommand{\Q}{\mathbb{Q}}
\newcommand{\norm}[1]{\left\lVert#1\right\rVert}

\begin{document}

\section{Problem 1}

Let $x$ be any real and $\epsilon > 0$.
I'll prove that $\lim_{t \to x} f(t) = 0$.

Fix $\epsilon > 0$ and choose $N \in \N: \frac{1}{N} < \epsilon$.

For any $m < N$, by the Archimedean property $\exists n \in \Z: \frac{n}{m} \le x < \frac{n+1}{m}$.

Now we can choose $\delta$ like so:
\[\delta = \min_{1 \le m < N} \min\left(x-\frac{n}{m}, \frac{n+1}{m}-x\right)\]
With an edge case that if $x - \frac{n}{m} = 0$ we only consider the right term in the minimum.

Now consider any $y \ne x$ where $|x-y| < \delta$.
Suppose FSOC $y=\frac{a}{b}$ where $\frac{1}{b} \ge \epsilon$.
This means $b < N$ and going back to our construction we have an
$n \in \Z: \frac{n}{b} \le x < \frac{n+1}{b}$.
By our choice of $\delta$, this forces $\frac{n}{b} < \frac{a}{b} < \frac{n+1}{b} \implies a \notin \Z$,
which is a contradiction.

If $x \in \R \setminus \Q$, $f(x)=0$ so $f$ is continuous on all irrationals.
If not, $f(x) \ne 0$ even though the limit still exists, so there's a simple
discontinuity at all rationals. $\square$

\pagebreak

\section{Problem 2}

\subsection{Convexity Implies Continuous}

Fix $x \in (a, b)$ and $\epsilon > 0$.
We first take any $d$ s.t. $f(x+d)$ and $f(x-d)$ are both defined.
This is possible since $(a, b)$ is an open interval.

Now by the inequality that's proved in \ref{sec:prob2p3}, for any $y \in (x-d, x)$
\begin{align*}
  \frac{f(x)-f(x-d)}{d} \le \frac{f(x)-f(y)}{x-y} \le \frac{f(x+d)-f(y)}{x+d-y}
\end{align*}

The left side inequality simplies like so:
\begin{align*}
             & \frac{f(x)-f(x-d)}{d} \le \frac{f(x)-f(y)}{x-y}    \\
  \implies{} & (f(x)-f(x-d))(x-y) \le d(f(x)-f(y))                \\
  \implies{} & d \cdot f(y) \le d \cdot f(x) - (f(x)-f(x-d))(x-y) \\
  \implies{} & f(y) \le f(x) - \frac{(f(x)-f(x-d))(x-y)}{d}
\end{align*}
Notice that everything here except $y$ is constant.
Thus, we can choose a $y$ close enough to $x$ s.t. the expression on the RHS is less than $x+\epsilon$.

We do some similar manipulation for the right inequality:
\begin{align*}
             & \frac{f(x)-f(y)}{x-y} \le \frac{f(x+d)-f(y)}{x+d-y}         \\
  \implies{} & f(x)(x+d-y)-f(y)(x+d)+f(y)y \le f(x+d)(x-y) - f(y)x + f(y)y \\
  \implies{} & f(x)(x+d-y)-d \cdot f(y) \le f(x+d)(x-y)                    \\
  \implies{} & f(y) \ge \frac{f(x)(x+d-y)-f(x+d)(x-y)}{d}                  \\
  \implies{} & f(y) \ge f(x)+(x-y) \cdot \frac{f(x)-f(x+d)}{d}
\end{align*}
where this time, if we choose an appropriate $y$, we can make the RHS greater than $x-\epsilon$.

Taking the max of the two $y$s will get us a $y_\text{left}$
where any $y \in (y_\text{left}, x)$ satsifies $|f(x)-f(y)| < \epsilon$.

We can do something similar for any $y \in (x, x+d)$ to get a range
in which $f(y)$ has to be within $x \pm \epsilon$ as well,
but that's just a butt-ton more algebraic manipulation.

This means we can set $\delta=\min(x-y_\text{left}, y_\text{right}-x)$ for everything to work out. $\square$

\pagebreak

\subsection{Convexity of Compositions}

\[\begin{aligned}
          & g(f(\lambda x (1-\lambda)y))                                                    \\
    \le{} & g(\lambda f(x) + (1 - \lambda) f(y))  &  & \text{$f$ convex and $g$ increasing} \\
    \le{} & \lambda g(f(x)) + (1-\lambda) g(f(y)) &  & \text{$g$ convex}\quad\square
  \end{aligned}\]

\subsection{Convex Inequality}\label{sec:prob2p3}

Since $s < t < u$, $\exists \lambda \in [0, 1]: t = \lambda s + (1 - \lambda)u$.

With some algebraic manipulation, we get
\begin{align*}
             & t = \lambda s + (1 - \lambda)u               \\
  \implies{} & t = \lambda(s-u)+u                           \\
  \implies{} & \frac{t-u}{s-u} = \frac{u-t}{u-s}= \lambda   \\
  \implies{} & 1-\frac{u-t}{u-s}=\frac{t-s}{u-s}= 1-\lambda
\end{align*}

Now, with the inequality $f(t) \le \lambda f(s) + (1 - \lambda)f(u)$, we have
\begin{align*}
             & f(t) \le \frac{u-t}{u-s}f(s) + \frac{t-s}{u-s}f(u)                                                      \\
  \implies{} & f(t)(u-s) \le f(s)(u-t) + f(u)(t-s)                                                                     \\
  \implies{} & f(t)u - f(t)s - f(s)u \le f(u)t - f(u)s - f(s)t \stepcounter{equation}\tag{\theequation}\label{checkpt} \\
  \implies{} & f(t)u - f(t)s - f(s)u + f(s)s \le f(u)t - f(u)s - f(s)t + f(s)s                                         \\
  \implies{} & (f(t)-f(s))(u-s) \le (f(u)-f(s))(t-s)                                                                   \\
  \implies{} & \frac{f(t)-f(s)}{t-s} \le \frac{f(u)-f(s)}{u-s}
\end{align*}
which is the first inequality we wanted to prove.

Starting from \ref{checkpt} again, we also have
\begin{align*}
             & f(s)t - f(s)u- f(u)t \le f(u)t + f(t)s - f(t)u                  \\
  \implies{} & f(s)t - f(s)u- f(u)t + f(u)u \le f(u)t + f(t)s - f(t)u  + f(u)u \\
  \implies{} & (f(u)-f(s))(u-t) \le (f(u)-f(t))(u-s)                           \\
  \implies{} & \frac{f(u)-f(s)}{u-s} \le \frac{f(u)-f(t)}{u-t}\quad\square
\end{align*}

\pagebreak

\section{Problem 3}

\subsection{True for a Certain Subset of \texorpdfstring{$[0, 1]$}{[0, 1]}}

I'll first show that $f(\lambda x + (1-\lambda)y) \le \lambda f(x) + (1-\lambda) f(y)$
for all $\lambda$ that can be expressed in the form of $\frac{m}{2^n}$, where $m, n \in \N$ and $0 \le m \le 2^n$.
Basically, any number that has a finite binary decimal expansion.

The cases where $m=0$ or $2^n$ are trivial.

For all other $m$ we proceed by induction on $n$.
The base case where $n=\frac{1}{2}$ and $m=1$ is already given by the inequality.

Before we do the inductive step, I'll just let
$g(\lambda)=f(\lambda x + (1-\lambda)y)$ for convenience.

For $n+1$ there's two cases:
\begin{enumerate}
  \item If $m$ is even, then the fraction simplifies to $\frac{m/2}{2^n}$ and that's covered by the hypothesis.
  \item If not, then we can take $a=\frac{m-1}{2^n}$ and $b=\frac{m+1}{2^n}$.
        The numerators in both are even, so they're also covered by the hypothesis.
        Now by the midpoint inequaltiy we have
        \begin{align*}
          g\left(\frac{m}{2^n}\right)
           & \le \frac{g\left(\frac{m-1}{2^n}\right)+g\left(\frac{m+1}{2^n}\right)}{2} \\
           & \le \frac{\frac{m-1}{2^n}f(x)+\left(1-\frac{m-1}{2^n}\right)f(y)
          +\frac{m+1}{2^n}f(x)+\left(1-\frac{m+1}{2^n}\right)f(y)}{2}                  \\
           & \le \frac{\frac{2m}{2^n}f(x)+2\left(1-\frac{m}{2^n}\right)f(y)}{2}        \\
           & = \frac{m}{2^n}f(x)+\left(1-\frac{m}{2^n}\right)f(y)\quad\square
        \end{align*}
\end{enumerate}

\subsection{True for Everything in There}

Let all the $\lambda \in [0, 1]$ we just proved that work in the inequality be in a set $B$.

Any $\lambda \notin B$ at least has an infinite binary deimcal expansion, so
for any $\lambda$ we can define a sequence $(a_n) \subseteq B$ that converges to it.
\begin{align*}
             & g(a_n) \le a_n \cdot f(x) + (1-a_n)f(y)                                                              \\
  \implies{} & \lim_{n \to \infty} g(a_n) \le a_n \cdot f(x) + (1-a_n)f(y)                                          \\
  \implies{} & g\left(\lim_{n \to \infty} a_n\right) \le a_n \cdot f(x) + (1-a_n)f(y) &  & \text{since $f$'s cont.} \\
  \implies{} & g(\lambda) \le a_n \cdot f(x) + (1-a_n)f(y)                            &  & \square
\end{align*}

\pagebreak

\section{Problem 4}

\subsection{Composition of UC Functions has to be UC}\label{sec:prob4lemma}

Consider two uniformly continuous functions $f: X \to Y$ and $g: Y \to Z$.

Fix an $\epsilon > 0$. We know the following since $f$ and $g$ are both UC:
\begin{itemize}
  \item $\exists \delta_Y:\ \forall a, b \in Y\ d_Y(a, b) < \delta_Y \implies d_Z(g(a), g(b)) < \epsilon$
  \item $\forall \epsilon_Y > 0\ \exists \delta_X: \forall a, b \in X\ d_X(a, b) < \delta_X \implies d_Y(f(a), f(b)) < \epsilon_Y$
\end{itemize}
With this, take $\delta_Y$ and its corresponding $\delta_X$ if we set $\epsilon_Y=\delta_Y$.

Now, $d_X(a, b) < \delta_X \implies d_Y(f(a), f(b)) < \delta_Y \implies d_Z(g(f(a)), g(f(b))) < \epsilon$.

\subsection{\texorpdfstring{$h$}{h} UC Implies \texorpdfstring{$f$}{f} UC}

In accordance with the hint, notice that $g^{-1}$ has a compact domain $g(Y)$
since continuous functions map compact sets to compact sets.
Also, since $g$ is one-to-one, $f=g^{-1} \circ h$.

Notice that $g^{-1}$ is continuous.
This is because its inverse, $g$, has a compact domain $Y$.
Any closed set in $Y$ must be compact, so $g(Y)$ has to be compact and therefore closed,
implying the continuity of $g^{-1}$.
This also means it's uniformly continuous, since its domain is compact.

But with this, we see that since $f=g^{-1} \circ h$, by \ref{sec:prob4lemma}
it must be uniformly continuous. $\square$

\subsection{\texorpdfstring{$h$}{h} Continuous Implies \texorpdfstring{$f$}{f} Continuous}

The composition of two continuous functions is continuous.

$g^{-1}$ and $h$ are only continuous here, but that's all we need to say that $f$ is too. $\square$

\subsection{\texorpdfstring{$Y$}{Y} Has to be Compact}

Let $X$ and $Z$ be the set of all points on the unit circle, and $Y$ be the interval $[0, 2\pi)$.

As for the functions, let $g(t)=(\sin t, \cos t)$ and $f(t)=g^{-1}(t)$.

As established in the example given in Rudin, $g$ is both one-to-one and onto.
Though $h=g \circ f$ is the identity function and continuous, $f$ isn't,
as demonstrated in the book.

\pagebreak

\section{Problem 5}

\subsection{Bounded to Bounded in \texorpdfstring{$\R^k$}{R\^k}}

BWOC let $f(E)$ be unbounded.

This means we can construct $(e_n) \subseteq E$ s.t. $d_Y(f(e_n), f(e_m)) > 1\ \forall n \ne m$.

Since $E \subseteq \R^k$, there must exist some Cauchy subsequence $\left(e_{k_n}\right)$.

$f$ is also uniformly continuous, so
\[\exists \delta > 0: d(e_{k_n}, e_{k_m}) < \delta \implies d_Y(f(e_{k_n}), f(e_{k_m})) < 1\]

Since $\left(e_{k_n}\right)$ is Cauchy, we can also find an
$N \in \N: d(d_{k_n}, e_{k_m}) < \delta\ \forall n, m \ge N$.

This should imply $d_Y(f(e_{k_n}), f(e_{k_m})) < 1$, but
by our construction of $(e_n)$ the distance between any two elements has
to be at least $1$, which is a contradiction. $\square$

\subsection{In General}

If we replace Euclidean distance with the discrete metric $d_\text{diff}$, the statement doesn't hold.

Consider $f: (\R, d_\text{diff}) \to \R_{|\cdot|}$ defined by $f(x)=x$.
Though the domain is bounded (since the maximum distance between any two points is $1$),
the range certainly isn't.

\pagebreak

\section{Problem 6}

I'll abbreviate "pathwise connected" as PC.

\subsection{Balls are PC}\label{sec:prob6lemma}

I'll first show that any ball in $\R^d$ is PC.

Fix $c \in \R^d$ and $r > 0$, as well as $x, y \in B_r(c)$.
I'll write $x$ as $(x_1, x_2, \cdots, x_d)$ and $y$ similarly.

I propose that $\forall \lambda \in [0, 1]$
\[z=(\lambda x_1 + (1-\lambda) y_1, \lambda x_2 + (1-\lambda) y_2, \cdots, \lambda x_d + (1-\lambda) y_d)\]
is in $B_r(c)$ as well.

This is because
\begin{align*}
  \sum_{i=1}^{d} (z_i-c_i)^2
   & = \sum_{i=1}^{d} (\lambda x_i + (1-\lambda) y_i - c_i)^2                      \\
   & = \sum_{i=1}^{d} (\lambda (x_i - c_i) + (1-\lambda)(y_i-c_i))^2               \\
   & = \lambda \sum_{i=1}^{d} (x_i-c_i)^2 + (1-\lambda) \sum_{i=1}^{d} (y_i-c_i)^2 \\
   & < \lambda r^2 + (1-\lambda) r^2                                               \\
   & = r^2
\end{align*}
and taking the square root of both sides gives us
\[\sqrt{\sum_{i=1}^{d} (z_i-c_i)^2} < r \implies z \in B_r(c)\]

Thus, we can use
\[f(t)=\text{the $z$ associated with $\lambda=t$}\]
as a continuous path from $x$ to $y$.

\pagebreak

\subsection{Actual Proof}

Fix a point $x_0$.
It STP any point can be connected to $x_0$ with a continuous path.

Let $S$ be the set of all points in $E$ that satisfy the above condition.
$S \ne \varnothing$, as $x_0 \in S$.

Consider any $s \in S$ and its associated $B_r(s) \subseteq E$.
$x_0$ and $s$ are PC, and by \ref{sec:prob6lemma}
$s$ and any $x \in B_r(s)$ are connected too,
which means we can pathwise connect $x_0$ and $x$.
Thus, $B_r(s) \subseteq S$ and $S$ is open in $E$.

Now we shift our attention to $S' = E \setminus S$, which is
all the points that \textit{can't} be connected to $x_0$ with a continuous path.
For any $s \in S' \subseteq E$, there's still an $r > 0: B_r(s) \subseteq E$.

Notice that $B_r(s) \subseteq S'$ as well, which implies $S'$ is open.
To prove this, suppose FSOC $\exists e \in B_r(s): e \notin S' \implies e \in S$.
$e$ and $x_0$ are PC, and by \ref{sec:prob6lemma}
$e$ and $s$ are PC as well.
This is a contradiction, since it means $x_0$ and $s$ are PC.

If $S'$ is open in $E$, then $S=E \setminus S'$ must be closed in $E$ as well.

Since $S \subseteq E$ is nonempty, open, and closed, and $E$ is connected, $S=E$. $\square$

\pagebreak

\section{Problem 7}

\subsection{A Quick Tangent}

I'll first prove that if $X$ is a compact set, $f: X \to \R$
must attain a global minimum at some point.
We're in $\R$, so we just have to show $\inf f(X) \in f(X)$.

BWOC say $\inf f(X) \notin X$.

We then construct $(a_n)$ where $a_n$ is any $y \in f(X): y < \inf f(X) + \frac{1}{n}$.
Such a $y$ is guaranteed to exist by the definition of the supremum.

Though this sequence converges to $\inf f(X)$, the limit isn't actually in $f(X)$.
This contradicts the fact that $f(X)$ is compact, so
\[\inf f(X) \in f(X) \implies \exists x \in X: f(x) = \inf f(X)\]

\subsection{The Actual Proof}

By our assumptions, $\exists r_1, r_2$ s.t.
$f(x_1)=\max \{f(x) \mid x \in B_{r_1}(x_1)\}$ and similarly for $f(x_2)$.
Make sure to choose them so that $x_1+r_1 < x_2-r_2$,
which is possible since if a certain $r$ works any smaller $r$ is guaranteed to work as well.

For $x_1$, there's two cases.
\begin{enumerate}
  \item Everything in $(x_1, x_1+r_1)$ is equal to $f(x_1)$, in which case
        we can just pick anything in the interval as our local minimum since the function is constant.
  \item The function is nonconstant, in which case we can pick $y_1 \in (x_1, x_1+r_1]$
        s.t. $f(y_1)=\inf \{f(x) \mid x_1 < x \le x_1+r_1\}$.
        Such a $y_1$ is guaranteed to exist because $[x_1, x_1+r_1]$ is compact
        and the only way the global minimum is $x_1$ is if the function is constant.
\end{enumerate}

We can pick a similar $y_2 \in [x_2-r_2, x_2)$.
It suffices to find a local minimum in $[y_1, y_2] \subset (x_1, x_2)$.

Let $Y=[y_1, y_2]$ and $y \in Y$ be the location of the global minimum.

Here, again we have two cases:
\begin{enumerate}
  \item $y \in (y_1, y_2)$, which means it's a local minimum in $[a, b]$ as well.
  \item $y=y_1$ or $y_2$, which is still fine as we defined them to be
        lesser than everything on the left/right side of the interval as well.
\end{enumerate}
Either way, we have a local minimum. $\square$

\end{document}
