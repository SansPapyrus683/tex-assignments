\documentclass[12pt]{article}

% a template that a friend gave, it's worked well enough for me
% i have added some packages and stuff that have proved useful

\usepackage{fancyhdr}
\usepackage{tipa}
\usepackage{fontspec}
\usepackage{amsfonts}
\usepackage{enumitem}
\usepackage[margin=1in]{geometry}
\usepackage{graphicx}
\usepackage{float}
\usepackage{amsmath}
\usepackage{braket}
\usepackage{amssymb}
\usepackage{booktabs}
\usepackage{hyperref}
\usepackage{mathtools}
\usepackage{xcolor}
\usepackage{float}
\usepackage{algpseudocodex}
\usepackage{titlesec}
\usepackage{bbm}

\pagestyle{fancy}
\fancyhf{} % sets both header and footer to nothing
\lhead{Kevin Sheng}
\setmainfont{Comic Neue}
\renewcommand{\headrulewidth}{1pt}
\setlength{\headheight}{0.75in}
\setlength{\oddsidemargin}{0in}
\setlength{\evensidemargin}{0in}
\setlength{\voffset}{-.5in}
\setlength{\headsep}{10pt}
\setlength{\textwidth}{6.5in}
\setlength{\headwidth}{6.5in}
\setlength{\textheight}{8in}
\renewcommand{\headrulewidth}{0.5pt}
\renewcommand{\footrulewidth}{0.3pt}
\setlength{\textwidth}{6.5in}
\usepackage{setspace}
\usepackage{multicol}
\usepackage{float}
\setlength{\columnsep}{1cm}
\setlength\parindent{24pt}
\usepackage [english]{babel}
\usepackage [autostyle, english = american]{csquotes}
\MakeOuterQuote{"}

\setlength{\parskip}{6pt}
\setlength{\parindent}{0pt}

\titlespacing\section{0pt}{12pt plus 4pt minus 2pt}{0pt plus 2pt minus 2pt}
\titlespacing\subsection{0pt}{12pt plus 4pt minus 2pt}{0pt plus 2pt minus 2pt}
\titlespacing\subsubsection{0pt}{12pt plus 4pt minus 2pt}{0pt plus 2pt minus 2pt}

\hypersetup{colorlinks=true, urlcolor=blue}

\newcommand{\correction}[1]{\textcolor{red}{#1}}


\rhead{Math 131BH}

\makeatletter
\def\@seccntformat#1{%
  \expandafter\ifx\csname c@#1\endcsname\c@section\else
  \csname the#1\endcsname\quad
  \fi}
\makeatother

\DeclareMathOperator{\Fr}{Fr}
\newcommand{\lra}{\xLeftrightarrow}
\newcommand{\ra}{\xRightarrow}
\newcommand{\N}{\mathbb{N}}
\newcommand{\R}{\mathbb{R}}
\newcommand{\Z}{\mathbb{Z}}
\newcommand{\Q}{\mathbb{Q}}
\newcommand{\norm}[1]{\left\lVert#1\right\rVert}
\newcommand*{\ditto}{---''---}

\begin{document}

\section{Problem 1}

I can represent $f$ as $\braket{f_1(x), f_2(x), f_3(x)}$.

With this, we have
\begin{align*}
             & \norm{f(x)}=1                                    \\
  \implies{} & \sqrt{f_1(x)^2+f_2(x)^2+f_3(x)^2}=1              \\
  \implies{} & f_1(x)^2+f_2(x)^2+f_3(x)^2=1                     \\
  \implies{} & \frac{d}{dx} \text{\ditto} = \frac{d}{dx} 1      \\
  \implies{} & 2f_1(x)f_1'(x)+2f_2(x)f_2'(x)+2f_3(x)f_3'(x) = 0 \\
  \implies{} & f_1(x)f_1'(x)+f_2(x)f_2'(x)+f_3(x)f_3'(x) = 0    \\
  \implies{} & f(x) \cdot \nabla f(x) = 0\quad\square
\end{align*}
Geometrically this means that if a function is constant magnitude then
the function is orthogonal to its gradient.

\pagebreak

\section{Problem 2}

I'll just prove that $f$ isn't 1-1 in any neighborhood of $0$.

If it \textit{were} injective, then there'd be a neighborhood around $0$ where $f$ is strictly monotonic.
This then implies that $f'$ is always nonnegative or always nonpositive in that neighborhood as well.

However, the derivative of $f$ for nonzero values is
\[f'(x)=1+4x\sin\left(\frac{1}{x}\right)-2\cos\left(\frac{1}{t}\right)\]
and for all $n \in \N$, setting $x=\frac{1}{2\pi n}$ gives
\begin{align*}
  f'(x)
   & =1+4x\sin(2\pi n)-2\cos(2\pi n) \\
   & = 1+0-2                         \\
   & = -1
\end{align*}
and setting $x=\frac{1}{\pi(2n-1)}$ gives
\begin{align*}
  f'(x)
   & =1+4x\sin(\pi(2n-1))-2\cos(\pi(2n-1)) \\
   & = 1+0+2                               \\
   & = 3
\end{align*}
so no matter how small our neighborhood is, there will always be two derivatives of different signs.
This means $f$ can't be monotonic, so by extension it can't be 1-1. $\square$

\pagebreak

\section{Problem 3}

\subsection{Critical Points}

The gradient is
\[\begin{bmatrix}
    6x^2-6x & 6y^2+6y
  \end{bmatrix}\]
and setting both elements equal to $0$ gives the following solutions:
\begin{align*}
  (0, 0) &  & (1, 0) &  & (0, -1) &  & (1, -1)
\end{align*}

I'll just go through each of the points one by one and see if they're an extreme point.
\begin{enumerate}
  \item $(0, 0)$ is neither.
        While $f(0, y) > f(0, 0) = 0\ \forall y > 0$,
        $f(0, 0) \le f(x, 0)\ \forall x < \frac{3}{2}$ since
        \begin{align*}
          f(x, 0)
           & = 2x^3-3x^2 \\
           & = x^2(2x-3) \\
           & \le 0
        \end{align*}
  \item $(1, 0)$ is a local minimum.
        This is because $f(x, y)=h(x)+g(y)$,
        where $1$ is a local minimum of $h(x)=2x^3-3x^2$
        and $0$ is a local minimum of $g(y)=2y^3+3y^2$.
        Both of these can be verified by the second derivative test.
  \item $(0, -1)$ is a local maximum by similar reasoning as for $(1, 0)$.
  \item $(1, -1)$ is neither.
        For all $x > 1$, we have
        \begin{align*}
          f(x, -1)
           & = 2x^3-3x^2+1 \\
           & = x^2(2x-3)+1 \\
           & > 0
        \end{align*}
        while for all $y < -1$ we have
        \begin{align*}
          f(1, y)
           & = -1+2y^3+3y^2 \\
           & = y^2(3+2y)-1  \\
           & < 0
        \end{align*}
\end{enumerate}

\pagebreak

\subsection{No Good Neighborhoods}

To be able to solve for $f(x, y)=0$ in a neighborhood of some point $(x, y)$,
we can use the Implict Function Theorem which requires at least one of these
terms to be nonzero:
\[f'(x, y)=\begin{bmatrix}
    6x^2-6x & 6y^2+6y
  \end{bmatrix}\]
The only points that don't satisfy the conditions are $(0, 0)$, $(1, -1)$, $(0, -1)$, and $(1, 0)$.

Notice that
\[2x^3-3x^2+2y^3+3y^2=(x+y)\left(2x^2-2xy-3x+2y^2+3y\right)\]
So for $f(x, y)=0$ either $y=-x$ or they satisfy that weird quadratic thing.

$f(0, -1)$ and $f(1, 0)$ are both nonzero, so the only sus points are $(0, 0)$ and $(1, -1)$.

To show that they're in $S$, we need to show that for any $x$ in a neighborhood
around them there's more than one $y$ s.t. $f(x, y)=0$.
We're guaranteed at least one since $y=-x$ is always a solution.

To see that the quadratic also gives a solution notice that by the quadratic formula
\[y=\frac{2x-3 \pm \sqrt{(2x-3)^2-8\left(2x^2-3x\right)}}{4}\]

For $(0, 0)$, if we take the positive solution and take $x$ to $0$,
since the expression is continuous on $\left(-\frac{1}{2}, \frac{3}{2}\right)$
$y$ tends to $0$ as well.
We know this $y$ isn't equal to $-x$ since the only points
where the two curves intersect are $(0, 0)$ and $(1, -1)$ anyways.

What this means is that for any neighborhood $B_r((0, 0))$,
we can find an $x \ne 0$ that has two candidates for $y$:
\begin{itemize}[nolistsep]
  \item $y=-x$
  \item $y=$ that quadratic formula above using $+$
\end{itemize}
All this winds up implying that $(0, 0) \in S$.

For $(1, -1)$ we can do something similar to see that it has no good neighborhoods either.

Thus, $S=\{(0, 0), (1, -1)\}$.

\pagebreak

\section{Problem 4}

$f$ is differentiable by inspection; all the partials (below) are clearly continuous:
\[f'(x)=\begin{bmatrix}
    2xy_1+e^x & x^2 & 1
  \end{bmatrix}\]
$f(0, 1, -1)=0$ and $f'(0, 1, -1)_1=e^0\ne 0$, we can use the Implicit Function Theorem.

There's a $C^1$ mapping from a neighborhood of $(1, -1)$ to $\R$ (call it $g$) s.t.
\[f(x, y_1, y_2)=0\text{ in the neighborhood} \iff x=g(y_1, y_2)\]
Since $f(0, 1, -1)=0$, $g(1, -1)=0$.
Let $G=g(y_1, y_2)$ for convenience.

$f(G, y_1, y_2)=0$, so its partials w.r.t. $y_1$ is $0$ and
\begin{align*}
  \frac{\partial}{\partial y_1} f(G, y_1, y_2)
   & = \frac{\partial f}{\partial y_1}(G, y_1, y_2) + \frac{\partial f}{\partial x}(G, y_1, y_2)\frac{\partial g}{\partial y_1}(y_1, y_2) \\
   & = 0
\end{align*}
We can then isolate $\frac{\partial g}{\partial y_1}$ to evaluate it at $(1, -1)$:
\begin{align*}
  \frac{\partial g}{\partial y_1}(1, -1)
   & = -\frac{\frac{\partial f}{\partial y_1}(G, 1, -1)}{\frac{\partial f}{\partial x}(G, 1, -1)} \\
   & = -\frac{\frac{\partial f}{\partial y_1}(0, 1, -1)}{\frac{\partial f}{\partial x}(0, 1, -1)} \\
   & = \boxed{0}
\end{align*}

We can do something similar to see that
\begin{align*}
  \frac{\partial g}{\partial y_2}(1, -1)
   & = -\frac{\frac{\partial f}{\partial y_2}(0, 1, -1)}{\frac{\partial f}{\partial x}(0, 1, -1)} \\
   & = \boxed{-1}
\end{align*}

\pagebreak

\section{Problem 5}

\subsection{Contraction}

Lemme first do some stuff with the elementwise difference of $\phi(x)$ and $\phi(y)$:
\begin{align*}
  |\phi(x)_i-\phi(y)_i|
   & = \left|\int_{0}^{1} \frac{d}{dt} \phi(x+t(y-x))_i\,dt\right| \\
   & = \left|\int_{0}^{1} (\phi'(x+t(y-x))(y-x))_i\,dt\right|
\end{align*}
$\phi'(x)=f'(x)-I$, so it should be small as we get really close to $x_0$.

$f$ is $C^1$, so every element of the matrix $f'(x)$ is a continuous function;
since $f'(x_0)=I$, we can choose $\delta$ s.t.
\[\norm{x-x_0} < \delta \implies |f'(x)_{ij}-I_{ij}| < \frac{1}{2n^2}\]

If $x$ and $y$ are both in $B_\delta(x_0)$, then $x+t(y-x)$ should be in there too
for any $0 \le t \le 1$ since balls are convex.
Thus, $\norm{(x+t(y-x))-x_0} < \delta$ and $|f'(x+t(y-x))_{ij}-I_{ij}| < \frac{1}{2n^2}$.

Call that resulting matrix $A$.
For any vector $v$ and index $i$,
\begin{align*}
  |(Av)_i|
   & = \left|\sum_{j=1}^{n} A_{ij}v_j\right| \\
   & \le \sum_{j=1}^{n} |A_{ij}||v_j|        \\
   & \le \sum_{j=1}^{n} \frac{|v_j|}{2n^2}
\end{align*}

So now we can continue bounding like so:
\begin{align*}
  \left|\int_{0}^{1} (\phi'(x+t(y-x))(y-x))_i\,dt\right|
   & \le \int_{0}^{1} |((f'(x+t(y-x))-I)(y-x))_i|\,dt           \\
   & \le \int_{0}^{1} \sum_{i=1}^{n} \frac{|(y-x)_i|}{2n^2}\,dt \\
   & = \sum_{i=1}^{n} \frac{|(y-x)_i|}{2n^2}                    \\
   & = \frac{1}{2n^2} \sum_{i=1}^{n} |(y-x)_i|
\end{align*}

That inequality's only for one component; now we actually bound the norm:
\begin{align*}
  \norm{\phi(x)-\phi(y)}
   & = \sqrt{\sum_{i=1}^{n} (\phi(x)_i-\phi(y)_i)^2}                                  \\
   & \le \sqrt{\sum_{i=1}^{n} \left(\frac{1}{2n^2} \sum_{j=1}^{n} |(y-x)_j|\right)^2} \\
   & = \frac{1}{2n^2}\sqrt{\sum_{i=1}^{n} \left(\sum_{j=1}^{n} |(y-x)_j|\right)^2}    \\
   & = \frac{1}{2n^2} \sqrt{n\left(\sum_{j=1}^{n} |(y-x)_j|\right)^2}                 \\
   & = \frac{\sqrt{n}}{2n^2} \norm{x-y}                                               \\
   & \le \frac{1}{2} \norm{x-y}\quad\square
\end{align*}

\pagebreak

\subsection{And Invertible}

\subsubsection{Dets are Continuous}

If we let the distance $d$ between two matrices be the squre root of the sum
of squared differences between all elements, then the determinant is actually continuous.

The proof is by induction on $n$ where $n$ is the size of the matrix.
The base case where $n=1$ is trivial since in that case $\det A=A_{11}$.

For the inductive step, assuming the determinant of any $n \times n$ matrix is continuous,
we use the cursed alternating sums approach on an $(n+1) \times (n+1)$ matrix $A$:
\[\det A = \sum_{i=1}^{n} (-1)^{i+1} A_{1i} \det \overline{A_{1i}}\]
where $\overline{A_{1i}}$ indicates $A$ with the first row \& $i$-th column removed.

Fix an $\epsilon$.

By our inductive hypothesis, $\exists \delta_i$ for each $i$ s.t.
$d(A', \overline{A_{1i}}) < \delta_i \implies |\det A' - \det \overline{A_{1i}}| < \frac{\epsilon}{n}$.

Let $\delta=\min\left(\min_{i=1, \cdots, n} \delta_i, \frac{\epsilon}{n}\right)$;
notice that $d(A, A') < \delta \implies d(\overline{A_{1i}}, \overline{A'_{1i}}) < \delta$ for each $i$.
Also, the individual elements of $A$ and $A'$ can't differ by more than $\frac{\epsilon}{n}$.

Now we can bound the difference like so:
\begin{align*}
  |\det A - \det A'|
  &= \left|\sum_{i=1}^{n} (-1)^{i+1} A_{1i} \det \overline{A_{1i}} - \sum_{i=1}^{n} (-1)^{i+1} A'_{1i} \det \overline{A'_{1i}}\right| \\
  &\le \sum_{i=1}^{n} \left|A_{1i} \det \overline{A_{1i}} - A'_{1i} \det \overline{A'_{1i}}\right| \\
  &\le n \cdot \text{some small term surely}
\end{align*}
which should conclude the proof :clueless:.

\subsubsection{Actual Proof}

With that, we know that $\exists \delta: d(A, I) < \delta \implies |\det A - \det I| < \frac{1}{2} \implies A \ne 0$.

$f'(x_0)=I$, so this $\delta$ should work for our purposes.
By the $C^1$-ness of $f$ pick $r$ s.t. $\norm{x-x_0} < r \implies d(f'(x), f'(x_0)) < \delta$. 
This makes it so the determinant of any $f'(x)$ within $B_r(x_0)$ can't be nonzero,
which means they're invertible. $\square$

\pagebreak

\section{Problem 6}

\subsection{Dot Product Equality}

Consider the single-variable function
\[g(t)=\vec{v} \cdot (f(t(y-x)+x)-f(x))\]
so $g(0)=\vec{v} \cdot f(x)$ and $g(1)=\vec{v} \cdot f(y)$.

By the normal MVT, $\exists 0 < t < 1: g'(t)=g(1) - g(0)$.

The RHS of this expression is the LHS of the equality we need to prove,
so now with a little bit of derivative wrangling we have that
\begin{align*}
  g'(t)
   & = \frac{d}{dt} \vec{v} \cdot f(t(y-x)+x) - \frac{d}{dt} \vec{v} \cdot f(x) \\
   & = \vec{v} \cdot (f'(t(y-x)+x)(y-x))                                        \\
   & = \vec{v} \cdot (f'(z)(y-x))
\end{align*}
where $z$ is some point on the segment between $x$ and $y$. $\square$

\subsection{True Equality?}

Unfortunately, this doesn't mean there exists a $z$ s.t. $f(y)-f(x)=f'(z)(y-x)$.

Consider $f: \R \to \R^2$ defined by
\[f(x)=\Braket{-x(x-2), x\sin\left(\frac{\pi}{2}x\right)}\]
with $x=0$ and $y=2$.
$f(x)=f(y)=\vec{0}$, so we'd want a $z$ s.t. $2f'(z)=\vec{0}$ as well.

However, $\frac{d}{dx} f(x)_1=-2x+2$ is only $0$ at $x=1$.

When we plug that into
$\frac{d}{dx} f(x)_2=\frac{\pi x}{2}\cos\left(\frac{\pi x}{2}\right)+\sin\left(\frac{\pi x}{2}\right)$,
the number isn't $0$.

Thus, there's no point at which $2f'(z)=0$.

\pagebreak

\section{Problem 7}

\subsection{Main Body}

WLOG assume $x_0=\vec{0}$ and $f: \R^n \to \R$.
If the output is multidimensional we can rerun this argument
for each output component and it would be the same thing.

It STP that both of the partials are equivalent to the following:
\[\lim_{h \to 0} \frac{f(he_1+he_2)-f(he_2)-f(he_1)+f(0)}{h^2}\]

Regrouping the top gives $(f(he_1+he_2)-f(he_2))-(f(he_1)-f(\vec{0}))$,
which can then be written as $g(h)-g(0)$ where $g(x)=f(he_1+xe_2)-f(xe_2)$.

With this, by the MVT $\exists 0 < a < h$ s.t. $g(h)-g(0)=g'(a)h$ and
\begin{align*}
  & \lim_{h \to 0} \frac{(f(he_1+he_2)-f(he_2))-(f(he_1)-f(\vec{0}))}{h^2} \\
  ={} & \lim_{h \to 0} \frac{g'(a)}{h} \\
  ={} & \lim_{h \to 0} \frac{\partial x_2 f(he_1+ae_2)-\partial x_2 f(ae_2)}{h} \\
  ={} & \lim_{h \to 0} \frac{D(\partial x_2 f)(\vec{0})(he_1)+o(h)}{h} \\
  ={} & \lim_{h \to 0} \frac{D(\partial x_2 f)(\vec{0})(he_1)}{h} \\
  ={} & D(\partial x_2 f)(\vec{0})(e_1) \\
  ={} & \partial x_1 \partial x_2 f(\vec{0})
\end{align*}

Notice that if we grouped the initial term into $(f(he_1+he_2)-f(he_1))-(f(he_2)-f(\vec{0}))$
then we could run basically the same steps to get that it was equal to $\partial x_2 \partial x_2 f(\vec{0})$,
so the two second derivatives are equal. $\square$

Some of the steps here likely need further elaboration though.

\subsection{Some Other Stuff I Probably Should Justify}

\subsubsection{The Lil O}

I first need to convince myself that
\[\partial x_2 f(he_1+ae_2)-\partial x_2 f(ae_2)
=D(\partial x_2 f)(\vec{0})(he_1)+o(h)\]
or in other words
\[\lim_{h \to 0} \frac{\partial x_2 f(he_1+ae_2)-\partial x_2 f(ae_2)-D(\partial x_2 f)(\vec{0})(he_1)}{h}=0\]

Abbreviating $\delta x_2 f$ as $F$ and $D(\partial x_2 f)(\vec{0})$ as $A$, we have
\begin{align*}
  \text{above limit}
  &= \lim_{h \to 0} \frac{F(he_1+ae_2)-F(0)-A(he_1+ae_2)}{h} - \frac{F(0)-f(ae_2)+A(ae_2)}{h} \\
  &= \lim_{h \to 0} \frac{F(he_1+ae_2)-F(0)-A(he_1+ae_2)}{h} - \lim_{h \to 0} \frac{F(ae_2)-F(0)-A(ae_2)}{h}
\end{align*}
As long as both of those go to $0$ we're chilling.

The first one goes to $0$ since by definition
\[\lim_{h \to 0} \frac{|F(he_1+ae_2)-F(0)-A(he_1+ae_2)|}{\left|\sqrt{a^2+h^2}\right|}=0\]
and for small enough $h$
\[\frac{|F(he_1+ae_2)-F(0)-A(he_1+ae_2)|}{\left|h\right|}
< \frac{|F(he_1+ae_2)-F(0)-A(he_1+ae_2)|}{\left|\sqrt{a^2+h^2}\right|}\]

The second one also goes to $0$ since
$\left|\frac{F(ae_2)-F(0)-A(ae_2)}{h}\right|
< \left|\frac{F(ae_2)-F(0)-A(ae_2)}{a}\right|$.

In both these cases, the upper bound goes to $0$, which forces the
lower term to be $0$ too as everything is wrapped in absolute values.

\subsubsection{And the Final Limit}

Lemme also show
\[\lim_{h \to 0} \frac{D(\partial x_2 f)(\vec{0})(he_1)}{h}
= D(\partial x_2 f)(\vec{0})(e_1)\]
This is true since $D(\partial x_2 f)(\vec{0})$ is a linear transformation, so
\begin{align*}
  \lim_{h \to 0} \frac{D(\partial x_2 f)(\vec{0})(he_1)}{h}
  &= \lim_{h \to 0} \frac{D(\partial x_2 f)(\vec{0})(e_1) \cdot h}{h} \\
  &= D(\partial x_2 f)(\vec{0})(e_1)
\end{align*}

\end{document}
