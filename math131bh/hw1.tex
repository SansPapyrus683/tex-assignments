\documentclass[12pt]{article}

% a template that a friend gave, it's worked well enough for me
% i have added some packages and stuff that have proved useful

\usepackage{fancyhdr}
\usepackage{tipa}
\usepackage{fontspec}
\usepackage{amsfonts}
\usepackage{enumitem}
\usepackage[margin=1in]{geometry}
\usepackage{graphicx}
\usepackage{float}
\usepackage{amsmath}
\usepackage{braket}
\usepackage{amssymb}
\usepackage{booktabs}
\usepackage{hyperref}
\usepackage{mathtools}
\usepackage{xcolor}
\usepackage{float}
\usepackage{algpseudocodex}
\usepackage{titlesec}
\usepackage{bbm}
\usepackage{pythonhighlight}

\pagestyle{fancy}
\fancyhf{} % sets both header and footer to nothing
\lhead{Kevin Sheng}
\setmainfont{Comic Neue}
\renewcommand{\headrulewidth}{1pt}
\setlength{\headheight}{0.75in}
\setlength{\oddsidemargin}{0in}
\setlength{\evensidemargin}{0in}
\setlength{\voffset}{-.5in}
\setlength{\headsep}{10pt}
\setlength{\textwidth}{6.5in}
\setlength{\headwidth}{6.5in}
\setlength{\textheight}{8in}
\renewcommand{\headrulewidth}{0.5pt}
\renewcommand{\footrulewidth}{0.3pt}
\setlength{\textwidth}{6.5in}
\usepackage{setspace}
\usepackage{multicol}
\usepackage{float}
\setlength{\columnsep}{1cm}
\setlength\parindent{24pt}
\usepackage [english]{babel}
\usepackage [autostyle, english = american]{csquotes}
\MakeOuterQuote{"}

\setlength{\parskip}{6pt}
\setlength{\parindent}{0pt}

\titlespacing\section{0pt}{12pt plus 4pt minus 2pt}{0pt plus 2pt minus 2pt}
\titlespacing\subsection{0pt}{12pt plus 4pt minus 2pt}{0pt plus 2pt minus 2pt}
\titlespacing\subsubsection{0pt}{12pt plus 4pt minus 2pt}{0pt plus 2pt minus 2pt}

\hypersetup{colorlinks=true, urlcolor=blue}

\newcommand{\correction}[1]{\textcolor{red}{#1}}


\rhead{Math 131BH}

\makeatletter
\def\@seccntformat#1{%
  \expandafter\ifx\csname c@#1\endcsname\c@section\else
  \csname the#1\endcsname\quad
  \fi}
\makeatother

\DeclareMathOperator{\Fr}{Fr}
\newcommand{\lra}{\xLeftrightarrow}
\newcommand{\ra}{\xRightarrow}
\newcommand{\N}{\mathbb{N}}
\newcommand{\R}{\mathbb{R}}
\newcommand{\Z}{\mathbb{Z}}
\newcommand{\Q}{\mathbb{Q}}
\newcommand{\norm}[1]{\left\lVert#1\right\rVert}

\begin{document}

\section{Problem 1}

\subsection{Construction}

Consider any sequence $((x_m)_n)$ of sequences in $E$.
Since $\sum_{i=1}^{\infty} n|x_n|^2 \le 1$, $|x_n| \le 1$
and the first terms of all the sequences must be bounded.

Let $k_n$ be the indices of the sequences we choose for our overall convergent subsequence.

We take a subsequence of this initial sequence s.t. their first terms converge.
Let $k_1$ be the first element of this subsequence.

Within this subsequence, there has to be another subsequence whose second terms converge as well.
Let $k_2$ be the second element of this subsequence to make it greater than $k_1$.

We can repeat this ad infinitum to get a value of $k_n$ for every $n$.
Each element of this sequence must converge- let this new sequence
of the limits of each of the invididual elements be $x^{(\infty)}$.

For convenience, I'll call the $n$th sequence in the subsequence $x^{(n)}$.

\subsection{Limit in \texorpdfstring{$E$}{E}}

We need to show that $x^{(\infty)}$ is actually in $E$.
Notice that for all $N, m \in \N$,
\begin{align*}
             & \sum_{n=1}^{N} n\left|x^{(m)}_n\right|^2 \le \sum_{n=1}^{\infty} n\left|x^{(m)}_n\right|^2 \le 1 \\
  \implies{} & \lim_{m \to \infty} \sum_{n=1}^{N} n\left|x^{(m)}_n\right|^2 \le 1                               \\
  \implies{} & \sum_{n=1}^{N} n\left|x^{(\infty)}_n\right|^2 \le 1                                              \\
  \implies{} & \sum_{n=1}^{\infty} n\left|x^{(\infty)}_n\right|^2 \le 1                                         \\
  \implies{} & x^{(\infty)} \in E
\end{align*}

\pagebreak

\subsection{Actual Convergence}

It STP
\[\lim_{n \to \infty} \sum_{i=1}^{\infty} \left(x^{(n)}_i-x^{(\infty)}_i\right)^2 = 0\]
which means for any $\epsilon$ we need to find an $N$ s.t. the distance is less than $\epsilon$ past $0$.

We can break down the summation into a head and tail:
\begin{align*}
  \sum_{i=1}^{\infty} \left(x^{(n)}_i-x^{(\infty)}_i\right)^2
   & = \sum_{i=1}^{M} \left(x^{(n)}_i-x^{(\infty)}_i\right)^2
  + \sum_{i=M+1}^{\infty} \left(x^{(n)}_i-x^{(\infty)}_i\right)^2                          \\
   & \le \sum_{i=1}^{M} \left(x^{(n)}_i-x^{(\infty)}_i\right)^2
  + 2\sum_{i=M+1}^{\infty} \left(x^{(n)}_i\right)^2+\left(x^{(\infty)}_i\right)^2               \\
   & \le \sum_{i=1}^{M} \left(x^{(n)}_i-x^{(\infty)}_i\right)^2
  + \frac{2}{M} \sum_{i=M+1}^{\infty} i\left(x^{(n)}\right)^2+i\left(x^{(\infty)}\right)^2 \\
   & \le \sum_{i=1}^{M} \left(x^{(n)}_i-x^{(\infty)}_i\right)^2 + \frac{4}{M}
\end{align*}

So if we're given an $\epsilon$, we first pick $M \in \N: \frac{4}{M} < \frac{\epsilon}{2}$.

Then, since $\lim_{n \to \infty} x^{(n)}_i = x^{(\infty)}_i$,
we can pick $N$ s.t. $\left(x^{(n)}_i-x^{(\infty)}_i\right)^2 < \frac{\epsilon}{2M}\ \forall n \ge N$.

Then for all $n \ge N$ we have
\begin{align*}
  \sum_{i=1}^{M} \left(x^{(n)}_i-x^{(\infty)}_i\right)^2 + \frac{1}{M}
   & < \sum_{i=1}^{M} \frac{\epsilon}{2M} + \frac{\epsilon}{2} \\
   & < \epsilon\quad\square
\end{align*}

\pagebreak

\section{Problem 2}

\subsection{Countable Base}

Consider any $r > 0$ and the set $C=\{B_r(k) \mid k \in K\}$.

I'll prove that there's a countable subcover in this set.
BWOC suppose there wasn't, so any countable subset of $C$ would be missing some elements.

Consider countable subset $S=\{B_r(x_1), B_r(k_2), \cdots\}$.

Take any point $x_1$ that's not covered by $S$ and add its ball to $S$.
A countable set with one more element is still countable, so by our assumption
there must still be some element $x_2$ that's missing.
Adding the ball of $x_2$, we can get an $x_3$, so on and so forth.

Notice that $\forall n, m \in \N: n < m$ we have $d(x_n, x_m) \ge r$.
When we added $x_m$ to $S$, $B_r(x_n)$ was already in there by construction,
so this means $x_m$ must be at least $r$ away from $x_n$. 

This contradicts sequential compactness, since any convergent subsequence must be Cauchy.
As we've just seen, all elements are at least $r$ away from each other, so
no Cauchy subsequence can exist.

With this, we can construct a countable base.
For all $r \in \Q^+$, add the countable subcover of $\{B_r(k) \mid k \in K\}$.
This works since the countable union of countable sets is still countable.

This set is also a base.
For any $k \in K$ and an open set $O$ s.t. $k \in O$,
we take the $\epsilon: B_\epsilon(k) \subseteq O$.

Taking any $r \in \Q$ s.t. $r < \frac{\epsilon}{2}$,
we know there's some set $B_r(k')$ in our base s.t. $k \in B_r(k')$.
$B_r(k') \subseteq B_\epsilon(k) \subseteq O$, so this set is indeed a base. $\square$

\subsection{Countable Subcover}

Consider any open cover $\{O_\alpha\}$ and countable base $\{B_n\}$.

For any $O_\alpha$ and $k \in K$, $\exists n \in \N: k \in B_n \subseteq O_\alpha$,
so the base itself covers $K$.

With this, we can define a subcover $\{Q_n\}$ like so:
\[Q_n = \text{any } O_\alpha: B_n \subseteq O_\alpha\]
Any $k \in K$ must be in some $B_n$ which is then contained in some $O_n$,
so this is indeed a countable subcover. $\square$

\pagebreak

\section{Problem 3}

\subsection{Seq. Cpt. Implies Limit Points}

Consider any infinite subset of $S \subseteq K$.
This set is nonempty, so we can construct an injection $f: \N \to S$
and define a sequence $x_n = f(n)$.

Since $K$ is sequentially compact, this sequence must have a subsequence
that converges to some $x \in K$.

This then implies
\[\forall \epsilon > 0, N \in \N\ \exists n > N: d(x_{k_n}, x) < \epsilon\]
Since $f$ is injective, $x$ itself occurs at most once in the sequence.

Let $N$ be s.t. $x_{k_N} = x$ if $x$ is in the sequence and $1$ otherwise.
Then, $\forall \epsilon > 0$ we can always find some other $x_{k_n}$
where $d(x_{k_n}, x) < \epsilon$ and $x \ne x_{k_n}$,
which satisfies the conditions for a limit point.

\subsection{Limit Points Imply Seq. Cpt.}

For any sequence $(x_n)$, consider the set $S=\{x_n \mid n \in \N\}$.

If there's only a finite number of distinct elements in the set,
there's definitely a convergent subsequence since some single
element must occur an infinite number of times.

If not, it's infinite and has a limit point $x$ in $K$, which implies
$\exists x \in K: \forall \epsilon > 0\ B_r(x) \cap S \ne \varnothing$.

We can then construct a subsequence $x_{k_n}$ like so:
\begin{itemize}
  \item $k_1$ is chosen s.t. $d(x_{k_1}, x) < 1$.
  \item $k_n$ is chosen s.t. $d(x_{k_n}, x) < \min\left(\min_{i=1, \cdots, k_n - 1} d(x_i, x), \frac{1}{n}\right)$.
\end{itemize}
This sequence pretty clearly converges to $x$
and by construction $k_n > k_{n-1}$ as well. $\square$

\pagebreak

\section{Problem 4}

\subsection{\texorpdfstring{$K$}{K} Compact}

BWOC say $\forall c > 0\ \exists x \in K: d(x, E) < c$.

Construct a sequence $x_n$ where the $n$th element is any $y \in K$ s.t. $d(y, E) < \frac{1}{n}$.

$K$ is compact, so there's some convergent subsequence $x_{k_n}$ that converges to a $k \in K$.

Now we need to prove a lemma which is that $d(k, E)=0$.
BWOC say $d(k, E)=\epsilon > 0$.
By construction, $\exists N \in \N: d(x_{k_n}, E) < \frac{\epsilon}{4}\ \forall n \ge N$.

By how we defined distance, this then means $\forall n \ge N\ \exists e \in E: d(x_{k_n}, e) < \frac{\epsilon}{2}$.
Then by the triangle inequality, across all $n \ge N$ we have
\begin{align*}
             & d(k, x_n) + d(x_n, e) \ge d(k, e)    \\
  \implies{} & d(k, x_n) \ge d(k, e) - d(x_n, e)    \\
  \implies{} & d(k, x_n) \ge \frac{\epsilon}{2} > 0
\end{align*}
which contradicts that $k=\lim_{n \to \infty} x_{k_n}$, so $d(k, E)=0$.

Now with this implication and the definition of the infimum, we have:
\begin{align*}
             & \forall r > 0\ \exists e \in E: d(k, e) < r \\
  \implies{} & B_r(k) \cap E \ne \varnothing               \\
  \implies{} & k \in \overline{E}=E
\end{align*}
which is a contradiction since $K \cap E = \varnothing$. $\square$

\subsection{\texorpdfstring{$K$}{K} Only Closed}

This statement isn't true if $K$ is just closed.

Consider $\R$ with the standard metric as well as the sets $A=\N$ and $B=\left\{n+\frac{1}{2n} \mid n \in \N\right\}$.

Though $A$ and $B$ are both closed and $A \cap B = \varnothing$, the distance
between them is $0$ since $B$ gets arbitrarily close to $A$ as $n \to \infty$.

\pagebreak

\section{Problem 5}

\subsection{A Quick Tangent}

Consider any $\epsilon$ and an associated $A_{\frac{\epsilon}{3}}$, which we'll call $A'$ here.

The set $\{B_{\frac{\epsilon}{3}}(a) \mid a \in A'\}$ is an open cover.
$A'$ is compact, so we can take a finiste set $S \subseteq A'$
s.t. $\forall a \in A'\ \exists s \in S: d(a, s) < \frac{\epsilon}{3}$.

By our assumptions, $\forall a \in A\ \exists a' \in A': d(a, a') < \frac{\epsilon}{3}$.

This combined with the triangle inequality gives us
$\forall a \in A\ \exists s \in S: d(a, s) < \frac{2\epsilon}{3}$.

We can actually extend this to all $a \in \overline{A}$.
If $a \in \overline{A}$, $\exists b \in A: d(a, b) < \frac{\epsilon}{3}$.
The distance between $b$ and its associated $s$ is less than $\frac{2\epsilon}{3}$, so $d(a, s) < \epsilon$.

Thus, for any $\epsilon$ we can find a finite set of $\epsilon$-balls that cover the entirety of $\overline{A}$.

\subsection{Constructing the Subsequence}

Consider any sequence $(x_n) \subseteq \overline{A}$.

Start with $\epsilon=1$.
By the above section, there's a finite set of points in $A$ that cover all of $\overline{A}$,
so some point $y_1$'s ball must contain an infinite number of elements in $(x_n)$.
Call the subsequence these points form $(x_{k^{(1)}_n})$.

$d\left(x_{k^{(1)}_n}, y_1\right) < 1\ \forall n \in \N$, so by the triangle inequality we have
$d\left(x_{k^{(1)}_n}, x_{k^{(1)}_m}\right) < 2\ \forall n, m \in \N$.

Now consider $\epsilon=\frac{1}{2}$ and \textit{its} finite set of points
whose balls cover all of $\overline{A}$.
There must be some point $y_\frac{1}{2}$ within this
s.t. $d\left(x_{k^{(1)}_n}, y_\frac{1}{2}\right) < \frac{1}{2}$ for an infinite number of $n$.
Call the subsequence this infinite set forms $(x_{k^{(2)}_n})$.

Here, by similar logic, we see that
$d\left(x_{k^{(2)}_n}, x_{k^{(2)}_m}\right) < 1\ \forall n, m \in \N$.

We can then repeat this process indefinitely to get $(x_{k^{(3)}_n})$, $(x_{k^{(4)}_n})$, until infinity.

What we have here is a series of nested subsequences in $\overline{A}$
whose diameter keeps on decreasing.
Since $\overline{A}$ is closed in a complete metric space, it's complete,
and by something we proved in AH the infinite intersection of these subsequences
must be have one (1) element which we'll call $x$.

By this process, we see that for any $\epsilon > 0$ we have an
infinite number of elements in $(x_n)$ s.t. $d(x_n, x) < \epsilon$,
making $x$ a limit point of $(x_n)$. $\square$

\pagebreak

\section{Problem 6}

Consider the following set:
\[A=\{0\} \cup
  \bigcup_{n \in \N \setminus \{1\}}
  \left(\left\{\frac{1}{n} + \frac{1}{x}\left(\frac{1}{n-1}-\frac{1}{n}\right)
  \mid x \in \N \setminus \{1\}\right\}
  \cup \left\{\frac{1}{n}\right\}\right)\]

For each $n$, the values in the parenthesis lie in the range $\left[\frac{1}{n}, \frac{1}{2}\left(\frac{1}{n-1}+\frac{1}{n}\right)\right]$.
The ranges go like $\left[\frac{1}{2}, \frac{3}{4}\right], \left[\frac{1}{3}, \frac{5}{12}\right], \cdots$
and by construction they're disjoint.

This set is bounded and closed in $\R$, so it's compact.

$A'=\left\{\frac{1}{n} \mid n \in \N \setminus \{1\}\right\} \cup \{0\}$,
so there's a countably infinite number of limit points as well.

\pagebreak

\section{Problem 7}

\subsection{Bounded}

For any $p_1, p_2 \in E$ we have
\begin{align*}
  2 < p_1^2 < 3 &  & -3 < -p_2^2 < -2
\end{align*}
so $-1 < p_1^2-p_2^2 < 1$ and the distance between any two elements is indeed bounded.

\subsection{Closed and open}

I'll just first only consider the positive portion of $E$.

$E$ is closed.
This is because its negation is the union of all rationals less than $\sqrt{2}$ and all rationals greater than $\sqrt{3}$.
Both these sets are open.
For the first one, if $p < \sqrt{2}$,
by the density of $\Q$ in $\R$ $\exists q: p < q < \sqrt{2}$ so we can take
$r=d(p, q)$ for everything in $B_r(p)$ to still be less than $\sqrt{2}$.

The second one with $\sqrt{3}$ is open for similar reasons.
The union of open sets is still open, so $E^C$ is open and $E$ must be closed.

$E$ is also open.
First assume $p > 0$.
Again by density, we can find:
\begin{itemize}[nolistsep]
  \item A $q_1$ s.t. $p^2 < q_1^2 < 3$
  \item A $q_2$ s.t. $2 < q_2^2 < p^2$
\end{itemize}
Taking $r=\min(d(p, q_1), d(p, q_2))$, we see that $B_r(p) \subseteq E$.

The case for the negatives goes the same as the positive case for both open/closed.
Closedness/openness stays the same under union, so we're fine.

\subsection{Not Compact}

Consider the sequence defined by $a_1=\frac{3}{2}$ and $a_{n+1}=\frac{1}{2}\left(a_n+\frac{2}{a_n}\right)$.

This sequence converges in the reals to $\sqrt{2}$.
Any subsequence must also converge to this same value, which isn't in $\Q$ and by extension $E$.
Thus, $E$ can't be compact.

\end{document}
