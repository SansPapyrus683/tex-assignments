\documentclass[12pt]{article}

% a template that a friend gave, it's worked well enough for me
% i have added some packages and stuff that have proved useful

\usepackage{fancyhdr}
\usepackage{tipa}
\usepackage{fontspec}
\usepackage{amsfonts}
\usepackage{enumitem}
\usepackage[margin=1in]{geometry}
\usepackage{graphicx}
\usepackage{float}
\usepackage{amsmath}
\usepackage{braket}
\usepackage{amssymb}
\usepackage{booktabs}
\usepackage{hyperref}
\usepackage{mathtools}
\usepackage{xcolor}
\usepackage{float}
\usepackage{algpseudocodex}
\usepackage{titlesec}
\usepackage{bbm}

\pagestyle{fancy}
\fancyhf{} % sets both header and footer to nothing
\lhead{Kevin Sheng}
\setmainfont{Comic Neue}
\renewcommand{\headrulewidth}{1pt}
\setlength{\headheight}{0.75in}
\setlength{\oddsidemargin}{0in}
\setlength{\evensidemargin}{0in}
\setlength{\voffset}{-.5in}
\setlength{\headsep}{10pt}
\setlength{\textwidth}{6.5in}
\setlength{\headwidth}{6.5in}
\setlength{\textheight}{8in}
\renewcommand{\headrulewidth}{0.5pt}
\renewcommand{\footrulewidth}{0.3pt}
\setlength{\textwidth}{6.5in}
\usepackage{setspace}
\usepackage{multicol}
\usepackage{float}
\setlength{\columnsep}{1cm}
\setlength\parindent{24pt}
\usepackage [english]{babel}
\usepackage [autostyle, english = american]{csquotes}
\MakeOuterQuote{"}

\setlength{\parskip}{6pt}
\setlength{\parindent}{0pt}

\titlespacing\section{0pt}{12pt plus 4pt minus 2pt}{0pt plus 2pt minus 2pt}
\titlespacing\subsection{0pt}{12pt plus 4pt minus 2pt}{0pt plus 2pt minus 2pt}
\titlespacing\subsubsection{0pt}{12pt plus 4pt minus 2pt}{0pt plus 2pt minus 2pt}

\hypersetup{colorlinks=true, urlcolor=blue}

\newcommand{\correction}[1]{\textcolor{red}{#1}}


\rhead{Math 131BH}

\makeatletter
\def\@seccntformat#1{%
  \expandafter\ifx\csname c@#1\endcsname\c@section\else
  \csname the#1\endcsname\quad
  \fi}
\makeatother

\DeclareMathOperator{\Fr}{Fr}
\newcommand{\lra}{\xLeftrightarrow}
\newcommand{\ra}{\xRightarrow}
\newcommand{\N}{\mathbb{N}}
\newcommand{\R}{\mathbb{R}}
\newcommand{\Z}{\mathbb{Z}}
\newcommand{\Q}{\mathbb{Q}}
\newcommand{\norm}[1]{\left\lVert#1\right\rVert}

\begin{document}

{
  \color{red}
  dropped bh, this probably isn't gonna be finished anytime soon
}

\section{Problem 1}

\section{Problem 2}

\subsection{Countable Base}

First consider the set $\{B_r(x) \mid x \in K\}$.

\subsection{Countable Subcover}

Consider any open cover $\{O_\alpha\}$ and countable base $\{B_n\}$.

For any $O_\alpha$ and $k \in K$, $\exists n \in \N: k \in B_n \subseteq O_\alpha$,
so the base itself covers $K$.

With this, we can define a subcover $\{Q_n\}$ like so:
\[Q_n = \text{any } O_\alpha: B_n \subseteq O_\alpha\]
Any $k \in K$ must be in some $B_n$ which is then contained in some $O_n$,
so this is indeed a countable subcover. $\square$

\section{Problem 3}

\subsection{SC Implies Limit Points}

\subsection{Limit Points Imply SC}

\pagebreak

\section{Problem 4}

\subsection{\texorpdfstring{$K$}{K} Compact}

BWOC say $\forall c > 0\ \exists x \in K: d(x, E) < c$.

Construct a sequence $x_n$ where the $n$th element is any $y \in K$ s.t. $d(y, E) < \frac{1}{n}$.

$K$ is compact, so there's some convergent subsequence $x_{k_n}$ that converges to a $k \in K$.

Now we need to prove a lemma which is that $d(k, E)=0$.
BWOC say $d(k, E)=\epsilon > 0$.
By construction, $\exists N \in \N: d(x_{k_n}, E) < \frac{\epsilon}{4}\ \forall n \ge N$.

By how we defined distance, this then means $\forall n \ge N\ \exists e \in E: d(x_{k_n}, e) < \frac{\epsilon}{2}$.
Then by the triangle inequality, across all $n \ge N$ we have
\begin{align*}
  & d(k, x_n) + d(x_n, e) \ge d(k, e) \\
  \implies{} & d(k, x_n) \ge d(k, e) - d(x_n, e) \\
  \implies{} & d(k, x_n) \ge \frac{\epsilon}{2} > 0
\end{align*}
which contradicts that $k=\lim_{n \to \infty} x_{k_n}$, so $d(k, E)=0$.

Now with this implication and the definition of the infimum, we have:
\begin{align*}
             & \forall r > 0\ \exists e \in E: d(k, e) < r \\
  \implies{} & B_r(k) \cap E \ne \varnothing               \\
  \implies{} & k \in \overline{E}=E
\end{align*}
which is a contradiction since $K \cap E = \varnothing$. $\square$

\subsection{\texorpdfstring{$K$}{K} Only Closed}

This statement isn't true if $K$ is just closed.

Consider $\R$ with the standard metric as well as the sets $A=\N$ and $B=\left\{n+\frac{1}{2n} \mid n \in \N\right\}$.

Though $A$ and $B$ are both closed and $A \cap B = \varnothing$, the distance
between them is $0$ since $B$ gets arbitrarily close to $A$ as $n \to \infty$.

\pagebreak

\section{Problem 5}

\section{Problem 6}

Consider the following set:
\[A=\{0\} \cup
  \bigcup_{n \in \N \setminus \{1\}}
  \left(\left\{\frac{1}{n} + \frac{1}{x}\left(\frac{1}{n-1}-\frac{1}{n}\right)
  \mid x \in \N \setminus \{1\}\right\}
  \cup \left\{\frac{1}{n}\right\}\right)\]

For each $n$, the values in the parenthesis lie in the range $\left[\frac{1}{n}, \frac{1}{2}\left(\frac{1}{n-1}+\frac{1}{n}\right)\right]$.
The ranges go like $\left[\frac{1}{2}, \frac{3}{4}\right], \left[\frac{1}{3}, \frac{5}{12}\right], \cdots$
and by construction they're disjoint.

This set is bounded and closed in $\R$, so it's compact.

$A'=\left\{\frac{1}{n} \mid n \in \N \setminus \{1\}\right\} \cup \{0\}$,
so there's a countably infinite number of limit points as well.

\pagebreak

\section{Problem 7}

\subsection{Bounded}

For any $p_1, p_2 \in E$ we have
\begin{align*}
  2 < p_1^2 < 3 &  & -3 < -p_2^2 < -2
\end{align*}
so $-1 < p_1^2-p_2^2 < 1$ and the distance between any two elements is indeed bounded.

\subsection{Closed and open}

I'll just first only consider the positive portion of $E$.

$E$ is closed.
This is because its negation is the union of all rationals less than $\sqrt{2}$ and all rationals greater than $\sqrt{3}$.
Both these sets are open.
For the first one, if $p < \sqrt{2}$,
by the density of $\Q$ in $\R$ $\exists q: p < q < \sqrt{2}$ so we can take
$r=d(p, q)$ for everything in $B_r(p)$ to still be less than $\sqrt{2}$.

The second one with $\sqrt{3}$ is open for similar reasons.
The union of open sets is still open, so $E^C$ is open and $E$ must be closed.

$E$ is also open.
First assume $p > 0$.
Again by density, we can find:
\begin{itemize}[nolistsep]
  \item A $q_1$ s.t. $p^2 < q_1^2 < 3$
  \item A $q_2$ s.t. $2 < q_2^2 < p^2$
\end{itemize}
Taking $r=\min(d(p, q_1), d(p, q_2))$, we see that $B_r(p) \subseteq E$.

The case for the negatives goes the same as the positive case for both open/closed.
Closedness/openness stays the same under union, so we're fine.

\subsection{Not Compact}

Consider the sequence defined by $a_1=\frac{3}{2}$ and $a_{n+1}=\frac{1}{2}\left(a_n+\frac{2}{a_n}\right)$.

This sequence converges in the reals to $\sqrt{2}$.
Any subsequence must also converge to this same value, which isn't in $\Q$ and by extension $E$.
Thus, $E$ can't be compact.

\end{document}
