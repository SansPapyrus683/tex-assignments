\documentclass[12pt]{article}

% a template that a friend gave, it's worked well enough for me
% i have added some packages and stuff that have proved useful

\usepackage{fancyhdr}
\usepackage{tipa}
\usepackage{fontspec}
\usepackage{amsfonts}
\usepackage{enumitem}
\usepackage[margin=1in]{geometry}
\usepackage{graphicx}
\usepackage{float}
\usepackage{amsmath}
\usepackage{braket}
\usepackage{amssymb}
\usepackage{booktabs}
\usepackage{hyperref}
\usepackage{mathtools}
\usepackage{xcolor}
\usepackage{float}
\usepackage{algpseudocodex}
\usepackage{titlesec}
\usepackage{bbm}

\pagestyle{fancy}
\fancyhf{} % sets both header and footer to nothing
\lhead{Kevin Sheng}
\setmainfont{Comic Neue}
\renewcommand{\headrulewidth}{1pt}
\setlength{\headheight}{0.75in}
\setlength{\oddsidemargin}{0in}
\setlength{\evensidemargin}{0in}
\setlength{\voffset}{-.5in}
\setlength{\headsep}{10pt}
\setlength{\textwidth}{6.5in}
\setlength{\headwidth}{6.5in}
\setlength{\textheight}{8in}
\renewcommand{\headrulewidth}{0.5pt}
\renewcommand{\footrulewidth}{0.3pt}
\setlength{\textwidth}{6.5in}
\usepackage{setspace}
\usepackage{multicol}
\usepackage{float}
\setlength{\columnsep}{1cm}
\setlength\parindent{24pt}
\usepackage [english]{babel}
\usepackage [autostyle, english = american]{csquotes}
\MakeOuterQuote{"}

\setlength{\parskip}{6pt}
\setlength{\parindent}{0pt}

\titlespacing\section{0pt}{12pt plus 4pt minus 2pt}{0pt plus 2pt minus 2pt}
\titlespacing\subsection{0pt}{12pt plus 4pt minus 2pt}{0pt plus 2pt minus 2pt}
\titlespacing\subsubsection{0pt}{12pt plus 4pt minus 2pt}{0pt plus 2pt minus 2pt}

\hypersetup{colorlinks=true, urlcolor=blue}

\newcommand{\correction}[1]{\textcolor{red}{#1}}


\rhead{Math 131BH}

\makeatletter
\def\@seccntformat#1{%
  \expandafter\ifx\csname c@#1\endcsname\c@section\else
  \csname the#1\endcsname\quad
  \fi}
\makeatother

\DeclareMathOperator{\Fr}{Fr}
\newcommand{\lra}{\xLeftrightarrow}
\newcommand{\ra}{\xRightarrow}
\newcommand{\N}{\mathbb{N}}
\newcommand{\R}{\mathbb{R}}
\newcommand{\Z}{\mathbb{Z}}
\newcommand{\Q}{\mathbb{Q}}
\newcommand{\norm}[1]{\left\lVert#1\right\rVert}

\begin{document}

\section{Problem 1}

\subsection{Limits are Continuous}

Consider any sequence $(x_n)$ that tends to $x$ but doesn't contain $x$.
By the definition of the limit, $\lim_{n \to \infty} f(x_n) = L(x)$.
It STP $\lim_{n \to \infty} L(x_n) = L(x)$.

Now let's construct a sequence $y_n$ as follows:
\begin{itemize}[nolistsep]
  \item $d_n$ chosen s.t. $|x-x_n| < d_n \land x \ne x_n \implies |f(x)-L(x_n)| < \frac{1}{n}$
  \item $\delta_n=\min\left(\frac{1}{n}, d_n\right)$
  \item $y_n$ is anything in $(x_n-\delta_n, x_n+\delta_n) \setminus \{x_n, x\}$.
\end{itemize}

Now, for the actual limit we want to show, fix an $\epsilon$.

By definition of the limit, $\exists \delta: |y-x| < \delta \land y \ne x \implies |f(y)-f(x)| < \frac{\epsilon}{2}$.

Choose $N_1 \in \N$ s.t. $|y_n-x| < \delta\ \forall n \ge N_1$.

Also choose $N_2 \in \N$ s.t. $|f(y_n)|-L(x_n)| < \frac{\epsilon}{2}\ \forall n \ge N_2$.

Now, taking $N$ as the larger of the two, we have for all $n \ge N$
\[|L(x_n)-L(x)| \le |L(x_n)-f(y_n)|+|f(y_n)-f(x)| \le \epsilon\quad\square\]

actually there's an edge case for if $x_n$ contains $x$ itself, uh,
in that case $L(x_n)=L(x)$ so like just omit the construction of $y_n$ for those terms ig lol

\subsection{Countable Discontinuities}

The set given in the question is just the set of all \textit{removable} discontinuities, right?

In that case, let $E_n$ be all RDs s.t. $|f(x)-L(x)| > \frac{1}{n}$.
We can get the actual set from taking $\bigcup_{n=1}^\infty E_n$,
so it STP that $E_n$ is countable.

Consider any $x$ where $|f(x)-L(x)| > \frac{1}{n}$.
By definition of the limit, we can take a $B_\delta(x)$
where $|f(y)-L(x)| < \frac{1}{4n}$ for all $y$ in the area that aren't $x$.

This means all the $y$ are at most $\frac{1}{2n}$ from each other.
Notice that none of these $y$ are in $E_n$, since $|L(y)-L(x)| < \frac{1}{4n}$
and $|f(y)-L(y)| < \frac{1}{2n}$, so $|L(y)-f(y)| < \frac{1}{n}$.

Now we can map $E_n$ to a set of disjoint open intervals
and then take a distinct rational from each of them.
Thus, each $E_n$ is at most countable. $\square$

\pagebreak

\section{Problem 2}

BWOC $\exists x: f(x) > 0$.
Take a $\delta$ s.t. $|x-y| < \delta \implies |f(x)-f(y)| < \frac{f(x)}{2} \implies f(y) > \frac{f(x)}{2}$.
With this, we have an interval $[c, d]$ where $f$ is at least $\frac{\epsilon}{2}$ on.

Then, no matter what partition $\{p_0, p_1, \cdots, p_n\}$
we pick, there's always $p_i < p_j$ s.t.:
\begin{itemize}[nolistsep]
  \item $p_i \le c < d \le p_j$
  \item $c < p_{i+1}$ and $p_{j-1} < d$
\end{itemize}
so
\begin{align*}
  \sum_{k=1}^{n} M_k \Delta x_k
   & \ge \sum_{k=i+1}^{j} M_k \Delta x_k                \\
   & \ge \sum_{k=i+1}^{j} \frac{\epsilon}{2} \Delta x_k \\
   & \ge (c-d) \cdot \frac{\epsilon}{2}                 \\
   & > (c-d) \cdot \frac{\epsilon}{4}
\end{align*}
and yeah, the infimum across all upper sums has to be strictly positive,
which contradicts that $\int_{a}^{b} f(x)\,dx=0$. $\square$

\pagebreak

\section{Problem 3}

\subsection{Not True for Squares}

No, $f(x)^2 \in \mathcal{R}$ doesn't imply $f(x) \in \mathcal{R}$.
Consider the following $f: [0, 1] \to \R$:
\[f(x)=\begin{cases}
    1  & x \in \Q    \\
    -1 & x \notin \Q
  \end{cases}\]
Although $f(x)^2=1$ is trivially integrable, $f(x)$ itself is not.

\subsection{But True for Cubes}

However, $f(x)^3 \in \mathcal{R} \implies f(x) \in \mathcal{R}$.

To show this, notice that $f(x)^3$ is bounded, so $\exists M: -M < f(x)^3 < M$.
The cube root function is defined on all of $\R$ and is continuous on the
closed interval $[-M, M]$ so it must also be UC on this interval.

For this problem, I'll take $C_i$/$c_i$ as the range sup/infs of $f(x)^3$,
and $F_i$/$f_i$ as the same thing but with $\sqrt[3]{f(x)^3}=f(x)$.

Fix $\epsilon$ and take $\delta$ s.t. $\delta < \epsilon$
and $|x-y| < \delta \implies \left|\sqrt[3]{x}-\sqrt[3]{y}\right| < \epsilon$.

$f(x)^3$ is integrable, so $\exists P$ s.t.
\[\sum_{i=1}^{n} (C_i-c_i)\Delta x_i < \delta^2\]
where $C_i$ and $c_i$ are the range sup/range infs of $f(x)^3$.

Let $i \in A$ if $C_i-c_i < \delta$ and $i \in B$ otherwise.
If $i \in A$, $F_i-f_i < \epsilon$.
For the bad ones in $B$, notice that
\[\delta\sum_{i \in B} \Delta x_i \le \sum_{i \in B} (C_i-c_i)\Delta x_i \le \delta^2
  \implies \sum_{i \in B} \Delta x_i \le \delta\]

Now, we have
\begin{align*}
  \sum_{i=1}^{n} (F_i-f_i)\Delta x_i
   & = \sum_{i \in A} (F_i-f_i) \Delta x_i + \sum_{i \in B} (F_i-f_i) \Delta x_i \\
   & < \sum_{i \in A} \epsilon \Delta x_i + \delta(\sup f - \inf f)              \\
   & \le \epsilon((b-a) + (\sup f - \inf f))
\end{align*}
so as long as we pick $\epsilon$ to be small enough initially,
$U(P, f) - L(P, f)$ will get tiny as well. $\square$

\pagebreak

\section{Problem 4}

\subsection{Integral to Series}

Notice that if $b \in \N \setminus \{1\}$,
\[\int_{1}^{b} f(x)\,dx = \sum_{i=1}^{b-1} \int_{i}^{i+1} f(x)\,dx\]

With that, since $f$ is monotnonically decreasing, we have
\begin{align*}
             & f(n+1) \le \int_{n}^{n+1} f(x)\,dx\ \forall n \in \N                        \\
  \implies{} & \sum_{i=1}^{N} f(i+1) \le \sum_{i=1}^{N} \int_{i}^{i+1} f(x)\,dx            \\
  \implies{} & \sum_{i=2}^{N} f(i) \le \int_{1}^{N} f(x)\,dx                               \\
  \implies{} & \sum_{i=2}^{\infty} f(i) \le \int_{1}^{\infty} f(x)\,dx                     \\
  \implies{} & \sum_{i=1}^{\infty} f(i) \le f(1) + \int_{1}^{\infty} f(x)\,dx \quad\square
\end{align*}

\subsection{Series to Integral}

We do basically the same thing except with the left end of each interval.
\begin{align*}
             & \int_{n}^{n+1} f(x)\,dx \le f(n)\ \forall n \in \N                  \\
  \implies{} & \sum_{i=1}^{N} \int_{i}^{i+1} f(x)\,dx \le \sum_{i=1}^{N} f(i)      \\
  \implies{} & \int_{1}^{N+1} f(x)\,dx \le \sum_{i=1}^{N} f(i)                     \\
  \implies{} & \int_{1}^{\infty} f(x)\,dx \le \sum_{i=1}^{\infty} f(i)\quad\square
\end{align*}

\pagebreak

\section{Problem 5}

It STP $\int_{a}^{b} f(x) \,dx-\int_{a}^{b} g(x)\,dx=\int_{a}^{b} f(x)-g(x)\,dx=0$.
$f$ and $g$ are both integrable, so $f-g$ should be too, surely.

Notice that the set where $f(x)-g(x)=0$ is dense in $\R$ as well.

Thus, $L(P, f-g)=0\ \forall P$ since $\forall u < v \in [a, b]\ \exists x \in [u, v]: f(x)-g(x)=0$.

With that,
\[\int_{a}^{b} f(x)-g(x)\,dx=\sup_{\text{all }P} L(P, f-g)=0\quad\square\]

\section{Problem 6}

\subsection{Continuity of \texorpdfstring{$f^{-1}$}{f inverse}}

It's given that $f$ attains every value in $\R$, so $f^{-1}$ certainly is well-defined.

Anyways, fix an $x$ and $\delta > 0$ (yeah, i'm reversing variables too).

If we let $f(x-\delta)=f(x)-\epsilon_1$, $f(x+\delta)=f(x)+\epsilon_2$.
Since $f$ is \textit{strictly} increasing, both $\epsilon$s are positive.
If $\epsilon$ is the smaller of the two, I propose that
$|f(x')-f(x)| < \epsilon \implies|x'-x| < \delta$.

I'll show the contrapositive.
If $|x-x'| \ge \delta$, then either:
\begin{enumerate}[nolistsep]
  \item $x-x' \ge \delta \implies x' \le x-\delta \implies f(x)-f(x') \ge \epsilon_1$
  \item $x'-x \ge \delta \implies x+\delta \le x' \implies f(x')-f(x) \ge \epsilon_2$
\end{enumerate}
so in either case $|f(x)-f(x')| \ge \epsilon$. $\square$

\pagebreak

\subsection{Weird Integral Thing}

I'm pretty sure $f^{-1}$ has the same properties $f$ does (strictly increasing, hits all the reals, etc.),
so WLOG let $b \le f(a)$.

It STP
\begin{gather}
  \label{inteq} \int_{0}^{a} f(x)\,dx + \int_{0}^{f(a)} f^{-1}(x)\,dx = af(a)
\end{gather}
since if $b < f(a)$ the integral will only be less than what it is now.

Conside any partition $P$ for on $[0, a]$.
Since $f$ is strictly increasing, we can map this to a partition $P^{-1}$ where $x_i=f(x_i)$.

First, we see that
\begin{align*}
  U(P, f)
   & =\sum_{i=1}^{n} F_i \Delta x_i         \\
   & = \sum_{i=1}^{n} f(x_i)(x_i - x_{i-1})
\end{align*}
since $f$ increases and thus achieves the supremum on the right end.

Second, we have (if you'll excuse the abuse of notation)
\begin{align*}
  L\left(P^{-1}, f^{-1}\right)
   & = \sum_{i=1}^{n} f^{-1}_i \Delta x_i                   \\
   & = \sum_{i=1}^{n} f^{-1}(f(x_{i-1}))(f(x_i)-f(x_{i-1})) \\
   & = \sum_{i=1}^{n} x_{i-1}(f(x_i)-f(x_{i-1}))
\end{align*}

Adding these two together gives
\begin{align*}
      & \sum_{n=1}^{n} f(x_i)(x_i-x_{i-1})+x_{i-1}(f(x_i)-f(x_{i-1})) \\
  ={} & \sum_{n=1}^{n} x_i f(x_i) - x_{i-1} f(x_{i-1})                \\
  ={} & x_n f(x_n)                                                    \\
  ={} & a f(a)
\end{align*}
Since both $f$ and $f^{-1}$ are integrable, the sum of these
two sums must approach the sums of the actual integrals.
As we see, the sum is a constant $a f(a)$, so yeah. $\square$

\pagebreak

\subsection{That Solves a Rudin Problem??}

\subsubsection{Raw Inequality}

Let $f(x)=x^{p-1}$.
Notice that $f^{-1}(x)=x^{\frac{1}{p-1}}=x^{q-1}$.
These functions are obviously continuous, and they're increasing as well due to that $p, q > 1$.

Now for some integrals:
\begin{align*}
  \int_{0}^{u} f(x)\,dx = \left.\left(\frac{x^p}{p}\right)\right|^u_0=\frac{u^p}{p} &  &
  \int_{0}^{v} f(x)\,dx = \left.\left(\frac{x^q}{q}\right)\right|^v_0=\frac{v^q}{q}
\end{align*}
so by what we just proved, $\frac{u^p}{p}+\frac{v^q}{q} \ge uv$.

\subsubsection{When Equality is Achieved}

If $u^p=v^q$, then
\[f(u)=u^{p-1}=\frac{v^q}{u}=\frac{v^q}{\left(v^q\right)^{1/p}}=v^{q-\frac{q}{p}}=v\]
which by the equality in \eqref{inteq} means $\frac{u^p}{p}+\frac{v^q}{q} = uv$.

OTOH, if $u^p \ne v^q$, WLOG assume $u^p > v^q$.

Since we earlier had that $f(u)=v$ if $u^p=v^q$, $u$ being a greater value
combined with the monotonicity of $f$ imply that $f(u) > v$ in this scenario.

This forces $\frac{u^p}{p}+\frac{v^q}{q} < uv$ now, again since
equality is achieved only when \eqref{inteq} happens. $\square$

\pagebreak

\section{Problem 7}

Let $F_i$/$f_i$ indicate the range sup/infs for $f$,
and $A_i$/$a_i$ be the same for $\alpha$.

Since both are monotnonically increasing, the max and mins must occur
at the right and left endpoints respectively.

\subsection{Actually Integrable}

Since $f \in \mathcal{R}(\alpha)$, $\forall \epsilon\ \exists P$ s.t.
\begin{align*}
  \epsilon & > \sum_{i=1}^{n} (F_i-f_i)\Delta \alpha_i                         \\
           & = \sum_{i=1}^{n} (f(x_i)-f(x_{i-1}))(\alpha(x_i)-\alpha(x_{i-1})) \\
           & = \sum_{i=1}^{n} \Delta f_i \cdot (A_i-a_{i-1})                   \\
           & = U(P, \alpha, f) - L(P, \alpha, f)
\end{align*}
and, wow, just like that, we've found a partition that implies $\alpha \in \mathcal{R}(f)$.

\subsection{But to What Value?}

Take $U(P, f, \alpha) + L(P, \alpha, f)$.
As $P$ gets finer, it should go to the sum of the intergrals.

\begin{align*}
  U(P, f, \alpha) + L(P, \alpha, f)
  &= \sum_{i=1}^{n} F_i \Delta \alpha_i + \sum_{i=1}^{n} a_i \Delta f_i \\
  &= \sum_{i=1}^{n} f(x_i)(\alpha(x_i)-\alpha(x_{i-1}))+\alpha(x_{i-1})(f(x_i)-f(x_{i-1})) \\
  &= \sum_{i=1}^{n} f(x_i)\alpha(x_i) - f(x_{i-1})\alpha(x_{i-1}) \\
  &= f(x_n)\alpha(x_n) - f(x_0)\alpha(x_0) \\
  &= f(b)\alpha(b)-f(a)\alpha(a)\quad\square
\end{align*}

\end{document}
