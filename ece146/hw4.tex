\documentclass[11pt]{article}

\usepackage{fancyhdr}
\usepackage{tipa}
\usepackage{fontspec}
\usepackage{amsfonts}
\usepackage{enumitem}
\usepackage[margin=1in]{geometry}
\usepackage{graphicx}
\usepackage{float}
\usepackage{amsmath}
\usepackage{braket}
\usepackage{amssymb}
\usepackage{booktabs}
\usepackage{hyperref}
\usepackage{mathtools}
\usepackage{float}
\usepackage{algpseudocodex}
\usepackage{titlesec}
\usepackage{bbm}

\newcommand{\cnum}{ECE M146}
\newcommand{\ced}{Spring 2023}
\newcommand{\ctitle}[4]{\title{\vspace{-0.5in}\cnum, \ced\\Problem Set #1: #2}\author{\vspace{-0.35in}\\Name: #3, UID: #4}}
\newcommand{\solution}[1]{{{\color{blue}{\bf Solution:} {#1}}}}

\renewcommand*{\theenumi}{\alph{enumi}}
\renewcommand*\labelenumi{(\theenumi)}
\renewcommand*{\theenumii}{\roman{enumii}}
\renewcommand*\labelenumii{\theenumii.}

\newcommand{\R}{\mathbb{R}}
\newcommand{\norm}[1]{\left\lVert#1\right\rVert}

\DeclareMathOperator{\Cov}{Cov}
\DeclareMathOperator{\Var}{Var}
\DeclareMathOperator{\E}{E}
\DeclareMathOperator{\Tr}{Tr}
\DeclareMathOperator{\sign}{sign}


\begin{document}
\ctitle{4}{K-Means and PCA}{Kevin Sheng}{406 196 414}
\date{}
\maketitle

\section{Problem 1}

\subsection{Problem 1a}

\solution{
      If $K=n$, then the optimal assignment is $C_i=\{x_i\}$.
      Each sample would be its own cluster's center, and
      so the best objective value would be \boxed{0}.
}

\subsection{Problem 1b}

\solution{
      I'll remove the subscripts from $\mu_i$ and $C_i$
      since we're only talking about a single cluster.

      Anyways, we take gradients and get
      \begin{align*}
            \nabla_\mu \lambda\norm{\mu}_2^2 + \sum_{x \in C} \norm{x-\mu}_2^2
             & = \nabla_\mu \lambda\norm{\mu_2}^2 + \sum_{x \in C} \nabla_\mu \norm{x-\mu}_2^2 \\
             & = 2\lambda\mu + \sum_{x \in C} -2(x - \mu)                                      \\
             & = 2(\lambda + |C|)\mu - 2 \sum_{x \in C} x
      \end{align*}

      And setting this equal to $\mathbf{0}$ gives
      \[ \mu(\lambda + |C|) = \sum_{x \in C} x
            \implies \boxed{\mu^* = \frac{\sum_{x \in C} x}{\lambda + |C|}}\]
}

\subsection{Problem 1c}

\solution{
      Since we have to include the distance the TAs walk,
      we have to do regularized K-Means with $\lambda=1$.

      Applying the formula for the optima we just derived makes the problem
      \begin{gather*}
            \min_{C_1, C_2, \cdots, C_K}
            \sum_{i=1}^{K} \left(\norm{\mu_i}_2^2 + \sum_{x \in C_i} \norm{x-\mu_i}_2^2\right)
            \text{ where} \\
            \mu_i=\frac{\sum_{x \in C_i} x}{1 + |C|}
      \end{gather*}
}

\pagebreak

\section{Problem 2}

\subsection{Problem 2a}

\solution{
      I take it we already assume the stuff in Claim 1 is true?

      Anyways, if so, we can change the objective function like so:
      \begin{align*}
            \sum_{i=1}^{N} p_i\norm{x_i-\tilde{x}_i}_2^2
             & = \sum_{i=1}^{N} p_i\norm{x_i - UWx_i}_2^2                                                 \\
             & = \sum_{i=1}^{N} p_i\norm{x_i - UU^Tx_i}_2^2                                               \\
             & = \sum_{i=1}^{N} p_i\left(\norm{x_i}_2^2 - \Tr\left(x_i^TUU^Tx_i\right)\right)\quad\square
      \end{align*}
}

\subsection{Problem 2b}

\solution{
      We have
      \begin{align*}
            \Tr\left(x_i^TUU^Tx_i\right)
             & = \Tr\left(\left(x_i^TU\right)\left(U^Tx_i\right)\right) \\
             & = \Tr\left(\left(U^Tx_i\right)\left(x_i^TU\right)\right) \\
             & = \Tr\left(U^Tx_ix_i^TU\right)\quad\square
      \end{align*}
      ($U^Tx_i \in \R^{l \times 1}$ and $x_i^TU \in \R^{1 \times l}$, so the dimensions check out just fine.)
}

\subsection{Problem 2c}

\solution{
      Minimizing the current objective is the same as maximizing its negative;
      that makes our new objective
      \[\max_{U \in \R^{d \times l}, UU^T=I_l}
            \sum_{i=1}^N p_i\left(\Tr\left(x_i^TUU^Tx_i\right)-\norm{x_i}_2^2\right)\]
      The second part of the summation doesn't change no matter how we pick $U$, so we can remove it.
      After that, our objective function is
      \begin{align*}
            \sum_{i=1}^N p_i\Tr\left(x_i^TUU^Tx_i\right)
             & = \sum_{i=1}^N p_i\Tr\left(U^Tx_ix_i^TU\right)                          \\
             & = \sum_{i=1}^N \Tr\left(p_iU^Tx_ix_i^TU\right)                          \\
             & = \Tr\left(\sum_{i=1}^N p_iU^Tx_ix_i^TU\right)                          \\
             & = \Tr\left(U^T\left(\sum_{i=1}^N p_ix_ix_i^T\right)U\right)\quad\square
      \end{align*}
}

\subsection{Problem 2d}

\solution{
      Now that the summation in the middle is neatly wrapped up
      in a matrix $\tilde{A}$ with eigendecomposition $\tilde{A}=V\Lambda V^T$.

      Assuming $l \le d$, we take the eigenvalues $\lambda_1$ through $\lambda_l$ and let
      \[W=\begin{bmatrix}v_1 & v_2 & \cdots & v_l\end{bmatrix}\]
      where $v_i$ is the eigenvector corresponding to the eigenvalue $\lambda_i$.

      The only difference between this solution and the one derived in class is $\tilde{A}$.
}

\end{document}
