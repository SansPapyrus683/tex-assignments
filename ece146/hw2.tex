\documentclass[11pt]{article}

\usepackage{fancyhdr}
\usepackage{tipa}
\usepackage{fontspec}
\usepackage{amsfonts}
\usepackage{enumitem}
\usepackage[margin=1in]{geometry}
\usepackage{graphicx}
\usepackage{float}
\usepackage{amsmath}
\usepackage{braket}
\usepackage{amssymb}
\usepackage{booktabs}
\usepackage{hyperref}
\usepackage{mathtools}
\usepackage{float}
\usepackage{algpseudocodex}
\usepackage{titlesec}
\usepackage{bbm}

\newcommand{\cnum}{ECE M146}
\newcommand{\ced}{Spring 2023}
\newcommand{\ctitle}[4]{\title{\vspace{-0.5in}\cnum, \ced\\Problem Set #1: #2}\author{\vspace{-0.35in}\\Name: #3, UID: #4}}
\newcommand{\solution}[1]{{{\color{blue}{\bf Solution:} {#1}}}}

\renewcommand*{\theenumi}{\alph{enumi}}
\renewcommand*\labelenumi{(\theenumi)}
\renewcommand*{\theenumii}{\roman{enumii}}
\renewcommand*\labelenumii{\theenumii.}


\DeclareMathOperator{\Cov}{Cov}
\DeclareMathOperator{\Var}{Var}
\DeclareMathOperator{\E}{E}
\DeclareMathOperator{\Tr}{Tr}
\DeclareMathOperator{\sign}{sign}

\newcommand{\correction}[1]{\textcolor{red}{#1}}


\begin{document}
\ctitle{2}{Decision Trees, KNN, etc.}{Kevin Sheng}{406 196 414}
\date{}
\maketitle

\section{Problem 1}

\subsection{Problem 1a}

\solution{
      For convenience, I'll represent a sample of $a$ $0$s and $b$ $1$s as $[a, b]$.

      If we split on $U$, we get a $[2, 1]$ and a $[0, 2]$.
      The first branch's entropy is
      \[-\left(\frac{0}{0+2}\log\frac{0}{0+2}+\frac{2}{0+2}\log\frac{2}{0+2}\right)=0\]
      while the second's is
      \[-\left(\frac{1}{1+2}\log\frac{1}{1+2}+\frac{2}{1+2}\log\frac{2}{1+2}\right) \approx 0.918\]
      making the conditional entropy $\frac{3}{5} \cdot H[Y \mid U = 1] \approx 0.55$.

      If we split on $V$, we get a $[1, 1]$ and a $[1, 2]$.
      The entropy of the first is $1$ and the second is the same as the value
      we got for $H[Y \mid U = 1]$, so our entropy for this variable is
      \[\frac{2}{5} \cdot 1 + \frac{3}{5} \cdot H[Y \mid V = 0] \approx 0.951\]

      Finally, if we split on $W$ we get a $[2, 2]$ and a $[0, 1]$, so the total entropy is
      \[\frac{4}{5} \cdot 1 + \frac{1}{5} \cdot 0 = 0.8\]

      Splitting on $\boxed{U}$ gives the lowest entropy, so ID3 is gonna pick that one.
}

\pagebreak

\subsection{Problem 1b}

\solution{
      After splitting on $U$, we have two nodes, one of which already has entropy $0$.

      We just need to split the $[2, 1]$ branch on instances where $U=0$.
      Luckily, choosing $W$ as our next feature properly classifies the remaining instances.

      Here's the decision tree in the form of a bunch of if-else statements:
      \begin{algorithmic}
            \If{$U=0$}
            \If{$W=0$}
            \State \Return $Y=0$
            \ElsIf{$W=1$}
            \State \Return $Y=1$
            \EndIf
            \ElsIf{$U=1$}
            \State \Return $Y=1$
            \EndIf
      \end{algorithmic}
}

\subsection{Problem 1c}

\solution{
      The initial entropy is $-\frac{2}{5}\log\frac{2}{5}-\frac{3}{5}\log\frac{3}{5} \approx 0.971$.
      Splitting on $U$ reduced that to around $0.55$, so we gained around $0.42$ bits of information.

      To force a decision tree of depth $0$, we would need an $\epsilon$ value of at least $0.42$.
      Since the training data has more $1$s than it does $0$s,
      this tree would predict $Y=1$ on the new sample.
}

\section{Problem 2}

\subsection{Problem 2a}

\solution{
      Points 2, 3, and 5 are misclassified, so the test error is $\boxed{0.6}$.
}

\subsection{Problem 2b}

\solution{
      The smaller the angle between two points, the greater their cosine similarity.

      Now, just points $1$ and $3$ are misclassified, resulting in a test error of $\boxed{0.4}$.
}

\subsection{Problem 2c}

\solution{
      If $k$ is too large, the algorithm might underfit,
      while if $k$ is too small, it might overfit.
}

\pagebreak

\section{Problem 3}

\subsection{Problem 3a}

\solution{
      If $k$ was a valid kernel function, then
      \[\begin{bmatrix}
                  k(x_1, x_1) & k(x_1, x_2) \\
                  k(x_2, x_1) & k(x_2, x_2)
            \end{bmatrix}\]
      must be PSD for across all $x_1$ and $x_2$.

      Consider $x_1=\begin{bmatrix}0 \\ 0\end{bmatrix}$ and $x_2=\begin{bmatrix}0 \\ 1\end{bmatrix}$.
      For them, the matrix is
      \[\begin{bmatrix}
                  0 & 1 \\
                  1 & 0
            \end{bmatrix}\]
      which isn't PSD, as
      \[\begin{bmatrix}1 & -1\end{bmatrix}
      \begin{bmatrix}
            0 & 1 \\
            1 & 0
      \end{bmatrix}
      \begin{bmatrix}1 \\ -1\end{bmatrix}=-2\]
      which is negative.

      Thus, $k$ cannot be a valid kernel function.
}

\subsection{Problem 3b}

\solution{
      It suffices to find an inner product representation.

      Since $k_0$ is a valid kernel, $\exists \phi_0: \R^D \to \R$ s.t. $k_0(x, x')=\phi_0(x)^T\phi_0(x)$.

      Consider $\phi(x)=f(x)\phi(Ax)$.
      Then we have
      \begin{align*}
            k(x, x') & =f(x)k_0(Ax, Ax')f(x')                       \\
                     & = f(x)\phi_0(Ax)^T\phi_0(Ax)f(x')            \\
                     & = (\phi_0(x)f(Ax))^T \cdot (\phi_0(Ax)f(x')) \\
                     & = \phi(x)^T\phi(x)\quad\square
      \end{align*}
}

\end{document}
