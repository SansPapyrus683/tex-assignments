\documentclass[12pt]{article}

% a template that a friend gave, it's worked well enough for me
% i have added some packages and stuff that have proved useful

\usepackage{fancyhdr}
\usepackage{tipa}
\usepackage{fontspec}
\usepackage{amsfonts}
\usepackage{enumitem}
\usepackage[margin=1in]{geometry}
\usepackage{graphicx}
\usepackage{float}
\usepackage{amsmath}
\usepackage{braket}
\usepackage{amssymb}
\usepackage{booktabs}
\usepackage{hyperref}
\usepackage{mathtools}
\usepackage{xcolor}
\usepackage{float}
\usepackage{algpseudocodex}
\usepackage{titlesec}
\usepackage{bbm}

\pagestyle{fancy}
\fancyhf{} % sets both header and footer to nothing
\lhead{Kevin Sheng}
\setmainfont{Comic Neue}
\renewcommand{\headrulewidth}{1pt}
\setlength{\headheight}{0.75in}
\setlength{\oddsidemargin}{0in}
\setlength{\evensidemargin}{0in}
\setlength{\voffset}{-.5in}
\setlength{\headsep}{10pt}
\setlength{\textwidth}{6.5in}
\setlength{\headwidth}{6.5in}
\setlength{\textheight}{8in}
\renewcommand{\headrulewidth}{0.5pt}
\renewcommand{\footrulewidth}{0.3pt}
\setlength{\textwidth}{6.5in}
\usepackage{setspace}
\usepackage{multicol}
\usepackage{float}
\setlength{\columnsep}{1cm}
\setlength\parindent{24pt}
\usepackage [english]{babel}
\usepackage [autostyle, english = american]{csquotes}
\MakeOuterQuote{"}

\setlength{\parskip}{6pt}
\setlength{\parindent}{0pt}

\titlespacing\section{0pt}{12pt plus 4pt minus 2pt}{0pt plus 2pt minus 2pt}
\titlespacing\subsection{0pt}{12pt plus 4pt minus 2pt}{0pt plus 2pt minus 2pt}
\titlespacing\subsubsection{0pt}{12pt plus 4pt minus 2pt}{0pt plus 2pt minus 2pt}

\hypersetup{colorlinks=true, urlcolor=blue}

\newcommand{\correction}[1]{\textcolor{red}{#1}}


\lhead{406-196-414}
\rhead{CS 180}

\begin{document}

\section{Topological Ordering or Cycle}

\subsection{Algorithm}

\begin{algorithmic}[1]
    \State Run the standard DAG toposort algorithm
    \If{All nodes have been added}
        \State \Return computed topological order
    \EndIf
    \State Do another run of the toposort algorithm on the \textit{reversed} graph

    \item[]
    \State $v \gets$ any node that is in neither toposort's already sorted nodes
    \State $\texttt{frontier} \gets \text{empty queue}$
    \State $\texttt{comeFrom} \gets \text{array of size }n\text{, all nil}$
    \State Add $v$ to $\texttt{frontier}$
    \While{\texttt{comeFrom}[v] is nil}
        \State $\texttt{curr} \gets \text{popped front of }\texttt{frontier}$
        \ForAll{$n \in$ \texttt{curr}'s neighbors}
            \If{$\texttt{comeFrom}[n]$ is nil}
                \State $\texttt{comeFrom}[n] = \texttt{curr}$
                \State Add $n$ to $\texttt{frontier}$
            \EndIf
        \EndFor
    \EndWhile

    \item[]
    \State Trace back cycle starting from $v$
    \State \Return the computed cycle
\end{algorithmic}

\subsection{Complexity}

The first section runs in $O(m+n)$ time, as has already been established in class.
The second section runs a multi-source BFS from all the neighbors of $v$.
Besides that the initial queue has more than one node, all other aspects
of the BFS are identical to the one we already came up with,
so the time complexity of this section is $O(m+n)$.

Finally, tracing back the cycle will take time proportional to the length of the cycle.
Since the cycle can have at most $n$ nodes, the complexity of this is $O(n)$.

Adding these complexities together,
we see that the overall time complexity is $O(m+n)$.

\pagebreak

\section{Deleting a Node to Destroy All Paths}

\subsection{Proof of Existence}

SWOC the $v$ supposed in the problem doesn't exist.
Then, there must be two disjoint paths from $s$ to $t$.
Each path must have strictly greater than $\frac{n}{2}-1$ intermediary nodes,
and since none of the paths intersect, this means
\[|V| > 2\left(\frac{n}{2}-1\right)+2 = n\]
Contradiction, since $|V|=n$ by definition. $\square$

\subsection{Algorithm}

\begin{algorithmic}[1]
    \State $L_0 \gets \{s\}$
    \State $\texttt{dist} \gets 0$
    \State Mark $s$ as explored
    \While{$L_{\texttt{dist}}$ isn't empty}
        \State Increment \texttt{dist}
        \State $L_{\texttt{dist}} \gets \{\}$
        \ForAll{$\texttt{curr} \in L_{\texttt{dist} - 1}$}
            \ForAll{$n \in$ \texttt{curr}'s neighbors}
                \If{$n$ isn't explored}
                    \State $L_{\texttt{dist}} \gets L_{\texttt{dist}} \cup \{n\}$
                \EndIf
            \EndFor
        \EndFor

        \item[]
        \If{$L_{\texttt{dist}} = 1$}
            \State \Return the only element in $L_{\texttt{dist}}$
        \EndIf
    \EndWhile
\end{algorithmic}

\subsection{Complexity}

This algorithm is a basic BFS.
The only modification is that it stops early if it sees that the current
$L_{\texttt{dist}}$ is of size exactly $1$, in which case it stops early.
Thus, we see that the overall time complexity is $O(m+n)$.

\pagebreak

\section{Butterfly Classification}

\subsection{Algorithm}

\begin{algorithmic}[1]
    \State $\texttt{comp} \gets \text{array of size }n\text{, all nil}$
    \State $\texttt{currComp} \gets 0$
    \For{$b \gets 1, n$}
        \If{$\texttt{comp}[i]$ isn't nil}
            \State \textbf{continue}
        \EndIf

        \item[]
        \State $\texttt{comp}[b] \gets \texttt{currComp}$
        \State $\texttt{frontier} \gets \text{empty queue}$
        \State Add $b$ to \texttt{frontier}
        \While{\texttt{frontier} isn't empty}
            \State $\texttt{curr} \gets \text{popped front of }\texttt{frontier}$
            \ForAll{$n \in L_{\texttt{curr},\text{same}}$}
                \If{$\texttt{comp}[n]$ is nil}
                    \State $\texttt{comp}[n] = \texttt{currComp}$
                    \State Add $n$ to $\texttt{frontier}$
                \EndIf
            \EndFor
        \EndWhile
        \State Increment \texttt{currComp}
    \EndFor

    \item[]
    \For{$b \gets 1, n$}
        \ForAll{$n \in L_{b,\text{diff}}$}
            \If{$\texttt{comp}[b] \ne \texttt{comp}[n]$}
                \State \Return not consistent
            \EndIf
        \EndFor
    \EndFor
    \State \Return consistent
\end{algorithmic}

\pagebreak

\subsection{Correctness}

This algorithm just runs a basic DFS followed
by a for loop through all of the different judgements,
so it definitely terminates.

Suppose the judgements are consistent.
Let $G$ be a graph where there's an edge from butterly $i$ to $j$ if
the two were judged to be the same.
Notice that for all pairs of butterflies $a$ and $b$, they're the same species
iff there exists a path from $a$ to $b$ in $G$.
From this, it follows that if two butterflies
are judged differently, then they must be in different connected
components of the graph, since undirected connectivity goes both ways.

With these two facts, we can now say that there is an inconsistency in
the judgmenets iff there exists two butterflies $a$ and $b$
that are in the same connected component but have been explicitly
judged to be different.

The proposed algorithm computes the connected components
and checks all the "different" judgements by construction, so it is correct.

\subsection{Complexity}

The first section does a DFS floodfill of all connected components.
Each edge ("same" judgement) is traversed at most once and each node (butterfly)
is explored at most once, so the complexity of this part is $O(n+m)$.

The second section does an $O(1)$ operation for all the "different" judgements,
so the complexity of that part is $O(m)$.
Combined with the first part, we get $\boxed{O(n+m)}$.

\pagebreak

\section{Number of Shortest Paths}

\subsection{Algorithm}

\begin{algorithmic}[1]
    \State $L_0 \gets \{v\}$
    \State $\texttt{dist} \gets 0$
    \State $\texttt{numWays} \gets \text{array of size }n\text{, all }0$
    \State Mark $v$ as explored
    \State $\texttt{numWays}[v] \gets 1$

    \item[]
    \While{$w \notin L_{\texttt{dist}}$}
        \State Increment \texttt{dist}
        \State $L_{\texttt{dist}} \gets \{\}$
        \ForAll{$\texttt{curr} \in L_{\texttt{dist} - 1}$}
            \ForAll{$n \in$ \texttt{curr}'s neighbors}
                \If{$n$ isn't explored}
                    \State $L_{\texttt{dist}} \gets L_{\texttt{dist}} \cup \{n\}$
                \EndIf
                \If{$n \in L_{\texttt{dist}}$}
                    \State $\texttt{numWays}[n] \gets \texttt{numWays}[n] + \texttt{numWays}[\texttt{curr}]$
                \EndIf
            \EndFor
        \EndFor
    \EndWhile

    \item[]
    \State \Return $\texttt{numWays}[w]$
\end{algorithmic}

\subsection{Correctness}

We'll prove by induction that by the end of the algorithm,
$\texttt{numWays}[n]$ contains the number of shortest $v-n$ paths in $G$.
Our base case, $v-v$, is trivially true, since there's only one
shortest path from $v$ to $v$.
With this, we assume that $\texttt{numWays}[n]$ contains the number
of shortest paths for all $n \in L_d$.
It remains to prove that the same is true for all $n \in L_{d+1}$.

Any shortest path from $v$ to $n \in L_{d+1}$ has to pass
through a $n' \in L_d$ before getting to $n$.
From this, we can find that
\[\texttt{numWays}[n]=\sum_{(n', n) \in E \land n' \in L_d} \texttt{numWays}[n']\]
This formula is precisely what the algorithm calculates,
so the proof is complete.

\subsection{Complexity}

Once again, we have a standard BFS with the addition of \texttt{numWays}.
If we store $L_{\texttt{dist}}$ with a hash set, we can do the membership check
in the else if statement in constant time.
Thus, this algorithm has the same time complexity as a standard BFS,
which is $O(m+n)$.

\pagebreak

\section{Births and Deaths}

\subsection{Algorithm}

This problem reduces to topological sorting.

\begin{algorithmic}[1]
    \State $V \gets \{P_1, P_1', P_2, P_2', \cdots P_n, P_n'\}$
    \State $E \gets \{\}$
    \For{$p \gets 1, n$}
        \State $E \gets E \cup (P_p, P_p')$
    \EndFor

    \item[]
    \ForAll{$\texttt{fact} \in \texttt{facts}$}
        \If{\texttt{fact} is of the first type}
            \State $E \gets E \cup (P_i', P_j)$
        \ElsIf{\texttt{fact} is of the second type}
            \State $E \gets E \cup (P_i', P_j')$
        \ElsIf{\texttt{fact} is of the third type}
            \State $E \gets E \cup (P_i, P_j')$
            \State $E \gets E \cup (P_j, P_i')$
        \EndIf
    \EndFor

    \item[]
    \State Attempt to toposort $G = (V, E)$
    \If{valid ordering}
        \State $\texttt{allEvents} \gets []$
        \State $t \gets 0$
        \ForAll{$\texttt{event} \in \text{the given ordering}$}
            \If{\texttt{event} is $P_i$}
                \State Append $i$'s birth event at time $t$ to \texttt{allEvents}
            \ElsIf{\texttt{event} is $P_i'$}
                \State Append $i$'s death event at time $t$ to \texttt{allEvents}
            \EndIf
            \State $t \gets t + 1$
        \EndFor
        \State \Return \texttt{allEvents}
    \Else
        \State \Return inconsistent memories
    \EndIf
\end{algorithmic}

\pagebreak

\subsection{Correctness}

It's easily seen that this algorithm terminates.
It remains to show that a the existence of a consistent ordering
is equivalent with the algorithm outputting such a consistent ordering.

If there exists a valid order of birth/death events that align
with the collected facts, then the following things must be true:
\begin{itemize}[nolistsep]
    \item \textbf{Type 1}: $P_i$'s death occurred before $P_j$'s birth.
    \item \textbf{Type 2}: $P_i$'s death occurred before $P_j$'s death.
    \item \textbf{Type 3}: $P_i$'s birth occurred before $P_j$'s death
          and $P_j$'s birth occurred before $P_i$'s death.
\end{itemize}
By construction of a graph with edges between birth events ($P_i$ elements)
and death events ($P_i'$ elements) and topologically sorting it,
the algorithm guarantees an ordering that is consistent with the facts.
If there is no ordering consistent with the facts, the toposort
is guaranteed to reoprt that as well.

\subsection{Complexity}

The graph we construct has $2n$ nodes and at most $2m$ edges,
where $m$ is the number of facts.
The toposort algorithm will thus run in $O(2n+2m)=O(n+m)$ time.

The remaining section where we transform the ordering into a
log of the births and deaths takes $O(n)$, since that's just
a simple for loop through all the nodes in the topolgocial ordering.
Thus, the entire algorithm runs in time $O(m+n)$.

\end{document}
