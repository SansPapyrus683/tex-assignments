\documentclass[12pt]{article}

% a template that a friend gave, it's worked well enough for me
% i have added some packages and stuff that have proved useful

\usepackage{fancyhdr}
\usepackage{tipa}
\usepackage{fontspec}
\usepackage{amsfonts}
\usepackage{enumitem}
\usepackage[margin=1in]{geometry}
\usepackage{graphicx}
\usepackage{float}
\usepackage{amsmath}
\usepackage{braket}
\usepackage{amssymb}
\usepackage{booktabs}
\usepackage{hyperref}
\usepackage{mathtools}
\usepackage{xcolor}
\usepackage{float}
\usepackage{algpseudocodex}
\usepackage{titlesec}
\usepackage{bbm}

\pagestyle{fancy}
\fancyhf{} % sets both header and footer to nothing
\lhead{Kevin Sheng}
\setmainfont{Comic Neue}
\renewcommand{\headrulewidth}{1pt}
\setlength{\headheight}{0.75in}
\setlength{\oddsidemargin}{0in}
\setlength{\evensidemargin}{0in}
\setlength{\voffset}{-.5in}
\setlength{\headsep}{10pt}
\setlength{\textwidth}{6.5in}
\setlength{\headwidth}{6.5in}
\setlength{\textheight}{8in}
\renewcommand{\headrulewidth}{0.5pt}
\renewcommand{\footrulewidth}{0.3pt}
\setlength{\textwidth}{6.5in}
\usepackage{setspace}
\usepackage{multicol}
\usepackage{float}
\setlength{\columnsep}{1cm}
\setlength\parindent{24pt}
\usepackage [english]{babel}
\usepackage [autostyle, english = american]{csquotes}
\MakeOuterQuote{"}

\setlength{\parskip}{6pt}
\setlength{\parindent}{0pt}

\titlespacing\section{0pt}{12pt plus 4pt minus 2pt}{0pt plus 2pt minus 2pt}
\titlespacing\subsection{0pt}{12pt plus 4pt minus 2pt}{0pt plus 2pt minus 2pt}
\titlespacing\subsubsection{0pt}{12pt plus 4pt minus 2pt}{0pt plus 2pt minus 2pt}

\hypersetup{colorlinks=true, urlcolor=blue}

\newcommand{\correction}[1]{\textcolor{red}{#1}}


\lhead{406-196-414}
\rhead{CS 180}

\newcommand{\what}{\stackrel{?}=}

\begin{document}

\section{A General Recurrence Relation}

My guy, this is just the
\href{https://en.wikipedia.org/wiki/Master_theorem_(analysis_of_algorithms)}{Master Theorem}...

\subsection{Proof of Lemma 0.3}

The proof is by induction.

\subsubsection{Base Case}

Set $n=b=b^1$.
Then, $\log_b^n=1$ and
\begin{align*}
    T(n)
     & = a^1T(1)+cn^d\left(\sum_{j=0}^{0}\left(\frac{a}{b^d}\right)^j\right) \\
     & = aT\left(\frac{n}{b}\right)+cn^d
\end{align*}

We now assume this is true for all $n=b^i$.

\subsubsection{Inductive Step}

It remains to prove that the statement being true for $n=b^i$ implies the same for $n=b^{i+1}$.
\begin{align*}
    T\left(b^{i+1}\right)
     & = aT\left(b^i\right)+cb^{i+1}d                                                                         \\
     & = a\left[a^iT(1)+cb^{id}\left(\sum_{j=0}^{i-1} \left(\frac{a}{b^d}\right)^j\right)\right]+cb^{d(i+1)}  \\
     & = a^{i+1}T(1)+acb^{id}\left(\sum_{j=0}^{i-1} \left(\frac{a}{b^d}\right)^j\right)+cb^{d(i+1)}           \\
     & = a^{i+1}T(1)+cb^{(i+1)d}\left[\left(\sum_{j=0}^{i-1} \left(\frac{a}{b^d}\right)^{j+1}\right)+1\right] \\
     & = a^{i+1}T(1)+cb^{(i+1)d}\left[\left(\sum_{j=1}^{i} \left(\frac{a}{b^d}\right)^j\right)+1\right]       \\
     & = a^{i+1}T(1)+cb^{(i+1)d}\left(\sum_{j=0}^{i} \left(\frac{a}{b^d}\right)^j\right)\quad\square
\end{align*}

The final line is due to that $\left(\frac{a}{b^d}\right)^0=1$.

\subsection{Proof of Lemma 0.4}

For all the below, proofs, we can use lemma 0.1
to change $a^{\log_b n}$ to $n^{\log_b a}$.

\subsubsection{Case 1: $d=\log_b a$}

Note that $d=\log_b a \rightarrow a=b^d \rightarrow \frac{a}{b^d}=1$.
\begin{align*}
    T(n)
    &= n^{\log_b a} T(1) + cn^d \left(\sum_{j=0}^{\log_b n - 1} 1\right) \\
    &= n^{\log_b a} T(1)+cn^d\left(\log_b n - 1\right) \\
    &= n^d\left(T(1)+c\left(\log_b n - 1\right)\right) \\
    &= O\left(n^d \log_b n\right)
\end{align*}

\subsubsection{Case 2: $d>\log_b a$}

The logarithm means $\frac{a}{b^d}<1$ and we can treat the sum as if it converges.
\begin{align*}
    T(n)
    &< n^{\log_b a}T(1)+cn^d \cdot \frac{a/b^d}{1-a/b^d} \\
    &= n^{\log_b a}T(1)+cn^d \cdot \frac{a}{b^d-a} \\
    &= O\left(n^d\right)
\end{align*}
We cast out $n^{\log_b a}$ since $n^{\log_b a}<n^d$.

\subsubsection{Case 3: $d<\log_b a$}

We use the formula for the sum of a geometric series provided in lemma 0.2.
\begin{align*}
    T(n)
    &= n^{\log_b a}T(1)+cn^d \cdot \frac{a}{b^d} \cdot \frac{1-\left(\frac{a}{b^d}\right)^{\log_b n}}{1-\frac{a}{b^d}} \\
    &= n^{\log_b a}T(1)+cn^d \cdot \frac{a}{b^d} \cdot \frac{1-n^{\log_b a/b^d}}{1-\frac{a}{b^d}} \\
    &= n^{\log_b a}T(1)+\frac{ac}{b^d} \cdot n^d \cdot \frac{1-n^{(\log_b a) - d}}{1-\frac{a}{b^d}} \\
    &= n^{\log_b a}T(1)+\frac{ac}{b^d} \cdot \frac{n^d-n^{\log_b a}}{1-\frac{a}{b^d}} \\
    &= n^{\log_b a}T(1)+\frac{ac}{b^d} \cdot \frac{n^{\log_b a}-n^d}{\frac{a}{b^d}-1} \\
    &= O\left(n^{\log_b a}\right)
\end{align*}

\section{Array Shift}\label{arrshift}

\subsection{Algorithm}

Binary search on the greatest $i$ s.t. $A[i] \ge A[0]$.
If the array is 0-indexed, the answer is $n-(i+1)$.

\subsection{Correctness}

It is well-established that binary search terminates
and returns the greatest or lowest element such that it makes some
monotonic function that only returns $0$ or $1$ return $1$.
Now we just have to prove that $f(i)=A[i] \ge A[0]$ is monotonic.

Note that due to the structure of the array,
we have two strictly increasing sequences of numbers side by side,
with the largest of the second sequence lesser than the smallest of the first sequence.

Any $i$ that resides within the first sequence is trivially
going to be greater than or equal to $A[0]$, since $A[0]$ is the smallest element
in the first sequence.
As for any $i$ in the second sequence, $A[i]<A[n-1]<A[0]$.

Since all elements in the first sequence occur before the second sequence (obviously),
$f$ will return $1$ for some left segment of the search space and $0$ everywhere else.
Thus, binary search will return the correct answer. $\square$

\subsection{Time Complexity}

The computation within the binary search is constant time and the length of the
initial range is $n$, so our overall complexity is $\boxed{O(\log n)}$.

\pagebreak

\section{k-th Smallest Element}

\subsection{Algorithm}

With binary search we can find the number of elements that are strictly
less than or less than or equal to a specified value $k$,
so I'll calculate that stuff without any elaboration.

The algorithm itself is also a binary search.
The pseudocode for the comparison function is as follows:
\begin{algorithmic}[1]
    \Procedure{\texttt{beforeK}}{i}
        \State $\texttt{pos1} \gets $ number of elements in array 1 that are strictly lesser than $i$
        \State $\texttt{pos2} \gets $ same but for array 2
        \State \Return $\texttt{pos1}+\texttt{pos2} < k$
    \EndProcedure
\end{algorithmic}
We binary search for the largest $i$ that makes \texttt{beforeK} return true.

\subsection{Correctness}

We first prove that $\texttt{beforeK}$ is monotonic just like in \ref{arrshift}.
Notice that \texttt{pos1} and \texttt{pos2} are both nondecreasing (ND),
since the number of elements a value is greater than can only increase
as the value itself increases.
The sum of two ND sequences is also ND, so $\texttt{pos1}+\texttt{pos2}$ is ND.

It remains to prove that the breakpoint is the value of the $k$th element $x$.
By the premise of the problem, this is strictly greater than at \textit{most}
$k-1$ other elements across both arrays.
Thus, the value of the $k$-th element will make $\texttt{beforeK}$ return true.

Now consider $x+1$.
$x+1>x$, so it's greater than the $k$-th element of the array and has
is also strictly greater than at least $k$ elements across both arrays.
As such, $\texttt{beforeK}$ will output false for $x+1$
(and any value greater than that). $\square$

\subsection{Time Complexity}

Our initial range is the smallest element in the array to the largest element.
These two can be computed in constant time
by looking at the starts and ends of both arrays.

Next, the binary search in $\texttt{beforeK}$ runs in $O(\log n)$,
since it has to calculate \texttt{pos1} and \texttt{pos2}.
Our final complexity is then $\boxed{O(\log (\texttt{max}-\texttt{min})\log n)}$,
where \texttt{max} and \texttt{min} are the largest and smallest elements
across both arrays respectively.

\pagebreak

\section{Majority Bank Card}

\subsection{Algorithm}

To avoid the time consuming process of splitting it in half,
this can also be implemented with a single global array with
two indices to tell the function what range it should look at.

\begin{algorithmic}[1]
    \LComment{Returns the card of the majority (if there is one) and how many times it occurs}
    \Procedure{\texttt{majority}}{\texttt{cards}}
        \State $\texttt{sz} \gets $ length of cards
        \Comment{just a shorthand}
        \If{\texttt{cards} only has one card}
            \State \Return $(\text{that card}, 1)$
        \EndIf

        \item[]
        \State Split \texttt{cards} into \texttt{half1} and \texttt{half2}
        \State $(c_1, s_1) \gets \Call{\texttt{majority}}{\texttt{half1}}$
        \State $(c_2, c_2) \gets \Call{\texttt{majority}}{\texttt{half2}}$
        \If{$c_1 \ne \varnothing$}
            \State $\texttt{other} \gets $ \# of cards in $\texttt{half2}$ equivalent to $c_1$
            \If{$s_1+\texttt{other}>\frac{\texttt{sz}}{2}$}
                \State \Return $(c_1, s_1+\texttt{other})$
            \EndIf
        \ElsIf{$c_2 \ne \varnothing$}
            \State $\texttt{other} \gets $ \# of cards in $\texttt{half1}$ equivalent to $c_2$
            \If{$s_2+\texttt{other}>\frac{\texttt{sz}}{2}$}
                \State \Return $(c_2, s_2+\texttt{other})$
            \EndIf
        \EndIf

        \item[]
        \State \Return $\varnothing$
    \EndProcedure

    \item[]
    \State $(c, s) \gets \Call{\texttt{majority}}{\text{the initial set of cards}}$
    \State \Return true if $c \ne \varnothing$, false otherwise
\end{algorithmic}

\subsection{Correctness}

Since $\textproc{\texttt{majority}}$ always splits $\texttt{cards}$
into smaller and smaller halves with a base case of $1$,
this algorithm clearly terminates.

For actual correctness,
the key observation is that if more than $\frac{n}{2}$ cards are identical,
then no matter how we split the array there will always be at least one
half that also has a strict majority of these identical cards.

To prove this, AFSOC we can split the cards in two halves of size $x_1$ and $x_2$
such that no half has a strict majority.
Then, $s_1 \le \frac{x_1}{2}$ and $s_2 \le \frac{x_2}{2}$,
where the $s$ variables indicate the amount of the majority card
that occurs in the half.

But adding these two inequalities gives us
\[s_1+s_2 \le \frac{x_2+x_2}{2} = \frac{n}{2}\]
which is a contradiction, since we know that the majority
card occurs strictly more than $\frac{n}{2}$ times.

Now we move on to the actual correctness of our algorithm.
It is trivial to see that the base case is true,
as any set of size $1$ has a strict majority with its only element.

As for the recurrence relation, there's two cases.
If a strict majority exists in $\texttt{cards}$,
at least one of the halves we recursed on would return a non-null value
by the observation we made.
We also perform a linear search on the other half to find the remaining
cards that are in the majority, so no card is excluded from calculation.

No strict majority existing would also force the algorithm to return $\varnothing$,
since it only returns a non-null value if it found some card
that occurred more than $\frac{n}{2}$ times.

It remains to show that this procedure makes $O(n \log n)$ queries.
Notice that at each level we make at most $O(n)$ queries
where $n$ is the size of the array passed as a parameter.
Letting $Q(n)$ be the number of queries we make, we have the relation
\[Q(n)=2Q\left(\frac{n}{2}\right)+O(n)\]
which is easily seen to be equivalent to $O(n \log n)$
as it's no different from the mergesort time complexity recurrence.

\subsection{Time Complexity}

We break problems of size $n$ into 2 problems of size $\frac{n}{2}$,
and then we combine them through a time proportional to $n$
as a linear search to get $\texttt{other}$ is required.

This recurrence relation is equivalent to merge sort and
thus our algorithm runs in $\boxed{O(n \log n)}$ time, assuming
that a query for card equality runs in constant time.

\pagebreak

\section{Hidden Surface Removal}

\subsection{Algorithm}

\begin{algorithmic}[1]
    \State Sort all the lines from lowest slope to highest

    \item[]
    \Procedure{\texttt{visible}}{\texttt{lines}}
        \If{\texttt{lines} has only one line}
            \State \Return \texttt{lines}
        \EndIf
    
        \item[]
        \LComment{Again, we can use indices instead of actually creating new arrays if need be.}
        \State Split \texttt{lines} into \texttt{half1} and \texttt{half2}
        \State $l_1 \gets \Call{\texttt{visible}}{\texttt{half1}}$
        \State $l_2 \gets \Call{\texttt{visible}}{\texttt{half2}}$
        \State $\texttt{intersection}_1 \gets$ intersection points of adjacent lines in $l_1$
        \State $\texttt{intersection}_2 \gets$ same thing for $l_2$

        \item[]
        \If{last line of $l_1$ is above any points in $\texttt{intersection}_2$}
            \State Get the last point in $\texttt{intsersection}_2$ that's below said last line
            \State \Return $l_1$ concatenated with all of $l_2$'s lines that are after that point
        \ElsIf{first line of $l_2$ is above any points in $\texttt{intersection}_1$}
            \State Get the \textit{first} point in $\texttt{intsersection}_1$ that's below said first line
            \State \Return All of $l_1$'s lines before that point concatenated with $l_2$
        \Else
            \State \Return $l_1$ concatenated with $l_2$
        \EndIf
    \EndProcedure

    \item[]
    \State \Return \Call{\texttt{visible}}{all lines}
\end{algorithmic}

It should be trivial to compute if a point is above or below a line
given that they're givein in slope-intercept form, so I'm omitting the pseudocode for that.

\pagebreak

\subsection{Correctness}

A key observation is that the set of visible lines form
a convex parabola of sorts where as we move from segment to segment
the slope always increases.

At each step we maintain this piecewise linear function
that we obtain from merging the two smaller functions.
For our base case, this invariant is obviously true
since there's only one line and it can't be blocked by anything.

Notice that since we've sorted the slopes so that the largest slope
in $l_1$ is strictly smaller than the smallest slope in $l_2$,
the only lines that can block out lines in the merged hull
can be the last line in $l_1$ and the first line in $l_2$.

As the line functions are piecewise and convex,
the two lines mentioned above will block out a continuous prefix
or suffix of the array they are interfering with.
Thus, we can see that this function correctly merges
the two sets and ultimately returns the set of all visible lines.

\subsection{Time Complexity}

Just like with merge sort we split each problem
into two smaller problems of size around $\frac{n}{2}$ and then
merge them in $O(n)$, making the total time complexity $\boxed{O(n \log n)}$.

\end{document}
