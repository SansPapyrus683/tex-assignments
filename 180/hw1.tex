\documentclass[12pt]{article}

% a template that a friend gave, it's worked well enough for me
% i have added some packages and stuff that have proved useful

\usepackage{fancyhdr}
\usepackage{tipa}
\usepackage{fontspec}
\usepackage{amsfonts}
\usepackage{enumitem}
\usepackage[margin=1in]{geometry}
\usepackage{graphicx}
\usepackage{float}
\usepackage{amsmath}
\usepackage{braket}
\usepackage{amssymb}
\usepackage{booktabs}
\usepackage{hyperref}
\usepackage{mathtools}
\usepackage{xcolor}
\usepackage{float}
\usepackage{algpseudocodex}
\usepackage{titlesec}
\usepackage{bbm}

\pagestyle{fancy}
\fancyhf{} % sets both header and footer to nothing
\lhead{Kevin Sheng}
\setmainfont{Comic Neue}
\renewcommand{\headrulewidth}{1pt}
\setlength{\headheight}{0.75in}
\setlength{\oddsidemargin}{0in}
\setlength{\evensidemargin}{0in}
\setlength{\voffset}{-.5in}
\setlength{\headsep}{10pt}
\setlength{\textwidth}{6.5in}
\setlength{\headwidth}{6.5in}
\setlength{\textheight}{8in}
\renewcommand{\headrulewidth}{0.5pt}
\renewcommand{\footrulewidth}{0.3pt}
\setlength{\textwidth}{6.5in}
\usepackage{setspace}
\usepackage{multicol}
\usepackage{float}
\setlength{\columnsep}{1cm}
\setlength\parindent{24pt}
\usepackage [english]{babel}
\usepackage [autostyle, english = american]{csquotes}
\MakeOuterQuote{"}

\setlength{\parskip}{6pt}
\setlength{\parindent}{0pt}

\titlespacing\section{0pt}{12pt plus 4pt minus 2pt}{0pt plus 2pt minus 2pt}
\titlespacing\subsection{0pt}{12pt plus 4pt minus 2pt}{0pt plus 2pt minus 2pt}
\titlespacing\subsubsection{0pt}{12pt plus 4pt minus 2pt}{0pt plus 2pt minus 2pt}

\hypersetup{colorlinks=true, urlcolor=blue}

\newcommand{\correction}[1]{\textcolor{red}{#1}}


\lhead{406-196-414}
\rhead{CS 180}

\begin{document}

\section{Proof Techniques}

\subsection*{Part A}

We start with the trivial base case $n=1$.
\[1=\frac{1 \cdot (3 - 1)}{2}\]
We now assume $\sum_{i=1}^{n} (3i-2)=\frac{n(3n-1)}{2}$ and try to prove
\[\sum_{i=1}^{n+1} (3i-2) = \frac{(n+1)(3n+2)}{2}\]
which can be done by plugging in our inducitve hypothesis like so:
\begin{align*}
    \sum_{i=1}^{n+1} (3i-2) & = (3n+1) + \sum_{i=1}^{n} (3i-2)    \\
                            & = 3n+1+\frac{n(3n-1)}{2}            \\
                            & = \frac{3n^2-n+6n+2}{2}             \\
                            & = \frac{(n+1)(3n+2)}{2}\quad\square
\end{align*}

\subsection*{Part B}

BWOC let $x$ and $y$ both be integers.
Since $x^2-y^2=1$, $(x+y)(x-y)=1$.
There's two ways for this to be true.

The first is if $x+y=x-y=1$.
This necessitates that $x=1$ and $y=0$, which contradicts that $y>0$.

The second is if $x+y>1$ and $0<x-y<1$.
The latter inequality implies $y<x<y+1$.
Since $y$ is in an integer, $x$ has to be a noninteger since it's between
two consecutive integers, contradicting our initial assumption. $\square$

\section{Order Notation}

\begin{table}[H]
    \centering
    \begin{tabular}{|c|c|c|c|c|c|c|}
        \hline
        $A$              & $B$             & $A \in o(B)$ & $A \in O(B)$ & $A \in \Theta(B)$ & $A \in \Omega(B)$ & $A \in \omega(B)$ \\
        \hline
        $\log_2 n$       & $\log_5 n$      & no           & yes          & yes               & yes               & no                \\
        \hline
        $\log \log n$    & $\sqrt{\log n}$ & yes          & yes          & no                & no                & no                \\
        \hline
        $n^{4/3} \log n$ & $n^{\log n}$    & yes          & yes          & no                & no                & no                \\
        \hline
        $2^{\log_6 n}$   & $n^6$           & yes          & yes          & no                & no                & no                \\
        \hline
        $n^3 2^n$        & $3^n$           & yes          & yes          & no                & no                & no                \\
        \hline
    \end{tabular}
\end{table}

\textcolor{red}{you stupid bitch, it was $2^{\log^6 n}$, not $2^{\log_6 n}$}

\section{Runtime Analysis}

\subsection*{Part A}

The first loop runs in proportion to $n$, and the second one is the same.
Despite incrementing $j$ until it reaches $i$, $i$ goes all the way from $1$ to $n$.
Thus, the runtime complexity is $\Theta(n^2)$

\subsection*{Part B}

If we double $i$ each time until it reaches $n$, then it would take $\log n$ iterations.
However, here we're \textit{squaring} $i$.
After $x$ iterations of the loop, $i$ becomes $2^{2^{x+1}}$.
Setting that equal to $n$ and isolating $x$, we see that the runtime
complexity is $\Theta(\log \log n)$.

\section{Smallest Element with Min Frequency}

Let $\texttt{freq}$ be a hashmap that represents the frequency of each element.
It starts with $0$ for each value.

\begin{algorithmic}[1]
    \State $\texttt{freq} \gets \{\}$
    \ForAll{$i \in \{1, \cdots , n\}$}
    \State $\texttt{freq}_{A_i} \gets \texttt{freq}_{A_i} + 1$
    \EndFor

    \State $\texttt{minFreq} \gets \infty$
    \State $\texttt{bestNum} \gets -1$
    \ForAll{\texttt{value}, \texttt{frequency} $\in$ \texttt{freq}}
    \If{$\texttt{frequency} < \texttt{minFreq}$}
    \State $\texttt{minFreq} \gets \texttt{frequency}$
    \ElsIf{$\texttt{frequency} = \texttt{minFreq}$}
    \State $\texttt{bestNum} \gets \min(\texttt{bestNum}, \texttt{value})$
    \EndIf
    \EndFor

    \State \Return $\texttt{bestNum}$
\end{algorithmic}

This algorithm first iterates through the entire array of size $n$.
It then iterates through $\texttt{freq}$, which is equivalent to going through
every distinct value in the array.
Since an array of size $n$ has at most $n$ distinct values,
the complexity of this algorithm is $O(N)$.

\section{Stable Matching Variation}

\subsection{Algorithm}

Let's suppose an array $S$ where $S_i$ is the number of slots for med students
the $i$th hospital has.
The algorithm for this problem is rather similar to the GS algorithm discussed in class:
\begin{algorithmic}[1]
    \State $\texttt{matchings} \gets \{\}$
    \Comment{Set of ordered pairs of hospitals and students.}
    \State $\texttt{matchNum} \gets \sum_{i=1}^{m} S_i$
    \Comment{$m$ is the number of hospitals.}
    \While{$|\texttt{matchings}| < \texttt{matchNum}$}
    \State Find a hospital $h$ where $|\{(a, b) \in \texttt{matchings}\,|\,a = h\}| < S_h$
    \State $s \gets$ the highest student on $h$'s preference list that hasn't been proposed to
    \item[]
    \If{$\exists (h', s') \in \texttt{matchings}$ s.t. $s'=s$}
    \If{$s$ prefers $h$ to $h'$}
    \State $\texttt{matchings} \gets (\texttt{matchings} \setminus (h', s)) \cup (h, s)$
    \EndIf
    \Else
    \State $\texttt{matchings} \gets \texttt{matchings} \cup (h, s)$
    \EndIf
    \item[]
    \State Mark that $h$ has now proposed to $s$
    \EndWhile
\end{algorithmic}

\subsection{Correctness}

We'll go through both types of instabilities and show why they can't exist
in the output of our algorithm.

\textbf{Type 1}:
Suppose that there do exist students $s$ and $s'$ and a hospital $h$
that meet the conditions described in the instability.
Since $h$ prefers $s'$ to $s$, it must have proposed to $s'$ at some point
during the algorithm since all hospitals propose in order of their preference list.
However, the fact that $s'$ is matchless implies that no hospital
ever proposed to it in the first place.
Contradiction- no instabilities of the first type can exist.

\textbf{Type 2}:
There can't be instabilities of this type for the same reason
the GS algorithm can't have instabilities of this type.
I believe the proof of this was already gone over in class.

\subsection{Complexity}

Suppose there are $n$ students and $m$ hospitals.
During each iteration of the while loop, at least one student is proposed to
by a hospital, and by design this algorithm never has that hospital propose
to the student ever again.
Thus, the while loop runs at most $nm$ times and the final complexity
of this algorithm is $O(mn)$ with appropriate data structures.

\section{Peripatetic Shipping Lines}

This problem reduces to the stable matching problem.
Have each ship's "preference list" be the ports they visit in the order of their schedule,
with ports they visit earlier being higher priority.
On the other hand, each port's "preference list" consists of the ships that visit,
which ships that visit them \textit{later} being of higher precedence.

It remains to show that the GS algorithm will output a valid matching of
ships to ports.
To do this, we'll show that any unstable pairing with the above preference
lists will result in two ships being in the same port on the same day.

Say our current set of matchings from ships to ports is $M$.
An instability occurs when $\exists (s, p), (s', p') \in M$ s.t.:
\begin{itemize}
    \item $p'$ is higher on $s$'s preference list than $p$.
    \item $s$ is higher on $p'$'s preference list than $s'$.
\end{itemize}
The meaning of the first condition is that $s$ comes to stop at $p'$
before their schedule is truncated at their assigned $p$.
The second condition means that $s$ comes to port $p'$ after $s'$
has already stopped at $p'$ for repairs.
These two conditions combined imply that $s$ and $s'$ will meet at $p'$.

With this and the knowledge that the GS algorithm will always output a matching
with no instabilities, we have shown that a set of truncations can always be found.
The algorithm is basically just the same GS algorithm we always use,
except with a bit of custom preference list creation.

\pagebreak

\section{Truthfulness in Stable Matching}

Yes, a woman is able to better her outcome if she swaps two men in her preference list.

Consider the following preference lists with $N=3$.
The men are letters, and the women are numbers:

\begin{table}[H]
    \parbox{.45\linewidth}{
        \centering
        \begin{tabular}{|c|c|c|c|}
            \hline
            A & 0 & 1 & 2 \\
            \hline
            B & 0 & 2 & 1 \\
            \hline
            C & 1 & 0 & 2 \\
            \hline
        \end{tabular}
        \caption{Men Preferences}
    }
    \hfill
    \parbox{.45\linewidth}{
        \centering
        \begin{tabular}{|c|c|c|c|}
            \hline
            0 & C & A & B \\
            \hline
            1 & A & B & C \\
            \hline
            2 & A & C & B \\
            \hline
        \end{tabular}
        \caption{Women Preferences}
    }
\end{table}

If we run the GS algorithm on this table as is, a stable matching is as follows:

\begin{table}[H]
    \centering
    \begin{tabular}{|c|c|c|c|}
        \hline
        Men & A & B & C \\
        \hline
        Women & 0 & 2 & 1 \\
        \hline

    \end{tabular}
\end{table}

However, if woman 0 decides to lie and swap the positions of men A and B on her list,
the algorithm outputs the following matching:

\begin{table}[H]
    \centering
    \begin{tabular}{|c|c|c|c|}
        \hline
        Men & A & B & C \\
        \hline
        Women & 1 & 2 & 0 \\
        \hline
    \end{tabular}
\end{table}

Woman 0 is now paired with man C, her first choice.
This inadvertently makes it so that woman 1 is now paired with man A,
\textit{her} first choice as well.

\end{document}
