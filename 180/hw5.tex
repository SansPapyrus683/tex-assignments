\documentclass[12pt]{article}

% a template that a friend gave, it's worked well enough for me
% i have added some packages and stuff that have proved useful

\usepackage{fancyhdr}
\usepackage{tipa}
\usepackage{fontspec}
\usepackage{amsfonts}
\usepackage{enumitem}
\usepackage[margin=1in]{geometry}
\usepackage{graphicx}
\usepackage{float}
\usepackage{amsmath}
\usepackage{braket}
\usepackage{amssymb}
\usepackage{booktabs}
\usepackage{hyperref}
\usepackage{mathtools}
\usepackage{xcolor}
\usepackage{float}
\usepackage{algpseudocodex}
\usepackage{titlesec}
\usepackage{bbm}

\pagestyle{fancy}
\fancyhf{} % sets both header and footer to nothing
\lhead{Kevin Sheng}
\setmainfont{Comic Neue}
\renewcommand{\headrulewidth}{1pt}
\setlength{\headheight}{0.75in}
\setlength{\oddsidemargin}{0in}
\setlength{\evensidemargin}{0in}
\setlength{\voffset}{-.5in}
\setlength{\headsep}{10pt}
\setlength{\textwidth}{6.5in}
\setlength{\headwidth}{6.5in}
\setlength{\textheight}{8in}
\renewcommand{\headrulewidth}{0.5pt}
\renewcommand{\footrulewidth}{0.3pt}
\setlength{\textwidth}{6.5in}
\usepackage{setspace}
\usepackage{multicol}
\usepackage{float}
\setlength{\columnsep}{1cm}
\setlength\parindent{24pt}
\usepackage [english]{babel}
\usepackage [autostyle, english = american]{csquotes}
\MakeOuterQuote{"}

\setlength{\parskip}{6pt}
\setlength{\parindent}{0pt}

\titlespacing\section{0pt}{12pt plus 4pt minus 2pt}{0pt plus 2pt minus 2pt}
\titlespacing\subsection{0pt}{12pt plus 4pt minus 2pt}{0pt plus 2pt minus 2pt}
\titlespacing\subsubsection{0pt}{12pt plus 4pt minus 2pt}{0pt plus 2pt minus 2pt}

\hypersetup{colorlinks=true, urlcolor=blue}

\newcommand{\correction}[1]{\textcolor{red}{#1}}


\lhead{406-196-414}
\rhead{CS 180}

\begin{document}

\section{Coin Game}

\subsection{Algorithm}

Let $\texttt{dp}[s][e]$ be the maximums core we can achieve if the starting state
consisted only of the coins from $v_s$ to $v_e$, inclusive and we go first.
For simplicity, let $\texttt{dp}[s][e]=0$ if $s>e$.
The base case is $\texttt{dp}[i][i]=v_i\ \forall i=1,\cdots, n$.

Starting from ranges of length $2$, we slowly move on to bigger and bigger ranges
with the following transition:

\begin{algorithmic}[1]
    \State $\texttt{removeLeft} \gets \left(\sum_{i=s+1}^{e} v_i\right)-\texttt{dp}[s+1][e]$
    \State $\texttt{removeRight} \gets \left(\sum_{i=s}^{e-1} v_i\right)-\texttt{dp}[s][e-1]$
    \State $\texttt{dp}[s][e] \gets \max(\texttt{removeLeft}, \texttt{removeRight})$
\end{algorithmic}

Our answer is $\texttt{dp}[1][n]$.

\subsection{Correctness}

The proof is by induction on the size of the range, namely $e-s+1$.
Our base case is trivially true since there's only one coin we can take.

Now assume we have a range of size $n$ from $v_s$ to $v_e$.
There's two moves we can make.
If we take from the start, the range reduces to $v_{s+1}, \cdots, v_e$,
while taking from the end reduces it to $v_s, \cdots, v_{e+1}$.

Notice that the turn has also changed.
Assuming both players play optimally, the second player will try to maximize
the value of the coins they take, which is exactly how $\texttt{dp}$ is defined.
The first player has to take whatever the second player leaves us with,
which is the sum of the coins in the remaining range
minus $\texttt{dp}[s+1][e]$ or $\texttt{dp}[s][e-1]$.

As the first player, we obviously want to maximize our value,
so we take the maximum of \texttt{removeLeft} and \texttt{removeRight}
given their definition above in the pseudocode. $\square$

\subsection{Time Complexity}

The amount of DP states is equivalent to the number of continuous ranges
in the array, which is on the order of $O(n^2)$.
The transition can be done in constant time, since we can compute
continuous range sum queries with prefix sums.
Past that all there is is some arithmetic and a maximum operation,
all of which is $O(1)$.

$O(n^2)$ states and an $O(1)$ transition make for a total TC of $\boxed{O(n^2)}$.

\pagebreak

\section{Number of Shortest Paths}

\subsection{Algorithm}

For simplicity, I'll assume that $E$
has both $(a, b)$ and $(b, a)$ in itself for each undirected edge.

\begin{algorithmic}[1]
    \State $\texttt{dist} \gets$ array of size $|V|$, initially all $\infty$
    \State $\texttt{dist}[v] \gets 0$
    \State $\texttt{comeFrom} \gets$ array of sets of size $|V|$
    \ForAll{$i \in \{1, \cdots, |V|-1\}$}
        \Comment{$i$ isn't used for anything}
        \ForAll{$(\texttt{from}, \texttt{to}, w) \in E$}
            \If{$\texttt{dist}[\texttt{from}]+w<\texttt{dist}[\texttt{to}]$}
                \State $\texttt{dist}[\texttt{to}]=\texttt{dist}[\texttt{from}]+w$
            \EndIf
            \If{$\texttt{dist}[\texttt{from}]+w=\texttt{dist}[\texttt{to}]$}
                \State $\texttt{comeFrom}[\texttt{to}]=\texttt{comeFrom}[\texttt{to}] \cup \{\texttt{from}\}$
            \EndIf
        \EndFor
    \EndFor

    \item[]
    \State $\texttt{numWays} \gets$ array of size $|V|$, initially all $0$
    \Comment{Holds the \# of SPs to each node}
    \State $\texttt{numWays}[v] \gets 1$
    \ForAll{$n \in $ nodes of the graph in topological order}
        \ForAll{$\texttt{from} \in \texttt{comeFrom}[n]$}
            \State $\texttt{numWays}[n]=\texttt{numWays}[n]+\texttt{numWays}[\texttt{from}]$
        \EndFor
    \EndFor

    \item[]
    \State \Return $\texttt{numWays}[w]$
\end{algorithmic}

\subsection{Correctness}

The first part of the algorithm is just
\href{https://en.wikipedia.org/wiki/Bellman%E2%80%93Ford_algorithm#Algorithm}{Bellman-Ford},
so it definitely works.

We can iterate through the nodes in topological order
for the second loop because there can't be any cycles in the directed
graph indicated by \texttt{comeFrom} where $\texttt{comeFrom}[i]$
gives all the nodes one can reach from node $i$.
If there were, that would mean a nonpositive cycle in $G$,
which the problem has stated can't happen.

With this, we can see that the DP invariant mentioned in the pseudocode comment
holds true after we're done looping over each node.
The base case is true (since there's obviously only one way to get from $v$ to $v$),
When we're at node $n$, all SPs to it must come from a node in $\texttt{comeFrom}[n]$,
and we add them all up to get the correct value for $\texttt{numWays}[n]$.

\subsection{Time Complexity}

BF takes $O(|V| \cdot |E|)$ time.
The DP part does a toposort then a for loop that goes
through each edge at most once, and that takes $O(|V|+|E|)$ time.

This algorithm's time complexity is $\boxed{O(|V| \cdot |E|)}$.

\section{Word Segmentation}

\subsection{Algorithm}

Let $\texttt{dp}[i]$ be the best quality of $y_1 \cdots y_i$,
with a base case $\texttt{dp}[0]=0$.

We can fill out this array as follows:
\begin{algorithmic}
    \ForAll{$i \in \{1, \cdots, n\}$}
        \State $\texttt{dp}[i] \gets 0$
        \ForAll{$\texttt{prev} \in \{1, \cdots, i\}$}
            \State $\texttt{new} \gets \texttt{dp}[\texttt{prev}-1]+\Call{\texttt{quality}}{y_\texttt{prev} \cdots y_i}$
            \State $\texttt{dp}[i] \gets \max(\texttt{dp}[i], \texttt{new})$
        \EndFor
    \EndFor
\end{algorithmic}

The final answer we return is $\texttt{dp}[n]$.

\subsection{Correctness}

Consider the optimal partitioning of the words among all prefixes.
We'll prove that \texttt{dp} corresponds to the best quality
given by this optimal partitioning for all $i$.

For our base case, $\texttt{dp}=0$ trivially
since there aren't any characters to partition in the first place.

At each step $i$, the final character must be part of some word segment.
To be more formal, if we know the optimal partition
has an ending segment from $j$ to $i$, its value must be
$\texttt{dp}[j-1]+\textproc{\texttt{quality}}(y_j \cdots y_i)$.
We can use $\texttt{dp}[j-1]$ because of our inductive hypothesis.
Taking the maximum over all possible $j$ has to catch the optimal last segment,
so $\texttt{dp}[i]$ must be optimal as well. $\square$

\subsection{Time Complexity}

There's exactly $n+1$ states including the base case.
When computing $\texttt{dp}[i]$, we have to check all positions
we could extend from to $i$, so our DP transition takes $O(n)$ steps.

Combining the two gives a complexity of $\boxed{O(n^2)}$.
Note that this assumes calls to $\textproc{\texttt{quality}}$ takes $O(1)$ time.

\pagebreak

\section{Signal Untangling}

\subsection{Algorithm}

For simplicity, we'll pad $x$ and $y$ with repetitions
until they're at least as long as $s$.

Our array $\texttt{dp}[a][b]$ will indicate whether
we can interleave the first $a$ characters of $x$ and the first $b$ characters
of $y$ to form a prefix of $s$.

The transition is as follows:
\begin{algorithmic}
    \State $\texttt{dp}[0][0]=\texttt{true}$
    \ForAll{$a \in \{0, \cdots, n\}$}
        \ForAll{$b \in \{0, \cdots, n\}$}
            \State $\texttt{dp}[a][b]=\texttt{false}$
            \LComment{Assumes $x$, $y$, and $s$ are 1-indexed.}
            \If{$a>0$ and $s_{a+b}=y_a$}
                \State $\texttt{dp}[a][b] \gets \texttt{dp}[a][b] \lor \texttt{dp}[a-1][b]$
            \EndIf
            \If{$b>0$ and $s_{a+b}=y_b$}
                \State $\texttt{dp}[a][b] \gets \texttt{dp}[a][b] \lor \texttt{dp}[a][b-1]$
            \EndIf
        \EndFor
    \EndFor
\end{algorithmic}

Finally we go through all pairs of $a$ and $b$ that sum to $|s|$,
checking if $\texttt{dp}[a][b]$ is true for any of them.
If there is a true one, we return true.
Otherwise, we output false.

\subsection{Correctness}

We induct on $a+b$. The base case is trivial because two empty strings make another.

Now assume $\texttt{dp}[a][b]$ is valid for all $a+b=n$
and consider the case where $a+b=n+1$.
There's two cases:
\begin{enumerate}[nosep]
    \item In the interleaving of, the last character of $s$ corresponds to the last character of $x$.
          Then we must also be able to construct a prefix from $x_1 \cdots x_{a-1}$ and $y_1 \cdots y_b$,
          which corresponds to $\texttt{dp}[a-1][b]=\texttt{true}$.

    \item Same as the last case, but now it's the last character of $y$.
          In this case, we can interleave a prefix from $x_1 \cdots x_a$ and $y_1 \cdots y_{b-1}$,
          which corresponds to $\texttt{dp}[a][b-1]=\texttt{true}$
\end{enumerate}

By our inductive hypothesis, $\texttt{dp}[a][b]$ is guranteed to have been filled out and valid,
so our algorithm won't rely on any missing states.
This completes the inductive step, and our proof is complete. $\square$

\subsection{Time Complexity}

The initial padding is $O(|s|)$ time.
For the DP, $O(|s|^2)$ states and an $O(1)$ transition make
for an $\boxed{O(|s|^2)}$ algorithm total (as quadratic dominates linear).

\pagebreak

\section{Computing Plan}

\subsection{Incorrect Greedy}

Consider $A=[10, 10]$ and $B=[1, 100]$.
The algorithm starts with $A$, which immediately casts it as suboptimal.
The optimal choice is to stick with $B$ and get $101$ steps.

\subsection{Correct DP}

\subsubsection{Algorithm}

Let $\texttt{dp}[i][m]$ be the most steps given that you're at minute $i$
and have to end the computation on machine $m$.
$\texttt{step}[i][m]$ will contain the move you made to get to that point.

\begin{algorithmic}[1]
    \State $\texttt{dp}[1][a] \gets A[1]$
    \State $\texttt{dp}[1][b] \gets B[1]$
    \State $\texttt{step}[1][a] \gets \varnothing$
    \State $\texttt{step}[1][b] \gets \varnothing$

    \item[]
    \ForAll{$i \in \{2, \cdots, n\}$}
        \ForAll{$m \in \{A, B\}$}
            \State $\texttt{stay} \gets m[i]+\texttt{dp}[i-1][m]$
            \State $\texttt{switch} \gets 0$
            \If{$i>2$}
                \State $\texttt{switch} \gets m[i]+\texttt{dp}[i-2][\text{the other machine}]$
            \EndIf

            \item[]
            \If{$\texttt{stay}>\texttt{switch}$}
                \State $\texttt{dp}[i][m] \gets \texttt{stay}$
                \State $\texttt{step}[i][m] \gets \text{stay}$
            \Else
                \State $\texttt{dp}[i][m] \gets \texttt{switch}$
                \State $\texttt{step}[i][m] \gets \text{switch}$
            \EndIf
        \EndFor
    \EndFor

    \item[]
    \State Backtrack from $\max(\texttt{dp}[n][a], \texttt{dp}[n][b])$ and construct movelist as you go along
\end{algorithmic}

The problem asks for what we should do at any point in time.
While I feel like that's redundant since we can uniquely reconstruct
a series of steps just from the instructions "stay at the current machine"
or "move to the next machine", it's possible to do this in $O(n)$ time,
just really tedious and implementation-heavy.

\pagebreak

\subsubsection{Correctness}

The base cases only allow for one option,
since if we end at any machine on minute $1$ we'd have to
have started there in the first place.

Inducting from there, we see that to end at machine $m$ at step $i$,
there's only two ways we could've gotten there:
\begin{enumerate}[nosep]
    \item We were at $m$ at step $i-1$ and remained at the same machine.
          This case is simple- our new value would just be $dp[i-1][m]+m[i]$.
    \item We switched to it from the other machine.
          In this case we have to transition from $dp[i-2][\text{the other machine}]$,
          since minute $i-1$ was spent moving.
          This makes our value $dp[i-2][\text{the other machine}]+m[i]$.
\end{enumerate}
To get the actual optimal value for the current state, we just take the larger of the two.

Since we also log which one was bigger for each step,
we can backtrack along the auxiliary \texttt{step} array to get
the actual operations we have to do to achieve the optimal schedule.

\subsubsection{Time Complexity}

$n-1$ iterations of a for loop with a constant amount of operations
within means that our time complexity is $\boxed{O(n)}$.

\end{document}
