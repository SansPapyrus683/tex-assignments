\documentclass[12pt]{article}

% a template that a friend gave, it's worked well enough for me
% i have added some packages and stuff that have proved useful

\usepackage{fancyhdr}
\usepackage{tipa}
\usepackage{fontspec}
\usepackage{amsfonts}
\usepackage{enumitem}
\usepackage[margin=1in]{geometry}
\usepackage{graphicx}
\usepackage{float}
\usepackage{amsmath}
\usepackage{braket}
\usepackage{amssymb}
\usepackage{booktabs}
\usepackage{hyperref}
\usepackage{mathtools}
\usepackage{xcolor}
\usepackage{float}
\usepackage{algpseudocodex}
\usepackage{titlesec}
\usepackage{bbm}

\pagestyle{fancy}
\fancyhf{} % sets both header and footer to nothing
\lhead{Kevin Sheng}
\setmainfont{Comic Neue}
\renewcommand{\headrulewidth}{1pt}
\setlength{\headheight}{0.75in}
\setlength{\oddsidemargin}{0in}
\setlength{\evensidemargin}{0in}
\setlength{\voffset}{-.5in}
\setlength{\headsep}{10pt}
\setlength{\textwidth}{6.5in}
\setlength{\headwidth}{6.5in}
\setlength{\textheight}{8in}
\renewcommand{\headrulewidth}{0.5pt}
\renewcommand{\footrulewidth}{0.3pt}
\setlength{\textwidth}{6.5in}
\usepackage{setspace}
\usepackage{multicol}
\usepackage{float}
\setlength{\columnsep}{1cm}
\setlength\parindent{24pt}
\usepackage [english]{babel}
\usepackage [autostyle, english = american]{csquotes}
\MakeOuterQuote{"}

\setlength{\parskip}{6pt}
\setlength{\parindent}{0pt}

\titlespacing\section{0pt}{12pt plus 4pt minus 2pt}{0pt plus 2pt minus 2pt}
\titlespacing\subsection{0pt}{12pt plus 4pt minus 2pt}{0pt plus 2pt minus 2pt}
\titlespacing\subsubsection{0pt}{12pt plus 4pt minus 2pt}{0pt plus 2pt minus 2pt}

\hypersetup{colorlinks=true, urlcolor=blue}

\newcommand{\correction}[1]{\textcolor{red}{#1}}


\begin{document}
\begin{enumerate}
    \item \begin{enumerate}
              \item It means that there's exactly one set of values $a_i$ can take on s.t. $\sum_{i=1}^n a_i v_i=s$.
              \item By the definition of the span, we know that there is
                    \textit{at least one} set of values for $a_i$ s.t. the given equation is satisfied.
                    It remains to prove that this solution is unique.

                    Suppose for the sake of contradiction we have another two sets of values $a_i$ and $a_i'$
                    that differ for at least one $i$, and that both give a linear combination that results in $s$.
                    If this is the case, we can write
                    \[\sum_{i=1}^n a_i v_i=\sum_{i=1}^n a_i' v_i\]
                    With some algebraic manipulation, we can then turn this into
                    \[\sum_{i=1}^n a_i v_i-\sum_{i=1}^n a_i' v_i=\sum_{i=1}^n (a_i-a_i')v_i=0\]
                    Since we know $\exists i: a_i' \ne a_i$, there is at least one nonzero scalar in that summation.
                    However, by the definition of linear independence, we know that the only solution
                    to $\sum_{i=1}^n a_i v_i=0$ is $a_i=0$. Contradiction.

                    Thus, there must be \textit{exactly one} solution to $s=\sum_{i=1}^n a_i v_i$. $\square$
              \item Let's make our mapping from $F^n$ to $\text{span}(S)$ as the following:
              \[\begin{bmatrix}
                a_1 \\ a_2 \\ \vdots \\ a_n
              \end{bmatrix} \rightarrow \sum_{i=1}^n a_i v_i\]
              This function is clearly surjective, since we can collect all the coefficients
              of any summation on the RHS and collect them into a vector to get the inverse mapping.

              It's also injective, because we know that any $v \in V$ has only one way to be
              expressed in terms a linear combination of the vectors in $S$.
              Thus, if we change any $a_i$ then $\sum_{i=1}^n a_i v_i$ would also change.
          \end{enumerate}
    \item \begin{enumerate}
              \item \textbf{Base case $n=1$}: \[\sum_{k=1}^1 (2k-1)=1=1^2\]
                    Our inductive hypothesis and the statement we wish to prove are the following:
                    \[\sum_{k=1}^n (2k-1)=n^2 \rightarrow \sum_{k=1}^{n+1} (2k-1)=(n+1)^2\]
                    \textbf{Inductive step:}
                    We can prove the inductive step by adding $2(n+1)-1$ to both sides of the hypothesis:
                    \begin{gather*}
                        \sum_{k=1}^n (2k-1)+2(n+1)-1=n^2+2(n+1)-1 \\
                        \sum_{k=1}^{n+1} (2k-1)=n^2+2n+1 \\
                        \sum_{k=1}^{n+1} (2k-1)=(n+1)^2\quad\square
                    \end{gather*}
              \item \textbf{Base case $n=1$:} \\
              If there's only one set the formula reduces to a trivial equality:
              \[T \cap S_1=T \cap S_1\]
              Our inductive hypothesis is then that
              \[T \cap \left(\bigcup_{i=1}^n S_i\right)=\bigcup_{i=1}^n (T \cap S_i)\]
              and we wish to prove
              \[T \cap \left(\bigcup_{i=1}^{n+1} S_i\right)=\bigcup_{i=1}^{n+1} (T \cap S_i)\]

              \textbf{Inductive step:} \\
              Let's expand the sides of the formula to make the $n+1$th set explicit:
              \[T \cap \left(S_{n+1} \cup \bigcup_{i=1}^n S_i\right)=(T \cap S_{n+1}) \cup \bigcup_{i=1}^n (T \cap S_i)\]
              To prove this, we'll have to use that for any three sets $A,B$, and $C$,
              \[A \cap (B \cup C)=(A \cap B) \cup (A \cap C)\]
              Thus, the LHS of the expanded form can be transformed like so:
              \[(T \cap S_{n+1}) \cup \left(T \cap \bigcup_{i=1}^n S_i\right)=(T \cap S_{n+1}) \cup \bigcup_{i=1}^n (T \cap S_i)\]
              The first terms are equal by inspection, and the second terms are equal by our inductive hypothesis. $\square$
          \end{enumerate}
\end{enumerate}
\end{document}
