\documentclass[12pt]{article}

% a template that a friend gave, it's worked well enough for me
% i have added some packages and stuff that have proved useful

\usepackage{fancyhdr}
\usepackage{tipa}
\usepackage{fontspec}
\usepackage{amsfonts}
\usepackage{enumitem}
\usepackage[margin=1in]{geometry}
\usepackage{graphicx}
\usepackage{float}
\usepackage{amsmath}
\usepackage{braket}
\usepackage{amssymb}
\usepackage{booktabs}
\usepackage{hyperref}
\usepackage{mathtools}
\usepackage{xcolor}
\usepackage{float}
\usepackage{algpseudocodex}
\usepackage{titlesec}
\usepackage{bbm}

\pagestyle{fancy}
\fancyhf{} % sets both header and footer to nothing
\lhead{Kevin Sheng}
\setmainfont{Comic Neue}
\renewcommand{\headrulewidth}{1pt}
\setlength{\headheight}{0.75in}
\setlength{\oddsidemargin}{0in}
\setlength{\evensidemargin}{0in}
\setlength{\voffset}{-.5in}
\setlength{\headsep}{10pt}
\setlength{\textwidth}{6.5in}
\setlength{\headwidth}{6.5in}
\setlength{\textheight}{8in}
\renewcommand{\headrulewidth}{0.5pt}
\renewcommand{\footrulewidth}{0.3pt}
\setlength{\textwidth}{6.5in}
\usepackage{setspace}
\usepackage{multicol}
\usepackage{float}
\setlength{\columnsep}{1cm}
\setlength\parindent{24pt}
\usepackage [english]{babel}
\usepackage [autostyle, english = american]{csquotes}
\MakeOuterQuote{"}

\setlength{\parskip}{6pt}
\setlength{\parindent}{0pt}

\titlespacing\section{0pt}{12pt plus 4pt minus 2pt}{0pt plus 2pt minus 2pt}
\titlespacing\subsection{0pt}{12pt plus 4pt minus 2pt}{0pt plus 2pt minus 2pt}
\titlespacing\subsubsection{0pt}{12pt plus 4pt minus 2pt}{0pt plus 2pt minus 2pt}

\hypersetup{colorlinks=true, urlcolor=blue}

\newcommand{\correction}[1]{\textcolor{red}{#1}}


\begin{document}
\begin{enumerate}
      \item \begin{enumerate}
                  \item \begin{enumerate}
                              \item If $\beta_i$ is the basis for $E_{\lambda_i}$, then $\text{span}(\beta_i)=E_{\lambda_i}$.
                                    Then, \begin{align*}
                                          V & = \sum_{i=1}^{r} E_{\lambda_i}                      \\
                                            & = \sum_{i=1}^{r} \text{span}(\beta_i)               \\
                                            & = \text{span}\left(\bigcup_{i=1}^{r} \beta_i\right)
                                    \end{align*}
                              \item Each $v$ in $\bigcup_{i=1}^{r} \beta_i$ must be a member of one of the $\beta_i$s, which then
                                    means $v \in E_{\lambda_i}$, thus implying that $v$ is an eigenvector with eigenvalue $\lambda_i$.
                              \item By the replacement theorem, we know that any spanning set of a vector space contains a basis for it as well.
                                    $\bigcup_{i=1}^{r} \beta_i$ is a spanning set, and so it must contain a basis as well.
                                    Since every vector in this set is also an eigenvector, this basis must also consist of only eigenvectors
                                    and is thus an eigenbasis.
                        \end{enumerate}
                  \item \begin{enumerate}
                              \item $v \in \beta \rightarrow T(v)=\lambda_i v \therefore v \in E_{\lambda_i}$.
                              \item As we have a basis that consists completely of eigenvectors,
                                    we can construct any $v \in V$ from a linear combination of eigenvectors as well.
                                    Since all these eigenvectors must lie in some eigen\textit{space},
                                    $v \in \sum_{i=1}^{r} E_{\lambda_i}$ as well.
                              \item Any sum of subspaces must be a subspace as well, which by definition if a subset of $V$.
                                    Since all $E_{\lambda_i}$ are subspaces, their sum is a subspace and thus $\sum_{i=1}^{r} E_{\lambda_r} \subset V$.
                                    We've proved that these two sets are contained in each other, so $\sum_{i=1}^{r} E_{\lambda_r} = V$.
                        \end{enumerate}
            \end{enumerate}

            \pagebreak

      \item For these problems, I'll let $x=a+bi$ and $y=c+di$.
            \begin{enumerate}
                  \item \begin{align*}
                              \overline{x+y} & = \overline{(a+bi)+(c+di)}  & \overline{xy} & = \overline{(a+bi)(c+di)}        \\
                                             & =(a+c)-(b+d)i               &               & = \overline{ac-bd+i(bc+ad)}      \\
                                             & = (a-bi)+(c-di)             &               & = (ac-bd)-i(bc+ad)               \\
                                             & = \overline{x}+\overline{y} &               & =(a-bi)(c-di)                    \\
                                             &                             &               & =\overline{x} \cdot \overline{y}
                        \end{align*}
                  \item $a$ is already taken, so I'll let the scalar be $c$ instead.
                        \begin{align*}
                              \overline{cx} & = \overline{ca+cbi} & \overline{\overline{x}} & = \overline{\overline{a+bi}} \\
                                            & = ca-cbi            &                         & = \overline{a-bi}            \\
                                            & = c \cdot (a-bi)    &                         & = a+bi                       \\
                                            & = c\overline{x}     &                         & = x
                        \end{align*}
                  \item \begin{align*}
                              x\overline{x} & = (a+bi)(a-bi)      \\
                                            & =a^2-b^2i^2-abi+abi \\
                                            & = a^2+b^2
                        \end{align*}
                        $a, b \in \mathbb{R}$, so $x\overline{x}=a^2+b^2 \in \mathbb{R}$.
                        Also, $x^2, y^2 \ge 0$, so $x^2+y^2 \ge 0$ as well.

                        If $x=0+0i$, then $x\overline{x}=0^2+0^2=0$.
                        On the other hand, if $x\overline{x}=0$, then $a^2+b^2=0$.
                        Both terms have to be positive since they're squares,
                        and so the only way they can sum to $0$ is if they're both $0$
                        in the first place, i.e. $a=b=0$. \label{list:2c}

                  \item \begin{enumerate}
                              \item \begin{align*}
                                          \braket{x+y,z} & = \sum_{i=1}^{n} (x_i+y_i)\overline{z_i}                                \\
                                                         & = \sum_{i=1}^{n} x_i \overline{z_i} + y_i \overline{z_i}                \\
                                                         & = \sum_{i=1}^{n} x_i \overline{z_i} + \sum_{i=1}^{n} y_i \overline{z_i} \\
                                                         & = \braket{x,z}+\braket{y,z}
                                    \end{align*}
                              \item \begin{align*}
                                          \braket{\lambda x, y} & = \sum_{i=1}^{n} \lambda x_i \overline{y_i} \\
                                                                & = \lambda \sum_{i=1}^{n} x_i \overline{y_i} \\
                                                                & = \lambda \braket{x, y}
                                    \end{align*}
                              \item \begin{align*}
                                          \overline{\braket{x, y}} & = \overline{\sum_{i=1}^{n} x_i \overline{y_i}} \\
                                                                   & = \sum_{i=1}^{n} \overline{x_i \overline{y_i}} \\
                                                                   & = \sum_{i=1}^{n} \overline{x_i} y_i            \\
                                                                   & = \sum_{i=1}^{n} y_i \overline{x_i}            \\
                                                                   & = \braket{y, x}
                                    \end{align*}
                              \item \[\braket{x, x}=\sum_{i=1}^{n} x_i \overline{x_i}\]
                                    In \ref{list:2c}, we proved that $x\overline{x}$ is a nonnegative real number
                                    and is $0$ iff $x=\vec{0}$.
                                    Since it's given that $x \ne \vec{0}$, $x_i\overline{x_i}>0$ and their sum is too.
                        \end{enumerate}
            \end{enumerate}
\end{enumerate}
\end{document}
