\documentclass[12pt]{article}

% a template that a friend gave, it's worked well enough for me
% i have added some packages and stuff that have proved useful

\usepackage{fancyhdr}
\usepackage{tipa}
\usepackage{fontspec}
\usepackage{amsfonts}
\usepackage{enumitem}
\usepackage[margin=1in]{geometry}
\usepackage{graphicx}
\usepackage{float}
\usepackage{amsmath}
\usepackage{braket}
\usepackage{amssymb}
\usepackage{booktabs}
\usepackage{hyperref}
\usepackage{mathtools}
\usepackage{xcolor}
\usepackage{float}
\usepackage{algpseudocodex}
\usepackage{titlesec}
\usepackage{bbm}

\pagestyle{fancy}
\fancyhf{} % sets both header and footer to nothing
\lhead{Kevin Sheng}
\setmainfont{Comic Neue}
\renewcommand{\headrulewidth}{1pt}
\setlength{\headheight}{0.75in}
\setlength{\oddsidemargin}{0in}
\setlength{\evensidemargin}{0in}
\setlength{\voffset}{-.5in}
\setlength{\headsep}{10pt}
\setlength{\textwidth}{6.5in}
\setlength{\headwidth}{6.5in}
\setlength{\textheight}{8in}
\renewcommand{\headrulewidth}{0.5pt}
\renewcommand{\footrulewidth}{0.3pt}
\setlength{\textwidth}{6.5in}
\usepackage{setspace}
\usepackage{multicol}
\usepackage{float}
\setlength{\columnsep}{1cm}
\setlength\parindent{24pt}
\usepackage [english]{babel}
\usepackage [autostyle, english = american]{csquotes}
\MakeOuterQuote{"}

\setlength{\parskip}{6pt}
\setlength{\parindent}{0pt}

\titlespacing\section{0pt}{12pt plus 4pt minus 2pt}{0pt plus 2pt minus 2pt}
\titlespacing\subsection{0pt}{12pt plus 4pt minus 2pt}{0pt plus 2pt minus 2pt}
\titlespacing\subsubsection{0pt}{12pt plus 4pt minus 2pt}{0pt plus 2pt minus 2pt}

\hypersetup{colorlinks=true, urlcolor=blue}

\newcommand{\correction}[1]{\textcolor{red}{#1}}


\begin{document}
\begin{enumerate}
    \item \begin{enumerate}
              \item The system has its equilibrium points when $y=0$ and $-\cos x-\frac{y}{2}=0$.
                    Solving this system gives us our set of equilibrium points
                    $\{\left(\frac{\pi}{2}+n\pi, 0\right): n \in \mathbb{Z}\}$.
              \item The Jacobian is
                    \[\begin{bmatrix}
                            0      & 1    \\
                            \sin x & -0.5
                        \end{bmatrix}\]
                    Notice that when $n$ is even, $\sin \left(\frac{\pi}{2}+n\pi\right)=1$,
                    and when $n$ is odd, the expression evaluates to $-1$ instead.
                    Thus, all our equilibrium points fall into one of two cases.

                    \textbf{$n$ even:} \\
                    When $n$ is even, the trace of the matrix is $0=0.5=-0.5$
                    and the determinant is $0 \cdot -0.5 - 1 \cdot 1=-1$.
                    From this, we can see that the equilibrium point is
                    an unstable saddle when $n$ is even.

                    \textbf{$n$ odd:} \\
                    OTOH, when $n$ is odd, the while the trace is still $-0.5$,
                    the determinant now becomes $0 \cdot -0.5 - 1 \cdot -1=0.5$.
                    Plotting this on the trace-determinant plane,
                    we see that the equilibrium point is
                    a stable spiral sink when $n$ is odd.
          \end{enumerate}
    \item First we solve the system
          \begin{gather*}
              x+y=0 \\
              y(1-x^2)=0
          \end{gather*}
          The second equation is true if and only if one of the coefficients
          is $0$, which means that either $y=0$, or $x=\pm 1$.
          Plugging these three cases back into the first equation gives us
          the equilibrium points $(0, 0)$, $(1, -1)$, and $(-1, 1)$.

          The Jacobian of this system $J(\vec{x})$ is
          \[\begin{bmatrix}
                  1    & 1     \\
                  -2xy & 1-x^2
              \end{bmatrix}\]

          We then plug in each equilibrium point and see what type they are.
          \begin{itemize}
              \item $(0, 0)$: $\det J((0, 0)) = 1$, while $\text{tr} J((0, 0))=2$.
                    Since the discriminant is zero and the trace is positive, this point is an unstable improper node.
              \item $(1, -1)$: $\det J((1, -1))=-2$ and $\text{tr} J((1, -1)) = 1$.
                    Since the determinant is negative, this point is an unstable saddle.
              \item $(-1, 1)$: Since the resulting Jacobian is the same as the previous point,
                    this point is an unstable saddle as well.
          \end{itemize}
    \item \begin{enumerate}
              \item Adding all three derivatives together, we see that
                    \[S'(t)+I'(t)+R'(t)=-\frac{\beta S(t)I(t)}{N}+\frac{\beta S(t)I(t)}{N}-\alpha I(t)+\alpha I(t)=0\]
                    Since the derivative is always $0$, we can infer that the value of $S(t)+I(t)+R(t)$ stays constant.
              \item To do this, we solve the system
                    \begin{gather*}
                        -\frac{\beta S(t) I(t)}{ N} = 0 \\
                        \frac{\beta S(t) I(t)}{ N} - \alpha I(t) = 0
                    \end{gather*}
                    Adding the first to the second, we get that $\alpha I(t)=0$ and by extension $I(t)=0$.
                    This alone satisfies both equalities, so $S \in [0, N]$.

                    Thus, our equilibrium points are where $I(t)=0$ and $0 \le S(t) \le N$.
              \item The Jacobian is
                    \begin{align*}
                        \begin{bmatrix}
                            \frac{\partial S'}{\partial S} & \frac{\partial S'}{\partial I} \\
                            \frac{\partial I'}{\partial S} & \frac{\partial I'}{\partial I}
                        \end{bmatrix}
                         & =\begin{bmatrix}
                                -\frac{\beta I(t)}{N} & -\frac{\beta S(t)}{N}       \\
                                \frac{\beta I(t)}{N}  & \frac{\beta S(t)}{N}-\alpha
                            \end{bmatrix} \\
                         & =\begin{bmatrix}
                                0 & -\frac{\beta S(t)}{N}       \\
                                0 & \frac{\beta S(t)}{N}-\alpha
                            \end{bmatrix}
                    \end{align*}
                    For there to be a positive eigenvalue, $\frac{\beta S(t)}{N}-\alpha > 0$.
                    Isolating $S(t)$, this inequality goes to $\boxed{S > \frac{N\alpha}{\beta}}$.
              \item The lowest $S$ can be while having the entire system
                    tend toward the equilibrium point is $\frac{N\alpha}{\beta}$.
                    Thus, \[\lim_{t \to \infty} \frac{R(t)}{N}=1-\lim_{t \to \infty} \frac{S(t)}{N}=\boxed{\frac{\alpha}{\beta}}\]
          \end{enumerate}
\end{enumerate}
\end{document}
