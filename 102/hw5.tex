\documentclass[12pt]{article}

% a template that a friend gave, it's worked well enough for me
% i have added some packages and stuff that have proved useful

\usepackage{fancyhdr}
\usepackage{tipa}
\usepackage{fontspec}
\usepackage{amsfonts}
\usepackage{enumitem}
\usepackage[margin=1in]{geometry}
\usepackage{graphicx}
\usepackage{float}
\usepackage{amsmath}
\usepackage{braket}
\usepackage{amssymb}
\usepackage{booktabs}
\usepackage{hyperref}
\usepackage{mathtools}
\usepackage{xcolor}
\usepackage{float}
\usepackage{algpseudocodex}
\usepackage{titlesec}
\usepackage{bbm}

\pagestyle{fancy}
\fancyhf{} % sets both header and footer to nothing
\lhead{Kevin Sheng}
\setmainfont{Comic Neue}
\renewcommand{\headrulewidth}{1pt}
\setlength{\headheight}{0.75in}
\setlength{\oddsidemargin}{0in}
\setlength{\evensidemargin}{0in}
\setlength{\voffset}{-.5in}
\setlength{\headsep}{10pt}
\setlength{\textwidth}{6.5in}
\setlength{\headwidth}{6.5in}
\setlength{\textheight}{8in}
\renewcommand{\headrulewidth}{0.5pt}
\renewcommand{\footrulewidth}{0.3pt}
\setlength{\textwidth}{6.5in}
\usepackage{setspace}
\usepackage{multicol}
\usepackage{float}
\setlength{\columnsep}{1cm}
\setlength\parindent{24pt}
\usepackage [english]{babel}
\usepackage [autostyle, english = american]{csquotes}
\MakeOuterQuote{"}

\setlength{\parskip}{6pt}
\setlength{\parindent}{0pt}

\titlespacing\section{0pt}{12pt plus 4pt minus 2pt}{0pt plus 2pt minus 2pt}
\titlespacing\subsection{0pt}{12pt plus 4pt minus 2pt}{0pt plus 2pt minus 2pt}
\titlespacing\subsubsection{0pt}{12pt plus 4pt minus 2pt}{0pt plus 2pt minus 2pt}

\hypersetup{colorlinks=true, urlcolor=blue}

\newcommand{\correction}[1]{\textcolor{red}{#1}}


\rhead{ECE 102}

\newcommand{\rect}{\operatorname{rect}}
\newcommand{\sinc}{\operatorname{sinc}}

\begin{document}

\begin{enumerate}
      \item \begin{enumerate}
                  \item If $f(t)$ is purely imaginary, then $\frac{f(t)}{i} \in \mathbb{R}$.
                        Let's call this new function $g(t)$.
                        \begin{gather*}
                              \Re(c_k)=\frac{1}{T_0} \int_{-\infty}^{\infty} g(t)\sin(k\omega_0 t)\,dt \\
                              \Im(c_k)=\frac{1}{T_0} \int_{-\infty}^{\infty} g(t)\cos(k\omega_0 t)\,dt
                        \end{gather*}

                        Because of that,
                        \begin{itemize}
                              \item $\Re(c_k)=-\Re(c_{-k})$, since sine is odd.
                              \item $\Im(c_k)=\Im(c_{-k})$, since cosine is odd.
                              \item $c_k^* = -c_{-k}$, since $c_k$ is just $c_{-k}$ reflected across the y-axis.
                              \item $|c_k|=|c_{-k}|$ by the same reason.
                              \item $\angle c_k=\pi - \angle c_k$ because ditto.
                        \end{itemize}
                  \item From the get-go, we know $x$ takes the form of
                        \[x(t)=\sum_{k=-1}^{1} a_k \exp\left(jk\omega_0t\right)\]
                        where $a_0=4$ and $\omega_0=\frac{\pi}{8}$.
                        It remains to figure out $a_1$ and $a_{-1}$.

                        Since $x(t)$ is even, we have
                        \[a_{-1}e^{-j\omega_0t}+4+a_1e^{j\omega_0t}=a_{-1}e^{j\omega_0t}+4+a_1e^{-j\omega_0t}\]
                        Expanding with Euler's formula, we get $a_1=a_{-1}$, which from now on will be $a$.

                        The only piece of information we haven't used yet is the integral:
                        \begin{gather*}
                              \int_{0}^{4} \left|x(t)-\left(\sum_{l=0}^{\infty} a_{2l}\right)\right|^2\,dt  = 2 \\
                              \int_{0}^{4} (x(t)-4)^2\,dt = 2 \\
                              \int_{0}^{4} \left(a(\exp(j\omega_0t)+\exp(-j\omega_0t))\right)^2\,dt = 2 \\
                              a^2=2 \div \int_{0}^{4} \left(\exp(j\omega_0t)+\exp(-j\omega_0t)\right)^2\,dt \\
                              a^2=2 \div \int_{0}^{4} 4\cos^2(\omega_0 t)\,dt
                        \end{gather*}

                        Evaluating that and getting $a=\frac{1}{2}$, we get \[x(t)=4+\cos\left(\frac{\pi}{8}t\right)\]
            \end{enumerate}
      \item \begin{enumerate}
                  \item \begin{enumerate}
                              \item a, d, and e.
                              \item f only.
                              \item c and e.
                              \item a and b.
                              \item e only.
                              \item f only.
                              \item d only.
                              \item b only.
                        \end{enumerate}
                  \item \begin{enumerate}
                              \item Convolution in the time domain is multiplication in the frequency domain,
                                    so this is equivalent to multiplying a real even signal with an imaginary odd signal,
                                    which clearly produces an \boxed{\text{odd}} signal. (true)
                              \item $F(j\omega)$ is real and odd for both signals,
                                    and multiplying an odd signal by its reversed version
                                    produces an \boxed{\text{even}} signal. (false)
                        \end{enumerate}
                  \item \begin{enumerate}
                              \item \[\begin{aligned}
                                                X(j\omega)
                                                 & = \int_{-\infty}^{\infty} x(t)e^{-j\omega t}\,dt                             \\
                                                 & = \int_{0}^{\infty} x(t)e^{-j\omega t}+x^*(t)e^{j\omega t}\,dt               \\
                                                 & = \int_{0}^{\infty} \cos(\omega t)(x(t)-x^*(t))+j\sin(\omega t)(x^*(t)-x(t)) \\
                                                 & = \int_{0}^{\infty} 2\Re(x(t))\cos(\omega t)+2\Im(x(t))\sin(\omega t)\,dt    \\
                                                 & \in \mathbb{R}\quad\square
                                          \end{aligned}\]
                              \item Proof for even component:
                                    \begin{align*}
                                          X_e(j\omega)
                                           & = \int_{-\infty}^{\infty} x_e(t)e^{-j\omega t}\,dt                                                                     \\
                                           & = \int_{-\infty}^{\infty} \frac{x(t)+x(-t)}{2}e^{-j\omega t}\,dt                                                       \\
                                           & = \left(\int_{-\infty}^{\infty} x(t)e^{-j\omega t}\,dt + \int_{-\infty}^{\infty} x(-t)e^{-j\omega t}\,dt\right) \div 2 \\
                                           & = \frac{X(j\omega)+X(-j\omega)}{2}                                                                                     \\
                                           & = \frac{X(j\omega)+X^*(j\omega)}{2}                                                                                    \\
                                           & = \Re(X(j\omega))\quad\square
                                    \end{align*}

                                    For the odd component, we can do the same steps to obtain
                                    \[X_o(j\omega)=\frac{X(j\omega)-X(-j\omega)}{2}=j\Im(X(j\omega))\quad\square\]
                        \end{enumerate}
            \end{enumerate}
      \item \begin{enumerate}
                  \item The box is doing an FT on the input function.
                  \item \begin{enumerate}
                              \item $X(0)$ is the area of the signal, which is \boxed{5}.
                              \item Taking the area is equivalent to convolving the FT with $F(t)=1$.
                                    The inverse FT of $F$ is \[f(t)=\frac{1}{2\pi}\int_{-\infty}^{\infty} e^{j\omega t}\,d\omega =\frac{\delta(t)}{2\pi}\]

                                    Then, by the multiplication principle,
                                    \begin{align*}
                                          (F * X)(t)
                                           & = 2\pi\mathcal{F}[x(t)f(t)]                              \\
                                           & = 2\pi\int_{-\infty}^{\infty} x(t)f(t)e^{-j\omega t}\,dt \\
                                           & = x(0)                                                   \\
                                           & = \boxed{1}
                                    \end{align*}
                              \item By Parseval's Theorem,
                                    \begin{align*}
                                          \int_{-\infty}^{\infty} |X(j\omega)|^2\,d\omega
                                           & = 2\pi \int_{-\infty}^{\infty} |x(t)|^2\,dt                \\
                                           & = 2\pi \cdot 2 \cdot \left(\frac{1}{3}+\frac{26}{6}\right) \\
                                           & = \boxed{\frac{56\pi}{3}}
                                    \end{align*}
                              \item $x(t+1)+x(-t+1)$ is real and even, so $X(j\omega)$ is also real and even. $\Im(X(j\omega))=\boxed{0}$.
                              \item $\mathcal{F}[f(at-b)]=\frac{1}{|a|}e^{-j\omega\frac{b}{a}}\mathcal{F}[f(t)]$, so
                                    we should input $x(t)=\boxed{x\left(5t-10j\right)}$
                        \end{enumerate}
                  \item $x^4$ doesn't have an FT.
                        I could've at least plugged in the integral into Wolfram Alpha to see that the
                        integral for the FT had no chance of converging.
            \end{enumerate}
      \item \begin{enumerate}
                  \item \begin{enumerate}
                              \item $\rect$ is an even function, so $\rect\left(\frac{-t-3}{2}\right)=\rect\left(\frac{t+3}{2}\right)$.
                                    \begin{gather*}
                                          \mathcal{F}\left[\rect\left(\frac{t}{2}\right)\right] = 2\sinc\left(\frac{\omega}{\pi}\right) \\
                                          \mathcal{F}\left[\rect\left(\frac{t+3}{2}\right)\right] = 2e^{3j\omega}\sinc\left(\frac{\omega}{\pi}\right) \\
                                          \begin{align*}
                                                \mathcal{F}[x_1(t)]
                                                 & = 2 \cdot \frac{1}{2}\left(2e^{3j(\omega-10\pi)}\sinc\left(\frac{\omega}{\pi}-10\right)+2e^{3j(\omega+10\pi)}\sinc\left(\frac{\omega}{\pi}+10\right)\right) \\
                                                 & = 2e^{3j(\omega-10\pi)}\sinc\left(\frac{\omega}{\pi}-10\right)+2e^{3j(\omega+10\pi)}\sinc\left(\frac{\omega}{\pi}+10\right)                                 \\
                                                 & = \boxed{2e^{3j\omega}\left(e^{-30j\pi}\sinc\left(\frac{\omega}{\pi}-10\right)+e^{30j\pi}\sinc\left(\frac{\omega}{\pi}+10\right)\right)}
                                          \end{align*}
                                    \end{gather*}
                              \item \[\begin{aligned}
                                                \mathcal{F}[x_2(t)]
                                                 & = \int_{-\infty}^{\infty} \exp(t(2+3j))u(-t+1)\exp(-j\omega t)\,dt                  \\
                                                 & = \int_{-\infty}^{1} \exp(t(2+3j-j\omega))\,dt                                      \\
                                                 & = \left.\left(\frac{1}{2+3j-j\omega}\exp(t(2+3j-j\omega))\right)\right|^1_{-\infty} \\
                                                 & = \boxed{\frac{\exp(2+3j-j\omega)}{2+3j-j\omega}}
                                          \end{aligned}\]
                              \item Notice that the piecewise part can be simplified to multiplication by $\rect\left(\frac{t}{2}\right)$.
                                    \begin{align*}
                                          x_3(t)
                                           & = \rect\left(\frac{t}{2}\right)(1+\cos(\pi t))                                                                                                 \\
                                           & = \rect\left(\frac{t}{2}\right)+\rect\left(\frac{t}{2}\right)\cos(\pi t)                                                                       \\
                                           & \rightarrow 2\sinc\left(\frac{\omega}{T}\right)+\mathcal{F}\left[\rect\left(\frac{t}{2}\right)\cos(\pi t)\right]                               \\
                                          % jesus christ this line is long
                                           & =2\sinc\left(\frac{\omega}{\pi}\right)+\frac{1}{2}\left(2\sinc\left(\frac{\omega}{\pi}-1\right)+2\sinc\left(\frac{\omega}{\pi}+1\right)\right) \\
                                           & =\boxed{2\sinc\left(\frac{\omega}{\pi}\right)+\sinc\left(\frac{\omega}{\pi}-1\right)+\sinc\left(\frac{\omega}{\pi}+1\right)}
                                    \end{align*}
                              \item \[\begin{aligned}
                                                \mathcal{F}[x_4(t)]
                                                 & = \int_{-\infty}^{\infty} te^{-2t}u(t)e^{-j\omega t}\,dt \\
                                                 & = \int_{0}^{\infty} t\exp(-t(2+j\omega))\,dt             \\
                                          \end{aligned}\]
                                    Before we continue, let's assign $a=2+j\omega$ for brevity.
                                    \begin{align*}
                                          \int t\exp(-ta)\,dt
                                          &= t \cdot \left(-\frac{1}{a}\right)e^{-ta}-\int \left(-\frac{1}{a}\right)e^{-ta}\,dt \\
                                          &= -\frac{t}{a}e^{-ta}-\frac{1}{a^2}e^{-ta} \\
                                          &= -\frac{1}{a}e^{-ta}\left(t+\frac{1}{a}\right)
                                    \end{align*}

                                    Evaluating this integral from $0$ to $\infty$,
                                    we have \[\boxed{F(j\omega)=\frac{1}{(2+j\omega)^2}}\]
                        \end{enumerate}
                  \item \begin{enumerate}
                              \item Let's FT $f_1$:
                                    \begin{gather*}
                                          \mathcal{F}\left[\frac{1}{4\pi}\rect\left(\frac{t}{4\pi}\right)\right]=\sinc(2\omega) \\
                                          \mathcal{F}[\sinc(2t)]=2\pi \cdot \frac{1}{4\pi}\rect\left(\frac{-\omega}{4\pi}\right)=\frac{1}{2}\rect\left(\frac{\omega}{4\pi}\right) \\
                                    \end{gather*}

                                    and then $f_2$:
                                    \begin{gather*}
                                          \mathcal{F}\left[\frac{1}{2\pi}\rect\left(\frac{t}{2\pi}\right)\right]=\sinc(\omega) \\
                                          \mathcal{F}[\sinc(t)]=\rect\left(\frac{\omega}{2\pi}\right) \\
                                          \mathcal{F}[\sinc(t)\cos(3.1\pi t)]=\frac{1}{2}\left(\rect\left(\frac{\omega}{2\pi}+\frac{3.1}{2}\right)+\rect\left(\frac{\omega}{2\pi}-\frac{3.1}{2}\right)\right)
                                    \end{gather*}

                                    By the convolution theorem, the FT of $f$ is just the product of these two FTS.
                                    However, there's no space where both of them are nonzero,
                                    so the end result is just... $\boxed{F(j\omega)=0}$.
                              \item $f(t)=0$
                        \end{enumerate}
            \end{enumerate}
\end{enumerate}
\end{document}
