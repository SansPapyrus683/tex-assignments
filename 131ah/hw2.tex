\documentclass[12pt]{article}

% a template that a friend gave, it's worked well enough for me
% i have added some packages and stuff that have proved useful

\usepackage{fancyhdr}
\usepackage{tipa}
\usepackage{fontspec}
\usepackage{amsfonts}
\usepackage{enumitem}
\usepackage[margin=1in]{geometry}
\usepackage{graphicx}
\usepackage{float}
\usepackage{amsmath}
\usepackage{braket}
\usepackage{amssymb}
\usepackage{booktabs}
\usepackage{hyperref}
\usepackage{mathtools}
\usepackage{xcolor}
\usepackage{float}
\usepackage{algpseudocodex}
\usepackage{titlesec}
\usepackage{bbm}

\pagestyle{fancy}
\fancyhf{} % sets both header and footer to nothing
\lhead{Kevin Sheng}
\setmainfont{Comic Neue}
\renewcommand{\headrulewidth}{1pt}
\setlength{\headheight}{0.75in}
\setlength{\oddsidemargin}{0in}
\setlength{\evensidemargin}{0in}
\setlength{\voffset}{-.5in}
\setlength{\headsep}{10pt}
\setlength{\textwidth}{6.5in}
\setlength{\headwidth}{6.5in}
\setlength{\textheight}{8in}
\renewcommand{\headrulewidth}{0.5pt}
\renewcommand{\footrulewidth}{0.3pt}
\setlength{\textwidth}{6.5in}
\usepackage{setspace}
\usepackage{multicol}
\usepackage{float}
\setlength{\columnsep}{1cm}
\setlength\parindent{24pt}
\usepackage [english]{babel}
\usepackage [autostyle, english = american]{csquotes}
\MakeOuterQuote{"}

\setlength{\parskip}{6pt}
\setlength{\parindent}{0pt}

\titlespacing\section{0pt}{12pt plus 4pt minus 2pt}{0pt plus 2pt minus 2pt}
\titlespacing\subsection{0pt}{12pt plus 4pt minus 2pt}{0pt plus 2pt minus 2pt}
\titlespacing\subsubsection{0pt}{12pt plus 4pt minus 2pt}{0pt plus 2pt minus 2pt}

\hypersetup{colorlinks=true, urlcolor=blue}

\newcommand{\correction}[1]{\textcolor{red}{#1}}


\rhead{Math 131AH}

\makeatletter
\def\@seccntformat#1{%
  \expandafter\ifx\csname c@#1\endcsname\c@section\else
  \csname the#1\endcsname\quad
  \fi}
\makeatother

\newcommand{\lra}{\xLeftrightarrow}
\newcommand{\ra}{\xRightarrow}

\begin{document}

\section{Problem 1}

I'll assume that $2x$ really means $x+x$.
For the first statement:
\begin{align*}
      (a-b)(a-b) \ge 0
       & \ra{D} a(a-b)-b(a-b) \ge 0      \\
       & \ra{D} a^2-ab-ba+b^2 \ge 0      \\
       & \ra{A2} a^2-ab-ab+b^2 \ge 0     \\
       & \ra{A3} a^2+b^2-2ab \ge 0       \\
       & \ra{01} a^2+b^2-2ab+2ab \ge 2ab \\
       & \ra{A4} a^2+b^2 \ge 2ab
\end{align*}
which is equivalent to what we wanted to prove,
it's just a matter of direction/notation.

For the second statement, allow me to first prove
$a<b \land c<d \ra\  a+c<b+d$:
\begin{gather*}
      a<b \ra{01} a+d<b+d \\
      c<d \ra{01} c+a<d+a \\
      a+c=c+a<d+a=a+d<b+d
\end{gather*}
The last line is by additive commutatitivity and ordered transitivity.

Now by the first statement, we can get
\begin{align*}
      2bc \le b^2+c^2 & 2ac \le a^2+c^2
\end{align*}
Adding the three inequalities together
(which we know we can do by what we just proved), we have
\begin{align*}
              & 2ab+2bc+2ac \le a^2+b^2+b^2+c^2+a^2+c^2        \\
      \ra\ {} & 2ab+2bc+2ac \le 2a^2+2b^2+2c^2                 \\
      \ra\ {} & 2^{-1}(2ab+2bc+2ac) \le 2^{-1}(2a^2+2b^2+2c^2) \\
      \ra\ {} & ab+bc+ac \le a^2+b^2+c^2\quad\square
\end{align*}

\pagebreak

\section{Problem 2}

We first do the multiplication table, where we show
\[b \ne c \ra\  ab \ne ac\ \forall a \in F \setminus \{0\}\]
We argue by contradiction.
If $ab=ac$, we times both sides by $a^{-1}$ to get $b=c$,
which conflicts with $b \ne c$.

$ab \ne 0$, since this forces either $a$ or $b$ to be
$0$, which they aren't.
It can't be $a$ either, because $ab=a \ra\  b=1$, which is false.
$b$ is banned by similar reasoning, leaving $ab=1$ as the only viable option.
The rest of the table is filled out either by identities or process of elimination:
\begin{center}
      \begin{tabular}{ c|c|c|c|c }
            $\cdot$ & 0 & 1 & a & b \\ \hline
            0       & 0 & 0 & 0 & 0 \\ \hline
            1       & 0 & 1 & a & b \\ \hline
            a       & 0 & a & b & 1 \\ \hline
            b       & 0 & b & 1 & a
      \end{tabular}
\end{center}

For the addition table, that rows and columns must be unique is equivalent
to showing
\[b \ne c \ra\  a+b \ne a+c\ \forall a \in F\]
We argue by contradiction.
If $a+b=a+c$, then we can add $-a$ to each side to get $b=c$,
which clearly contradicts the premise.

The first row and column are easily filled out due to $0$
being the additive identity.
For the second row/column, there's three possible options:
\begin{center}
      \hfill
      \begin{tabular}{ c|c|c|c|c }
            + & 0 & 1 & a & b \\ \hline
            0 & 0 & 1 & a & b \\ \hline
            1 & 1 & a & 0 & b \\ \hline
            a & a & 0 &   &   \\ \hline
            b & b & b &   &
      \end{tabular}
      \hfill
      \begin{tabular}{ c|c|c|c|c }
            + & 0 & 1 & a & b \\ \hline
            0 & 0 & 1 & a & b \\ \hline
            1 & 1 & a & b & 0 \\ \hline
            a & a & b &   &   \\ \hline
            b & b & 0 &   &
      \end{tabular}
      \hfill
      \begin{tabular}{ c|c|c|c|c }
            + & 0 & 1 & a & b \\ \hline
            0 & 0 & 1 & a & b \\ \hline
            1 & 1 & 0 & b & a \\ \hline
            a & a & b &   &   \\ \hline
            b & b & a &   &
      \end{tabular}
      \hfill \mbox{}  % WHAT???
\end{center}
The first option can't be because it results in
\begin{align*}
      b(a+(-a)) & = b \cdot 0 & b(a+(-a)) & = b(a+1) \\
                & = 0         &           & = ba+b   \\
                &             &           & = 1+b    \\
                &             &           & = b
\end{align*}
which doesn't reconcile with the distributive property.

The second option can't be because then we would have
\begin{align*}
      a(b+(-b)) & = a \cdot 0 & a(b+(-b)) & = a(b+1) \\
                & = 0         &           & = ab+a   \\
                &             &           & = 1+a    \\
                &             &           & = b
\end{align*}
which again violates the distributive property.

This leaves only the third table.
For the final square, we have two options:
\begin{center}
      \hfill
      \begin{tabular}{ c|c|c|c|c }
            + & 0 & 1 & a & b \\ \hline
            0 & 0 & 1 & a & b \\ \hline
            1 & 1 & 0 & b & a \\ \hline
            a & a & b & 1 & 0 \\ \hline
            b & b & a & 0 & 1
      \end{tabular}
      \hfill
      \begin{tabular}{ c|c|c|c|c }
            + & 0 & 1 & a & b \\ \hline
            0 & 0 & 1 & a & b \\ \hline
            1 & 1 & 0 & b & a \\ \hline
            a & a & b & 0 & 1 \\ \hline
            b & b & a & 1 & 0
      \end{tabular}
      \hfill \mbox{}
\end{center}
The first square is invalid because it results in
\begin{align*}
      a(b+(-b)) & = a \cdot 0 & a(b+(-b)) & = a(b+1) \\
                & = 0         &           & = ab+a   \\
                &             &           & = 1+a    \\
                &             &           & = b
\end{align*}
breaking the distributive property.

This leaves the second square as our only option.
Our final tables are below:
\begin{center}
      \hfill
      \begin{tabular}{ c|c|c|c|c }
            + & 0 & 1 & a & b \\ \hline
            0 & 0 & 1 & a & b \\ \hline
            1 & 1 & 0 & b & a \\ \hline
            a & a & b & 0 & 1 \\ \hline
            b & b & a & 1 & 0
      \end{tabular}
      \hfill
      \begin{tabular}{ c|c|c|c|c }
            $\cdot$ & 0 & 1 & a & b \\ \hline
            0       & 0 & 0 & 0 & 0 \\ \hline
            1       & 0 & 1 & a & b \\ \hline
            a       & 0 & a & b & 1 \\ \hline
            b       & 0 & b & 1 & a
      \end{tabular}
      \hfill \mbox{}
\end{center}

\pagebreak

\section{Problem 3}

\subsection{Well-defined}

\textbf{Addition} \\
Say $a \sim b$ and $c \sim d$.
We WTS $a+c \sim b+d$.

By the definition of divisibility, $\exists n_1, n_2 \in \mathbb{Z}: qn_1=b-a, qn_2=d-c$.
Adding the two equations,
\[qn_1+qn_2=b-a+d-c \therefore q(n_1+n_2)=(b+d)-(a+c)\]
$n_1+n_2 \in \mathbb{Z}$, so $q \mid (b+d)-(a+c)$ and $a+c \sim b+d$. $\square$

\textbf{Multiplication} \\
The premise is the same, excect now we WTS $ac \sim bd$.
This time, let's define the following:
\begin{align*}
      a=qn_1+m_1 &  & b=qn_2+m_1 \\
      c=qn_3+m_2 &  & d=qn_4+m_2
\end{align*}
This gives us
\begin{gather*}
      ac=q^2n_1n_3+qn_1m_2+qn_3m_1+m_2m_1 \\
      bd=q^2n_2n_4+qn_2m_2+qn_4m_1+m_2m_1 \\
      ac-bd=q(qn_1n_3+n_1m_2+n_3m_1-qn_2n_4+n_2m_2+n_4m_1)
\end{gather*}
As we can see, $q \mid ac-bd$, so $ac \sim bd$. $\square$

\pagebreak

\subsection{Axioms}

\textbf{A. Commutativity}
\[C(a)+C(b)=C(a+b)=C(b+a)=C(b)+C(a)\]

\textbf{A. Associativity}
\begin{align*}
      (C(a)+C(b))+C(c)
       & = C(a+b)+C(c)      \\
       & = C(a+b+c)         \\
       & = C(a+(b+c))       \\
       & = C(a)+C(b+c)      \\
       & = C(a)+(C(b)+C(c))
\end{align*}

\textbf{A. Identity}
\[C(a)+C(0)=C(a+0)=C(a)\]

\textbf{A. Inverse}
\[C(a)+C(-a)=C(a+(-a))=C(0)\]

\textbf{M. Commutativity}
\[C(a) \cdot C(b)=C(ab)=C(ba)=C(b) \cdot C(a)\]

\textbf{M. Associativity}
\begin{align*}
      (C(a) \cdot C(b)) \cdot C(c)
       & = C(a \cdot b) \cdot C(c) \\
       & = C(abc)                  \\
       & = C(a(b \cdot c))         \\
       & = C(a) \cdot C(bc)
\end{align*}

\textbf{M. Identity}
\[C(a) \cdot C(1)=C(a \cdot 1)=C(a)\]

\textbf{M. Inverse} \\
For each $a$, we WTS $\exists b, n \in \mathbb{Z}: ab=nq+1$.
Rearranging, we have $ab-nq=1$.
Notice that since $q$ is prime, $\gcd(a, q)=1$.
By Bezout's identity, there must exist integers $b$ and $n$
that satisfy the given equality.

\textbf{Distributivity}
\begin{align*}
      (C(a)+C(b)) \cdot C(c)
       & = C(a+b) \cdot C(c)                 \\
       & = C((a+b) \cdot c)                  \\
       & = C(ac+bc)                          \\
       & = C(ac)+C(bc)                       \\
       & = C(a) \cdot C(c) + C(b) \cdot C(c)
\end{align*}

All the axioms are fulfilled;
$\mathbb{Z}/q\mathbb{Z}$ with the given operations is a field. $\square$

\subsection{No Ordering}

BWOC let we have an ordering relation on $\mathbb{Z}/q\mathbb{Z}$.

By Theorem 5.2, this implies a $P \in \mathbb{Z}/q\mathbb{Z}$
that satisfies all the relevant properties.
Out of $C(x)$ and $-C(x)=C(q-x)$, exactly
one has to be in $P$ (besides $C(0)$).
In particular, either $C(1)$ or $C(q-1)=C(-1)$ have to be in $P$.

If $C(1) \in P$, we can add $C(1)$ to itself $q-1$ times to get $C(q)=0$,
which definitely isn't in $P$.
OTOH, if $C(-1) \in P$, we can again add $C(-1)$
to itself $q-1$ times to get $C(-q)=C(0)$, which results in the same thnig.

Both cases result in a contradiction, and thus
such a $P$ satisfying the properties (and an associated ordering)
cannot exist. $\square$

\pagebreak

\section{Problem 4}

\subsection{Axioms}

\textbf{A. Commutativity}
\[(a,b)+(c,d)=(a+b,c+d)=(b+a,d+c)=(b,d)+(c,a)\]

\textbf{A. Associativity}
\begin{align*}
      ((a,b)+(c,d))+(e,f)
       & = (a+c,b+d)+(e,f)     \\
       & = (a+c+e,b+d+f)       \\
       & = (a+(c+e),b+(d+f))   \\
       & = (a,b)+(c+e,d+f)     \\
       & = (a,b)+((c,d)+(e,f))
\end{align*}

\textbf{A. Identity}
\[(a,b)+(0,0)=(a+0,b+0)=(a,b)\]

\textbf{A. Inverse}
\[(a,b)+(-a,-b)=(a-a,b-b)=(0,0)\]

\textbf{M. Commutativity}
\[(a,b)\cdot(c,d)=(ac-bd,ad+bc)=(ca-db,cb+da)=(c,d)\cdot(a,b)\]

\textbf{M. Associativity}
\begin{align*}
      ((a,b)\cdot(c,d))\cdot(e,f)
       & = (ac-bd,ad+bc)\cdot(e,f)           \\
       & = (a+c+e,b+d+f)                     \\
       & = (eac-adf-bcf-ebd,acf+ead+ebc-bdf) \\
       & = (a,b)\cdot(ec-df,cf+de)           \\
       & = (a,b)\cdot((c,d)\cdot(e,f))
\end{align*}

\textbf{M. Identity}
\[(a,b)\cdot(1,0)=(1 \cdot a - 0 \cdot b, a \cdot 0 + b \cdot 1)=(a,b)\]

\textbf{M. Inverse} \\
I propose that the inverse for $(a, b)$ is $\left(\frac{a}{a^2+b^2},-\frac{b}{a^2+b^2}\right)$.
We can verify like so:
\begin{align*}
      (a,b) \cdot \left(\frac{a}{a^2+b^2},-\frac{b}{a^2+b^2}\right)
       & = \left(\frac{a^2}{a^2+b^2}-\frac{-b^2}{a^2+b^2}, -\frac{ab}{a^2+b^2}+\frac{ab}{a^2+b^2}\right) \\
       & = \left(\frac{a^2+b^2}{a^2+b^2}, 0\right)                                                       \\
       & = (1, 0)
\end{align*}

\textbf{Distributivity}
\begin{align*}
      (a,b) \cdot ((c,d)+(e,f))
       & = (a,b) \cdot (c+e,d+f)                 \\
       & = (ac+ae+bd+bf, ad+af+bc+be)            \\
       & = (ac+bd, ad+bc)+(ae+bf, af+be)         \\
       & = (a,b) \cdot (c,d) + (a,b) \cdot (e,f)
\end{align*}

All the axioms are fulfilled; $F$ with the given operations is a field. $\square$

\subsection{No Ordering}

BWOC let there exists an ordering on $F$.

Again, by Theorem 5.2, this implies a $P \in \mathbb{Z}/q\mathbb{Z}$
that satisfies all the relevant properties.

Note that either $i \in P, -i \notin P$ or vice versa.
If $i \in P$, $i \cdot i \cdot i=-i$, which violates closure under multiplication.
OTOH, if $-i \in P$, $-i \cdot -i \cdot -i=i$, which is also wrong.

Either way, we get a contradiction- no ordering can exist. $\square$

\pagebreak

\section{Problem 5}

\subsection{Axioms}

\textbf{A. Commutativity}
\[(a,b)+(c,d)=(a+b,c+d)=(b+a,d+c)=(b,d)+(c,a)\]

\textbf{A. Associativity}
\begin{align*}
      ((a,b)+(c,d))+(e,f)
       & = (a+c,b+d)+(e,f)     \\
       & = (a+c+e,b+d+f)       \\
       & = (a+(c+e),b+(d+f))   \\
       & = (a,b)+(c+e,d+f)     \\
       & = (a,b)+((c,d)+(e,f))
\end{align*}

\textbf{A. Identity}
\[(a,b)+(0,0)=(a+0,b+0)=(a,b)\]

\textbf{A. Inverse}
\[(a,b)+(-a,-b)=(a-a,b-b)=(0,0)\]

\textbf{M. Associativity}
\begin{align*}
      ((a,b)\cdot(c,d))\cdot(e,f)
       & = (ac+2bd,ad+bc)\cdot(e,f)               \\
       & = (eac+2(bde+fad+fbc), ead+ebc+fac+2bdf) \\
       & = (a,b)\cdot(ec+2df,cf+de)               \\
       & = (a,b)\cdot((c,d)\cdot(e,f))
\end{align*}

\textbf{M. Identity}
\[(a,b)\cdot(1,0)=(1 \cdot a + 2 \cdot 0 \cdot b, a \cdot 0 + b \cdot 1)=(a,b)\]

\textbf{Distributivity P1}
\begin{align*}
      (a,b) \cdot ((c,d)+(e,f))
       & = (a,b) \cdot (c+e,d+f)                 \\
       & = (ac+ae+2bd+2bf,ad+af+bc+be)           \\
       & = (ac+2bd, ad+bc)+(ae+2bf, af+be)       \\
       & = (a,b) \cdot (c,d) + (a,b) \cdot (e,f)
\end{align*}

\textbf{Distributivity P2}
\begin{align*}
      ((c,d)+(e,f))\cdot
       & = (c+e,d+f)\cdot(a,b)               \\
       & = (ac+ae+2bd+2bf,ad+af+bc+be)       \\
       & = (ac+2bd, ad+bc)+(ae+2bf, af+be)   \\
       & = (c,d)\cdot(a,b) + (e,f)\cdot(a,b)
\end{align*}

All the axioms are fulfilled; $R$ with the given operations is a ring. $\square$

\pagebreak

\subsection{Order Relation}

\subsubsection{Trichotomy}

Notice that with $(a,b)$ and $(c,d)$ we're really comparing
$a+b\sqrt{2}$ and $c+d\sqrt{2}$.
These two quantities are in $\mathbb{R}$, and given that the ordering on $\mathbb{R}$
is a valid one, the trichotomy from there carries over.

\subsubsection{Transitivity}

We're given $(a,b)<(c,d)$ and $(c,d)<(e,f)$, which means
$a+b\sqrt{2}<c+d\sqrt{2}$ and $c+d\sqrt{2}<e+f\sqrt{2}$.
By transitivity in $\mathbb{R}$, this means $a+b\sqrt{2}<e+f\sqrt{2}$ as well.

\subsubsection{Adding to Both Sides}

If $(a,b)<(c,d)$, then we have
\begin{align*}
              & a+b\sqrt{2}<c+d\sqrt{2}                         \\
      \ra\ {} & a+b\sqrt{2}+e+f\sqrt{2}<c+d\sqrt{2}+e+f\sqrt{2} \\
      \ra\ {} & (a+e)+\sqrt{2}(b+f)<(c+e)+\sqrt{2}(d+f)
\end{align*}
which is equivalent to saying $(a,b)+(e,f)=(a+e,b+f)<(c+e,d+f)=(c,d)+(e,f)$.

\subsubsection{Multiplying Both Sides}

$(e,f)>(0,0)$, so $e+f\sqrt{2}>0$ and we can multiply both sides of an inequality by it.
\begin{align*}
              & a+b\sqrt{2}<c+d\sqrt{2}                                                                             \\
      \ra\ {} & \left(e+f\sqrt{2}\right)\left(a+b\sqrt{2}\right) < \left(e+f\sqrt{2}\right)\left(c+d\sqrt{2}\right) \\
      \ra\ {} & ae+2bf+\sqrt{2}(af+be) < ce+2df+\sqrt{2}(cf+de)                                                     \\
      \ra\ {} & (a,b) \cdot (e, f) < (c, d) \cdot (e, f)
\end{align*}

All four ordering properties are satisfied.
This is a valid ordering. $\square$

\pagebreak

\section{Problem 6}

\subsection{}

I'll just prove that all lower bounds are smaller than all upper bounds.
The infinimum and supremum are included in those, so it's implied.

Suppose we have any LB $m$ and any UB $M$.
Since $S \ne \varnothing$, we know there's an $x$ to use the transitive property on,
so $m \le x \le M \ra\ m \le M$. $\square$

\subsection{}

If $\inf S = \sup S$, then we can choose lower and upper bounds $m$/$M$
such that $m=M$ and $m \le x \le M$.
The two are equal, meaning $x=m$, so there can only be exactly one number
in the subset.

\pagebreak

\section{Problem 7}

\subsection{}

We've already proven $\inf S \le \sup S$, so we just WTS $\inf T \le \inf S$ and $\sup S \le \sup T$.

BWOC let $\inf T > \inf S$.
This means $\inf T$ isn't a valid LB for $S$, meaning $\exists x \in S: \inf T > x$.
But $S \subseteq T$, so $\exists x \in T: \inf T > x$,
which means $\inf T$ isn't even a valid LB for it's own set.
Contradiction.

The proof for the other inequality goes much the same way.
BWOC let $\sup S > \inf T$.
This means $\sup T$ isn't a valid UB for $S$, so $\exists x \in S: \sup T < x$.
Again, $S \subseteq T$, so $\exists x \in T: \sup T < x$,
contradicting the definition of the supremum and proving just what we wanted. $\square$

\subsection{}

Let's prove this first by $\sup(S \cup T) \le \max(\sup S, \sup T)$ and then
by showing that strict inequality can't happen.

\subsubsection{At Least}

It suffices to show that $\max(\sup S, \sup T)$ is a valid UB for $S \cup T$.

If $\sup S \ge \sup T$, then $x \le \sup S\ \forall x \in S$ and $x \le \sup T \le \sup S\ \forall x \in T$,
which covers all the elements.
This makes $\max(\sup S, \sup T)=\sup T$ a valid UB for $S \cup T$.

OTOH, if $\sup S < \sup T$, then $x \le \sup S < \sup T\ \forall x \in S$ and $x \le \sup T\ \forall x \in T$.
Again, this covers every element, so $\max(\sup S, \sup T)=\sup S$ in this case is still a valid UB.

\subsubsection{No Strict}

BWOC let $\sup(S \cup T) < \max(\sup S, \sup T)$.
This implies an $x \in \mathbb{R}$ that's a lower UB for $S \cup T$.

Again we casework on which value the max takes.
If $\sup S \ge \sup T$, $i \le s < \sup S\ \forall i \in S \cup T$.
$S \subseteq S \cup T$, so $s$ is a lower UB for $S$ than $\sup S$, which violates the
definition of the supremum.

If $\sup S < \sup T$, $i \le s < \sup T\ \forall i \in S \cup T$.
$T \subseteq S \cup T$, so in this case $s$ is a lower UB for $T$ than $\sup T$,
again contradicting how the supremum is defined.

Both cases result in a contradiction, so we only have $\sup(S \cup T)=\max(\sup S, \sup T)$. $\square$

\pagebreak

\section{Problem 8}

We'll prove this by showing $-\sup(-A) \le \inf A$ and vice versa.

\subsection{One Way}

All we have to do is show that $-\sup(-A)$ is a valid LB for $A$.

By definition $x \le \sup(-A)\ \forall x \in -A$.
$x \in -A \iff -x \in A$, so we can turn this into
\begin{gather*}
      -x \le \sup(-A)\ \forall x \in A \\
      x \ge -\sup(-A)\ \forall x \in A
\end{gather*}
showing that $-\sup(-A)$ is a valid LB for $A$.

\subsection{Then the Other}

$-\sup(-A) \ge \inf A \iff \sup(-A) \le -\inf A$,
which means now we have to show $-\inf A$ is a valid UB for $-A$.
\begin{gather*}
      x \ge \inf A\ \forall x \in A \\
      -x \ge \inf A\ \forall x \in -A \\
      x \le -\inf A\ \forall x \in -A
\end{gather*}
showing that $-\inf A$ is indeed a valid LB.

With inequalities in both directions proved, we finally have $-\sup(-A)=\inf A$. $\square$

\end{document}
