\documentclass[12pt]{article}

% a template that a friend gave, it's worked well enough for me
% i have added some packages and stuff that have proved useful

\usepackage{fancyhdr}
\usepackage{tipa}
\usepackage{fontspec}
\usepackage{amsfonts}
\usepackage{enumitem}
\usepackage[margin=1in]{geometry}
\usepackage{graphicx}
\usepackage{float}
\usepackage{amsmath}
\usepackage{braket}
\usepackage{amssymb}
\usepackage{booktabs}
\usepackage{hyperref}
\usepackage{mathtools}
\usepackage{xcolor}
\usepackage{float}
\usepackage{algpseudocodex}
\usepackage{titlesec}
\usepackage{bbm}
\usepackage{pythonhighlight}

\pagestyle{fancy}
\fancyhf{} % sets both header and footer to nothing
\lhead{Kevin Sheng}
\setmainfont{Comic Neue}
\renewcommand{\headrulewidth}{1pt}
\setlength{\headheight}{0.75in}
\setlength{\oddsidemargin}{0in}
\setlength{\evensidemargin}{0in}
\setlength{\voffset}{-.5in}
\setlength{\headsep}{10pt}
\setlength{\textwidth}{6.5in}
\setlength{\headwidth}{6.5in}
\setlength{\textheight}{8in}
\renewcommand{\headrulewidth}{0.5pt}
\renewcommand{\footrulewidth}{0.3pt}
\setlength{\textwidth}{6.5in}
\usepackage{setspace}
\usepackage{multicol}
\usepackage{float}
\setlength{\columnsep}{1cm}
\setlength\parindent{24pt}
\usepackage [english]{babel}
\usepackage [autostyle, english = american]{csquotes}
\MakeOuterQuote{"}

\setlength{\parskip}{6pt}
\setlength{\parindent}{0pt}

\titlespacing\section{0pt}{12pt plus 4pt minus 2pt}{0pt plus 2pt minus 2pt}
\titlespacing\subsection{0pt}{12pt plus 4pt minus 2pt}{0pt plus 2pt minus 2pt}
\titlespacing\subsubsection{0pt}{12pt plus 4pt minus 2pt}{0pt plus 2pt minus 2pt}

\hypersetup{colorlinks=true, urlcolor=blue}

\newcommand{\correction}[1]{\textcolor{red}{#1}}


\rhead{CS 118}

\begin{document}

\section{Intersymbol Interference}

\begin{enumerate}
      \item The outputs are $2 V$, $0 V$, and $2 V$ respectively.
            The receiver will then output the bit sequence $\texttt{101}$.
      \item Here, the outputs are still $2 V$, $0 V$, and $2 V$, resulting in the same bits.
      \item The outputs are $2 V$, $2.75 V$, and $2 V$, so now the bits are $\texttt{111}$.
      \item When the input sequence dips down to $0$, the output frequency has to also have a
            corresponding sort of "dip" in the middle.
            Otherwise, the receiver won't be able to detect the change.
\end{enumerate}

\pagebreak

\section{Clock Recovery}

\begin{enumerate}
      \item I've marked the times in multiples of $t$ where $t$ is the sender's clock rate.
            Here, $t=1.09 s$.
            \begin{center}
                  \includegraphics[width=15cm]{img/hw1/signal}
            \end{center}
            The green marks indicate the ideal sampling times.

      \item If the receiver doesn't run the clock recovery, then it'll sample at $9.5 s$ when the true ideal sample time is $9.5t$, or $10.355 s$.
            Here, the sampling is off by a whole $0.855 s$.

      \item The sampled signal is still correct- \texttt{0101110010}.

            It's sampled at the following seconds:
            \[.5, 1.5, 2.59, 3.68, 4.77, 5.86, 6.95, 8.04, 9.13, 10.22\]

            The value of \texttt{lag} is only $0$ when the loop initially starts.
            After, it takes on a constant value of $0.9$ for the rest of the code.

      \item A transition already occurs during the time from $t=1.5 s$ to $t=2.59 s$,
            so it would depend on which transition the edge detector recognizes.

            If it doesn't use the spike from the noise, the worst case scenario is that
            the third sampled bit might be a \texttt{1} instead of a \texttt{0}.

            OTOH, if the spike is the actual recognized transition, then \texttt{lag}
            would take on a much higher value and the sampling times would be delayed.

      \item No transition occurs due to the signal here.
            However, in this case \texttt{lag} would wind up taking a negative value
            for some time and squish the later sampling instances together.
\end{enumerate}

\pagebreak

\section{code here bc idk where else to put it}

\begin{python}
bits = "0101110010"
send_interval = 1.09
transitions = [i * send_interval for i in [1, 2, 3, 6, 8, 9]]

lag = 0
at = 1 / 2
pred = 0
sampled = []
lags = []
for _ in range(10):
    print(at)
    sampled.append(bits[int(at // send_interval)])
    lags.append(lag)

    pred += 1 + lag
    next_at = at + 1 + lag

    for t in transitions:
        if at <= t <= next_at:
            lag = t - pred

    at = next_at

print(sampled)
print(lags)
\end{python}

\end{document}
