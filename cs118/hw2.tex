\documentclass[12pt]{article}

% a template that a friend gave, it's worked well enough for me
% i have added some packages and stuff that have proved useful

\usepackage{fancyhdr}
\usepackage{tipa}
\usepackage{fontspec}
\usepackage{amsfonts}
\usepackage{enumitem}
\usepackage[margin=1in]{geometry}
\usepackage{graphicx}
\usepackage{float}
\usepackage{amsmath}
\usepackage{braket}
\usepackage{amssymb}
\usepackage{booktabs}
\usepackage{hyperref}
\usepackage{mathtools}
\usepackage{xcolor}
\usepackage{float}
\usepackage{algpseudocodex}
\usepackage{titlesec}
\usepackage{bbm}
\usepackage{pythonhighlight}

\pagestyle{fancy}
\fancyhf{} % sets both header and footer to nothing
\lhead{Kevin Sheng}
\setmainfont{Comic Neue}
\renewcommand{\headrulewidth}{1pt}
\setlength{\headheight}{0.75in}
\setlength{\oddsidemargin}{0in}
\setlength{\evensidemargin}{0in}
\setlength{\voffset}{-.5in}
\setlength{\headsep}{10pt}
\setlength{\textwidth}{6.5in}
\setlength{\headwidth}{6.5in}
\setlength{\textheight}{8in}
\renewcommand{\headrulewidth}{0.5pt}
\renewcommand{\footrulewidth}{0.3pt}
\setlength{\textwidth}{6.5in}
\usepackage{setspace}
\usepackage{multicol}
\usepackage{float}
\setlength{\columnsep}{1cm}
\setlength\parindent{24pt}
\usepackage [english]{babel}
\usepackage [autostyle, english = american]{csquotes}
\MakeOuterQuote{"}

\setlength{\parskip}{6pt}
\setlength{\parindent}{0pt}

\titlespacing\section{0pt}{12pt plus 4pt minus 2pt}{0pt plus 2pt minus 2pt}
\titlespacing\subsection{0pt}{12pt plus 4pt minus 2pt}{0pt plus 2pt minus 2pt}
\titlespacing\subsubsection{0pt}{12pt plus 4pt minus 2pt}{0pt plus 2pt minus 2pt}

\hypersetup{colorlinks=true, urlcolor=blue}

\newcommand{\correction}[1]{\textcolor{red}{#1}}


\rhead{CS 118}

\renewcommand{\c}{\texttt}  % short for \code

\begin{document}

\section{HDLC Bit-stuffing Code}

\begin{enumerate}[label=\Alph*.]
    \item The code works by scanning through all consecutive "runs" of \c{1}s in the provided array.

          The two main variables it uses to keep track of it's position are
          \c{i}, it's position in the array, and \c{j}, it's position
          in the buffer it's going to print out.

          It does this by running a for loop through the array itself and then
          running a subclause as soon as it encounters a \c{1}.
          After this, it sets \c{i} to the index immediately following the run
          before continuing on to the outer for loop.

          Within the subclause is another for loop that keeps a counter of how
          many \c{1}s it's seen inside the run.
          As soon as the counter hits $5$, it's reset and an additional $0$
          is added alongside all the original bits.

    \item The output is \boxed{\c{111110011111010}}

    \item \begin{algorithmic}[1]
              \LComment{Returns the destuffed code or reports an error if the encoded string is invalid}
              \Procedure{\c{destuff}}{\c{arr}}
                  \State $\c{result} \gets []$
                  \State $\c{ones} \gets 0$
                  \State $i \gets 0$
                  \While{$i <$ length of \c{arr}}
                      \If{$\c{arr}[i] = 1$}
                        \State Append $1$ to \c{result}
                        \State $\c{ones} \gets \c{ones} + 1$
                        \If{$\c{ones} = 5$}
                            \If{$\c{arr[i + 1]} = 1$}
                                \State report error!
                                \Comment{$6$ consecutive $1$s can't exist}
                            \EndIf
                            \State $i \gets i + 1$
                            \Comment{Move $i$ past the stuffed $0$}
                            \State $\c{ones} \gets 0$
                        \EndIf
                      \item[]
                      \ElsIf{$\c{arr}[i] = 0$}
                        \State Append $0$ to \c{result}
                      \EndIf

                      \item[]
                      \State $i \gets i + 1$
                      \Comment{Move on to the next bit}
                  \EndWhile
                  
                  \item[]
                  \State \Return \c{result}
              \EndProcedure
          \end{algorithmic}
\end{enumerate}

\pagebreak

\section{Error Detection - CRCs}

\begin{enumerate}[label=\Alph*.]
    \item The two errors are these:
          \begin{enumerate}[label=\arabic*.]
              \item There's an indexing error in the for loop.
                    It should stop when $\c{i} < \c{input\_len}$,
                    not when $\c{i} \le \c{input\_len}$.

              \item The dividend is wrong.
                    It should be the input left-shifted by $3$ bits, not the input as is.
          \end{enumerate}

    \item We want to find the mod-2 remainder when \c{110010000} is divided by \c{1001}.

          I'll just go through the bits one by one and XOR when necessary.
          New lines indicate when a XOR has occurred.
          \begin{align*}
                    & \c{1} \to \c{11} \to \c{110} \to \c{1100} \\
              \to{} & \c{101} \to \c{1011}                      \\
              \to{} & \c{10} \to \c{100} \to \c{1000}           \\
              \to{} & \c{1} \to \c{10} \to \c{100}
          \end{align*}
          So our overall remainder is \boxed{\c{100}}.

    \item This code is usually run in the physical layer and thus very close to the hardware.
          Thus, ChatGPT chose C, the most popular hardware-adjacent language.

    \item The raw bitstring \c{1011011} must be divisible by \c{1001} for things to work out.

          We do the division and see if it all checks out:
          \begin{align*}
                  & \c{1} \to \c{10} \to \c{101} \to \c{1011} \\
              \to & \c{10} \to \c{100} \to \c{1001}           \\
              \to & \c{0} \to \c{01} = \c{1}
          \end{align*}
          The remainder is nonzero, so an error is detected.

    \item Even though the client might not detect any errors, this doesn't mean
          no corruption occurred.
          This is fine, though, as the chance of random errors still preserving
          the divisiblity is next to nothing.

    \item The longer a CRC generator is, the better it is at detecting burst errors.
          However, it also means we have to send more bits over the wire,
          so we can't just pick arbitrarily long generator polynomials.
\end{enumerate}

\pagebreak

\section{Error Recovery}

\begin{enumerate}[label=\Alph*.]
    \item The \textbf{STATUS} packets are needed due to the following hypothetical situation:
          \begin{itemize}
              \item Sender has $5$ total packets to send.
              \item The first $4$ packets are fine and are received in order.
              \item The fifth and last packet is lost.
              \item The receiver doesn't know anything, so no \textbf{NACK} is sent.
              \item The sender also doesn't know anything, so the last one isn't retransmitted.
          \end{itemize}

    \item Suppose there do exist data packets that haven't been properly received yet.
          Since the status timer is running, the sender will keep on sending \textbf{STATUS}
          packets to check on the receiver.

          Eventually, by the law of large numbers
          one \textbf{STATUS} packet and one corresponding \textbf{NACK} packet will make it through.
          This means that the sender will then retransmit the missing data packet.
          If that one doesn't make it, the whole status thing does it all over again. $\square$

    \item The timer must be stopped only when a \textbf{NACK} has been received
          that acknowledges all the data that needed to be sent.

          It must be started in the following scenarios:
          \begin{itemize}
              \item When the last \textbf{STATUS} packet was sent
              \item When the last data packet was sent
          \end{itemize}

    \item Let the status timeout be $S$ in seconds and the one-way delay be $T$.

          The worst case scenario is the following:
          \begin{itemize}
              \item Sender sends second packet and receiver transmits a \textbf{NACK} pack- $2T s$.
              \item Sender retransmits first packet and starts status timer.
              \item No other packets are lost, so it takes $S$ seconds for the status timer to go off.
              \item Sender sends off a \textbf{STATUS} packet and gets a \textbf{NACK} back- $2T s$.
          \end{itemize}
          Assuming $S > T$ which I don't think is unreasonable,
          this has a total latency of $\boxed{4T+S}$ seconds.
\end{enumerate}

\end{document}
