\documentclass[12pt]{article}

% a template that a friend gave, it's worked well enough for me
% i have added some packages and stuff that have proved useful

\usepackage{fancyhdr}
\usepackage{tipa}
\usepackage{fontspec}
\usepackage{amsfonts}
\usepackage{enumitem}
\usepackage[margin=1in]{geometry}
\usepackage{graphicx}
\usepackage{float}
\usepackage{amsmath}
\usepackage{braket}
\usepackage{amssymb}
\usepackage{booktabs}
\usepackage{hyperref}
\usepackage{mathtools}
\usepackage{xcolor}
\usepackage{float}
\usepackage{algpseudocodex}
\usepackage{titlesec}
\usepackage{bbm}
\usepackage{pythonhighlight}

\pagestyle{fancy}
\fancyhf{} % sets both header and footer to nothing
\lhead{Kevin Sheng}
\setmainfont{Comic Neue}
\renewcommand{\headrulewidth}{1pt}
\setlength{\headheight}{0.75in}
\setlength{\oddsidemargin}{0in}
\setlength{\evensidemargin}{0in}
\setlength{\voffset}{-.5in}
\setlength{\headsep}{10pt}
\setlength{\textwidth}{6.5in}
\setlength{\headwidth}{6.5in}
\setlength{\textheight}{8in}
\renewcommand{\headrulewidth}{0.5pt}
\renewcommand{\footrulewidth}{0.3pt}
\setlength{\textwidth}{6.5in}
\usepackage{setspace}
\usepackage{multicol}
\usepackage{float}
\setlength{\columnsep}{1cm}
\setlength\parindent{24pt}
\usepackage [english]{babel}
\usepackage [autostyle, english = american]{csquotes}
\MakeOuterQuote{"}

\setlength{\parskip}{6pt}
\setlength{\parindent}{0pt}

\titlespacing\section{0pt}{12pt plus 4pt minus 2pt}{0pt plus 2pt minus 2pt}
\titlespacing\subsection{0pt}{12pt plus 4pt minus 2pt}{0pt plus 2pt minus 2pt}
\titlespacing\subsubsection{0pt}{12pt plus 4pt minus 2pt}{0pt plus 2pt minus 2pt}

\hypersetup{colorlinks=true, urlcolor=blue}

\newcommand{\correction}[1]{\textcolor{red}{#1}}


\rhead{CS 118}

\renewcommand{\c}{\texttt}

\begin{document}

\section{Bridging and Loops}

\begin{enumerate}
      \item Bridges can keep some cache of all the packets they've seen before.
            If a duplicate packet is detected, a bridge just drops it.

      \item Obviously this approach won't work if packets can be dropped and need to be transmitted
            between different LANS.
\end{enumerate}

\section{IP Broadcast Storms; Bridges VS Routers}

\begin{enumerate}
      \item This is what happens:
            \begin{itemize}
                  \item D broadcasts an ARP request to the entire network.
                        However, since this specifies the IP address (not the data link address),
                        only A pays attention.
                  \item A gives an ARP response.
                  \item D now sends the packet to A's data link address.
                        However, since this address was set to the broadcast address
                        of all 1s, every node on the network gets the packet.
            \end{itemize}

      \item If the bridge is replaced by an IP router, then D shouldn't know A's MAC address;
            all it should do is send to the IP router and let it to the rest of the work.

            The IP router still gives the packet to everyone in A's network, but
            at least D's network is spared.
            This results in $T-M-1$ less endnodes that have to deal with
            packets that weren't meant for them.
\end{enumerate}

\pagebreak

\section{IP Forwarding; LMP w/ Tries}

\begin{enumerate}
      \item If a child is \c{NULL} I didn't draw it.
            \begin{center}
                  \includegraphics[width=10cm]{img/hw3/trie}
            \end{center}

            \pagebreak

      \item Say our IP address went like \c{000}.
            It would stop after going to the \c{0} child of \c{00},
            since that's a null pointer.

            However, it would then have to know what the longest prefix it matched was.
            Just knowing the last node traversed wouldn't suffice, since \c{00}
            doesn't corrspond with any entry in the forwarding table.

            Thus, we need an \c{is\_end\_of\_prefix} field to keep track
            of which nodes can be thought of as the "last matched entry".

      \item When adding a prefix, we need future lookups to know that
            this node is now a valid entry in the forwarding table.
            To indicate this, we set \c{is\_end\_of\_prefix} to be $1$.

      \item Well, there's many ways to implement it and with some
            pointer magic we could have just kept track of the next hop + interface
            instead of the actual last matching node.

            An edge case that could be possible in multithreaded systems
            is if the next hop of an entry changes during a walk.
            If this is the case, then keeping track of the next hop and interface directly
            wouldn't suffice as the values would then become outdated.

      \item \c{extract\_destination\_ip} reads bytes $16$ through $19$ inclusive
            and concatenates in big-endian order to recover the destination IP.

            It looks in these indices because according to the IP spec
            that's where the bytes of the destination IP are.
            Nothing more to it, really

      \item The code could probably include a default IP to forward things to
            if no matching entry is found.
            It should also perhaps check if the destination IP is in its own local network
            and act accordingly if it is.
\end{enumerate}

\pagebreak

\section{Modifying Routing to Avoid Fragmentation}

\begin{enumerate}
      \item The shortest path is
      \[\c{R1} \to \c{R2} \to \c{R4} \to \c{R3} \to \c{P}\]
      The largest packet size that can get through is \boxed{512} bytes.

      \item Define the following functions:
      \begin{itemize}
            \item $d(a, b)$ is the distance from $a$ to $b$, and it's $\infty$ if the two aren't adjacent.
            \item $m(a, b)$ is the max packet size from $a$ to $b$, and it's $-\infty$ if they aren't adjacent.
      \end{itemize}
      
      The equation for \c{Distance} is
      \[\c{Distance}(P, R) = \min_{N} (\c{Distance}(P, N) + d(R, N))\]
      and for \c{MinMaxPacketSize} it's
      \[\c{MinMaxPacketSize}(P, R) = \max_{N} \min(\c{MinMaxPacketSize}(P, N), m(R, N))\]

      \pagebreak

      \item The system takes $3$ time steps to converge.
      
      $t=1$:
      \begin{center}
            \includegraphics[width=13cm]{img/hw3/t1}
      \end{center}

      $t=2$:
      \begin{center}
            \includegraphics[width=13cm]{img/hw3/t2}
      \end{center}

      $t=3$:
      \begin{center}
            \includegraphics[width=13cm]{img/hw3/t3}
      \end{center}

      \item Assuming the connection fails at the start of $t=7$,
      the system takes another $2$ time steps to converge.

      $t=7$:
      \begin{center}
            \includegraphics[width=13cm]{img/hw3/t7}
      \end{center}

      $t=8$:
      \begin{center}
            \includegraphics[width=13cm]{img/hw3/t8}
      \end{center}
\end{enumerate}

\end{document}
