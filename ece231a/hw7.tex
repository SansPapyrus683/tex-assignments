\documentclass[12pt]{article}

% a template that a friend gave, it's worked well enough for me
% i have added some packages and stuff that have proved useful

\usepackage{fancyhdr}
\usepackage{tipa}
\usepackage{fontspec}
\usepackage{amsfonts}
\usepackage{enumitem}
\usepackage[margin=1in]{geometry}
\usepackage{graphicx}
\usepackage{float}
\usepackage{amsmath}
\usepackage{braket}
\usepackage{amssymb}
\usepackage{booktabs}
\usepackage{hyperref}
\usepackage{mathtools}
\usepackage{xcolor}
\usepackage{float}
\usepackage{algpseudocodex}
\usepackage{titlesec}
\usepackage{bbm}
\usepackage{pythonhighlight}

\pagestyle{fancy}
\fancyhf{} % sets both header and footer to nothing
\lhead{Kevin Sheng}
\setmainfont{Comic Neue}
\renewcommand{\headrulewidth}{1pt}
\setlength{\headheight}{0.75in}
\setlength{\oddsidemargin}{0in}
\setlength{\evensidemargin}{0in}
\setlength{\voffset}{-.5in}
\setlength{\headsep}{10pt}
\setlength{\textwidth}{6.5in}
\setlength{\headwidth}{6.5in}
\setlength{\textheight}{8in}
\renewcommand{\headrulewidth}{0.5pt}
\renewcommand{\footrulewidth}{0.3pt}
\setlength{\textwidth}{6.5in}
\usepackage{setspace}
\usepackage{multicol}
\usepackage{float}
\setlength{\columnsep}{1cm}
\setlength\parindent{24pt}
\usepackage [english]{babel}
\usepackage [autostyle, english = american]{csquotes}
\MakeOuterQuote{"}

\setlength{\parskip}{6pt}
\setlength{\parindent}{0pt}

\titlespacing\section{0pt}{12pt plus 4pt minus 2pt}{0pt plus 2pt minus 2pt}
\titlespacing\subsection{0pt}{12pt plus 4pt minus 2pt}{0pt plus 2pt minus 2pt}
\titlespacing\subsubsection{0pt}{12pt plus 4pt minus 2pt}{0pt plus 2pt minus 2pt}

\hypersetup{colorlinks=true, urlcolor=blue}

\newcommand{\correction}[1]{\textcolor{red}{#1}}


\rhead{ECE 231A}

\newcommand{\ind}{\perp\!\!\!\!\perp} 
\newcommand*{\ditto}{-''-}
\renewcommand{\c}{\teXttt}
\DeclareMathOperator{\E}{E}
\DeclareMathOperator{\Var}{Var}

\begin{document}

\section{Resources}

I've used my own brain, posted slides/resources and a Discord chat with Kevin Zhao.

\section{4-ary Hamming Distortion}

We have
\begin{align*}
    I(\hat{X}; X)
     & = H(X) - H(X \mid \hat{X})                                              \\
     & \ge 2 - (H(P(\hat{X} \ne X)) + P(\hat{X} \ne X)H(X \mid X \ne \hat{X})) \\
     & \ge \boxed{2-H(D)-D\log 3}
\end{align*}

To achieve this lower bound, consider the following test channel going from $\hat{X}$ to $X$:
\[P=\begin{bmatrix}
        1-D & D/3 & D/3 & D/3 \\
        D/3 & 1-D & D/3 & D/3 \\
        D/3 & D/3 & 1-D & D/3 \\
        D/3 & D/3 & D/3 & 1-D
    \end{bmatrix}\]
This transition matrix is cyclically symmetric, so it's maximized when the input distribution is symmetric.
This also implies that the output distribution is symmetric.

The mutual information is as follows:
\begin{align*}
    I(X; \hat{X})
     & = H(X)-H(X \mid \hat{X})                                       \\
     & = 2-H_4\left(1-D, \frac{D}{3}, \frac{D}{3}, \frac{D}{3}\right) \\
     & = 2-\left(-D\log\left(\frac{D}{3}\right)-(1-D)\log(1-D)\right) \\
     & = 2-(-D \log D-(1-D)\log(1-D)+D\log 3)                         \\
     & = 2-(H(D)+D\log 3)
\end{align*}
which is exactly what we want, thus implying a $P(\hat{X} \mid X)$ that satisfies all the constraints.

\pagebreak

\section{\texorpdfstring{$m$}{m}-ary Hamming Distortion}

By engineer's induction, I propose that $R(D)$ for arbitrary $m$ is
\[R(D)=\log m - H(D) - D\log(m-1)\]
That this is a lower bound can be proved in basically the same way that we proved it for $m=4$
with a couple of changes to the numbers.

We can construct a test channel for this similar to the way we did it for $m-4$:
\[P_{ij}=\begin{cases}
        1-D           & i=j              \\
        \frac{D}{m-1} & \text{otherwise}
    \end{cases}\]
This matrix is also cyclically symmetric,
so the channel achieves capacity with $\hat{X}$ is distributed uniformly.

The capacity is as follows:
\begin{align*}
    I(X; \hat{X})
     & = H(X)-H(X \mid \hat{X})                                              \\
     & = \log m-H_m\left(1-D, \frac{D}{m-1}, \cdots, \frac{D}{m-1}\right)    \\
     & = \log m-\left(-D\log\left(\frac{D}{m-1}\right)-(1-D)\log(1-D)\right) \\
     & = \log m-(-D \log D-(1-D)\log(1-D)+D\log(m-1))                        \\
     & = \log m-(H(D)+D\log(m-1))
\end{align*}
showing that this lower bound is indeed achievable.

\section{Scaled Hamming Distortion}

The only thing that's changed about this cost function is that we need $2P(\hat{X} \ne X) \le D$
instead of $P(\hat{X} \ne X)$.

This also means that $\boxed{R(D)=R_3\left(\frac{D}{2}\right)}$, where $R_3$ is the rate distortion
function calculated in the previous section when $m=3$.

\pagebreak

\section{RD Function With Infinite Distortion}

If $d(x, \hat{X})=\infty$ when $\hat{X}=1$ and $x=0$, this forces $P(\hat{X}=0 \mid x=0)=1$.

Past that, the only variable we can change is $P(\hat{X}=1 \mid x=1)$, which I'll denote as $p$.

The only place where we have a nonzero distortion is when
we send a $0$ instead of a $1$ with a $\frac{1-p}{2}$ chance.

Our distortion rate is bounded by $D$, so let's set it at equality: $\frac{1-p}{2}=D$.
This is because the more $0$s we send instead of $1$s, the farther we make the distribution
of $\hat{X}$ from the uniform and the lower we make the entropy.

With that set,
\begin{align*}
    I(\hat{X}; X)
     & = H(X)-H(X \mid \hat{X})                                            \\
     & = 1-P(\hat{X}=0)H(X \mid \hat{X}=0)-P(\hat{X}=1)H(X \mid \hat{X}=1) \\
     & = 1-\frac{2-p}{2}H(X \mid \hat{X}=0)-\frac{p}{2}H(X \mid \hat{X}=1) \\
     & = 1-\frac{2-p}{2}H(X \mid \hat{X}=0)                                \\
     & = 1-\left(\frac{1}{2}+D\right)H(X \mid \hat{X}=0)                   \\
     & = 1-\left(\frac{1}{2}+D\right)H\left(\frac{1}{2-p}\right)           \\
     & = 1-\left(\frac{1}{2}+D\right)H\left(\frac{1}{2D+1}\right)
\end{align*}

\pagebreak

\section{Erasure Distortion}

I'm going to assume the middle column is some "error" state, since the indices don't 100\% match up.
Otherwise, $d(1, 1)=1$, and there's no universe in which that makes sense.

\subsection{Part A}

We can lower bound $R(D)$ as follows:
\begin{align*}
          & I(\hat{X}; X)                                                                                           \\
    ={}   & H(X)-H(X \mid \hat{X})                                                                                  \\
    ={}   & 1 - H(X \mid \hat{X}=0)P(\hat{x}=0) - H(X \mid \hat{X}=e)P(\hat{x}=e) - H(X \mid \hat{X}=1)P(\hat{x}=1) \\
    ={}   & 1- H(X \mid \hat{X}=e)P(\hat{x}=e)                                                                      \\
    \ge{} & 1-H(X \mid \hat{X}=e)D                                                                                  \\
    \ge{} & \boxed{1-D}
\end{align*}

I'll show how to achieve this LB in the next part.

\subsection{Part B}

Here, instead of $p$ being the chance that we send a $0$ instead of a $1$,
we let $p$ be the chance we send an $e$ instead of any symbol.
By symmetry, this should be the same regardless of the original symbol.

Now, since $p$ applies to both symbols, we set $p=D$ to get the following mutual info:
\begin{align*}
        & I(\hat{X}; X)                                                                                           \\
    ={} & H(X)-H(X \mid \hat{X})                                                                                  \\
    ={} & 1 - H(X \mid \hat{X}=0)P(\hat{x}=0) - H(X \mid \hat{X}=e)P(\hat{x}=e) - H(X \mid \hat{X}=1)P(\hat{x}=1) \\
    ={} & 1- H(X \mid \hat{X}=e)P(\hat{x}=e)                                                                      \\
    ={} & 1-p                                                                                                     \\
    ={} & 1-D
\end{align*}
In this case, $H(X \mid \hat{X}=e)=1$ because it was equally likely that we got a $0$ or a $1$ when we have an $e$.

\pagebreak

\section{Conditional Probability That Achieves \texorpdfstring{$R(D)$}{R(D)}}

didn't we do this in lecture what

Regardless, from the hint we can represent the conditional distribution using a BSC with crossover probability $D$.

The output and conditionals are alrady known:
\begin{gather*}
    P(X=0)=p \\
    P(X=1)=1-p \\
    P(X=0 \mid \hat{X}=1)=P(X=1 \mid \hat{X}=0)=D \\
    P(X=0 \mid \hat{X}=0)=P(X=1 \mid \hat{X}=1)=1-D
\end{gather*}
Letting $a=P(\hat{X}=0)$, we have
\begin{align*}
               & (1-D)a + D(1-a)=p               \\
    \implies{} & a(1-2D)=p-D                     \\
    \implies{} & P(\hat{X}=0)=\frac{p-D}{1-2D}   \\
    \implies{} & P(\hat{X}=1)=\frac{1-p-D}{1-2D}
\end{align*}


From this we can find $P(\hat{X} \mid X)$ using Bayes' rule:
\begin{align*}
    P(\hat{X}=0 \mid X=0)
     & = \frac{P(X=0 \mid \hat{X}=0)P(\hat{X}=0)}{P(X=0)} \\
     & = \frac{(1-D)\frac{p-D}{1-2D}}{\frac{1}{2}}        \\
     & = \frac{2(1-D)(p-D)}{1-2D}
\end{align*}
By a similar process, we have
\begin{gather*}
    P(\hat{X}=1 \mid X=0)=\frac{2D(p-D)}{1-2D} \\
    P(\hat{X}=1 \mid X=1)=\frac{2(1-D)(1-p-D)}{1-2D} \\
    P(\hat{X}=0 \mid X=1)=\frac{2D(1-p-D)}{1-2D}
\end{gather*}

\pagebreak

\section{Bounds on \texorpdfstring{$R(D)$}{R(D)} for MSE}

\subsection{Lower Bound}

Since we're using the squared error distortion, we have the constraint that $\E[(X-\hat{X})^2] \le D$.

And now,
\begin{align*}
    I(X; \hat{X})
     & = h(X) - h(X \mid \hat{X})                      \\
     & = h(X) - h(X - \hat{X} \mid \hat{X})            \\
     & \ge h(X) - h(X - \hat{X})                       \\
     & \ge h(X) - h(\mathcal{N}(0, \E[(X-\hat{X})^2])) \\
     & \ge h(X) - h(\mathcal{N}(0, D))                 \\
     & = h(X) - \frac{1}{2}\log(2\pi eD)
\end{align*}

\subsection{Upper Bound}

Following the hint, consider the following channel:
\begin{gather*}
    Z \sim \mathcal{N}\left(0, \frac{D\sigma^2}{\sigma^2-D}\right) \\
    \hat{X}=\frac{\sigma^2-D}{\sigma^2}(X+Z)
\end{gather*}
Since $X \parallel Z$,
\begin{align*}
    \Var(\hat{X})
     & = \frac{(\sigma^2-D)^2}{\sigma^4}(\Var(X)+\Var(Z))                                  \\
     & = \frac{(\sigma^2-D)^2}{\sigma^4}\left(\frac{D\sigma^2}{\sigma^2-D}+\sigma^2\right) \\
     & = \frac{(\sigma^2-D)^2}{\sigma^4} \cdot \frac{\sigma^4}{\sigma^2-D}                 \\
     & = \sigma^2-D
\end{align*}
and
\[h(\hat{X}) \le \frac{1}{2}\log(2 \pi e(\sigma^2-D))\]
This allows us to bound $I(X; \hat{X})$ like so:
\begin{align*}
    I(X; \hat{X})
     & = h(\hat{X}) - h(\hat{X} \mid X)                                                                         \\
     & = h(\hat{X}) - h(Z \mid X)                                                                               \\
     & = h(\hat{X}) - h(Z)                                                                                      \\
     & \le \frac{1}{2}\log(2 \pi e(\sigma^2-D)) - h(Z)                                                          \\
     & = \frac{1}{2}\log(2 \pi e(\sigma^2-D)) - \frac{1}{2}\log\left(2\pi e \frac{D\sigma^2}{\sigma^2-D}\right) \\
     & = \frac{1}{2}\log \frac{2\pi e (\sigma^2-D)}{2\pi e \frac{D\sigma^2}{\sigma^2-D}}                        \\
     & = \frac{1}{2}\log \frac{(\sigma^2-D)^2}{D\sigma^2}                                                       \\
     & \le \frac{1}{2}\log \frac{\sigma^2}{D}\quad\square
\end{align*}
The last line is true (assuming $D < \sigma^2$) because
\begin{align*}
               & \sigma^2 - D \le \sigma^2                               \\
    \implies{} & (\sigma^2-D)^2 \le \sigma^4                             \\
    \implies{} & \frac{(\sigma^2-D)^2}{\sigma^2} \le \sigma^2            \\
    \implies{} & \frac{(\sigma^2-D)^2}{D\sigma^2} \le \frac{\sigma^2}{D}
\end{align*}

It's not unreasonable to assume $D < \sigma^2$ because the past that
we can just always predict $\E[X]$ and on average we'll be off by the variance, which is $\sigma^2$.

\section{Conditional That Achieves \texorpdfstring{$R(D)$}{R(D)}}

uuuhhhh my initial answer was wronger than wrong so\dots

\pagebreak

\section{Simplicity is Best}

\subsection{Part A}\label{sec:10a}

The best we can do is $\frac{1}{2}\log\left(1+\frac{P}{N}\right)$ bits per channel use.

Since $R(D)=\frac{1}{2}\log\frac{P}{D}$, we can set the two to be equivalent to obtain
\begin{align*}
               & \frac{1}{2}\log\frac{P}{D}=\frac{1}{2}\log\left(1+\frac{P}{N}\right) \\
    \implies{} & \log\frac{P}{D}=\log\left(1+\frac{P}{N}\right)                       \\
    \implies{} & \frac{P}{D}=1+\frac{P}{N}                                            \\
    \implies{} & D=\boxed{\frac{NP}{N+P}}
\end{align*}

\subsection{Part B}

We have
\begin{align*}
    \lim_{P/N \to \infty} \frac{NP}{N+P}
     & = \lim_{P/N \to \infty} \frac{P}{1+\frac{P}{N}}                   \\
     & = \lim_{P/N \to \infty} N \cdot \frac{\frac{P}{N}}{1+\frac{P}{N}} \\
     & = N
\end{align*}
which is equivalent to the error when just sending $X$ raw.

\subsection{Part C}

We want to choose a to minimize this weird expression:
\begin{align*}
    \E\left[(a(X+Z)-X)^2\right]
    &= \E\left[a^2(X+Z)^2+X^2-2aX(X+Z)\right] \\
    &= \E\left[X^2(a^2+1-2a)+(2a^2-2a)XZ+a^2Z^2\right] \\
    &= (a-1)^2\E\left[X^2\right]+(2a^2-2a)\E[X]\E[Z]+a^2\E\left[Z^2\right]\\
    &= (a-1)^2\E\left[X^2\right]+a^2\E\left[Z^2\right] \\
    &= (a-1)^2P+a^2N
\end{align*}
Taking the derivative and setting it to $0$ gives $\boxed{a=\frac{P}{P+N}}$.
Plugging it back into the expression we got in \ref{sec:10a} indeed gives
an expected value of $\frac{NP}{N+P}$.

\pagebreak

\section{Properties of an Optimal Code}

I'll name the inequalities by their label in the textbook:
\begin{itemize}
    \item \textbf{10.58} pushes the output codewords to all be equally likely or
          as close to equally likely as possible.
    \item \textbf{10.59} forces $f_n$ to be deterministic, so knowing $X^n$ completely determines $f(X^n)$.
    \item \textbf{10.61} requires $g_n$ to not map multiple sequences of bits to the same codeword
          so no two elements in $f_n(X^n)$'s PMF are merged together.
    \item \textbf{10.65} wants $X_i$ to only depend on $\hat{X}_i$, so compression should be done by symbol.
          Not sure how this meshes with what we got at the end of lecture 13 with the Gaussian input.
    \item \textbf{10.67} forces $I(X; \hat{X})$ to be the lowest possible given its expected distortion.
    \item \textbf{10.69}, assuming $R(d)$ isn't linear anywhere, requires $\E[d(X_i, \hat{X}_i)]$ to be independent of $i$.
          In other words, it's unoptimal to heap distortion onto a single part of the codeword.
\end{itemize}

\end{document}
