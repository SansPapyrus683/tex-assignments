\documentclass[12pt]{article}

% a template that a friend gave, it's worked well enough for me
% i have added some packages and stuff that have proved useful

\usepackage{fancyhdr}
\usepackage{tipa}
\usepackage{fontspec}
\usepackage{amsfonts}
\usepackage{enumitem}
\usepackage[margin=1in]{geometry}
\usepackage{graphicx}
\usepackage{float}
\usepackage{amsmath}
\usepackage{braket}
\usepackage{amssymb}
\usepackage{booktabs}
\usepackage{hyperref}
\usepackage{mathtools}
\usepackage{xcolor}
\usepackage{float}
\usepackage{algpseudocodex}
\usepackage{titlesec}
\usepackage{bbm}
\usepackage{pythonhighlight}

\pagestyle{fancy}
\fancyhf{} % sets both header and footer to nothing
\lhead{Kevin Sheng}
\setmainfont{Comic Neue}
\renewcommand{\headrulewidth}{1pt}
\setlength{\headheight}{0.75in}
\setlength{\oddsidemargin}{0in}
\setlength{\evensidemargin}{0in}
\setlength{\voffset}{-.5in}
\setlength{\headsep}{10pt}
\setlength{\textwidth}{6.5in}
\setlength{\headwidth}{6.5in}
\setlength{\textheight}{8in}
\renewcommand{\headrulewidth}{0.5pt}
\renewcommand{\footrulewidth}{0.3pt}
\setlength{\textwidth}{6.5in}
\usepackage{setspace}
\usepackage{multicol}
\usepackage{float}
\setlength{\columnsep}{1cm}
\setlength\parindent{24pt}
\usepackage [english]{babel}
\usepackage [autostyle, english = american]{csquotes}
\MakeOuterQuote{"}

\setlength{\parskip}{6pt}
\setlength{\parindent}{0pt}

\titlespacing\section{0pt}{12pt plus 4pt minus 2pt}{0pt plus 2pt minus 2pt}
\titlespacing\subsection{0pt}{12pt plus 4pt minus 2pt}{0pt plus 2pt minus 2pt}
\titlespacing\subsubsection{0pt}{12pt plus 4pt minus 2pt}{0pt plus 2pt minus 2pt}

\hypersetup{colorlinks=true, urlcolor=blue}

\newcommand{\correction}[1]{\textcolor{red}{#1}}


\rhead{ECE 231A}

\newcommand{\ind}{\perp\!\!\!\!\perp} 
\newcommand{\ceil}[1]{\left\lceil#1\right\rceil}
\newcommand*{\ditto}{-''-}
\newcommand{\N}{\mathbb{N}}
\renewcommand{\c}{\texttt}
\DeclareMathOperator{\E}{E}

\begin{document}

\section{Resources}

I've used my own brain, posted slides/resources and a Discord chat with Kevin Zhao.

\section{Additive Noise Channel}

There's some casework to be done.

If $a \notin \{-1, 1\}$ then the possible outputs for $0$ and $1$ are disjoint and $\boxed{C=1}$.

If not, then WLOG consider $a=1$.
If $X$ takes $0$ with probability $p$, then the output probabilities are as follows:
\begin{itemize}
    \item $Y=0$ with probability $\frac{p}{2}$
    \item $Y=1$ with probability $\frac{p}{2}+\frac{1-p}{2}=\frac{1}{2}$
    \item $Y=2$ with probability $\frac{1-p}{2}$.
\end{itemize}
Then, calculating the mutual information gives us
\begin{align*}
    I(X; Y)
     & = H(Y) - H(Y \mid X)                                                                        \\
     & = -\frac{1}{2}\log\frac{1}{2}-\frac{p}{2}\log\frac{p}{2}-\frac{1-p}{2}\log\frac{1-p}{2} - 1 \\
     & = H(p)-\frac{1}{2}                                                                          \\
     & \le \frac{1}{2}
\end{align*}
Setting $p=\frac{1}{2}$ achieves this upper bound of $\boxed{C=\frac{1}{2}}$.

Notice that if $a=-1$, then it's the same thing with $0$ as the common value instead of $1$.

\pagebreak

\section{Inverse Erasure Channel}

Letting $p_i$ be the chance that $X$ takes on $i$, we have
\begin{align*}
    H(Y \mid X)
     & = p_1 \cdot 0 + p_2 \cdot 1 + p_3 \cdot 0 \\
     & = p_2
\end{align*}
and thus
\begin{align*}
    I(X; Y)
     & = H(Y) - H(Y \mid X) \\
     & = H(Y) - p_2         \\
     & \le 1 - p_2          \\
     & \le \boxed{1}
\end{align*}
This upper bound is easily achieved by setting $p_2=0$
(I don't think the values for $p_1$ and $p_3$ matter that much).

\section{Cyclic Symmetry}\label{sec:cyclic}

\subsection{Part A}

We have
\[P=\begin{bmatrix}
        1-a & a & 0   \\
        0   & a & 1-a
    \end{bmatrix}\]
where the columns correspond to $0$, $e$, and $1$ in that order.

\subsection{Part B}

The sets are
\begin{align*}
    S_1 = \left\{[1-a, 0]^T, [0, 1-a]^T\right\} &  & S_2 = \left\{[a, a]^T\right\}
\end{align*}

\pagebreak

\subsection{Part C}

Consider two distributions $x_1, \cdots, x_n$ and $x_1', \cdots, x_n'$
where $x_i'=x_{i-1}'$ (where subtraction is done cyclically, you get the idea).

It STP that these two distributions give us the same mutual information.

First of all, notice that since all the rows are permutations of each other,
\[H(Y \mid X) = \sum_{i=1}^{n} x_i H(\text{one row}) = H(\text{one row})\]
and the same goes for $X'$.

The resulting expression has no dependence on the pmf of the input distribution,
so now all we have to show is that $H(Y)$ for the two are the same.

Each $S_i$ corresponds to a nonempty set of output variables,
so I'll show that the elements within each $S_i$ don't change on a shift.

WLOG number the output alphabet within a single $S_i$ $y_1, \cdots, y_m$.
Let $p$ be the function that indicates a symbol's probability under $X$
and $p'$ be the same but for $X'$.

Consider any output element $y_g$.
Under $X$, the chance of it occuring is
\[p(y_g)=\sum_{i=1}^{n} p(x_i)M_{ig}\]
Now by our premise, there's a column (let's call it column $h$) that's a cyclic shift of $g$.

Notice that since $M_{ig}=M_{(i-1)h}$,
\begin{align*}
    p'(y_h)
     & =\sum_{i=1}^{n} p(x_i')M_{ih}         \\
     & = \sum_{i=1}^{n} p(x_{i-1})M_{(i-1)h} \\
     & = \sum_{i=1}^{n} P(x_i)M_{ig}         \\
     & = p(y_g)
\end{align*}
so each $y_g$'s probability under the original distribution is equivalent to
some $y_h$'s probability under the new distribution.

Since this is basically a "shifting" of the set's values, the overall probabilities of each $S_i$
and by extension $Y$ as a whole stay the same.
Thus, their entropies and mutual information are the same as well. $\square$

\pagebreak

\section{Errors, Erasures, \& Symmetry}

\subsection{Part A}

The transition matrix is
\[P=\begin{bmatrix}
        1-\alpha-p & \alpha & p          \\
        p          & \alpha & 1-\alpha-p
    \end{bmatrix}\]

\subsection{Part B}

As this matrix satisfies the conditions laid out in \ref{sec:cyclic},
the channel is indeed cyclically symmetric.

\subsection{Part C}

The only way this channel can be weakly symmetric is if
\[1-\alpha-p+p = 2\alpha \implies \alpha=\frac{1}{3}\]

\subsection{Part D}

Since this channel is cyclically symmetric, we can use the uniform distribution.
\begin{align*}
    I(X; Y)
     & = H(Y) - H(Y \mid X)                                                                                     \\
     & = h_3\left(\frac{1-\alpha}{2}, \alpha, \frac{1-\alpha}{2}\right) - h_3\left(1-\alpha-p, \alpha, p\right) \\
     & = -(1-\alpha)\log\frac{1-\alpha}{2}+p\log p+(1-\alpha-p)\log(1-\alpha-p)
\end{align*}

\subsection{Part E}

If $\alpha=0$, it's the BSC and the capacity becomes
\[-\log\frac{1}{2}+p\log p+(1-p)\log(1-p)=1-H(p)\]
which is the same as what the slides derived.

If $p=0$, it's the BEC.
The capacity becomes $C=1-\alpha$, which is also consistent.

\pagebreak

\section{Symmetric Channel?}

\subsection{Part A}

A channel's weakly symmetric if all the rows of its transition matrix
are permutations of each other and all the columns have the same sum.

For this channel,
\[P=\begin{bmatrix}
        \frac{1}{2} & \frac{1}{4} & \frac{1}{4} & 0           \\
        \frac{1}{2} & \frac{1}{4} & \frac{1}{4} & 0           \\
        0           & \frac{1}{4} & \frac{1}{4} & \frac{1}{2} \\
        0           & \frac{1}{4} & \frac{1}{4} & \frac{1}{2}
    \end{bmatrix}\]
The rows all have the same values, and all columns sum to $1$,
so this channel is weakly symmetric.

\subsection{Part B}

No, the channel isn't cyclically symmetric.

Consider $X=\left[\frac{1}{2}, \frac{1}{2}, 0, 0\right]$.

Here, the PMF of $Y=\left[\frac{1}{2}, \frac{1}{4}, \frac{1}{4}, 0\right]$.

Now, shifting $X$ to the right by one gives $X=\left[0, \frac{1}{2}, \frac{1}{2}, 0\right]$.

In this case, the PMF is $Y=\left[\frac{1}{4}, \frac{1}{4}, \frac{1}{4}, \frac{1}{4}\right]$.

\subsection{Part C}

Since this channel is weakly symmetric, we can use the uniform distribution:
\begin{align*}
    I(X; Y)
     & = H(Y) - H(Y \mid X)                                                                             \\
     & = -4 \cdot \frac{1}{4}\log\frac{1}{4} - h_4\left(\frac{1}{2}, \frac{1}{4}, \frac{1}{4}, 0\right) \\
     & = 2-\frac{3}{2}                                                                                  \\
     & = \boxed{\frac{1}{2}}
\end{align*}

\pagebreak

\section{Symmetric Channel Part II}

\subsection{Part A}

Here,
\[P=\begin{bmatrix}
        \frac{1}{2} & \frac{3}{8} & 0           & \frac{1}{8} \\
        0           & \frac{1}{2} & \frac{3}{8} & \frac{1}{8} \\
        \frac{3}{8} & 0           & \frac{1}{2} & \frac{1}{8}
    \end{bmatrix}\]
Since the first and last column sum to $\frac{7}{8}$ and $\frac{3}{8}$,
this channel isn't weakly symmetric.

\subsection{Part B}

However, since the first three columns are cyclic shifts of $\left[\frac{1}{2}, 0, \frac{3}{8}\right]$
and the last column is uniform.

This, combined with that all the rows are permutations of each other,
means that the conditions in \ref{sec:cyclic} are met and that
this channel is cyclically symmetric.

So regardless, we can use the uniform distribution:
\begin{align*}
    I(X; Y)
     & = H(Y) - H(Y \mid X)                                                           \\
     & = \boxed{h_4\left(\frac{7}{24}, \frac{7}{24}, \frac{7}{24}, \frac{1}{8}\right)
        -h_4\left(\frac{1}{2}, \frac{3}{8}, 0, \frac{1}{8}\right)}
\end{align*}

\subsection{Part C}

Since the rows are all permutations, $H(Y \mid X)$ is constant wrt $X$.

It remains to find an upper bound for $H(Y)$.
If we let the PMF for $X$ be $[a, b, c]$, then
\begin{align*}
    H(Y)
     & = h_4\left(\frac{a}{2}+\frac{3c}{8}, \frac{3a}{8}+\frac{b}{2}, \frac{3b}{8}+\frac{c}{2}, \frac{a+b+c}{8}\right) \\
     & = h_4\left(\frac{a}{2}+\frac{3c}{8}, \frac{3a}{8}+\frac{b}{2}, \frac{3b}{8}+\frac{c}{2}, \frac{1}{8}\right)     \\
     & \le h_4\left(\frac{7}{24}, \frac{7}{24}, \frac{7}{24}, \frac{1}{8}\right)
\end{align*}
Since the last term is fixed, the best we can do is the uniformly distribute the remaining probability.

From the previous part, we see that this upper bound of $H(Y)$ and by extension $C$
is indeed achievable.

\pagebreak

\section{Joint Typicality}

\subsection{Part A}

By the joint AEP, the chance that $\left(x_1^n, y_1^n\right) \in A_\epsilon^{(n)}$ tends to $1$ as $n \to \infty$.

\subsection{Part B}

Since the subscripts are different, $x_1^n$ and $y_2^n$ are independent.

Thus, using something of the joint AEP again gives us
\[P\left(\left(x_1^n, y_2^n\right) \in A_\epsilon^{(n)}\right) \le 2^{-n(I(X; Y) - 3\epsilon)}\]

\subsection{Part C}

If $p=0.03113$, then $I(X; Y) = 1 - H(p) \approx 0.8$ and
\[2^{-n(I(X; Y) - 3\epsilon)} \approx \boxed{8.899 \cdot 10^{-16}}\]

\subsection{Part D}

As $n \to \infty$, the exponent becomes arbitrarily large
in the negative direction and the upper bound tends to \boxed{0}.

\subsection{Part E}

This upper bound on the probability of a sequence being in the typical set
also serves as a lower bound on the chance of us confusing one message for another.

\pagebreak

\section{JT on a Channel with Two Outupts}

\subsection{Part A}

\subsubsection{Bounds of Interest}

Lemme just derive some bounds first.

We have these two by a process nearly identical to what was done in lecture:
\begin{gather*}
    \left|-\frac{1}{n}\log p(x^n) - H(X)\right| < \epsilon
    \implies p(x^n) < 2^{-n(H(X)-\epsilon)} \\
    \left|-\frac{1}{n}\log p(y^n, z^n) - H(Y, Z)\right| < \epsilon
    \implies p(y^n, z^n) < 2^{-n(H(Y, Z)-\epsilon)}
\end{gather*}
And this one requires a bit more work but is similar to how the
single-variable AEP works:
\begin{align*}
    \left|-\frac{1}{n}\log p(x^n, y^n, z^n) - H(X, Y, Z)\right| < \epsilon
     & \implies p(x^n, y^n, z^n) > 2^{-n(H(X, Y, Z)+\epsilon)}             \\
     & \implies \left|A_\epsilon^{(n)}\right| < 2^{n(H(X, Y, Z)+\epsilon)}
\end{align*}

\subsubsection{Actual UB}

Let $X^n$ and $(Y^n, Z^n)$ be independent of each other.

In that case,
\begin{align*}
    P\left((X^n, Y^n, Z^n) \in A_\epsilon^{(n)}\right)
     & = \sum_{(x^n, y^n, z^n) \in A_\epsilon^{(n)}} p(x^n)p(y^n, z^n) \\
     &< \left|A_\epsilon^{(n)}\right| 2^{-n(H(X)-\epsilon)} 2^{-n(H(Y, Z)-\epsilon)} \\
     &< 2^{n(H(X, Y, Z)+\epsilon)} 2^{-n(H(X)-\epsilon)} 2^{-n(H(Y, Z)-\epsilon)} \\
     &= 2^{-n(H(X)+H(Y,Z)-H(X,Y,Z)-3\epsilon)} \\
     &= \boxed{2^{-n(I(X; Y, Z) - 3\epsilon)}}
\end{align*}

\subsection{Part B}

Fix an $\epsilon > 0$.

\subsubsection{Strategy}

Like in lecture, let's select a set of random codes according to $P_i(x)$.
In other words, the chance that we use the symbol $x_j$ to encode any letter
for any specific message is $p_i(x_j)$.

The receiver decodes a message as being number $i$ iff
\begin{gather*}
    (\text{codeword for $i$}, y^n, z^n) \in A_\epsilon^{(n)} \\
    (\text{codeword for all $j \ne i$}, y^n, z^n) \notin A_\epsilon^{(n)}
\end{gather*}

\subsubsection{Proof This Works and I'm Not Schizo}

Consider any rate $R < I(X; Y, Z)$.

The expected error across all codes we randomly generate is
\begin{align*}
    \sum_{C} P(C) \cdot \text{expected err for $C$}
    &= \sum_{C} P(C) \cdot \frac{1}{2^{nR}}\sum_{i=1}^{2^{nR}} \lambda_i \\
    &= \frac{1}{2^{nR}} \sum_{C} \sum_{i=1}^{2^{nR}} P(C)\lambda_i \\
    &= \frac{1}{2^{nR}} \sum_{i=1}^{2^{nR}} \sum_{C} P(C)\lambda_i \\
    &= \sum_{C} P(C)\lambda_1 \\
    &= \text{expected prob. of an err if we transmit $1$}
\end{align*}

Let's also define the following events:
\[E_i = (X^n(i), Y^n, Z^n) \in A_\epsilon^{(n)}\quad i = 1, \cdots, 2^{nR}\]
in other words, $E_i$ is the event that the received message
is in the typical set assuming that $i$ was the transmitted message.

The expected probability of an error is thus
\begin{align*}
    \E[\text{err} \mid 1]
    &= P(E_1^C \cup E_2 \cup \cdots \cup E_{2^{nR}}) \\
    &\le P(E_1^C) + \sum_{i=2}^{2^{nR}} P(E_i) \\
    &\le \epsilon + \sum_{i=2}^{2^{nR}} 2^{-n(I(X; Y, Z) - 3\epsilon)}
\end{align*}
as long as we choose $n$ sufficiently large by the AEP.

Since this is the expected error across all codes,
there must be at least one code that has an average error probability
less than this as well, thus proving that the rate is achievable. $\square$

\pagebreak

\section{All Codes are Good}

\subsection{Markov's Inequality}

ok this proof is for a continuous rv but, like, surely it's the same for discrete, right? :clueless:

Let's just chain some inequalities:
\begin{align*}
    \E[X]
     & = \int_{0}^{\infty} xp(x)\,dx                                            \\
     & = \int_{0}^{\delta} xp(x)\,dx + \int_{\delta}^{\infty} xp(x)\,dx         \\
     & \ge \int_{0}^{\delta} xp(x)\,dx + \int_{\delta}^{\infty} \delta p(x)\,dx \\
     & \ge \int_{\delta}^{\infty} \delta p(x)\,dx                               \\
     & = \delta \int_{\delta}^{\infty} p(x)\,dx                                 \\
     & = \delta P(X \ge \delta)
\end{align*}
This gives us that $\delta P(X \ge \delta) \le \E[X]$,
which when moving $\delta$ to the other side gives the desired inequality. $\square$

\subsection{Implications for Channel Errors}

The statement doesn't really define what "random" means, so I'm just gonna assume
we randomly generate them the same way we did in lecture.

Fix our error target $\delta > 0$.

If $X$ is the average error rate of a randomly generated code, then
$P(X \ge \delta)$ represents the probability that it exceeds the error target.

$\E[X]$ is the expected average error across all codes, and in lecture
we proved that we can make it arbitrarily small as the block length goes to infinity.

Since $\E[X] \to 0$ and $\delta$ is fixed, then $P(X \ge \delta)$ is bounded above
by some vanishing term according to Markov's inequality. $\square$

\end{document}
