\documentclass[12pt]{article}

% a template that a friend gave, it's worked well enough for me
% i have added some packages and stuff that have proved useful

\usepackage{fancyhdr}
\usepackage{tipa}
\usepackage{fontspec}
\usepackage{amsfonts}
\usepackage{enumitem}
\usepackage[margin=1in]{geometry}
\usepackage{graphicx}
\usepackage{float}
\usepackage{amsmath}
\usepackage{braket}
\usepackage{amssymb}
\usepackage{booktabs}
\usepackage{hyperref}
\usepackage{mathtools}
\usepackage{xcolor}
\usepackage{float}
\usepackage{algpseudocodex}
\usepackage{titlesec}
\usepackage{bbm}
\usepackage{pythonhighlight}

\pagestyle{fancy}
\fancyhf{} % sets both header and footer to nothing
\lhead{Kevin Sheng}
\setmainfont{Comic Neue}
\renewcommand{\headrulewidth}{1pt}
\setlength{\headheight}{0.75in}
\setlength{\oddsidemargin}{0in}
\setlength{\evensidemargin}{0in}
\setlength{\voffset}{-.5in}
\setlength{\headsep}{10pt}
\setlength{\textwidth}{6.5in}
\setlength{\headwidth}{6.5in}
\setlength{\textheight}{8in}
\renewcommand{\headrulewidth}{0.5pt}
\renewcommand{\footrulewidth}{0.3pt}
\setlength{\textwidth}{6.5in}
\usepackage{setspace}
\usepackage{multicol}
\usepackage{float}
\setlength{\columnsep}{1cm}
\setlength\parindent{24pt}
\usepackage [english]{babel}
\usepackage [autostyle, english = american]{csquotes}
\MakeOuterQuote{"}

\setlength{\parskip}{6pt}
\setlength{\parindent}{0pt}

\titlespacing\section{0pt}{12pt plus 4pt minus 2pt}{0pt plus 2pt minus 2pt}
\titlespacing\subsection{0pt}{12pt plus 4pt minus 2pt}{0pt plus 2pt minus 2pt}
\titlespacing\subsubsection{0pt}{12pt plus 4pt minus 2pt}{0pt plus 2pt minus 2pt}

\hypersetup{colorlinks=true, urlcolor=blue}

\newcommand{\correction}[1]{\textcolor{red}{#1}}


\rhead{ECE 231A}

\newcommand{\ind}{\perp\!\!\!\!\perp} 

\begin{document}

\section{Resources}

I've used my own brain, posted slides/resources and some yapping with Kevin Zhao.

\section{Coin Flips}

\subsection{Part A}

The chance that we need $n$ flips is $(1-p)^{n-1}p$.

Thus,
\begin{align*}
  H(X)
   & = -\sum_{n=1}^{\infty} (1-p)^{n-1}p \log \left((1-p)^{n-1}p\right)                                      \\
   & = -p \sum_{n=1}^{\infty} (1-p)^{n-1} \left(\log (1-p)^{n-1} + \log p\right)                             \\
   & = -p \left[\sum_{n=1}^{\infty} (n-1)(1-p)^{n-1}\log(1-p) + \log p\sum_{n=1}^{\infty} (1-p)^{n-1}\right] \\
   & = -p \left[\log(1-p)\sum_{n=1}^{\infty} n(1-p)^n + \log p \sum_{n=1}^{\infty} (1-p)^n + \log p\right]   \\
   & = -p \left[\log(1-p) \cdot \frac{1-p}{p^2} + \log p \cdot \frac{1-p}{p}+\log p\right]                   \\
   & = \frac{(p-1)\log(1-p)}{p}-\log p \\
   &= \frac{-q\log q - p\log p}{p} \\
   &= \boxed{\frac{H(p)}{p}}
\end{align*}

\pagebreak

\subsection{Part B}

Plugging $p=\frac{1}{2}$ into the expression above gives us
\[-\log \frac{1}{2} - \log \frac{1}{2} = \boxed{2}\]

\subsection{Part C}

Initially, the best question to ask is whether $X=1$ or not,
since we eliminate exactly half the possible outcomes in terms of probability.

Of the remaining outcomes, $X=2$ takes up another half, so if $X \ne 1$
the best question to ask after that is if it's $2$ or not.

We can follow this procedure to get what I believe is the most efficient way
to find the true value of $X$, as we eliminate in the worst case half the search space each time.

The expected number of questions is:
\begin{align*}
  E[\text{\# of Qs}]
   & = 1 \cdot \frac{1}{2} + 2 \cdot \frac{1}{2^2} + 3 \cdot \frac{1}{2^3} \cdots \\
   & = \sum_{n=1}^{\infty} n \cdot \frac{1}{2^n}                                  \\
   & = \frac{1/2}{1/4}                                                            \\
   & = \boxed{2}
\end{align*}
which is equivalent to the entropy we got above.

\pagebreak

\section{Entropy of Functions of an RV}

\subsection{Part A}

$H(X, g(X)) = H(X) + H(g(X) \mid X)$ by the chain rule for entropy.

\subsection{Part B}

We have that
\[H(g(X) \mid X) = -\sum_{x \in X} \sum_{y \in g(X)} p(y, x) \log p(y \mid x)\]
Notice that for the term in the summation, there's two possible cases:
\begin{enumerate}
  \item $g(x)=y$, so $p(y \mid x) = 1$ and $\log p(y \mid x) = 0$.
  \item $g(x) \ne y$, so $p(y \mid x) = 0$.
\end{enumerate}
Either way, all terms in the summation come out to $0$ and $H(g(X) \mid X) = 0$.

\subsection{Part C}

Here it doesn't matter in which order we apply the chain rule, so we take $H(g(X))$ out first.

\subsection{Part D}

Entropy is always nonnegative, so this inequality holds trivially.

\pagebreak

\section{Computing 2D Relative Entropy}

We can think of this as a single random variable with four outcomes instead of
two random variables with two outcomes.

Doing so allows us to just directly apply the formula like so:
\begin{align*}
  D(p \parallel q)
   & = \frac{1}{6}\log\frac{1/6}{1/4}+\frac{7}{12}\log\frac{7/12}{1/2}
  +\frac{1}{6}\log\frac{1/6}{1/12}+\frac{1}{12}\log\frac{1/12}{1/6}            \\
   & = \frac{1}{6}\log\frac{2}{3} + \frac{7}{12}\log\frac{7}{6}
  + \frac{1}{6}\log 2 + \frac{1}{12} \log \frac{1}{2}                          \\
   & = \frac{1}{6}\log\frac{2}{3} + \frac{7}{12}\log\frac{7}{6} + \frac{1}{12} \\
   & \approx \boxed{0.116}
\end{align*}

\section{Relative Entropy On a Line}

\subsection{Part A}

We start at the border and go to the center like so:
\begin{center}
  \includegraphics[width=5cm]{img/hw1/simplex}
\end{center}

\pagebreak

\subsection{Part B}

Applying the formula gets us
\begin{align*}
  D(p \parallel q_\lambda)
   & = \frac{1}{3}\left(\log\frac{1/3}{\lambda/3} + \log\frac{1/3}{(3-\lambda)/6} + \log\frac{1/3}{(3-\lambda)/6}\right) \\
   & = \frac{1}{3}\left(2 - \log \lambda - \log (3-\lambda) - \log (3-\lambda)\right)                                    \\
   & = \frac{1}{3}\left(2-\log \lambda - 2\log(3-\lambda)\right)
\end{align*}

Plotting it gives
\begin{center}
  \includegraphics[width=9cm]{img/hw1/p_diff_q}
\end{center}
At $\lambda=0$, the line tends towards positive infinity.

The code for this was pretty simple, just
\begin{verbatim}
x = np.linspace(0, 1, 10000)
y = 1/3 * (2 - np.log2(x) - 2 * np.log2(3 - x))

plt.plot(x, y)
plt.xlabel("lambda")
plt.ylabel("D(p || q)")
plt.show()
\end{verbatim}
followed by some Matplotlib commands.

\subsection{Part C}

This time we've
\begin{align*}
  D(q_\lambda \parallel p)
   & = \frac{\lambda}{3}\log\frac{\lambda/3}{1/3} + \frac{3-\lambda}{6}\log\frac{(3-\lambda)/6}{1/3} + \frac{3-\lambda}{6}\log\frac{(3-\lambda)/6}{1/3} \\
   & = \frac{\lambda}{3}\log \lambda + \frac{3-\lambda}{3}\log\frac{3-\lambda}{2}
\end{align*}
and plotting that gives
\begin{center}
  \includegraphics[width=9cm]{img/hw1/q_diff_p}
\end{center}
Here, the graph converges to $\log \frac{3}{2}$ instead of infinity.
This is because $\lim_{x \to 0} x \log x = 0$.

The code for this graph was the same as that in the previous part.
Just replace the second line with
\begin{verbatim}
y = x/3 * np.log2(x) + (3 - x) / 3 * np.log2((3 - x) / 2)
\end{verbatim}

\subsection{Part D}

While both have the same general shape
(decreasing as $\lambda$ increases and reaching $0$ when it hits $1$),
the two have different behavior as $\lambda$ becomes small.

This is because using $q_\lambda$ at small $\lambda$ to represent $p$ becomes
"impossible" in a sort of manner as sampling $x$ from $q$ becomes impossible.

On the other hand, using $p$ to represent $q_\lambda$ is still feasible
in some sort of way.
Here, sampling an $x$ from $p$ can be thought of as just "garbage" data.

\pagebreak

\section{Mutual Information?}

In this case, $D(p \parallel q) = I(p; q)$, since each entry in $q$
is equivalent to $p(x)p(y)$.

\section{Joint Entropy Example}

\subsection{Part A}

\begin{align*}
  H(X)
   & = -\frac{2}{3}\log\frac{2}{3}-\frac{1}{3}\log\frac{1}{3} \\
   & = -\frac{2}{3}(1 - \log 3) + \frac{1}{3}\log 3           \\
   & = \boxed{\log 3 - \frac{2}{3}}
\end{align*}

Since the distribution of $Y$ is functionally identical to that of $X$,
$H(Y)$ is equal to that as well.

\subsection{Part B}

We can use the result from \ref{sec:7c} and the chain rule to get
\[H(X \mid Y) = H(Y \mid X) = \boxed{\frac{2}{3}}\]

\subsection{Part C}\label{sec:7c}

\[H(X, Y) = -3 \cdot \frac{1}{3}\log\frac{1}{3} = \boxed{\log 3}\]

\subsection{Part D}\label{sec:7d}

\[H(Y) - H(Y \mid X) = \log 3 - \frac{4}{3}\]

\subsection{Part E}

$I(X; Y)$ is exactly what the expression in \ref{sec:7d} describes.

\pagebreak

\subsection{Part F}

The diagram on the slides already does this...
\begin{center}
  \includegraphics[width=10cm]{img/hw1/venn}
\end{center}

The only thing that maybe should be added is that $I(X; Y) = H(Y) - H(Y \mid X)$.

\pagebreak

\section{Mutual Info and the Weather}

\subsection{Part A}

\begin{align*}
  I(P_S; W)
   & = H(W) - H(W \mid P_S)                                       \\
   & = -(0.1\log 0.1 + 0.9\log 0.9) + (0.1\log 0.1 + 0.9\log 0.9) \\
   & = \boxed{0}
\end{align*}

\subsection{Part B}

The probability table looks like this:
\begin{table}[H]
  \centering
  \begin{tabular}{|c|c|c|}
    \hline
    $P_W$/$W$ & R    & S   \\
    \hline
    R         & 0.75 & 0   \\
    \hline
    S         & 0.15 & 0.1 \\
    \hline
  \end{tabular}
\end{table}

So Wendy's mutual information is:
\begin{align*}
  I(P_W; W)
   & = H(W) - H(W \mid P_W)                                                                                   \\
   & = -(0.1\log 0.1 + 0.9\log 0.9) + \left(0.15 \cdot \log\frac{3}{5} + 0.1 \cdot \log\frac{2}{5}\right) \\
   & \approx \boxed{0.226}
\end{align*}

\subsection{Part C}

Wendy probably provides more info about the weather since she has higher mutual info.

\subsection{Part D}

When Wendy predicts rain, I know 100\% it's going to be rain, so I'd go with her
over Stormy, who has a 10\% chance of being wrong.

\pagebreak

\section{Why Log?}

\subsection{Part A}

I'll use $S(n)$ as a shorthand for $\sum_{i=1}^{n} p_i$.

The proof is by induction on both $k$ and $m$; the base case is given as the axiom.

The truth of the statement depends on the sequence and amount we group.
I'll represent these two dependencies as a 2-tuple: $(\text{length}, \text{group amt})$.
I'll prove that if this property holds for $(m, k-1)$ and $(k, k-1)$,
then it holds for $(m, k)$.

Assuming truth for $k-1$, for $k$ we have
\begin{align*}
      & H_m(p_1, \cdots, p_m)                                                    \\
  ={} & H_{m-k+2}(S(k-1), p_k, \cdots, p_m)
  + S(k-1)H_{k-1}\left(\frac{p_1}{S(k-1)}, \cdots, \frac{p_{k-1}}{S(k-1)}\right) \\
  ={} & H_{m-k+1}(S(k), p_{k+1}, \cdots, p_m)
  + S(k)H_2\left(\frac{S(k-1)}{S(k)}, \frac{p_k}{S(k)}\right)
  + \text{that term above}
\end{align*}

The first term is what we want.
To combine the second terms, notice that
\begin{align*}
      & S(k)H_k\left(\frac{p_1}{S(k)}, \cdots, \frac{p_k}{S(k)}\right)                               \\
  ={} & S(k)\left(H_2\left(\frac{S(k-1)}{S(k)}, \frac{p_k}{S(k)}\right)
  + \frac{S(k-1)}{S(k)}H_{k-1}\left(\frac{p_1}{S(k-1)}, \cdots, \frac{p_{k-1}}{S(k-1)}\right)\right) \\
  ={} & S(k)H_2\left(\frac{S(k-1)}{S(k)}, \frac{p_k}{S(k)}\right)
  + S(k-1)H_{k-1}\left(\frac{p_1}{S(k-1)}, \cdots, \frac{p_{k-1}}{S(k-1)}\right)
\end{align*}
since we also assumed the property holds for $(k, k-1)$.

This implication is enough to carry truth over to all $2 \le k < m$,
since truth is assumed for all $m \in \mathbb{N}$ and $k=2$ initially. $\square$

\pagebreak

\subsection{Part B}

We repeatedly group terms like so:
\begin{align*}
  f(mn)
   & = H_{mn}\left(\frac{1}{mn}, \cdots, \frac{1}{mn}\right)                                                   \\
   & = H_{(m-1)n+1}\left(\frac{1}{m}, \frac{1}{mn}, \cdots, \frac{1}{mn}\right) + \frac{1}{m}f(n)              \\
   & = H_{(m-2)n+2}\left(\frac{1}{m}, \frac{1}{m}, \frac{1}{mn}, \cdots, \frac{1}{mn}\right) + \frac{2}{m}f(n) \\
   & = \cdots                                                                                                  \\
   & = H_m\left(\frac{1}{m}, \cdots, \frac{1}{m}\right) + \frac{m}{m}f(n)                                      \\
   & = f(m) + f(n)\quad\square
\end{align*}

\subsection{Part C}

If we were to use the natural logarithm, then
\[H\left(\frac{1}{2}, \frac{1}{2}\right) = 2 \cdot \frac{1}{2} \ln 2 = \ln 2\]
so instead of $1$ it would have to be $\ln 2$ instead.

\pagebreak

\section{Concavity of Entropy}

\subsection{Part A}

Nowhere did it say we couldn't use the second derivative test, so...
\begin{align*}
  \frac{d}{dx} \log_b x
   & = \frac{\frac{d}{dx} \ln x}{\ln b}     \\
   & = \frac{1}{x\ln b}                     \\
  \frac{d^2}{dx^2} \log_b x
   & = \frac{1}{\ln b} \cdot -\frac{1}{x^2}
\end{align*}
which is strictly negative for all $x > 0$. $\square$

\subsection{Part B}

Same thing.
\begin{align*}
  \frac{d}{dx} x \log_b x
   & = \frac{1}{\ln b} \frac{d}{dx} x \ln x       \\
   & = \frac{\ln x + 1}{\ln b}                    \\
  \frac{d^2}{dx^2} x \log_b x
   & = \frac{1}{x}                                \\
   & > 0\ \forall \text{positive $x$}\quad\square
\end{align*}

\subsection{Part C}

Letting the base be $b$ again, the first derivative is
\[H'(p) = -\frac{\ln p + 1}{\ln b} - \frac{-\ln(1-p)-1}{\ln b}\]
and the second is
\begin{align*}
  H''(p)
   & = -\frac{\frac{1}{p}}{\ln b} - \frac{\frac{1}{1-p}}{\ln b} \\
   & = \frac{\frac{1}{1-p}+\frac{1}{p}}{\ln b}
\end{align*}
When $0 < p < 1$, both of the addends in the numerator are positive. $\square$

\pagebreak

\section{Max and Min Entropy}

\subsection{Part A}

The maximum entropy achievable is $\log n$.
This is only achievable when $p_i=\frac{1}{n}$.

For $n=3$, the point is here:
\begin{center}
  \includegraphics[width=8cm]{img/hw1/max_entropy}
\end{center}

\subsection{Part B}

The minimum entropy is $0$, and it occurs when all $p_i$ are $0$ except for one.

On the simplex, these correspond to the vertices.

\pagebreak

\section{Urn Entropy}

Let $B_i$ denote the color of the $i$th ball drawn \textit{with} replacement
and $B_i'$ the same thing but \textit{without} replacement.

Notice that without any conditioning, $B_i$ and $B_i'$ have the same distribution
and thus the same entropy.

By the chain rule of entropy,
\begin{align*}
  H(B_1, \cdots, B_k)
  &= \sum_{i=1}^{k} H(B_i \mid B_1, \cdots, B_{i-1}) \\
  &= \sum_{i=1}^{k} H(B_i) \\
  &= kH(B_1)
\end{align*}
since all $B_i$s are independent and have the same distribution.

As for the one without replacement, we can only say that
\[H(B_1', \cdots, B_k') = \sum_{i=1}^{k} H(B_i' \mid B_1', \cdots, B_{i-1}')\]
However, since $B_i'$ and its predecessors aren't independent,
\[H(B_i' \mid B_1', \cdots, B_{i-1}') < H(B_i') = H(B_i)\]
so the entropy of drawing balls without replacement is lower
than that of drawing balls with.

\pagebreak

\section{Conditioning Increases Entropy?}

\subsection{Part A}

$X_2$ has a $\frac{2}{3}$ chance to be black and $\frac{1}{3}$ chance to be white, so
\[H(X_2)=-\frac{1}{3}\log\frac{1}{3}-\frac{2}{3}\log\frac{2}{3}=\boxed{\log 3 - \frac{2}{3}}\]

\subsection{Part B}

If the first ball is black, then
\[H(X_2 \mid X_1=\text{black})=-2 \cdot \frac{1}{2}\log \frac{1}{2} = 1\]
but if it was white, then there's only one outcome and
\[H(X_2 \mid X_1=\text{white})=-1\log 1 = 0\]
Something notable is that $H(X_2 \mid X_1=\text{black}) > H(X_2)$.

\subsection{Part C}

\[H(X_2 \mid X_1) = \frac{2}{3} \cdot 1 + \frac{1}{3} \cdot 0 = \boxed{\frac{2}{3}}\]

\pagebreak

\section{Conditional Mutual Information}

\subsection{Part A}\label{sec:14a}

By the chain rule of mutual information, we have
\[I(X; Y, Z) = I(X; Z) + I(X; Y \mid Z) = I(X; Y) + I(X; Z \mid Y)\]

Since $X \rightarrow Y \rightarrow Z$, $I(X; Z \mid Y) = 0$ and
\[I(X; Z) + I(X; Y \mid Z) = I(X; Y) \implies I(X; Y \mid Z) \le I(X; Y)\]
since mutual information is nonnegative. $\square$

\subsection{Part B}

No- here's a case where $I(X; Y \mid Z) > I(X; Y)$.

Consider the XOR example from class, where $X$ and $Y$ are both
random coin flips taking $0$ or $1$ and $Z = X \oplus Y$.

Since $X \ind Y$, $I(X; Y) = 0$.
However,
\begin{align*}
  I(X; Y \mid Z)
  &= H(X) - H(X \mid Y, Z) \\
  &= 1-0 \\
  &= 1
\end{align*}
since knowing $Y$ and $Z$ allows us to deduce with 100\% certainty what $X$ is.

\section{Find the Gap}

By that same equation we wrote in \ref{sec:14a},
\[I(X; Y) - I(X; Z) = I(X; Y \mid Z)\]

\pagebreak

\section{Mixing Increases Entropy}

The entropy of $(p_1, \cdots, p_i, \cdots, p_j, \cdots, p_m)$ is just $\sum_{k=1}^{m} p_k \log p_k$.

OTOH, the entropy of $\left(p_1, \cdots, \frac{p_i+p_j}{2}, \cdots, \frac{p_i+p_j}{2}, \cdots, p_m\right)$ is
basically the exact same, except for the terms at $i$ and $j$.
At those terms it's $\frac{p_i+p_j}{2}\log\frac{p_i+p_j}{2}$.

In other words, it suffices to show that
\[p_i \log p_i + p_j \log p_j \le 2 \cdot \frac{p_i+p_j}{2}\log\frac{p_i+p_j}{2}\]

Applying the Log-Sum inequality with the following sequences:
\begin{align*}
  a_1=p_i, a_2=p_j && b_1=b_2=1
\end{align*}
yields
\[p_i \log \frac{p_i}{1} + p_j \log \frac{p_j}{1} \ge (p_i+p_j)\log\frac{p_i+p_j}{1+1}\]
which when simplified yields exactly what we want. $\square$

\end{document}
