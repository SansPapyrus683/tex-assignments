\documentclass[12pt]{article}

% a template that a friend gave, it's worked well enough for me
% i have added some packages and stuff that have proved useful

\usepackage{fancyhdr}
\usepackage{tipa}
\usepackage{fontspec}
\usepackage{amsfonts}
\usepackage{enumitem}
\usepackage[margin=1in]{geometry}
\usepackage{graphicx}
\usepackage{float}
\usepackage{amsmath}
\usepackage{braket}
\usepackage{amssymb}
\usepackage{booktabs}
\usepackage{hyperref}
\usepackage{mathtools}
\usepackage{xcolor}
\usepackage{float}
\usepackage{algpseudocodex}
\usepackage{titlesec}
\usepackage{bbm}
\usepackage{pythonhighlight}

\pagestyle{fancy}
\fancyhf{} % sets both header and footer to nothing
\lhead{Kevin Sheng}
\setmainfont{Comic Neue}
\renewcommand{\headrulewidth}{1pt}
\setlength{\headheight}{0.75in}
\setlength{\oddsidemargin}{0in}
\setlength{\evensidemargin}{0in}
\setlength{\voffset}{-.5in}
\setlength{\headsep}{10pt}
\setlength{\textwidth}{6.5in}
\setlength{\headwidth}{6.5in}
\setlength{\textheight}{8in}
\renewcommand{\headrulewidth}{0.5pt}
\renewcommand{\footrulewidth}{0.3pt}
\setlength{\textwidth}{6.5in}
\usepackage{setspace}
\usepackage{multicol}
\usepackage{float}
\setlength{\columnsep}{1cm}
\setlength\parindent{24pt}
\usepackage [english]{babel}
\usepackage [autostyle, english = american]{csquotes}
\MakeOuterQuote{"}

\setlength{\parskip}{6pt}
\setlength{\parindent}{0pt}

\titlespacing\section{0pt}{12pt plus 4pt minus 2pt}{0pt plus 2pt minus 2pt}
\titlespacing\subsection{0pt}{12pt plus 4pt minus 2pt}{0pt plus 2pt minus 2pt}
\titlespacing\subsubsection{0pt}{12pt plus 4pt minus 2pt}{0pt plus 2pt minus 2pt}

\hypersetup{colorlinks=true, urlcolor=blue}

\newcommand{\correction}[1]{\textcolor{red}{#1}}


\rhead{ECE 231A}

\newcommand{\ind}{\perp\!\!\!\!\perp} 
\newcommand*{\ditto}{-''-}
\renewcommand{\c}{\texttt}
\DeclareMathOperator{\E}{E}

\begin{document}

\section{Resources}

I've used my own brain, posted slides/resources and a Discord chat with Kevin Zhao.

\section{Fano's Without Conditioning}

\subsection{Part A}

Assuming that $p_1 \ge \frac{1}{m}$, the best way to choose the remaining elements
is to set $p_i = \frac{1-p_1}{m-1}$.

The entropy of this is then
\begin{align*}
    H(X)
     & = -p_1 \log p_1 - (m-1) \cdot \frac{1-p_1}{m-1} \log\frac{1-p_1}{m-1} \\
     & = -p_1 \log p_1 - (1-p_1) \log\frac{1-p_1}{m-1}                       \\
     & = -p_1 \log p_1 - (1-p_1)\log(1-p_1) + (1-p_1)\log(m-1)               \\
     & = H(p_1) + P_e\log(m-1)                                               \\
     & < \boxed{H(P_e) + P_e\log m}
\end{align*}

\subsection{Part B}

$P_e$ represents a binary outcome, so its entropy is at most $1$.

Replacing $H(P_e)$ with that in the inequality gives
\begin{align*}
               & H(X) < 1 + P_e \log m       \\
    \implies{} & H(X) - 1 < P_e \log m       \\
    \implies{} & P_e > \frac{H(X)-1}{\log m}
\end{align*}

\pagebreak

\section{Differential Entropy}

I LOVE INTEGRALS

\subsection{Part A}

We have
\begin{align*}
    h(X)
     & = -\int_{0}^{\infty} \lambda e^{-\lambda x} \cdot \ln\left(\lambda e^{-\lambda x}\right)\,dx                \\
     & = -\int_{0}^{\infty} \lambda e^{-\lambda x} \cdot \left(-\lambda x + \ln \lambda\right)\,dx                 \\
     & = \int_{0}^{\infty} \lambda^2 x e^{-\lambda x}\,dx-\int_{0}^{\infty} \lambda \ln \lambda e^{-\lambda x}\,dx \\
     & = \int_{0}^{\infty} \lambda^2 x e^{-\lambda x}\,dx-\intval{-\ln \lambda e^{-\lambda x}}^\infty_0            \\
     & = \intval{-e^{-\lambda x}(\lambda x + 1)}^\infty_0 - \ln \lambda                                            \\
     & = \boxed{1 - \ln \lambda}
\end{align*}
where the left-hand integral was done using integration by parts.

\subsection{Part B}

This distribution is symmetric around $0$, so it suffices
to calculate the entropy for positive $x$:
\begin{align*}
    h(X)
     & = -\int_{0}^{\infty} \frac{1}{2} \lambda e^{-\lambda x} \cdot \ln\left(\frac{1}{2} \lambda e^{-\lambda x}\right)\,dx                      \\
     & = -\frac{1}{2} \int_{0}^{\infty} \lambda e^{-\lambda x} \cdot \left(-\lambda x + \ln \frac{\lambda}{2}\right)\,dx                         \\
     & = \frac{1}{2}\int_{0}^{\infty} \lambda^2 x e^{-\lambda x} - \frac{1}{2}\int_{0}^{\infty} \ln \frac{\lambda}{2} \lambda e^{-\lambda x}\,dx \\
     & = \frac{1}{2} - \frac{1}{2}\ln \frac{\lambda}{2}
\end{align*}
Doubling this yields the answer of \boxed{1-\ln \frac{\lambda}{2}}.

\subsection{Part C}

The sum of $X_1$ and $X_2$ is also normal, with mean $\mu_1 + \mu_2$
and variance $\sigma_1^2 + \sigma_2^2$.

Thus, the differential entropy is
\boxed{\frac{1}{2}\log\left(2\pi e \left(\sigma_1^2+\sigma_2^2\right)\right)}.

\section{Exponential Channel}

\subsection{Part A}

Let $p$ be the chance that $X=0$.

Then, surely we just combine the two:
\[f_Y(y)=\begin{cases}
        \frac{3pe^{-y}}{2}     & 0 \le y < \ln \frac{3}{2}     \\
        \frac{3e^{-y}}{2}      & \ln \frac{3}{2} \le y < \ln 3 \\
        \frac{3(1-p)e^{-y}}{2} & \ln 3 \le y                   \\
        0                      & \text{otherwise}
    \end{cases}\]

And now to show that this integrates to $1$:
\begin{align*}
        & \int_{0}^{\infty} f_Y(y)\,dy                     \\
    ={} & \int_{0}^{\ln 3/2} \frac{3pe^{-y}}{2}\,dy
    + \int_{\ln 3/2}^{\ln 3} \frac{3e^{-y}}{2}\,dy
    + \int_{\ln 3}^{\infty} \frac{3(1-p)e^{-y}}{2}\,dy     \\
    ={} & \intval{-\frac{3pe^{-y}}{2}}^{\ln 3/2}_0
    + \intval{-\frac{3e^{-y}}{2}}^{\ln 3}_{\ln 3/2}
    + \intval{-\frac{3(1-p)e^{-y}}{2}}^{\infty}_{\ln 3}    \\
    ={} & \left(\frac{3p}{2}-\frac{3p \cdot 2/3}{2}\right)
    + \left(\frac{3 \cdot 2/3}{2}-\frac{3 \cdot 1/3}{2}\right)
    + \frac{3(1-p) \cdot \frac{1}{3}}{2}                   \\
    ={} & \frac{p}{2} + \frac{1}{2} + \frac{1-p}{2}        \\
    ={} & 1
\end{align*}
Incredible, I'm the goat.

\subsection{Part B}

For the sake of consistency I'm going to use the natural log here:
\[H(X)=\ln 2 \approx 0.693\]

\pagebreak

\subsection{Part C}

\subsubsection{Case 1: \texorpdfstring{$0 \le y \le \ln \frac{3}{2}$}{0 <= y <= ln 3/2}}

$X$ is always going to be $0$ in this case, so $H(X)=0$.

\subsubsection{Case 2: \texorpdfstring{$\ln \frac{3}{2} \le y \le \ln 3$}{ln 3/2 <= y <= ln 3}}

This is the only case that's interesting.

The chance that $X$ is $0$ is
\[\frac{\frac{3pe^{-y}}{2}}{\frac{3pe^{-y}}{2}+\frac{3(1-p)e^{-y}}{2}} = \frac{3pe^{-y}}{3e^{-y}} = p\]
so in other words $H(X \mid Y = y)$ for $\ln \frac{3}{2} \le y \le \ln 3$ is just $H(X)$.

\subsubsection{Case 3: \texorpdfstring{$\ln 3 <= y$}{ln 3 <= y}}

$X$ is always going to be $1$ here, so $H(X)=0$.

\subsection{Part D}

We have
\begin{align*}
    P\left(\ln \frac{3}{2} \le Y \le \ln 3\right)
     & = \int_{\ln 3/2}^{\ln 3} \frac{3e^{-y}}{2}\,dy \\
     & = \frac{1}{2}
\end{align*}
and since the entropies in other regions are $0$, $H(X \mid Y) = \boxed{\frac{1}{2}H(X)}$.

\subsection{Part E}\label{sec:4e}

If $X=\left(\frac{1}{2}, \frac{1}{2}\right)$,
\begin{align*}
    I(X; Y)
     & = H(X) - H(X \mid Y)   \\
     & = \frac{\ln 2}{2}      \\
     & \approx \boxed{0.3466}
\end{align*}

\subsection{Part F}

Oh god.
I'm assuming $X$ is still equally likely to go either way.

Let's do the parts one by one:
\begin{align*}
    -\int_{0}^{\ln 3/2} \frac{3e^{-y}}{4} \ln \frac{3e^{-y}}{4}\,dy
     & = \intval{\frac{3}{4}e^{-y}\left(\ln\frac{3e^{-y}}{4}-1\right)}^{\ln 3/2}_0 \\
     & \approx 0.119
\end{align*}

The integral is evaluated similarly for the second one:
\begin{align*}
    -\int_{\ln 3/2}^{\ln 3} \frac{3e^{-y}}{2} \ln \frac{3e^{-y}}{2}\,dy
     & = \intval{\frac{3}{2}e^{-y}\left(\ln\frac{3e^{-y}}{2}-1\right)}^{\ln 3}_{\ln 3/2} \\
     &\approx 0.153
\end{align*}

and the third one:
\begin{align*}
    -\int_{\infty}^{\ln 3} \frac{3e^{-y}}{4} \ln \frac{3e^{-y}}{4}\,dy
     & = \intval{\frac{3}{4}e^{-y}\left(\ln\frac{3e^{-y}}{4}-1\right)}^{\infty}_{\ln 3} \\
     & \approx 0.597
\end{align*}

Adding these three together should get us the answer of
\[0.119+0.153+0.597=\boxed{0.869}\]

\pagebreak

\subsection{Part G}

There's two integrals we have evaluate to:
\begin{align*}
    h(Y \mid X = 0)
     & = -\int_{0}^{\ln 3} \frac{3e^{-y}}{2} \ln \frac{3e^{-y}}{2}\,dy        \\
     & = 1-\ln\frac{3}{2}+\frac{\ln 3}{2}                                     \\
     & \approx 0.0452                                                         \\
    h(Y \mid X = 1)
     & = -\int_{\ln 3/2}^{\infty} \frac{3e^{-y}}{2} \ln \frac{3e^{-y}}{2}\,dy \\
     & = 1
\end{align*}

The weighted sum of these two is then
\[h(Y \mid X) \approx \frac{0.0452+1}{2} = \boxed{0.5226}\]

\subsection{Part H}

Using the formula with $h(Y)$ and $h(Y \mid X)$ instead gives
\begin{align*}
    I(X; Y)
     & = h(Y) - h(Y \mid X)    \\
     & \approx 0.869 - 0.5226 \\
     & = \boxed{0.3464}
\end{align*}
which is around the same thing as what we got in \ref{sec:4e}.

\pagebreak

\section{Conditional Entropy of a Product}

\subsection{Part A}

$aY$ scales the outcomes, but it doesn't change the underlying probability
nor does it merge any two elements together since $a \ne 0$.
Thus, $H(aY)=H(Y)$.

\subsection{Part B}

Using the result from the previous part gives
\begin{align*}
    H(XY \mid X)
     & = \sum_{x \in X} p(X=x)H(XY \mid X = x) \\
     & = \sum_{x \in X} p(X=x)H(Y \mid x)             \\
     & = \boxed{H(Y \mid X)}
\end{align*}

\subsection{Part C}

If $X$ is continuous,
\begin{align*}
    h(XY \mid X)
    &= \int_{-\infty}^{\infty} f(x)h(XY \mid X=x) \\
    &= \int_{-\infty}^{\infty} f(x)(h(Y \mid X=x)+\log |x|) \\
    &= \int_{-\infty}^{\infty} f(x)h(Y \mid X=x) + f(x)\log |x| \\
    &= \int_{-\infty}^{\infty} f(x)h(Y \mid X=x) + \int_{-\infty}^{\infty} f(x)\log |x| \\
    &= \boxed{h(Y \mid X) + \E_X[\log |x|]}
\end{align*}
If $X$ is discrete, the expression turns out to be the same by means of a similar derivation.

\pagebreak

\section{Data Processing and Entropy}

\subsection{Part A}

We have both
\begin{align*}
    H(g(X), X)
     & = H(g(X))+H(X \mid g(X)) \\
     & = H(X)+H(g(X) \mid X)
\end{align*}
Since $H(g(X) \mid X) = 0$ and entropy is nonnegative, $H(X) \ge H(g(X))$. $\square$

\subsection{Part B}

If $g(x)=2x$, then $h(g(X))=h(X)+\log 2 > h(X)$.

\subsection{Part C}

Let the \textbf{floor distance} of $x$ be $x-\floor{x}$ and the PDF of $X$ be $f$.

Notice that all numbers whose floor distance is in $[0, 0.5]$ get mapped to $[0, 0.5]$ themselves,
and those whose distance is in $(0.5, 1)$ get mapped to $(-0.5, 0)$.

From this, we can see that the PDF of $g(X)$ (call it $g'$) is:
\[g'(x)=\sum_{i=-\infty}^{\infty} f(x+i)\label{eq:1}\]
for $x \in (-0.5, 0.5]$.

We have
\begin{align*}
    h(g(X))
     & = -\int_{-.5}^{.5} g'(x)\log g'(x) \,dx                                                                                \\
     & = -\int_{-.5}^{.5} \left(\sum_{i=-\infty}^{\infty} f(x+i)\right)\left(\log \sum_{i=-\infty}^{\infty} f(x+i)\right)\,dx \\
     & \le -\int_{-.5}^{.5} \sum_{i=-\infty}^{\infty} f(x+i)\log f(x+i)\,dx                                                   \\
     & = -\sum_{i=-\infty}^{\infty} \int_{-.5}^{.5} f(x+i)\log f(x+i)\,dx                                                     \\
     & = -\int_{-\infty}^{\infty} f(x)\log f(x)\,dx                                                                           \\
     & = h(X)
\end{align*}

\section{More Modulo Mischief}

\subsection{Part A}

Just some algebra:
\begin{align*}
    (a+b)\log(a+b)
     & = a\log(a+b)+b\log(a+b)             \\
     & \ge a \log a + b \log b\quad\square
\end{align*}

\subsection{Part B}

If $f$ is a PDF, then it's nonnegative for all $x$.

The proof is by induction(?) on the number of $i$ we consider.
The base case is pretty simple- consider just the set $\{0\}$:
\[f(x)\log f(x) \le f(x)\log f(x)\]

Now assume truth for all finite sets $S$.
Adding another element $j$ into it gives
\begin{align*}
          & f(x+j) \log f(x+j) + \sum_{i \in S} f(x+i) \log f(x+i)                                         \\
    \le{} & f(x+j) \log f(x+j) + \left(\sum_{i \in S} f(x+i)\right)\log \left(\sum_{i \in S} f(x+i)\right) \\
    \le{} & \left(\sum_{i \in S \cup \{j\}} f(x+i)\right)\log\left(\sum_{i \in S \cup \{j\}} f(x+i)\right)
\end{align*}
so the inequality still holds.

Assuming the summation converges, we can add more $j$s until we get all the integers. $\square$

\subsection{Part C}

We just, uh:
\begin{align*}
        & (a+b)\log(a+b)+a\log\frac{a}{a+b}+b\log\frac{b}{a+b}                                \\
    ={} & a\left(\log(a+b)+\log\frac{a}{a+b}\right)+b\left(\log(a+b)+\log\frac{b}{a+b}\right) \\
    ={} & a\log a + b \log b\quad\square
\end{align*}

\subsection{Part D}

Recall that we defined $g'(x)$, the PDF of $Y$, in \eqref{eq:1}.

We have
\begin{align*}
    h(X)
     & = \int_{-\infty}^{\infty} f(x) \log f(x)\,dx                                                                         \\
     & = \int_{-.5}^{.5} \sum_{i=-\infty}^{\infty} f(x+i)\log f(x+i)\,dx                                                    \\
     & = \int_{-.5}^{.5} g'(x)\log g'(x) + \sum_{i=-\infty}^{\infty} f(x+i)\log\frac{f(x+i)}{g'(x)}\,dx                     \\
     & = \int_{-.5}^{.5} g'(x)\log g'(x)\,dx + \int_{-.5}^{.5} \sum_{i=-\infty}^{\infty} f(x+i)\log\frac{f(x+i)}{g'(x)}\,dx \\
     & = h(Y) + \int_{-.5}^{.5} \sum_{i=-\infty}^{\infty} f(x+i)\log\frac{f(x+i)}{g'(x)}\,dx                                \\
     & = h(Y) + \int_{-.5}^{.5} g'(x) \sum_{i=-\infty}^{\infty} \frac{f(x+i)}{g'(x)}\log\frac{f(x+i)}{g'(x)}\,dx            \\
     & = h(Y) + \E_Y\left[\sum_{i=-\infty}^{\infty} P(Z=z \mid Y=y)\right]                                                  \\
     & = h(Y) + \E_Y[H(Z \mid Y = y)]                                                                                       \\
     & = h(Y) + H(Z \mid Y)\quad\square
\end{align*}

\pagebreak

\section{Mut. Info for a Mixed Distribution}

\subsection{Part A}

Same as in \ref{sec:4e}: $H(X)=\log 2$.

\subsection{Part B}

Since the outputs given an $X$ are mutually disjoint, $H(X \mid Y)=0$.

\subsection{Part C}

\begin{enumerate}[label=\roman*.]
    \item $h(Y \mid X=0)=-\infty$, since the distribution is discrete.
    \item $h(Y \mid X=1)=\log\left(\frac{3}{2}-\frac{1}{2}\right)=0$, since it's uniform.
    \item $h(Y \mid X)=\frac{-\infty+0}{2}=-\infty$. (real analysis is crying and sobbing rn)
\end{enumerate}

\subsection{Part D}

\[h(Y)=H(X)+h(Y \mid X)-H(X \mid Y)=-\infty\]

\subsection{Part E}

\[I(X;Y)=H(X)-H(X \mid Y)=\log 2\]

\end{document}
