\documentclass[12pt]{article}

% a template that a friend gave, it's worked well enough for me
% i have added some packages and stuff that have proved useful

\usepackage{fancyhdr}
\usepackage{tipa}
\usepackage{fontspec}
\usepackage{amsfonts}
\usepackage{enumitem}
\usepackage[margin=1in]{geometry}
\usepackage{graphicx}
\usepackage{float}
\usepackage{amsmath}
\usepackage{braket}
\usepackage{amssymb}
\usepackage{booktabs}
\usepackage{hyperref}
\usepackage{mathtools}
\usepackage{xcolor}
\usepackage{float}
\usepackage{algpseudocodex}
\usepackage{titlesec}
\usepackage{bbm}
\usepackage{pythonhighlight}

\pagestyle{fancy}
\fancyhf{} % sets both header and footer to nothing
\lhead{Kevin Sheng}
\setmainfont{Comic Neue}
\renewcommand{\headrulewidth}{1pt}
\setlength{\headheight}{0.75in}
\setlength{\oddsidemargin}{0in}
\setlength{\evensidemargin}{0in}
\setlength{\voffset}{-.5in}
\setlength{\headsep}{10pt}
\setlength{\textwidth}{6.5in}
\setlength{\headwidth}{6.5in}
\setlength{\textheight}{8in}
\renewcommand{\headrulewidth}{0.5pt}
\renewcommand{\footrulewidth}{0.3pt}
\setlength{\textwidth}{6.5in}
\usepackage{setspace}
\usepackage{multicol}
\usepackage{float}
\setlength{\columnsep}{1cm}
\setlength\parindent{24pt}
\usepackage [english]{babel}
\usepackage [autostyle, english = american]{csquotes}
\MakeOuterQuote{"}

\setlength{\parskip}{6pt}
\setlength{\parindent}{0pt}

\titlespacing\section{0pt}{12pt plus 4pt minus 2pt}{0pt plus 2pt minus 2pt}
\titlespacing\subsection{0pt}{12pt plus 4pt minus 2pt}{0pt plus 2pt minus 2pt}
\titlespacing\subsubsection{0pt}{12pt plus 4pt minus 2pt}{0pt plus 2pt minus 2pt}

\hypersetup{colorlinks=true, urlcolor=blue}

\newcommand{\correction}[1]{\textcolor{red}{#1}}


\rhead{ECE 231A}

\newcommand{\ind}{\perp\!\!\!\!\perp} 
\newcommand*{\ditto}{-''-}
\renewcommand{\c}{\teXttt}
\DeclareMathOperator{\E}{E}
\DeclareMathOperator{\Var}{Var}

\begin{document}

\section{Resources}

I've used my own brain, posted slides/resources and a Discord chat with Kevin Zhao.

\section{Unusual MAC}

\subsection{Part A}

For $(0, 1)$, we can send $X_1=0$ and $X_2 \sim \text{Bernoulli}\left(\frac{1}{2}\right)$.

Since $X_1$ is always $0$, $Y=X_2$ and $I(X_2; Y \mid X_1)=1$.

The channel is symmetric around $\mathcal{X}_1$ and $\mathcal{X}_2$,
so by similar reasoning $(1, 0)$ is achievable too.

\subsection{Part B}

Let $p_1$ be the chance $X_1=1$ and $p_2$ be the chance $X_2=1$.

Then,
\begin{align*}
    I(X_1, X_2; Y)
     & = H(Y) - H(Y \mid X_1, X_2)                         \\
     & = H\left(1-p_1-p_2+\frac{p_1p_2}{2}\right) - p_1p_2 \\
     & \le 1\quad\square
\end{align*}
where the final inequality comes from the fact that the entropy is of a binary variable.

\subsection{Part C}

I feel like the rate $\left(\frac{1}{2}, \frac{1}{2}\right)$ is \textit{not} achievable.

This rate would force both $p_1$ and $p_2$ to be nonzero, as they both would
have to send over some information each channel use.

However, as we saw in the previous part, this forces $I(X_1, X_2; Y)$
to be \textit{strictly} less than $1=\frac{1}{2}+\frac{1}{2}$.
Thus, the only way we can get this pair is with time sharing.

\pagebreak

\section{Multiplication MAC}

\subsection{Part A}

First, notice that $I(X_1; Y \mid X_2)=H(p)$, since $X_2 \ne 0$ and we can always recover $X_1$ from $Y$.

Also,
\begin{align*}
    I(X_2; Y \mid X_1)
     & = p \cdot I(X_2; Y \mid X_1=1)+(1-p) \cdot I(X_2; Y \mid X_1=0) \\
     & = pH(X_2) + (1-p) \cdot 0                                       \\
     & = pH(X_2)
\end{align*}
and
\begin{align*}
    I(X_1, X_2; Y)
     & = H(X_1, X_2) - H(X_1, X_2 \mid Y)                   \\
     & = H(X_1)+H(X_2) - H(X_1 \mid Y) - H(X_2 \mid X_1, Y) \\
     & = H(p)+H(X_2)
\end{align*}

No matter what $p$ is it's optimal to let $X_2$ be uniform across $\mathcal{X}_2$.

\subsection{Part B}

Everywhere at or below the blue line is good.
\begin{center}
    \includegraphics[width=10cm]{img/hw8/mul}
\end{center}

\subsection{Part C}

As the region is already convex, adding the convex hull doesn't increase its area.

\section{Modulo Addition MAC}

Knowing either $X_1$ or $X_2$ and $Y$ allows us to recover the other input, so
\begin{gather*}
    I(X_1; Y \mid X_2) = H(X_1) = \le 1 \\
    I(X_2; Y \mid X_1) = H(X_2)
\end{gather*}
and the joint mutual info is
\begin{align*}
    I(X_1, X_2; Y)
     & = H(Y) - H(Y \mid X_1, X_2) \\
     & = H(Y)
\end{align*}
Notice that if we let $X_2$ be uniform over its alphabet,
\begin{align*}
    P(Y=y)
     & = (1-p)P(X_2=y)+pP(X_2=y-1)                     \\
     & = (1-p) \cdot \frac{1}{4} + p \cdot \frac{1}{4} \\
     & = \frac{1}{4}
\end{align*}
and $H(Y)=2$, which is the largest it can possibly be.

Conveniently, this also maximizes $H(X_2)$, so there's no tradeoffs to be made here and
\begin{gather*}
    I_1 = 1 \\
    I_2 = I_3 = 2
\end{gather*}
giving us the following feasible region:
\begin{center}
    \includegraphics[width=10cm]{img/hw8/mod4}
\end{center}

\section{Binary Adder MAC}

\subsection{Part A}

As expected,
\begin{gather*}
    I(X_1; Y \mid X_2) = H(X_1) = H(p_1) \\
    I(X_2; Y \mid X_1) = H(X_2) = H(p_2)
\end{gather*}
and
\begin{align*}
        & I(X_1, X_2; Y)                                     \\
    ={} & H(Y) - H(Y \mid X_1, X_2)                          \\
    ={} & H(Y)                                               \\
    ={} & H_3(p_1p_2, (1-p_1)(1-p_2), p_1(1-p_2)+p_2(1-p_1)) \\
    ={} & H_3(p_1p_2, (1-p_1)(1-p_2), p_1+p_2-2p_1p_2)
\end{align*}

\subsection{Part B}

I mean, nothing special here, just:
\begin{align*}
        & H_4(p_1(1-p_2), p_1p_2, (1-p_1)(1-p_2), (1-p_1)p_2) \\
    ={} & H_3(p_1, (1-p_1)(1-p_2), (1-p_1)p_2) + p_1H(p_2)    \\
    ={} & H(p_1) + (1-p_1)H(p_2) + p_1H(p_2)                  \\
    ={} & H(p_1) + H(p_2)\quad\square
\end{align*}

\subsection{Part C}

If you'll excuse the dogwater LaTeX, notice that
\begin{align*}
        & H(p_1)+H(p_2)                                                                                          \\
    ={} & H_4(p_1(1-p_2), p_1p_2, (1-p_1)(1-p_2), (1-p_1)p_2)                                                    \\
    ={} & H_3(p_1p_2, (1-p_1)(1-p_2), p_1+p_2-2p_1p_2)                                                            \\
        & + (p_1+p_2-2p_1p_2)H_2\left(\frac{p_1(1-p_2)}{p_1+p_2-2p_1p_2}, \frac{(1-p_1)p_2}{p_1+p_2-2p_1p_2}\right)
\end{align*}
and thus
\[H(Y)=H(p_1)+H(p_2) - \alpha H(q)\]
where
\begin{gather*}
    \alpha = p_1+p_2-2p_1p_2 \\
    q = \frac{p_1(1-p_2)}{p_1+p_2-2p_1p_2}
\end{gather*}

\subsection{Part D}

I used Desmos this time, but the idea's there.

\begin{center}
    \includegraphics[width=10cm]{img/hw8/bin\_add}
\end{center}
The red region is $p_1=p_2=0.5$, and the green region is $p_1=0.3$ and $p_2=0.7$.
As we can see, the red completely contains the green.

\pagebreak

\section{TDMA vs. CDMA}

\subsection{Part A}

Let $C(x)=\frac{1}{2}\log(x)$ for convenience.

The pentagon is bounded above by the following inequalities:
\begin{gather*}
    R_1 < C\left(\frac{3N}{N}\right) = C(3) \\
    R_2 < C\left(\frac{3N}{N}\right) = C(3) \\
    R_1+R_2 < C\left(\frac{3N+3N}{N}\right) = C(6)
\end{gather*}
and when graphed is everything below the blue line:
\begin{center}
    \includegraphics[width=10cm]{img/hw8/two_normals}
\end{center}

\pagebreak

\subsection{Part B}

When the first channel sends with power $\frac{3N}{\lambda}$ for $\lambda$ of the time
and power $0$ for all other moments, its average power is just $\frac{3N}{\lambda} \cdot \lambda = 3N$.

Using the channel like this gives an average rate of
$\lambda C\left(\frac{3N/\lambda}{N}\right) = \lambda C\left(\frac{3}{\lambda}\right)$.

By similar logic, the second sender can get
$(1-\lambda)C\left(\frac{3}{1-\lambda}\right)$ bits per use on average.

Plotting these two rates against each other gives
\begin{center}
    \includegraphics[width=10cm]{img/hw8/time_sharing}
\end{center}

\subsection{Part C}

As we can see, there's only three $\lambda$s that achieve optimality: $0$, $1$, and $0.5$.

\subsection{Part D}

TSMA could have latency issues, since if you wanna send something right away
you would have to wait for your time share before actually beginning to transmit.
This is especially annoying if the other side isn't even using their time slice;
you're waiting in this case for essentially no reason at all.

\section{Noiseless MAC}

Really funny to ask this when I've actually used this in previous questions:
\begin{align*}
    R_1+R_2
     & < I(X_1, X_2; Y)            \\
     & = H(Y) - H(Y \mid X_1, X_2) \\
     & = H(Y)\quad\square
\end{align*}
Notice that since $f$ is a deterministic function, knowing $X_1$ and $X_2$
allows us to deduce with 100\% certainty $Y=f(X_1, X_2)$, so $H(Y \mid X_1, X_2)=0$.

\section{Slepian-Wolf For Deterministic \texorpdfstring{$f$}{f}}

We have
\begin{gather*}
    R_1 > H(X \mid Y) \\
    R_2 > H(Y \mid X) = 0 \\
    R_1+R_2 > H(X, Y) = H(X)
\end{gather*}
which results in the following region:
\begin{center}
    \includegraphics[width=10cm]{img/hw8/funny_region}
\end{center}
If $f$ is one-to-one, then the vertical dotted line would just be the same as the vertical axis.

\section{Slepian-Wolf}

Just some basic entropy calculations:
\begin{gather*}
    H(Y \mid X) = H(Z \oplus X \mid X) = H(Z) = H(r) \\
    H(X, Y) = H(X) + H(Y \mid X) = H(p) + H(r)
\end{gather*}
The last one is a bit more complicated though:
\begin{align*}
    H(X \mid Y)
     & = H(X, Y) - H(Y)             \\
     & = H(p) + H(r) - H(1-p-r+2pr)
\end{align*}

\section{Slepian-Wolf for Multiplication}

Some more number wrangling:
\begin{gather*}
    H(X_2 \mid X_1) = H(X_1 \times Z \mid X_1) = 2p \\
    H(X_1, X_2) = H(X_1) + H(X_2 \mid X_1) = H(p_1) + 2p
\end{gather*}
Also notice that if $X_2$ is nonzero, then $X_1$ has to be $1$.
The opposite is true if $X_2=0$.
Thus, $X_2$ gives us everything we need to know about $X_1$ and $H(X_1 \mid X_2) = 0$.

The plot for $p=0.5$ is as follows:
\begin{center}
    \includegraphics[width=10cm]{img/hw8/sw_mul}
\end{center}

\section{Slepian-Wolf for Distributed Sensors}

I think we just also have to communicate $e$ as an actual outcome, right?

Then
\begin{align*}
        & H(X_2 \mid X_1)                                                                                      \\
    ={} & \alpha_1 H(X_2 \mid X_1 = e) + (1-p)(1-\alpha_1)H(X_2 \mid X_1=0) + p(1-\alpha_1)H(X_2 \mid X_1 = 1) \\
    ={} & \alpha_1 H(X_2) + (1-\alpha_1)H(\alpha_2)
\end{align*}
and similarly $H(X_1 \mid X_2) = \alpha_2 H(X_1) + (1-\alpha_2)H(\alpha_1)$.

Finally,
\[H(X_1, X_2) = \alpha_2 H(X_1) + (1-\alpha_2)H(\alpha_1) + H(X_2)\]

For the sake of completeness, I'll write out the PMF of $X_1$:
\begin{itemize}
    \item $P(X_1=0)=(1-\alpha_1)(1-p)$
    \item $P(X_1=e)=\alpha_1$
    \item $P(X_1=1)=(1-\alpha_1)p$
\end{itemize}
The PMF of $X_2$ is analagous.

When all the parameters are $\frac{1}{2}$,
\begin{align*}
    H(X_2 \mid X_1) = H(X_1 \mid X_2) = 1.25 && H(X_1, X_2) = 2.75
\end{align*}
and the region looks like this:
\begin{center}
    \includegraphics[width=7cm]{img/hw8/sw_sensors}
\end{center}

\end{document}
