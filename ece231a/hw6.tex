\documentclass[12pt]{article}

% a template that a friend gave, it's worked well enough for me
% i have added some packages and stuff that have proved useful

\usepackage{fancyhdr}
\usepackage{tipa}
\usepackage{fontspec}
\usepackage{amsfonts}
\usepackage{enumitem}
\usepackage[margin=1in]{geometry}
\usepackage{graphicx}
\usepackage{float}
\usepackage{amsmath}
\usepackage{braket}
\usepackage{amssymb}
\usepackage{booktabs}
\usepackage{hyperref}
\usepackage{mathtools}
\usepackage{xcolor}
\usepackage{float}
\usepackage{algpseudocodex}
\usepackage{titlesec}
\usepackage{bbm}
\usepackage{pythonhighlight}

\pagestyle{fancy}
\fancyhf{} % sets both header and footer to nothing
\lhead{Kevin Sheng}
\setmainfont{Comic Neue}
\renewcommand{\headrulewidth}{1pt}
\setlength{\headheight}{0.75in}
\setlength{\oddsidemargin}{0in}
\setlength{\evensidemargin}{0in}
\setlength{\voffset}{-.5in}
\setlength{\headsep}{10pt}
\setlength{\textwidth}{6.5in}
\setlength{\headwidth}{6.5in}
\setlength{\textheight}{8in}
\renewcommand{\headrulewidth}{0.5pt}
\renewcommand{\footrulewidth}{0.3pt}
\setlength{\textwidth}{6.5in}
\usepackage{setspace}
\usepackage{multicol}
\usepackage{float}
\setlength{\columnsep}{1cm}
\setlength\parindent{24pt}
\usepackage [english]{babel}
\usepackage [autostyle, english = american]{csquotes}
\MakeOuterQuote{"}

\setlength{\parskip}{6pt}
\setlength{\parindent}{0pt}

\titlespacing\section{0pt}{12pt plus 4pt minus 2pt}{0pt plus 2pt minus 2pt}
\titlespacing\subsection{0pt}{12pt plus 4pt minus 2pt}{0pt plus 2pt minus 2pt}
\titlespacing\subsubsection{0pt}{12pt plus 4pt minus 2pt}{0pt plus 2pt minus 2pt}

\hypersetup{colorlinks=true, urlcolor=blue}

\newcommand{\correction}[1]{\textcolor{red}{#1}}


\rhead{ECE 231A}

\newcommand{\ind}{\perp\!\!\!\!\perp} 
\newcommand*{\ditto}{-''-}
\newcommand{\N}{\mathbb{N}}
\newcommand{\R}{\mathbb{R}}
\renewcommand{\c}{\texttt}
\DeclareMathOperator{\E}{E}

\begin{document}

\section{Resources}

I've used my own brain, posted slides/resources and a Discord chat with Kevin Zhao.

\section{A Truncated Gaussian}

\subsection{Part A}\label{sec:1a}

Consider any PDF $g(x)$ s.t. $\E_g[X^2]=\sigma^2$.

Then,
\begin{align*}
    0
     & \le D(g \parallel \phi)                                                                                \\
     & = \int g(x) \ln\frac{g(x)}{\phi(x)}\,dx                                                                \\
     & = \int g(x)\ln g(x)\,dx - \int g(x)\ln \phi(x)\,dx                                                     \\
     & = \int g(x)\ln g(x)\,dx + \int g(x)\left(\frac{1}{2}\ln(2\pi\sigma^2)+\frac{x^2}{2\sigma^2}\right)\,dx \\
     & = -h(g) + \frac{1}{2}\ln(2\pi\sigma^2)+\int g(x)\frac{x^2}{2\sigma^2}\,dx                              \\
     & = -h(g) + \frac{1}{2}\ln(2\pi\sigma^2)+\frac{\E_g[X^2]}{2\sigma^2}                                     \\
     & = -h(g) + \frac{1}{2}\ln(2\pi\sigma^2)+\frac{1}{2}                                                     \\
     & = -h(g) + h(\phi)
\end{align*}
This entire chain just says $0 \le -h(g)+h(\phi)$, which when rearranged gives $h(g) \le h(\phi)$. $\square$

\subsection{Part B}

I'm assuming we have the constraints that $\E[X^2] \le P$.

We have
\begin{align*}
    I(X; Y)
     & = h(Y) - h(Y \mid X) \\
     & = h(Y) - h(Z \mid X) \\
     & = h(Y) - h(Z)
\end{align*}

$h(Z)$ is constant w.r.t $X$, so it remains to maximize the differential entropy of $Y$.

We can bound that value like so:
\[h(Y)=\frac{1}{2}\log(2\pi e \cdot \E[Y^2])\]
The only thing we can control in that expression is $\E[Y^2]$:
\begin{align*}
    \E[Y^2]
     & = \E[X^2]+2\E[XZ]+\E[Z^2]     \\
     & = \E[X^2]+2\E[X]\E[Z]+\E[Z^2] \\
     & \le P + \sigma^2
\end{align*}
where the second equality is because $X \perp Z$.

Notice that this bound is met with equality when $X$ is Gaussian,
as the sum of $\mathcal{N}(0, P)$ and $\mathcal{N}(0, \sigma^2)$ is
$\mathcal{N}(0, P+\sigma^2)$. $\square$

\subsection{Part C}

The proof of this is similar to \ref{sec:1a}.

Consider any $g$ s.t. $g(x)=0\ \forall |x| > 1$ and $\E_g[x]=\sigma^2$:
\begin{align*}
    0
    &\le D(g \parallel \tau) \\
    &= \int_{-1}^{1} g(x)\ln\frac{g(x)}{\tau(x)}\,dx \\
    &= \int_{-1}^{1} g(x)\ln g(x)\,dx - \int_{-1}^{1} g(x)\ln \tau(x)\,dx \\
    &= -h(g) - \int_{-1}^{1} g(x)\left(\ln\frac{1}{K\sqrt{2\pi\gamma^2}}-\frac{x^2}{2\gamma^2}\right) \\
    &= -h(g) - \int_{-1}^{1} \tau(x)\ln\tau(x)\,dx \\
    &= -h(g) + h(\tau)
\end{align*}
Thus, $0 \le -h(g)+h(\tau) \implies h(g) \le h(\tau)$. $\square$

\subsection{Part D}

For the normal Gaussian channel, we relied on the fact that the sum of two
normal distributions is another normal distribution.

However, if we make $X$ a \textit{truncated} normal, we have no clue what
the resulting distribution will be when we sum it with $Z$.
At least that's what
\href{https://math.stackexchange.com/q/831714/713952}{this Stack Exchange thread}
told me.

\section{Shaping Gain}

Just some algebra and calc stuff:
\begin{align*}
    I(X_g; Y_g) - I(X_s; Y_s)
     & = (h(Y_g) - h(Y_g \mid X_g)) - (h(Y_s) - h(Y_s \mid X_s)) \\
     & = (h(Y_g) - h(Z \mid X_g)) - (h(Y_s) - h(Z \mid X_s))     \\
     & = (h(Y_g) - h(Z)) - (h(Y_s) - h(Z))                       \\
     & = h(Y_g) - h(Y_s)                                         \\
     & = D(Y_s \parallel Y_g)
\end{align*}

The last equality is due to two things:
\begin{enumerate}
    \item $Y_s$ and $Y_g$ have the same variance,
          as both input distributions have the same variance and are independent of $Z$.
    \item As we saw in \ref{sec:1a}, the relative entropy between a distribution
          and a Gaussian with the same variance is the same as the difference
          of their entropies.
\end{enumerate}

\section{BSC vs. Gaussian}

\subsection{Part A}

\begin{center}
    \includegraphics[width=10cm]{img/hw6/capacity}
\end{center}

\pagebreak

\subsection{Part B}

Let the quantizing process be a function $f: \R \to \left\{-\sqrt{P}, \sqrt{P}\right\}$.

Since $X \to Y \to f(Y)$, $I(X; f(Y)) \le I(X; Y)$, so in other words the channel
with binary PAM has a capacity that's at least that of the BSC.

Also, $I(X; Y) = H(X) - H(X \mid Y) = H(X) \le 1$, which is the same asymptote of the BSC.

\subsection{Part C}

There's two lines, trust me.
It's just that they're really, \textit{really} close together.

\begin{center}
    \includegraphics[width=10cm]{img/hw6/lb_ub}
\end{center}

\section{Shannon, Sensors, A/D Convertors}

\subsection{Part A}

As was mentioned in lecture, the capacity of this channel is
\boxed{\frac{1}{2}\log\left(1+\frac{P}{N}\right)} bits per use.

\subsection{Part B}

For cts. time, the capacity is \boxed{W\log\left(1+\frac{P}{N_0W}\right)} bits per second.

\subsection{Part C}

We take the limit as $W \to \infty$:
\begin{align*}
    \lim_{W \to \infty} W\log\left(1+\frac{P}{N_0W}\right)
     & = \lim_{W \to \infty} W\log e\ln\left(1+\frac{P}{N_0W}\right) \\
     & = \lim_{W \to \infty} W\log e \cdot \frac{P}{N_0W}            \\
     & = \log e\frac{P}{N_0}
\end{align*}
where the second equality is because $\ln(1+x)\approx x$ for small $x$.

\subsection{Part D}

To send one (1) bit at $\frac{1}{2}\log(1+P)$ bits per use takes \boxed{\frac{2P}{\log(1+P)} J}.

\subsection{Part E}

It takes $\frac{P}{W\log\left(1+\frac{P}{N_0W}\right)} J$ to send one bit over the cts. channel.
The denominator monotonically increases with $W$, so the theoretical limit is $\boxed{\frac{N_0}{\log e} J}$.
While this is more efficient energy-wise than the discrete channel,
it's not achievable since we'll always have finite bandwidth IRL.

\subsection{Part F}

We have $P=3 W$ and $N_0=10^{-9} \frac{W}{Hz}$.

Solving $\log\left(1+\frac{P}{N_0W}\right)= 1 \text{ bit}$ gives us $W=\boxed{3 \cdot 10^9\text{ Hz}}$.

\section{Parallel Channels and Waterfilling}

Let $P'=2P$ for convenience.

We need $\sigma_1^2-\sigma_2^2$ units of power before it starts to behave like a pair of channels.

(So in other words, the threshold is when $P=\frac{\sigma_1^2-\sigma_2^2}{2}$).

For any $P' \le \sigma_1^2-\sigma_2^2$, the optimal power allocation is always going to
be to just send everything down the second channel.

Letting $v=\sigma_2^2+P'$, we have
\[(v-\sigma_1^2)^+ + (v-\sigma_2^2)^+ = 0+P' = P'\]
which as we see, gives all power to the second channel.

Now when $P' > \sigma_1^2-\sigma_2^2$, we need to use both channels.

Here, $v=\sigma_1^2+\frac{P'-(\sigma_1^2-\sigma_2^2)}{2}$ and
\begin{align*}
    (v-\sigma_1^2)^+ + (v-\sigma_2^2)^+
    &= \frac{P'-(\sigma_1^2-\sigma_2^2)}{2} + \sigma_1^2 + \frac{P'-\sigma_1^2-\sigma_2^2}{2} \\
    &= P'
\end{align*}
In this case, both channels actually get used.

\pagebreak

\section{Split Ends}

\subsection{Part A}

I'm assuming we don't have to prove that $Z$ is a sufficient statistic. :clueless:

In other words, $X \to Z \to Y$, which means $I(X; Z) \ge I(X; Y)$.
Since $X \to Y \to Z$ as well, $I(X; Y) \ge I(X; Z) \implies I(X; Y) = I(X; Z)$.

This means that it suffices to maximize $I(X; Z)$:
\begin{align*}
    I(X; Z)
     & = H(X)-H(X \mid Z)              \\
     & = H(X)-H(X \mid Z=b)P(Z=b)      \\
     & = H(X)-\frac{1}{2}H(X \mid Z=b) \\
     & = H(X)-\frac{1}{2}H(X)          \\
     & = \frac{1}{2}H(X)
\end{align*}
This is maximized when $X$ is the uniform distribution,
so the capacity is \boxed{\frac{1}{2}}.

\subsection{Part B}

NOOOO I THOUGHT WE WOULDN'T HAVE TO PROVE THAT

It STP $P(Y \mid Z, X) = P(Y \mid Z)$.

Let's start with $P(Y=0)$.

The LHS simplifies like so:
\begin{align*}
    P(Y=0 \mid Z, X)
     & = \int_{0}^{1} \sum_{z \in \{a, b, c\}} f(x)P(Z=z)P(Y=0 \mid X=x, Z=z) \,dx \\
     & = \int_{0}^{1} f(x)\frac{p}{2} \cdot P(Y=0 \mid X=x, Z=a) \,dx              \\
     & = \int_{0}^{1} f(x)\frac{p}{2} \cdot \frac{1}{2}\,dx                        \\
     & = \frac{p}{4}
\end{align*}
while the RHS goes like this:
\begin{align*}
    P(Y=0 \mid Z)
     & = \sum_{z \in \{a, b, c\}} P(Z=z)P(Y=0 \mid Z=z) \\
     & = P(z=a)P(Y=0 \mid Z=a)                          \\
     & = \frac{p}{2} \cdot \frac{1}{2}                  \\
     & = \frac{p}{4}
\end{align*}
so the two are indeed the same.

By symmetry, the same holds for $1$, $4$, and $5$.

Now we move on to $P(Y=2)$, doing the LHS:
\begin{align*}
    P(Y=2 \mid Z, X)
     & = \int_{0}^{1} \sum_{z \in \{a, b, c\}} f(x)P(Z=z)P(Y=2 \mid X=x, Z=z) \,dx \\
     & = \int_{0}^{1} f(x)\frac{1}{2} \cdot P(Y=2 \mid X=x, Z=b) \,dx              \\
     & = \int_{0}^{1} f(x)\frac{1}{2} \cdot \frac{1}{2}\,dx                        \\
     & = \frac{1}{4}
\end{align*}
then the RHS
\begin{align*}
    P(Y=2 \mid Z)
     & = \sum_{z \in \{a, b, c\}} P(Z=z)P(Y=2 \mid Z=z) \\
     & = P(z=b)P(Y=2 \mid Z=a)                          \\
     & = \frac{1}{2} \cdot \frac{1}{2}                  \\
     & = \frac{1}{4}
\end{align*}
By symmetry this also applies to $p=3$.

So yeah, the PMF of $P(Y \mid Z, X)$ is the same as that of $P(Y \mid Z)$. $\square$

\section{Two Independent Looks At \texorpdfstring{$Y$}{Y}}

\subsection{Part A}\label{sec:8a}

We have
\begin{align*}
    I(X; Y_1, Y_2)
     & = H(Y_1, Y_2) - H(Y_1, Y_2 \mid X)                    \\
     & = H(Y_1, Y_2) - (H(Y_1 \mid X) + H(Y_2 \mid Y_1, X))  \\
     & = H(Y_1, Y_2) - (H(Y_1 \mid X) + H(Y_2 \mid X))       \\
     & = H(Y_1, Y_2) - 2H(Y_1 \mid X)                        \\
     & = H(Y_1) + H(Y_1 \mid Y_2) - 2H(Y_1 \mid X)           \\
     & = 2H(Y_1) - 2H(Y_1 \mid X) - H(Y_1) + H(Y_1 \mid Y_2) \\
     & = 2I(X; Y_1) - I(Y_1; Y_2)\quad\square
\end{align*}

\subsection{Part B}

The mutual info of the first channel is $I(X; Y_1, Y_2) = 2I(X; Y_1) - I(Y_1 \mid Y_2)$,
while that of the second channel is $I(X; Y_1)$.

Since mutual information is positive, the mutual information of the first channel
and therefore the capacity will always be less than twice that of the second channel.

\pagebreak

\section{Mutual Information for Correlated Normals}

Applying the formula for multivariate entropy from class gives us
\begin{align*}
    h(X, Y)
     & = \frac{1}{2}\log\left((2\pi e)^2 \cdot \det\begin{bmatrix}
                                                       \sigma^2     & \rho\sigma^2 \\
                                                       \rho\sigma^2 & \sigma^2
                                                   \end{bmatrix}\right) \\
     & = \frac{1}{2}\log\left((2\pi e)^2 \cdot \sigma^4(1-\rho^2)\right)       \\
     & = \log\left(2\pi e \sigma^2 \sqrt{1-p^2}\right)
\end{align*}

We also have $h(X)=h(Y)=\frac{1}{2}\log(2\pi e\sigma^2)$, so
\begin{align*}
    I(X; Y)
     & = h(X)+h(Y)-h(X, Y)                                                     \\
     & = \log(2\pi e \sigma^2) - \log\left(2\pi e \sigma \sqrt{1-p^2}\right)   \\
     & = \log\left(\frac{2\pi e \sigma^2}{2\pi e \sigma^2 \sqrt{1-p^2}}\right) \\
     & = \log\left(\frac{1}{\sqrt{1-p^2}}\right)                               \\
     & = -\frac{1}{2}\log \sqrt{1-p^2}
\end{align*}

Here's the value of this expression for some special values of $\rho$:
\begin{itemize}
    \item At $\rho=-1$, $X=-Y$ and $I(X; Y)=\infty$ since knowing one tells us everything about the other
          and the two are defined by PDFs, not PMFs.
    \item At $\rho=0$, $X \perp Y$ and their mutual info is naturally $0$.
    \item At $\rho=1$, $X=Y$ and $I(X; Y)=\infty$ in the same way as when $\rho=-1$.
\end{itemize}

\pagebreak

\section{The Two-Look Gaussian Channel}

\subsection{Part A}

If $\rho=1$ then this is equivalent to a single channel with noise $N$.

The capacity is thus $\frac{1}{2}\log\left(1+\frac{P}{N}\right)$.

\subsection{Part B}

If $\rho=0$, then sending $x$ gets us two independent samples of $\mathcal{N}(x, N)$.

Since the average of two normal samples is also distributed normally,
This is equivalent to having one sample with noise $\frac{N}{2}$.

Thus, the capacity is $\frac{1}{2}\log\left(1+\frac{2P}{N}\right)$.

\subsection{Part C}

If $\rho=-1$, $Z_1=-Z_2$ and we can recover $x$ exactly by taking the average of $Y_1$ and $Y_2$.

This is equivalent to a channel with $0$ noise which has infinite capacity.

\end{document}
