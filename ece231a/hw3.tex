\documentclass[12pt]{article}

% a template that a friend gave, it's worked well enough for me
% i have added some packages and stuff that have proved useful

\usepackage{fancyhdr}
\usepackage{tipa}
\usepackage{fontspec}
\usepackage{amsfonts}
\usepackage{enumitem}
\usepackage[margin=1in]{geometry}
\usepackage{graphicx}
\usepackage{float}
\usepackage{amsmath}
\usepackage{braket}
\usepackage{amssymb}
\usepackage{booktabs}
\usepackage{hyperref}
\usepackage{mathtools}
\usepackage{xcolor}
\usepackage{float}
\usepackage{algpseudocodex}
\usepackage{titlesec}
\usepackage{bbm}
\usepackage{pythonhighlight}

\pagestyle{fancy}
\fancyhf{} % sets both header and footer to nothing
\lhead{Kevin Sheng}
\setmainfont{Comic Neue}
\renewcommand{\headrulewidth}{1pt}
\setlength{\headheight}{0.75in}
\setlength{\oddsidemargin}{0in}
\setlength{\evensidemargin}{0in}
\setlength{\voffset}{-.5in}
\setlength{\headsep}{10pt}
\setlength{\textwidth}{6.5in}
\setlength{\headwidth}{6.5in}
\setlength{\textheight}{8in}
\renewcommand{\headrulewidth}{0.5pt}
\renewcommand{\footrulewidth}{0.3pt}
\setlength{\textwidth}{6.5in}
\usepackage{setspace}
\usepackage{multicol}
\usepackage{float}
\setlength{\columnsep}{1cm}
\setlength\parindent{24pt}
\usepackage [english]{babel}
\usepackage [autostyle, english = american]{csquotes}
\MakeOuterQuote{"}

\setlength{\parskip}{6pt}
\setlength{\parindent}{0pt}

\titlespacing\section{0pt}{12pt plus 4pt minus 2pt}{0pt plus 2pt minus 2pt}
\titlespacing\subsection{0pt}{12pt plus 4pt minus 2pt}{0pt plus 2pt minus 2pt}
\titlespacing\subsubsection{0pt}{12pt plus 4pt minus 2pt}{0pt plus 2pt minus 2pt}

\hypersetup{colorlinks=true, urlcolor=blue}

\newcommand{\correction}[1]{\textcolor{red}{#1}}


\rhead{ECE 231A}

\newcommand{\ind}{\perp\!\!\!\!\perp} 
\newcommand{\ceil}[1]{\left\lceil#1\right\rceil}
\newcommand*{\ditto}{-''-}
\newcommand{\N}{\mathbb{N}}
\renewcommand{\c}{\texttt}
\DeclareMathOperator{\E}{E}

\begin{document}

\section{Resources}

I've used my own brain, posted slides/resources and a Discord chat with Kevin Zhao.

\section{Code Analysis}

\subsection{Part A}

This code isn't instantaneous, as $0$ is a prefix of all the others.

However, it does satisfy the Kraft Inequality:
\[2^{-1}+2^{-2}+2^{-3}+2^{-6}+2^{-9} \approx 0.89 < 1\]
so an instantaneous binary source code with these lengths does exist.

\subsection{Part B}

The code isn't uniquely decodable either.

If the codewords were for the letters $a$ through $e$ inclusive, then
\[C(cd)=C(e)=001001001\]

\pagebreak

\section{Reversal}

\subsection{Part A}

This code isn't instantaneous, as $0$ is a prefix of all the others.

However, it does satisfy the Kraft Inequality:
\[2^{-1} + 2^{-2} + 2^{-3} + 2^{-4} + 2^{-5} + 2^{-6} + 2^{-6} = 1\]
so an instantaneous binary source code with these lengths does exist.

\subsection{Part B}

Despite not being instantaneous, it's still uniquely decodeable.

Any encoded string can be split into a $0$ followed by some amount of $1$s.
After "using up" as many of the $1$s as possible, whatever's left goes with the $0$.
This works because the codewords don't contain any run of $1$s after that's greater than $6$.

\section{Kraft Slackness}

Consider the codeword tree up to $l_{max}$, the longest out of all the word lengths.

As this code is instantaneous, no code word can be an ancestor of another.
Each code word thus eliminates exactly $D^{l_{max}-l_i}$ code words of length $l_{max}$.

In total, the number of codewords of length $l_{max}$ eliminated are:
\begin{align*}
    \sum_{i=1}^{m} D^{l_{max}-l_i}
     & = D^{l_{max}} \sum_{i=1}^{m} D^{-l_i} \\
     & < D^{l_{max}}
\end{align*}
where the inequality comes from what was given in the problem.

This means that there exists a sequence of symbols of length $l_{max}$ s.t.
it has no codeword as a parent.
Thus, it can't be decoded into any sequence the source alphabet. $\square$

\pagebreak

\section{One Bit Above Entropy}

Fix a positive $\epsilon$- I'm assuming $\epsilon < 1$.

For no reason in particular,
let's choose $n \in \N$ big enough s.t. $\frac{1}{D^{n+\epsilon}} < \epsilon$.

Consider the near-uniform distribution where all outcomes have probability $\frac{1}{D^{n+\epsilon}}$
except for one that has probability $p$.
If $m+1$ is the number of outcomes, these are the properties of the distribution:
\begin{align*}
    \sum_{i=1}^{m} \frac{1}{D^{n+\epsilon}}+p=1 &  &
    0 < p < \frac{1}{D^{n+\epsilon}}
\end{align*}

Then, using the encoding method described in class,
\begin{align*}
        & L-H(X)                                                                                                            \\
    ={} & \sum_{i=1}^{m} \frac{1}{D^{n+\epsilon}}\ceil{\log_D D^{n+\epsilon}} + p\ceil{\log_D p}
    - \sum_{i=1}^{n} \frac{1}{D^{n+\epsilon}} \log_D D^{n+\epsilon} - p \log_D p                                            \\
    ={} & m\left(\frac{1}{D^{n+\epsilon}}\left(\ceil{n+\epsilon}-n\right)\right) + p\left(\ceil{\log_D p} - \log_D p\right) \\
    ={} & \frac{m}{D^{n+\epsilon}} + p\left(\ceil{\log_D p} - \log_D p\right)                                               \\
    >{} & \frac{m}{D^{n+\epsilon}}                                                                                          \\
    ={} & 1 - p                                                                                                             \\
    >{} & 1-\frac{1}{D^{n+\epsilon}}                                                                                        \\
    >{} & 1-\epsilon
\end{align*}
So $L-H(X) > 1-\epsilon$.
Just move the $H(X)$ over to the other side and you have it. $\square$

\pagebreak

\section{Huffman Code}

The tree is as follows:
\begin{center}
    \includegraphics[width=10cm]{img/hw3/binary}
\end{center}
The codewords for each probability are as follows:
\begin{itemize}[nolistsep]
    \item $\frac{1}{2}$- \c{1}
    \item $\frac{1}{4}$- \c{01}
    \item $\frac{1}{8}$- \c{001}
    \item $\frac{1}{16}$- \c{0001}
    \item $\frac{1}{32}$- \c{00001}
    \item $\frac{1}{64}$- \c{000001}
    \item $\frac{1}{64}$- \c{000000}
\end{itemize}
The expected length of this code is
\[\frac{1}{2}+\frac{2}{4}+\frac{3}{8}+\frac{4}{16}+\frac{6}{32}+\frac{7}{64}+\frac{7}{64} = 2.03125\]
which is actually equivalent to the entropy of the PMF.

\pagebreak

\section{Non-Binary Huffman Code?}

\subsection{Part A}

The tree is as follows:
\begin{center}
    \includegraphics[width=10cm]{img/hw3/ternary_w_binary}
\end{center}
The codewords for each probability are as follows:
\begin{itemize}[nolistsep]
    \item $\frac{1}{3}$- \c{1}
    \item $\frac{1}{3}$- \c{01}
    \item $\frac{1}{9}$- \c{001}
    \item $\frac{1}{9}$- \c{0001}
    \item $\frac{1}{27}$- \c{00001}
    \item $\frac{1}{27}$- \c{000001}
    \item $\frac{1}{27}$- \c{000000}
\end{itemize}
It's average length is around $\boxed{2.407}$ bits.

\pagebreak

\subsection{Part B}

The tree is as follows:
\begin{center}
    \includegraphics[width=5cm]{img/hw3/ternary}
\end{center}
The codewords, listed in order from top to bottom, are:
\begin{align*}
    \c{2} && \c{1} && \c{02} && \c{01} && \c{002} && \c{001} && \c{000}
\end{align*}
The average length of this code is $\boxed{\frac{13}{9}}$.

\subsection{Part C}

The ternary's code average length is equivalent to the entropy in trits:
\begin{align*}
        & H_3(X)                                                                                                                                                                                                                       \\
    ={} & -\left(\frac{1}{3}\log_3\frac{1}{3} + \frac{1}{3}\log_3\frac{1}{3} + \frac{1}{9}\log_3\frac{1}{9} + \frac{1}{9}\log_3\frac{1}{9} + \frac{1}{27}\log_3\frac{1}{27} + \frac{1}{27}\log_3\frac{1}{27} + \frac{1}{27}\log_3\frac{1}{27}\right) \\
    ={} & 2 \cdot \frac{1}{3}+ 2 \cdot \frac{2}{9}+ 3 \cdot \frac{3}{27} \\
    ={} & \boxed{\frac{13}{9}}
\end{align*}
In this case, the ternary code is more efficient, since it gets closer to its entropy base.

\pagebreak

\section{A Sufficient Set of Huffman Codes}

\subsection{Part A}

There's only one tree possible with Huffman coding.

When there's only three symbols, the only way the algorithm can go about things
is to merge two nodes first and then merge the third one with the result.

\subsection{Part B}

If there's four nodes, then two trees are possible:
\begin{center}
    \hfill
    \includegraphics[width=5cm]{img/hw3/huffman4_1}
    \hfill
    \includegraphics[width=5cm]{img/hw3/huffman4_2}
    \hfill \mbox{}
\end{center}

\pagebreak

\section{Codeword Lengths}

\subsection{Part A}

We have
\[2^{-1} + 2^{-2} + 2^{-3} + 2^{-4} + 2^{-5}=\frac{31}{32} \le 1\]
which satisfies the Kraft inequality.
Thus, an instantaneous code exists with these lengths.

\subsection{Part B}

Consider the following set of codewords:
\begin{itemize}[nolistsep]
    \item \c{1}
    \item \c{00}
    \item \c{010}
    \item \c{0110}
    \item \c{01110}
\end{itemize}

\subsection{Part C}

Any Huffman code has to have at least two codewords with the same length.

All codewords are of distinct length here, so this can't be a Huffman code
no matter what PMF we start out with.

\section{Bad Codes}

$\{\c{00}, \c{01}, \c{10}, \c{110}\}$ can't be a Huffman code
because it only has one codeword of the longest length $3$.

$\{\c{01}, \c{10}\}$ also can't be a Huffman code because it's not optimal.
$\{\c{0}, \c{1}\}$ achieves a lower average length.

\pagebreak

\section{Arithmetic Coding}

\subsection{Initial Data Structures}

Both of my functions depend on this here bit of data:
\begin{python}
EOT = "$"
symbols = {"a": 0, "b": 1, "c": 2, EOT: 3}
pmf = [.4, .35, .15, .1]
pref = [0]

for i in pmf:
    pref.append(pref[-1] + i)
left_end, right_end = pref[:-1], pref[1:]
\end{python}

\subsection{Part A}

I defined and ran this function:
\begin{python}
def encode_str(string: str) -> list[int]:
    lo, hi = 0, 1
    transmission = []
    for c in string + EOT:
        id_ = symbols[c]
        lower, upper = left_end[id_], right_end[id_]

        scaling = hi - lo
        lo, hi = lo + lower * scaling, lo + upper * scaling

        while True:
            if hi <= 0.5:
                transmission.append(0)
                lo, hi = lo * 2, hi * 2
            elif lo > 0.5:
                transmission.append(1)
                lo, hi = (lo - 0.5) * 2, (hi - 0.5) * 2
            else:
                break

    return transmission + [1]


print("".join(str(i) for i in encode_str("caab")))
\end{python}
The output was \boxed{\c{110001001}}.

\subsection{Part B}

For decoding I defined this function here:
\begin{python}
def decode_bin(transmission: list[int]) -> str:
    val = 0
    for v, t in enumerate(transmission):
        val += 2 ** (-v - 1) * t

    ret = []
    while not ret or ret[-1] != EOT:
        for l, r, s in zip(left_end, right_end, symbols):
            if l <= val < r:
                ret.append(s)
                val = (val - l) / (r - l)
                break

    return "".join(ret[:-1])  # no need to return the <eot>


print(decode_bin([0, 0, 1, 1, 0, 1, 0, 1, 1, 1]))
\end{python}
The output was \boxed{\c{abac}}.

\pagebreak

\section{LZW}

\subsection{Part A}

The encoding goes as follows:
\begin{enumerate}[label=\roman*.]
    \item \c{c}- encoded as $3$, adds $ca$ as $4$.
    \item \c{a}- encoded as $1$, adds $ab$ as $5$.
    \item \c{b}- encoded as $2$, adds $bc$ as $6$.
    \item \c{c}- encoded as $3$, adds $cb$ as $7$.
    \item \c{bc}- encoded as $6$, adds $bcb$ as $8$.
    \item \c{bcb}- encoded as $8$, no more to encode.
\end{enumerate}
Our final sequence is \boxed{3, 1, 2, 3, 6, 8}.

\subsection{Part B}

The decoding goes as follows:
\begin{enumerate}[label=\roman*.]
    \item $3$- decoded as \c{c}, adds \c{c?} as $4$.
    \item $4$- decoded as \c{cc}, adds \c{cc?} as $5$.
    \item $5$- decoded as \c{ccc}, adds \c{ccc?} as $6$.
    \item Same thing goes on from $6$ to $9$.
    \item Finally, $1$ is decoded as just $a$.
\end{enumerate}

In total, the message is $28$ \c{c}s followed by one \c{a}.

\end{document}
