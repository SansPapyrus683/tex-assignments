\documentclass[12pt]{article}

% a template that a friend gave, it's worked well enough for me
% i have added some packages and stuff that have proved useful

\usepackage{fancyhdr}
\usepackage{tipa}
\usepackage{fontspec}
\usepackage{amsfonts}
\usepackage{enumitem}
\usepackage[margin=1in]{geometry}
\usepackage{graphicx}
\usepackage{float}
\usepackage{amsmath}
\usepackage{braket}
\usepackage{amssymb}
\usepackage{booktabs}
\usepackage{hyperref}
\usepackage{mathtools}
\usepackage{xcolor}
\usepackage{float}
\usepackage{algpseudocodex}
\usepackage{titlesec}
\usepackage{bbm}
\usepackage{pythonhighlight}

\pagestyle{fancy}
\fancyhf{} % sets both header and footer to nothing
\lhead{Kevin Sheng}
\setmainfont{Comic Neue}
\renewcommand{\headrulewidth}{1pt}
\setlength{\headheight}{0.75in}
\setlength{\oddsidemargin}{0in}
\setlength{\evensidemargin}{0in}
\setlength{\voffset}{-.5in}
\setlength{\headsep}{10pt}
\setlength{\textwidth}{6.5in}
\setlength{\headwidth}{6.5in}
\setlength{\textheight}{8in}
\renewcommand{\headrulewidth}{0.5pt}
\renewcommand{\footrulewidth}{0.3pt}
\setlength{\textwidth}{6.5in}
\usepackage{setspace}
\usepackage{multicol}
\usepackage{float}
\setlength{\columnsep}{1cm}
\setlength\parindent{24pt}
\usepackage [english]{babel}
\usepackage [autostyle, english = american]{csquotes}
\MakeOuterQuote{"}

\setlength{\parskip}{6pt}
\setlength{\parindent}{0pt}

\titlespacing\section{0pt}{12pt plus 4pt minus 2pt}{0pt plus 2pt minus 2pt}
\titlespacing\subsection{0pt}{12pt plus 4pt minus 2pt}{0pt plus 2pt minus 2pt}
\titlespacing\subsubsection{0pt}{12pt plus 4pt minus 2pt}{0pt plus 2pt minus 2pt}

\hypersetup{colorlinks=true, urlcolor=blue}

\newcommand{\correction}[1]{\textcolor{red}{#1}}


\rhead{ECE 231A}

\newcommand{\ind}{\perp\!\!\!\!\perp} 
\newcommand{\ceil}[1]{\left\lceil#1\right\rceil}
\newcommand*{\ditto}{-''-}
\DeclareMathOperator{\E}{E}

\begin{document}

\section{Resources}

I've used my own brain, posted slides/resources and a Discord chat with Kevin Zhao.

\section{Relative Entropy}

\subsection{Part A}

Using the law of large numbers gives us
\begin{align*}
    \lim -\frac{1}{n} \log q(X_1, \cdots, X_n)
     & = \lim -\frac{1}{n} \sum_{i=1}^{n} \log q(X_i)                     \\
     & = \E_{p(x)}[-\log q(x)]                                            \\
     & = \sum_{i=1}^{m} -p(x)\log q(x)                                    \\
     & = \sum_{i=1}^{m} -p(x)\left(\log\frac{q(x)}{p(x)}+\log p(x)\right) \\
     & = \sum_{i=1}^{m} p(x)\log\frac{p(x)}{q(x)}-p(x)\log p(x)           \\
     & = \boxed{D(p \parallel q) +H(p)}
\end{align*}

\subsection{Part B}

The computation basically goes the same way,
except now we don't have to add and subtract a $p(x)$.
Because of that, the answer now is just $\boxed{D(p \parallel q)}$.

\section{Take it to the Limit}

We have
\begin{align*}
    \lim \frac{1}{n} \log \frac{p\left(x^n, y^n\right)}{p\left(x^n\right)p\left(y^n\right)}
     & = E_{p(x, y)}\left[\log\frac{p(x, y)}{p(x)p(y)}\right]  \\
     & = \sum_{x} \sum_{y} p(x, y)\log\frac{p(x, y)}{p(x)p(y)} \\
     & = D(p(x, y) \parallel p(x)p(y))                         \\
     & = \boxed{I(X; Y)}
\end{align*}

\section{The AEP in Action}

\subsection{Part A}

We just apply the formula for entropy:
\begin{align*}
    H(X)
     & =-\frac{3}{4}\log\frac{3}{4}-\frac{1}{4}\log\frac{1}{4} \\
     & =-\frac{3}{4}(\log 3 - 2)-\frac{1}{4}(-2)               \\
     & =\frac{3}{2}-\frac{3}{4}\log 3+\frac{1}{2}              \\
     & =\boxed{2-\frac{3}{4}\log 3}
\end{align*}

\subsection{Part B}

Assuming the $x_i$s are i.i.d,
\begin{align*}
    -\frac{1}{n}\log p(x_1, \cdots, x_n)
     & = -\frac{1}{n}\sum_{i=1}^{n} \log p(x_i)                         \\
     & = -\frac{1}{n}\left(k\log\frac{3}{4}+(n-k)\log\frac{1}{4}\right) \\
     & = -\frac{1}{n}(k\log 3-2k-2+2k)                                  \\
     & = \boxed{2-\frac{k}{n}\log 3}
\end{align*}

\pagebreak

\subsection{Part C}

The values of $k$ for which $x_n \in A_\epsilon^{(n)}$ are $5$, $6$, and $7$.

Thus,
\[\left|A_\epsilon^{(n)}\right|=\binom{8}{5}+\binom{8}{6}+\binom{8}{7}=\boxed{92}\]
and
\[P\left(x_n \in A_\epsilon^{(n)}\right) = \sum_{k=5}^{7} \binom{8}{k}\frac{3^k}{4^8} \approx \boxed{0.786}\]

\subsection{Part D}

Any $k$ from $10$ to $14$ inclusive works.

Same thing here:
\begin{gather*}
    \left|A_\epsilon^{(n)}\right|=\sum_{k=10}^{14} \binom{16}{k} = \boxed{14876} \\
    P\left(x_n \in A_\epsilon^{(n)}\right) \approx \boxed{0.857}
\end{gather*}

\subsection{Part E}

$P\left(x_n \in A_\epsilon^{(n)}\right)$ did increase with $n$.

\subsection{Part F}

For $n=8$,
\[2^{n(H(X)+\epsilon)} \approx 272.52 \ge 92\]
and for $n=16$,
\[2^{n(H(X)+\epsilon)} \approx 74267.84 \ge 14876\]

\pagebreak

\section{The AEP and Source Coding}

\subsection{Part A}

There's $\sum_{k=0}^{3} \binom{100}{k}=166751$ sequences with $\le 3$ codewords,
so the minimum length is $\ceil{\log 166751}=\boxed{18}$.

\subsection{Part B}

This is much less than the $100$ bits needed for the naive representation.

\subsection{Part C}

The chance we get a source sequence \textit{with} a codeword is
\[\sum_{k=0}^{3} \binom{100}{k}(0.005)^k(0.995)^{100-k} \approx 0.998\]
so the chance we get a sequence for which no codeword has been assigned is around $\boxed{0.002}$.

\section{Time's Arrow}

The key thing to note is that since the process is stationary,
\[P(X_{-n}=x_1, \cdots, X_1=x_n) = P(X_1=x_1, \cdots, X_n=x_n)\]

By similar reasoning,
\begin{align*}
             & P(X_{-n}, \cdots, X_0) = P(X_0, \cdots, X_n)                       \\
    \implies & P(X_{-n}, \cdots, X_{-1}) \cdot P(X_0 \mid X_{-n}, \cdots, X_{-1})
    = P(X_1, \cdots, X_n) \cdot P(X_0 \mid X_1, \cdots, X_n)                      \\
    \implies & P(X_0 \mid X_{-n}, \cdots, X_{-1}) = P(X_0 \mid X_1, \cdots, X_n)
\end{align*}
In other words, the distribution for $P(X_0)$ is the same
whether you condition on the past or the future, as long as the outcomes
for both are the same.
This means that the entropy of the two are identical as well.

Thus,
\begin{gather*}
    \sum_{x^n} P(X_{-n}=x_1, \cdots, X_1=x_n)H(X_0 \mid \ditto) \\
    \qquad= \sum_{x^n} P(X_1=x_1, \cdots, X_n=x_n)H(X_0 \mid \ditto)
\end{gather*}
and by extension
\[H(X_0 \mid X_{-1}, \cdots, X_{-n})=H(X_0 \mid X_1, \cdots, X_n)\quad\square\]

\section{Average Entropy vs. Conditional Entropy}

Since the process is stationary, like in lecture
\[H(X_n \mid X_1, \cdots, X_{n-1}) \ge H(X_{n+1} \mid X_1, \cdots, X_n)\]
and so
\begin{align*}
    \frac{H(X_1, \cdots, X_n)}{n}
     & = \frac{\sum_{i=1}^{n} H(X_i \mid X_1, \cdots, X_{i-1})}{n}   \\
     & \ge \frac{\sum_{i=1}^{n} H(X_n \mid X_1, \cdots, X_{n-1})}{n} \\
     & = H(X_n \mid X_1, \cdots, X_{n-1})\quad\square
\end{align*}

\section{Adam's Seat Selection}

The transition matrix is
\[P=\begin{bmatrix}
        \frac{1}{4} & \frac{3}{4} & 0           \\
        0           & \frac{1}{3} & \frac{2}{3} \\
        \frac{1}{2} & 0           & \frac{1}{2}
    \end{bmatrix}\]

Along with summing to $1$, $\mu$ has to satisfy
\begin{align*}
    \frac{\mu_1}{4}+\frac{\mu_3}{2}=\mu_1  &  &
    \frac{3\mu_1}{4}+\frac{\mu_2}{3}=\mu_2 &  &
    \frac{2\mu_2}{3}+\frac{\mu_3}{2}=\mu_3
\end{align*}
solving this gets us
\[\mu=\begin{bmatrix}
        \frac{8}{29} &  & \frac{9}{29} &  & \frac{12}{29}
    \end{bmatrix}\]

The entropy rate for this is thus
\begin{align*}
               & H(X_2 \mid X_1)                                                                            \\
    ={}        & \frac{8}{29}H(X_2 \mid X_1=1)+\frac{9}{29}H(X_2 \mid X_1=2)+\frac{12}{29}H(X_2 \mid X_1=3) \\
    \approx {} & \boxed{0.923}
\end{align*}

\pagebreak

\section{Python Code}

I've written the following code to automatically calculate the rate of entropy
given an unweighted graph.
\pyth{nodes} contains the list of amounts of nodes, while \pyth{edges}
contains the number of outgoing edges from each type of node.
The example code uses the chessboard from the lecture 4 slides.

\begin{python}
from math import log2

nodes = [4, 24, 36]
edges = [3, 5, 8]

# 2W, where we sum up all edges
w2 = sum(n * e for n, e in zip(nodes, edges))
# the stationary distribution for each node
mu = [e / w2 for e in edges]

# calculate H(X2 | X1) according to the formula
entropy = 0
for i in range(len(nodes)):
    cond_entropy = log2(edges[i])  # since all edges are equally likely
    entropy += mu[i] * nodes[i] * cond_entropy

print(f"entropy rate: {round(entropy, 3)}")
\end{python}
Running this code produces $2.766$, which is what the slides also give.

\section{Random Chessboard Walk}

\subsection{King}

We have:
\begin{itemize}[nolistsep]
    \item $4$ nodes with $3$ edges
    \item $4$ nodes with $5$ edges
    \item $1$ node with $8$ edges
\end{itemize}
So the entropy rate is around $\boxed{2.236}$.

\subsection{Queen}

We have:
\begin{itemize}[nolistsep]
    \item $8$ nodes with $6$ edges
    \item $1$ node with $8$ edges
\end{itemize}
So the entropy rate is around $\boxed{2.644}$.

\subsection{Odd Bishop}

Only $5$ nodes are relevant for this bishop.
Of these,
\begin{itemize}[nolistsep]
    \item $4$ have $2$ edges
    \item $1$ has $4$ edges
\end{itemize}
So the entropy rate is $\boxed{\frac{4}{3}}$.

\subsection{Even Bishop}

Only $4$ nodes are relevant for this bishop.
All of them have $2$ edges, so the entropy is just $\boxed{1}$.

\subsection{Rook}

All $9$ nodes have $4$ edges for the rook, so the entropy is just $\boxed{2}$.

\section{Random Walk of Spider}

We have $6$ nodes with $3$ edges and $9$ nodes with $4$ edges,
so the entropy is around $\boxed{1.8617}$.

\end{document}
