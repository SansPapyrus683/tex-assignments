\documentclass[12pt]{article}

% a template that a friend gave, it's worked well enough for me
% i have added some packages and stuff that have proved useful

\usepackage{fancyhdr}
\usepackage{tipa}
\usepackage{fontspec}
\usepackage{amsfonts}
\usepackage{enumitem}
\usepackage[margin=1in]{geometry}
\usepackage{graphicx}
\usepackage{float}
\usepackage{amsmath}
\usepackage{braket}
\usepackage{amssymb}
\usepackage{booktabs}
\usepackage{hyperref}
\usepackage{mathtools}
\usepackage{xcolor}
\usepackage{float}
\usepackage{algpseudocodex}
\usepackage{titlesec}
\usepackage{bbm}
\usepackage{pythonhighlight}

\pagestyle{fancy}
\fancyhf{} % sets both header and footer to nothing
\lhead{Kevin Sheng}
\setmainfont{Comic Neue}
\renewcommand{\headrulewidth}{1pt}
\setlength{\headheight}{0.75in}
\setlength{\oddsidemargin}{0in}
\setlength{\evensidemargin}{0in}
\setlength{\voffset}{-.5in}
\setlength{\headsep}{10pt}
\setlength{\textwidth}{6.5in}
\setlength{\headwidth}{6.5in}
\setlength{\textheight}{8in}
\renewcommand{\headrulewidth}{0.5pt}
\renewcommand{\footrulewidth}{0.3pt}
\setlength{\textwidth}{6.5in}
\usepackage{setspace}
\usepackage{multicol}
\usepackage{float}
\setlength{\columnsep}{1cm}
\setlength\parindent{24pt}
\usepackage [english]{babel}
\usepackage [autostyle, english = american]{csquotes}
\MakeOuterQuote{"}

\setlength{\parskip}{6pt}
\setlength{\parindent}{0pt}

\titlespacing\section{0pt}{12pt plus 4pt minus 2pt}{0pt plus 2pt minus 2pt}
\titlespacing\subsection{0pt}{12pt plus 4pt minus 2pt}{0pt plus 2pt minus 2pt}
\titlespacing\subsubsection{0pt}{12pt plus 4pt minus 2pt}{0pt plus 2pt minus 2pt}

\hypersetup{colorlinks=true, urlcolor=blue}

\newcommand{\correction}[1]{\textcolor{red}{#1}}


\lhead{406-196-414}
\rhead{CS 181}

\begin{document}

% https://tex.stackexchange.com/a/336688
\hspace{0pt}
\vfill
\begin{center}
    For this homework, I worked with Andrew Sun and Uday Shukla.
\end{center}
\vfill
\hspace{0pt}

\pagebreak

\section{CFL Concatenation}

If $A$ and $B$ are regular, then they can be implemented by a DFA.
Let $A$ be defined by the standard tuple $(Q_A, F_A, q_{0A}, \Sigma, \delta_A)$
and $B$ be defined by something similar.
I'll assume the two have the same alphabet for convenience.

I found it easier instead to define a PDA with these rudimentary parameters:
\begin{itemize}
    \item $Q=\{q_0, q_{\text{fin}}\} \cup Q_A \cup Q_B$
    \item $F=\{q_{\text{fin}}\}$
    \item $\Sigma$ stays the same
    \item $\Gamma=\{\$, 0\}$
\end{itemize}
and a transition function
$\delta: Q \times \Sigma \times \{\text{push}, \text{pop}, \text{noop}\} \times \Gamma \rightarrow \mathbb{P}(Q)$
defined like so:
\begin{gather*}
    \delta(q_0, \epsilon, \text{push}, \$)=\{q_{0A}\} \\
    \delta(q \in Q_A, s, \text{push}, 0)=\{\delta_A(q, s)\} \\
    \delta(q \in F_A, \epsilon, \text{noop}, 0)=\{q_{0B}\} \\
    \delta(q \in Q_B, s, \text{pop}, 0)=\{\delta_B(q, s)\} \\
    \delta(q \in F_B, \epsilon, \text{pop}, \$)=\{q_{\text{fin}}\}
\end{gather*}
Everything that is not covered by the cases maps to $\varnothing$.
$A$ and $B$ are implemented by DFAs, so $\delta_{A\text{ or }B}(q, \epsilon)=\varnothing$

This is a machine that only accepts on an empty stack.
I've made it so that it can either be in the $A$-portion or the $B$-portion.
By construction, for every token in $x$ it consumes in the $A$-portion,
it has to consume a corresponding token in the $B$-portion.

When the machine is done with whatever part of the string $A$ should be,
it does an epsilon (and noop) transition into the $B$-portion of the machine.
After running through $B$, it needs to have an empty stack (which means $|x|=|y|$)
to actually go to the finishing state.

\pagebreak

\section{Proving CFLs}

\subsection{\texorpdfstring{$L_1$}{L\_1}}

For this language I found it easier to make two CFGs.
If either of them match, we accept; we can do this since context-free languages
are closed under union.

This first CFG accepts if $|x| \ne |y|$.
Setting $A$ as the root variable, we have
\begin{align*}
    A & \to XAX \mid C\$ \mid \$C \\
    C & \to XC \mid X             \\
    X & \to 0 \mid 1
\end{align*}
This grammar accepts the language
\[\{x\$y \mid x, y \in \{0, 1\}^* \text{ and } |x| \ne |y|\}\]
It keeps on stripping characters from the beginning and end until it hits the center \$.
If $|x| \ne |y|$, then either $|x| < |y|$ or vice versa,
so when the parser hits the center \$ exactly one side will have some
nonempty string consisting of 0s and 1s, which is exactly what $C$ matches.

Now we need another CFG that checks the language
\[\{x\$y \mid x, y \in \{0, 1\}^* \text{ and } \exists 1 \le k \le \min(|x|, |y|): x_k \ne y_k\}\]
This is becuase the case where $|x| \ne |y|$ is handled by the first language.

I propose the grammar
\begin{align*}
    S & \to A0C \mid B1C     \\
    A & \to XAX \mid 1C\$    \\
    B & \to XBX \mid 0C\$    \\
    C & \to XC \mid \epsilon \\
    X & \to 1 \mid 0
\end{align*}
Here, $C$ matches the regex $(0|1)^*$.

First consider the rule $A0C$, which makes $x_k=1$ and $y_k=0$ for some $k \ge 0$.

$A$'s base case is $1$ ($x_k$) followed by $C$ and \$.
We then pad both sides with an equal amount of arbitrary 0s and 1s.
Notice that since one character goes before $x_k$ and another goes after the \$,
each time we match on $XAX$ we're adding $1$ to $k$.

Then, when we recurse back up to $S$, we add a 0 to the end ($y_k$)
followed by another $C$.
This forces the $x$ and $y$ to differ at some index $k$.

$B$ works in basically the same way as well, except now $x_k=0$ and $y_k=1$.

\pagebreak

\subsection{\texorpdfstring{$L_2$}{L\_2}}

If we let $x$ be a shorthand for $\{0, 1\}$, notice that this language is equivalent to
\[\{x^j\ 1\ x^{j+k}\ 0\ x^k \mid j, k \ge 0\} \cup \{x^j\ 0\ x^{j+k}\ 1x^k \mid j, k \ge 0\}\]
since in the original language, $|x|=|y|$, so there must be some index
at which $x$ is a $1$ and $y$ is a $0$ or vice versa.

Notice that the left set can be written as $A \circ B$, where
\begin{gather*}
    A=\{x^j\ 1\ x^j \mid j \ge 0\} \\
    B=\{x^k\ 0\ x^k \mid k \ge 0\}
\end{gather*}
This is because the middle $x^{j+k}$ can be any string of characters,
so it doesn't particularly matter how we partition it so long as it has the proper length.

Similarly, the right set can be written as $B \circ A$.

These observations allow us to write the following CFG with $S$ as the root variable:
\begin{align*}
    S & \to AB \mid BA \\
    A & \to XAX \mid 1 \\
    B & \to XBX \mid 0 \\
    X & \to 0 \mid 1
\end{align*}
Variables $A$ and $B$ parse the \textit{languages} $A$ and $B$ described above.
We concatenate them in both orders and take the union to get our final
grammar for $L_2$.

\end{document}
