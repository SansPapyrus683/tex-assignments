\documentclass[12pt]{article}

% a template that a friend gave, it's worked well enough for me
% i have added some packages and stuff that have proved useful

\usepackage{fancyhdr}
\usepackage{tipa}
\usepackage{fontspec}
\usepackage{amsfonts}
\usepackage{enumitem}
\usepackage[margin=1in]{geometry}
\usepackage{graphicx}
\usepackage{float}
\usepackage{amsmath}
\usepackage{braket}
\usepackage{amssymb}
\usepackage{booktabs}
\usepackage{hyperref}
\usepackage{mathtools}
\usepackage{xcolor}
\usepackage{float}
\usepackage{algpseudocodex}
\usepackage{titlesec}
\usepackage{bbm}
\usepackage{pythonhighlight}

\pagestyle{fancy}
\fancyhf{} % sets both header and footer to nothing
\lhead{Kevin Sheng}
\setmainfont{Comic Neue}
\renewcommand{\headrulewidth}{1pt}
\setlength{\headheight}{0.75in}
\setlength{\oddsidemargin}{0in}
\setlength{\evensidemargin}{0in}
\setlength{\voffset}{-.5in}
\setlength{\headsep}{10pt}
\setlength{\textwidth}{6.5in}
\setlength{\headwidth}{6.5in}
\setlength{\textheight}{8in}
\renewcommand{\headrulewidth}{0.5pt}
\renewcommand{\footrulewidth}{0.3pt}
\setlength{\textwidth}{6.5in}
\usepackage{setspace}
\usepackage{multicol}
\usepackage{float}
\setlength{\columnsep}{1cm}
\setlength\parindent{24pt}
\usepackage [english]{babel}
\usepackage [autostyle, english = american]{csquotes}
\MakeOuterQuote{"}

\setlength{\parskip}{6pt}
\setlength{\parindent}{0pt}

\titlespacing\section{0pt}{12pt plus 4pt minus 2pt}{0pt plus 2pt minus 2pt}
\titlespacing\subsection{0pt}{12pt plus 4pt minus 2pt}{0pt plus 2pt minus 2pt}
\titlespacing\subsubsection{0pt}{12pt plus 4pt minus 2pt}{0pt plus 2pt minus 2pt}

\hypersetup{colorlinks=true, urlcolor=blue}

\newcommand{\correction}[1]{\textcolor{red}{#1}}


\lhead{406-196-414}
\rhead{CS 181}

\newcommand{\N}{\mathbb{N}}

\begin{document}

\section{Complementary Languages}

I'll show that the complement of this language is
recognizable and undecidable, since two recognizers imply a decider.

\subsection{Recognizability}

The complement of $\text{COMPL}_\text{TM}$ is
\[\{\Braket{M_1, M_2} \mid L(M_1) \ne \overline{L(M_2)}\}\]
A recognizer for this machine could go as follows:
\begin{algorithmic}[1]
    \Procedure{\texttt{NOT\_COMPL}}{$M_1$, $M_2$}
        \State Run $M_1$ and $M_2$ on all inputs in a triangular fashion
        \If{$\exists x: x \in L(M_1), L(M_2) \lor x \notin L_(M_1), L(M_2)$}
            \State Accept $\braket{M_1, M_2}$
        \Else
            \State The machine's going to loop forever
        \EndIf
    \EndProcedure
\end{algorithmic}

If $L(M_1) \ne \overline{L(M_2)}$, then there exists an input
that's in either both or neither of the languages, and this recognizer
will eventually detect it.

OTOH, if one is indeed the complement of the other, the machine's going
to keep on running the two for every single input until the heat death of the universe.

\subsection{Undecidability}

Suppose there existed a decider $D$ for $\text{COMPL}_\text{TM}$.

Then we can construct the following TM:
\begin{algorithmic}[1]
    \Procedure{M}{$x$}
        \State $\alpha \gets \braket{M}$
        \State $\beta \gets \braket{ACCEPT}$
        \Comment{Description of a TM that always accepts}
        \If{$D(\alpha, \beta)$}
            \State Accept $x$
            \Comment{$L(M)=L(ACCEPT)$ $\implies$ $D$ got it wrong.}
        \Else
            \State Reject $x$
            \Comment{$L(M)=L(REJECT)=\overline{L(ACCEPT)}$ $\implies$ $D$ got it wrong.}
        \EndIf
    \EndProcedure
\end{algorithmic}
whose behavior will always prove $D$ wrong no matter what it outputs. $\square$

\pagebreak

\section{Subset Languages}

Like always, we suppose the existence of a decider $D$.
\begin{algorithmic}[1]
    \Procedure{M}{$x$}
        \State $\alpha \gets \braket{M}$
        \State $\beta \gets \braket{ONLY\_\epsilon}$
        \Comment{This only accepts $\epsilon$}
        \If{$D(\alpha, \beta)$}
            \State Accept
            \Comment{$M$ accepts everything $\implies$ $L(M) \nsubseteq L(ONLY\_\epsilon)$}
        \Else
            \State Accept iff $x=\epsilon$
            \Comment{$L(M)=L(ONLY\_\epsilon) \implies \braket{M, ONLY\_\epsilon} \in \text{SUBSET}_\text{TM}$}
        \EndIf
    \EndProcedure
\end{algorithmic}
This $M$, no matter what, will contradict the behavior prescripted to it by $D$. $\square$

\section{Certified Languages}

\subsection{\texorpdfstring{$\text{Halt}_\epsilon$}{Halt\_e} Certifiability}

Consder the following certifier (or whatever idk what it's called):
\begin{algorithmic}[1]
    \Procedure{\texttt{HALT\_EPS\_CERT}}{$M$, $y$}
        \If{$M$ isnt' a valid TM}
            \State Reject
        \EndIf

        \item[]
        \State Run $M$ on $\epsilon$ for the amount of steps represented by $y$ in binary
        \If{$M$ is in an accepting state}
            \State Accept
        \Else
            \State Reject
        \EndIf
    \EndProcedure
\end{algorithmic}

If a TM $M$ is in $\text{Halt}_\epsilon$, then it will halt on $\epsilon$ after some
number of steps and $\exists y \in \{0, 1\}^*$ that makes the procedure above accept.
We can give the number of steps in binary to the procedure above to gurantee an acceptance. $\square$

\subsection{Screwed for \texorpdfstring{$\text{Halt}_\text{all}$}{Halt\_all}, Though...}

This sort of certifier won't work for $\text{Halt}_\text{all}$ though,
since we can't run the TM for $y$ steps on all inputs; that would take infinitely long.

\pagebreak

\subsection{Not Certifiable, BTW}

Obviously assume at the start that there exists a certifier $C$ for sake of contradiction.

\subsubsection{Implied Recognizability}

Notice that the existence of $C$ implies the existence of a recognizer $R$:
\begin{algorithmic}[1]
    \Procedure{R}{$M$}
        \State Run $C(M, y)$ on all $y \in \{0, 1\}^*$ with triangular scheduling
        
        \If{$\exists y: C(M, y)$ accepts}
            \State Accept
        \Else
            \LComment{If no $y$ works, $R$ will loop as it runs on all $y$ for infinitely long.}
        \EndIf
    \EndProcedure
\end{algorithmic}

\subsubsection{Unrecognizable!}

So certifiability implies recognizability.
Now we construct a bad $M$ which shows that $\text{Halt}_\text{all}$ is
unrecognizable, finally giving us our contradiction:
\begin{algorithmic}[1]
    \Procedure{M}{$x$}
        \State $\alpha \gets \Braket{M}$
        \State Run $R(\alpha)$ for $|x|$ steps
        \If{$R(\alpha)$ accepts}
            \State Loop forever :3
        \ElsIf{$R(\alpha)$ rejects}
            \State Accept
        \ElsIf{$R(\alpha)$'s still running}
            \State Accept
        \EndIf
    \EndProcedure
\end{algorithmic}

There's three cases for what can happen when we run $R(\alpha)$.
For convenience, let $y$ be the number of steps it finishes in.
\begin{itemize}[nolistsep]
    \item If $R(\alpha)$ accepts, $M(x)$ will loop on any $x$ s.t. $|x| \ge y$ will loop.
    \item If it rejects, we'll accept both all $x$ with $|x| < y$ and $|x| \ge y$:
          that is, we will halt and accept all possible $x$s.
    \item If it loops, for any $x$ $M$ will hit the third else-if and accept, so again
          we halt on all inputs.
\end{itemize}

In all three cases, we have a contradiction. $\square$

\end{document}
