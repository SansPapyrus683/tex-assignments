\documentclass[12pt]{article}

% a template that a friend gave, it's worked well enough for me
% i have added some packages and stuff that have proved useful

\usepackage{fancyhdr}
\usepackage{tipa}
\usepackage{fontspec}
\usepackage{amsfonts}
\usepackage{enumitem}
\usepackage[margin=1in]{geometry}
\usepackage{graphicx}
\usepackage{float}
\usepackage{amsmath}
\usepackage{braket}
\usepackage{amssymb}
\usepackage{booktabs}
\usepackage{hyperref}
\usepackage{mathtools}
\usepackage{xcolor}
\usepackage{float}
\usepackage{algpseudocodex}
\usepackage{titlesec}
\usepackage{bbm}

\pagestyle{fancy}
\fancyhf{} % sets both header and footer to nothing
\lhead{Kevin Sheng}
\setmainfont{Comic Neue}
\renewcommand{\headrulewidth}{1pt}
\setlength{\headheight}{0.75in}
\setlength{\oddsidemargin}{0in}
\setlength{\evensidemargin}{0in}
\setlength{\voffset}{-.5in}
\setlength{\headsep}{10pt}
\setlength{\textwidth}{6.5in}
\setlength{\headwidth}{6.5in}
\setlength{\textheight}{8in}
\renewcommand{\headrulewidth}{0.5pt}
\renewcommand{\footrulewidth}{0.3pt}
\setlength{\textwidth}{6.5in}
\usepackage{setspace}
\usepackage{multicol}
\usepackage{float}
\setlength{\columnsep}{1cm}
\setlength\parindent{24pt}
\usepackage [english]{babel}
\usepackage [autostyle, english = american]{csquotes}
\MakeOuterQuote{"}

\setlength{\parskip}{6pt}
\setlength{\parindent}{0pt}

\titlespacing\section{0pt}{12pt plus 4pt minus 2pt}{0pt plus 2pt minus 2pt}
\titlespacing\subsection{0pt}{12pt plus 4pt minus 2pt}{0pt plus 2pt minus 2pt}
\titlespacing\subsubsection{0pt}{12pt plus 4pt minus 2pt}{0pt plus 2pt minus 2pt}

\hypersetup{colorlinks=true, urlcolor=blue}

\newcommand{\correction}[1]{\textcolor{red}{#1}}


\lhead{406-196-414}
\rhead{CS 181}

\newcommand{\N}{\mathbb{N}}

\begin{document}

\section{Ordered Triples of Integers}

We start off with something like this:
\[\begin{array}{cccccc}
        0 & 1 & -1 & 2 & -2 & \cdots \\
        0 & 1 & -1 & 2 & -2 & \cdots \\
        0 & 1 & -1 & 2 & -2 & \cdots
    \end{array}\]
We proved in class that $|\mathbb{Z}|=|\N|$, so
we can indeed order the integers in each row.

Now, even though there's thric the number of integers,
we can still map $\N$ onto them:
\[\begin{array}{cccccc}
        1 & 4 & 7 & 10 & 13 & \cdots \\
        2 & 5 & 8 & 11 & 14 & \cdots \\
        3 & 6 & 9 & 12 & 15 & \cdots
    \end{array}\]
which shows that $|\N|=\left|\mathbb{Z}^3\right|$. $\square$

\section{More Unsolvable Languages}

Also, for convenience, I'll let $M_{ij}$ denote whether the $i$th language accepts
the $j$th string in the supposed enumeration the universe gives us,
and let $s_j$ denote the $j$th string as well.

\subsection{Another \texorpdfstring{$L^{DIAG}$}{L\^DIAG}}\label{sec:onemore}

Notice that $\exists i \in \N: M_{11} \ne M_{j1}$.
This is because there must exist an infinite number
of languages that accept $s_1$ (and the same can be said for rejecting $s_1$).

This allows us to construct $L_2^{DIAG}$ as follows:
\begin{enumerate}[nolistsep]
    \item Accept $s_1$ if and only if $M_{i1}=0$.
    \item Accept $s_i$ if and only if $M_{1i}=0$.
    \item For all other $x \ne 1, i$, we do the standard diagonalization.
\end{enumerate}
We just swap the order in which we go along all the strings and languages,
so by the same argument this $L_2^{DIAG}$ shouldn't be in the enumeration.

It also disagress with $L_1^{DIAG}$ on $s_1$.
While we accept iff $M_{i1}=0$, $L_1^{DIAG}$ accepts iff $M_{11}=0$,
which we know isn't equal to $M_{i1}$.

\pagebreak

\subsection{Infinitely More?}\label{sec:infmore}

(this is going to be basically the same as \ref{sec:onemore} but it is what it is)

There was nothing special about using $1$ for the first index.

Thus, $\forall i \in \N\ \exists j > i: M_{ii} \ne M_{ji}$, so we can construct $L_i^{DIAG}$ like so:
\begin{enumerate}[nolistsep]
    \item Accept $s_i$ if and only if $M_{ji}=0$.
    \item Accept $s_j$ if and only if $M_{ij}=0$.
    \item For all other $x \ne i, j$, we do the standard diagonalization.
\end{enumerate}

By the same logic as in \ref{sec:onemore}, we can see that none
of these languages are in the enumeration we were given.
It remains to show that none of them are equal to each other, either.

Consider $L_a^{DIAG}$ and $L_b^{DIAG}$, and WLOG assume $a < b$.

Since $a < b$, our rules dictate that $L_b^{DIAG}$ accepts $s_a$ iff $M_{aa}=0$.
However, $L_a^{DIAG}$ accepts the same string iff $M_{ja}=0$,
where $j$ was the succeeding language that we initially chose for $a$.

$M_{aa} \ne M_{ja}$, so the two languages differ in behavior on $s_a$ and are different overall.

\subsection{One More}

All our $L_i^{DIAG}$s relied on "swapping" two parts of the diagonal.
This means that the OG Cantor's diagonalization wasn't included in our
new infinite set of diagonal languages!.

\subsection{And Another!}

Unfortunately, we can't cheese this next part.

Take $1$ and its associated $j$ that we chose in \ref{sec:infmore}.
Then, take $j+1$ and its associated $k$ that was also chose back there.
Now we construct $L_2^{SUPERDIAG}$ as follows:
\begin{enumerate}[nolistsep]
    \item Accept $s_1$ iff $M_{j1}=0$, and $s_j$ iff $M_{1j}=0$.
    \item Accept $s_{j+1}$ iff $M_{k(j+1)}=0$, and $s_k$ iff $M_{(j+1)k}=0$.
    \item For all other $x \ne i, j$, we do the standard diagonalization.
\end{enumerate}
Still, this language isn't in the initial enumerated set of languages,
nor is it equal to the raw diagonalization we choose for $L_1^{SUPERDIAG}$
since we swapped two pairs.

For $L_i^{DIAG}$, if $i \ne 1$, the language's behavior must differ
from $L_2^{SUPERDIAG}$ on $s_1$ by the same argument as in \ref{sec:infmore}.
OTOH, if $i=1$, then it differs from $L_2^{SUPERDIAG}$ on accepting $s_{j+1}$.
These two cases cover all $i \in \N$, so $L_2^{SUPERDIAG}$
must be different from both the enumerated languages and the
enumerated DIAG languages.

\end{document}
