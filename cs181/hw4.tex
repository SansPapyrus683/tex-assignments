\documentclass[12pt]{article}

% a template that a friend gave, it's worked well enough for me
% i have added some packages and stuff that have proved useful

\usepackage{fancyhdr}
\usepackage{tipa}
\usepackage{fontspec}
\usepackage{amsfonts}
\usepackage{enumitem}
\usepackage[margin=1in]{geometry}
\usepackage{graphicx}
\usepackage{float}
\usepackage{amsmath}
\usepackage{braket}
\usepackage{amssymb}
\usepackage{booktabs}
\usepackage{hyperref}
\usepackage{mathtools}
\usepackage{xcolor}
\usepackage{float}
\usepackage{algpseudocodex}
\usepackage{titlesec}
\usepackage{bbm}
\usepackage{pythonhighlight}

\pagestyle{fancy}
\fancyhf{} % sets both header and footer to nothing
\lhead{Kevin Sheng}
\setmainfont{Comic Neue}
\renewcommand{\headrulewidth}{1pt}
\setlength{\headheight}{0.75in}
\setlength{\oddsidemargin}{0in}
\setlength{\evensidemargin}{0in}
\setlength{\voffset}{-.5in}
\setlength{\headsep}{10pt}
\setlength{\textwidth}{6.5in}
\setlength{\headwidth}{6.5in}
\setlength{\textheight}{8in}
\renewcommand{\headrulewidth}{0.5pt}
\renewcommand{\footrulewidth}{0.3pt}
\setlength{\textwidth}{6.5in}
\usepackage{setspace}
\usepackage{multicol}
\usepackage{float}
\setlength{\columnsep}{1cm}
\setlength\parindent{24pt}
\usepackage [english]{babel}
\usepackage [autostyle, english = american]{csquotes}
\MakeOuterQuote{"}

\setlength{\parskip}{6pt}
\setlength{\parindent}{0pt}

\titlespacing\section{0pt}{12pt plus 4pt minus 2pt}{0pt plus 2pt minus 2pt}
\titlespacing\subsection{0pt}{12pt plus 4pt minus 2pt}{0pt plus 2pt minus 2pt}
\titlespacing\subsubsection{0pt}{12pt plus 4pt minus 2pt}{0pt plus 2pt minus 2pt}

\hypersetup{colorlinks=true, urlcolor=blue}

\newcommand{\correction}[1]{\textcolor{red}{#1}}


\lhead{406-196-414}
\rhead{CS 181}


\begin{document}

\section{Ordered Triples of Integers}

It's been show in class that $|\mathbb{Z}|=|\N|$, it suffices to show a bijection
between $\N$ and $\N^3$.
This is because we can compose the mapping between $\N$ and $\N^3$ with
the bijection $f: \N \to \mathbb{Z}$ show in class to get the final
function from $\N$ to $\mathbb{Z}^3$.

\subsection{Just Pairs First}

First we show a bijection between $\N$ and $\N^2$.
If we think of $\N^2$ as a 2D grid indexed with natural numbers on both sides,
we can label all of the cells like so:
\begin{center}
    \begin{tabular}{c|cccc}
                   & \textbf{1} & \textbf{2} & \textbf{3} & \textbf{4} \\ \hline
        \textbf{1} & 1          & 2          & 4          & 7          \\
        \textbf{2} & 3          & 5          & 8          & \ldots     \\
        \textbf{3} & 6          & 9          & \ldots     & \ldots     \\
        \textbf{4} & 10         & \ldots     & \ldots     & \ldots     \\
    \end{tabular}
\end{center}
We go from one diagonal to the next and assigning each element the next natural number.

With this labelling, we now know there exists a bijection $f: \N \to \N^2$.

\subsection{Actual Triples}

Notice that each triple can be represented as a tuple of an ordered pair another numbers,
so we can break down $(a, b, c)$ into $((a, b), c)$.

Since we know all ordered pairs of natural numbers can be labelled with natural numbers,
we again construct a grid where the columns are indexed with the labels of $(a, b)$
and the rows are indexed with just $c$.


Now we do basically the exact same labelling:
\begin{center}
    \begin{tabular}{c|cccc}
                   & \textbf{(1, 1)} & \textbf{(1, 2)} & \textbf{(2, 1)} & \textbf{(1, 3)} \\ \hline
        \textbf{1} & 1               & 2               & 4               & 7               \\
        \textbf{2} & 3               & 5               & 8               & \ldots          \\
        \textbf{3} & 6               & 9               & \ldots          & \ldots          \\
        \textbf{4} & 10              & \ldots          & \ldots          & \ldots          \\
    \end{tabular}
\end{center}
and see that indeed, $\left|\N^3\right|=|\N|$. $\square$

\pagebreak

\section{More Unsolvable Languages}

Also, for convenience, I'll let $M_{ij}$ denote whether the $i$th language accepts
the $j$th string in the supposed enumeration the universe gives us,
and let $s_j$ denote the $j$th string as well.

\subsection{Another \texorpdfstring{$L^{DIAG}$}{L\^DIAG}}\label{sec:onemore}

Notice that $\exists i \in \N: M_{11} \ne M_{i1}$.
This is because there must exist an infinite number
of languages that accept $s_1$ (and the same can be said for rejecting $s_1$).

To construct $L_2^{DIAG}$, we do the following for each input $x$:
\begin{enumerate}[nolistsep]
    \item If $x=1$, accept $s_1$ if and only if $M_{i1}=0$.
    \item If $x=i$, accept $s_i$ if and only if $M_{1i}=0$.
    \item For all other $x \ne 1, i$, we do the standard diagonalization.
\end{enumerate}
We just swap the order in which we go along all the strings and languages,
so by the same argument this $L_2^{DIAG}$ shouldn't be in the enumeration.

It also disagress with $L_1^{DIAG}$ on $s_1$.
While we accept iff $M_{i1}=0$, $L_1^{DIAG}$ accepts iff $M_{11}=0$,
which we know isn't equal to $M_{i1}$.
\subsection{Infinitely More?}\label{sec:infmore}

(this is going to be basically the same as \ref{sec:onemore} but it is what it is)

There was nothing special about using $1$ for the first index.

Thus, $\forall i \in \N\ \exists j > i: M_{ii} \ne M_{ji}$, so we can construct $L_i^{DIAG}$ like so:
\begin{enumerate}[nolistsep]
    \item Accept $s_i$ if and only if $M_{ji}=0$.
    \item Accept $s_j$ if and only if $M_{ij}=0$.
    \item For all other $x \ne i, j$, we do the standard diagonalization.
\end{enumerate}

By the same logic as in \ref{sec:onemore}, we can see that none
of these languages are in the enumeration we were given.
It remains to show that none of them are equal to each other, either.

Consider $L_a^{DIAG}$ and $L_b^{DIAG}$, and WLOG assume $a < b$.

Since $a < b$, our rules dictate that $L_b^{DIAG}$ accepts $s_a$ iff $M_{aa}=0$.
However, $L_a^{DIAG}$ accepts the same string iff $M_{ja}=0$,
where $j$ was the succeeding language that we initially chose for $a$.

$M_{aa} \ne M_{ja}$, so the two languages differ in behavior on $s_a$ and are different overall.

\subsection{One More}

All our $L_i^{DIAG}$s relied on "swapping" two parts of the diagonal.
This means that the OG Cantor's diagonalization wasn't included in our
new infinite set of diagonal languages!.

\subsection{And Another!}

Unfortunately, we can't cheese this next part.

Take $1$ and its associated $j$ that we chose in \ref{sec:infmore}.
Then, take $j+1$ and its associated $k$ that was also chose back there.
Now we construct $L_2^{SUPERDIAG}$ as follows:
\begin{enumerate}[nolistsep]
    \item Accept $s_1$ iff $M_{j1}=0$, and $s_j$ iff $M_{1j}=0$.
    \item Accept $s_{j+1}$ iff $M_{k(j+1)}=0$, and $s_k$ iff $M_{(j+1)k}=0$.
    \item For all other $x \ne i, j$, we do the standard diagonalization.
\end{enumerate}
Still, this language isn't in the initial enumerated set of languages,
nor is it equal to the raw diagonalization we choose for $L_1^{SUPERDIAG}$
since we swapped two pairs.

For $L_i^{DIAG}$, if $i \ne 1$, the language's behavior must differ
from $L_2^{SUPERDIAG}$ on $s_1$ by the same argument as in \ref{sec:infmore}.
OTOH, if $i=1$, then it differs from $L_2^{SUPERDIAG}$ on accepting $s_{j+1}$.
These two cases cover all $i \in \N$, so $L_2^{SUPERDIAG}$
must be different from both the enumerated languages and the
enumerated DIAG languages.

\end{document}
