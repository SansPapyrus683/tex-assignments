\documentclass[12pt]{article}

% a template that a friend gave, it's worked well enough for me
% i have added some packages and stuff that have proved useful

\usepackage{fancyhdr}
\usepackage{tipa}
\usepackage{fontspec}
\usepackage{amsfonts}
\usepackage{enumitem}
\usepackage[margin=1in]{geometry}
\usepackage{graphicx}
\usepackage{float}
\usepackage{amsmath}
\usepackage{braket}
\usepackage{amssymb}
\usepackage{booktabs}
\usepackage{hyperref}
\usepackage{mathtools}
\usepackage{xcolor}
\usepackage{float}
\usepackage{algpseudocodex}
\usepackage{titlesec}
\usepackage{bbm}

\pagestyle{fancy}
\fancyhf{} % sets both header and footer to nothing
\lhead{Kevin Sheng}
\setmainfont{Comic Neue}
\renewcommand{\headrulewidth}{1pt}
\setlength{\headheight}{0.75in}
\setlength{\oddsidemargin}{0in}
\setlength{\evensidemargin}{0in}
\setlength{\voffset}{-.5in}
\setlength{\headsep}{10pt}
\setlength{\textwidth}{6.5in}
\setlength{\headwidth}{6.5in}
\setlength{\textheight}{8in}
\renewcommand{\headrulewidth}{0.5pt}
\renewcommand{\footrulewidth}{0.3pt}
\setlength{\textwidth}{6.5in}
\usepackage{setspace}
\usepackage{multicol}
\usepackage{float}
\setlength{\columnsep}{1cm}
\setlength\parindent{24pt}
\usepackage [english]{babel}
\usepackage [autostyle, english = american]{csquotes}
\MakeOuterQuote{"}

\setlength{\parskip}{6pt}
\setlength{\parindent}{0pt}

\titlespacing\section{0pt}{12pt plus 4pt minus 2pt}{0pt plus 2pt minus 2pt}
\titlespacing\subsection{0pt}{12pt plus 4pt minus 2pt}{0pt plus 2pt minus 2pt}
\titlespacing\subsubsection{0pt}{12pt plus 4pt minus 2pt}{0pt plus 2pt minus 2pt}

\hypersetup{colorlinks=true, urlcolor=blue}

\newcommand{\correction}[1]{\textcolor{red}{#1}}


\lhead{406-196-414}
\rhead{CS 181}

\begin{document}

\section{Pumping Lemma Limits}

\subsection{Part A}

Let's just let $q=5$ to make things more concrete.
It doesn't particularly matter which one we choose, though.

Notice that there's two types of strings we have to handle:
\begin{enumerate}[nolistsep]
    \item Those of the form $b^i$, where $i \ge 0$.
    \item And those of the form $a^i b^p$, where $i>0$ and $p$ is prime.
\end{enumerate}

If our string is of the first type, we can pick $x=\epsilon$, $y=b$, and $z=b^{i-1}$.
Then $xy^r z$=$b^{i-1+r}$, which still satisfies $L_2$.

If it's of the second type, we can pick $x=\epsilon$, $y=a$,
and $z$ as whatever's left to get $xy^r z=a^r a^{i-1} b^p=a^{i-1+r}b^p$.
$i-1+r \ge 0$, and $p$ is left untouched, so this new string is still in $L_2$.

As we can see, in both cases we can find a valid decomposition
of the string such that the Pumping Lemma still holds.

\subsection{Part B} \label{subsec:generalization}

$L$ is regular, which means we can create an FSM that implements it.
Let $q-1$ be the number of states in it.
That is, the $q$ we give is going to be one more than the number of states.

For any input string $w=xyz \in L$ with $|y| \ge q$, our machine must
go through the following sequence of states:
\begin{align*}
        & q_0 \to q_1 \to \cdots \to q_{|x|} \\
    \to{} & q_{|x|+1} \to \cdots \to q_{|xy|} \\
    \to{} & q_{|xy|+1} \to \cdots \to q_{|xyz|}
\end{align*}
Notice that when consuming $y$, the machine transitions through $q$ states.
By the pigeonhole principle, we can find $i$ and $j$ s.t. $1 \le i<j \le |y|$
and $q_{|x|+i}=q_{|y|+j}$.

We then decompose $y$ into $abc$, where:
\begin{itemize}[nolistsep]
    \item $a=y_1 \cdots y_i$
    \item $b=y_{i+1} \cdots y_j$, which we know is nonempty since again $i<j$.
    \item $c=y_{j+1} \cdots y_{|y|}$
\end{itemize}
Starting at $q_{|x|+i}$ and consuming $b$ brings it right back to where it was,
so we can inject $b$ any nonnegative number of times and the machine would still
come back to where it was.

Thus, for our given $q$, we've provided a decomposition of any $w \in L$
into $xabcz$ such that $|abc| \ge q$, $|b| \ge 1$, and $xab^i cz \in L\ \forall i \ge 0$. $\square$

\subsection{Part C}

BWOC say $L_2$ be a regular language.
This means the generalization of the Pumping Lemma we
\hyperref[subsec:generalization]{just proved} must hold.
Let $q$ be as described in the lemma.

Consider the string $ab^p$, where $p$ is any prime greater than $q$.
The generalized lemma holds true for any middle substring of length at least $q$,
so let's decompose it as follows:
\begin{itemize}[nolistsep]
    \item $x=a$
    \item $y=b^p$
    \item $z=\epsilon$
\end{itemize}
The lemma assures a valid breakdown of $y$ into $abc$ where $b \ne \epsilon$.
This means $ab^{p-|b|+i|b|} \in L$ as well for all nonnegative $i$.
Setting $i=p+1$, we have $ab^{p+p|b|} \in L$, which is a contradiction
since $p+p|b|=p \cdot (1+|b|)$, which clearly isn't prime.

Thus, $L_2$ cannot be regular. $\square$

\pagebreak

\section{\texorpdfstring{$\frac{1}{3}-\frac{1}{3}$}{1/3-1/3} Languages}

It suffices to give an $L$ s.t. $L_{\frac{1}{3}-\frac{1}{3}}$ isn't regular.

Let $\Sigma=\{0, 1, 2\}$ and $L=0^* 2 1^*$.
We've defined $L$ only using one regex, so by the equivalence between
regexes and regular languages proved in class it must be regular.

Now consider the $L_{\frac{1}{3}-\frac{1}{3}}$ built from this language and
let $t$ be its pumping length.
Consider the string $x=0^t 1^t \in L_{\frac{1}{3}-\frac{1}{3}}$.
We know that this string is in it because we can construct
$0^{t+\frac{t}{2}-1} 2 1^{t+\frac{t}{2}-1}$, give or take a couple
$0$s and $1$s in the middle to account for divisibility issues.

By the lemma, we have a decomposition of $x$ into $uvw$ s.t. $|uv| \le t$ and $v \ne \epsilon$.
Since the first $t$ characters of $x$ are all $0$s, $u=1^a$ and $v=1^b$.
We can repeat $v$ any number of times, so
\[\forall n \in \mathbb{Z}^+\ 1^{t+nb}0^t \in L_{\frac{1}{3}-\frac{1}{3}}\]

Now take $n$ to be incredibly large, say, $100t$.
We'd have at least $101t$ $1$s followed by only $t$ $0$s.
The condition of the language requires us to split this in half,
so the first third of our string in $L$ must be $51t$ $1$s
while the last third of it has to be $50t$ $1$s and $t$ $0$s.

Notice that since there's a $2$ between the $0^*$ and $1^*$ in our regex,
there can't be any substrings equivalent to $01$.
However, we've just shown that the last third of some string in $L$
contains $01$ as a substring by assuming the regularity of $L_{\frac{1}{3}-\frac{1}{3}}$.
Contradiction.

Thus, $L$ being regular doesn't force $L_{\frac{1}{3}-\frac{1}{3}}$ to be regular as well.

\end{document}
